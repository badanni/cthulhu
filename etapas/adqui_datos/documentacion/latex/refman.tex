\documentclass[a4paper]{book}
\usepackage{makeidx}
\usepackage{graphicx}
\usepackage{multicol}
\usepackage{float}
\usepackage{listings}
\usepackage{color}
\usepackage{ifthen}
\usepackage[table]{xcolor}
\usepackage{textcomp}
\usepackage{alltt}
\usepackage{ifpdf}
\ifpdf
\usepackage[pdftex,
            pagebackref=true,
            colorlinks=true,
            linkcolor=blue,
            unicode
           ]{hyperref}
\else
\usepackage[ps2pdf,
            pagebackref=true,
            colorlinks=true,
            linkcolor=blue,
            unicode
           ]{hyperref}
\usepackage{pspicture}
\fi
\usepackage[utf8]{inputenc}
\usepackage[spanish]{babel}
\usepackage{mathptmx}
\usepackage[scaled=.90]{helvet}
\usepackage{courier}
\usepackage{sectsty}
\usepackage[titles]{tocloft}
\usepackage{doxygen}
\lstset{language=C++,inputencoding=utf8,basicstyle=\footnotesize,breaklines=true,breakatwhitespace=true,tabsize=8,numbers=left }
\makeindex
\setcounter{tocdepth}{3}
\renewcommand{\footrulewidth}{0.4pt}
\renewcommand{\familydefault}{\sfdefault}
\begin{document}
\hypersetup{pageanchor=false}
\begin{titlepage}
\vspace*{7cm}
\begin{center}
{\Large Adquisicion de datos \\[1ex]\large 0.1 }\\
\vspace*{1cm}
{\large Generado por Doxygen 1.7.4}\\
\vspace*{0.5cm}
{\small Martes, 7 de Agosto de 2012 10:48:37}\\
\end{center}
\end{titlepage}
\clearemptydoublepage
\pagenumbering{roman}
\tableofcontents
\clearemptydoublepage
\pagenumbering{arabic}
\hypersetup{pageanchor=true}
\chapter{Adquisicion de datos para Pioneer P3-\/DX}
\label{index}\hypertarget{index}{}\input{index}
\chapter{Lista de tareas pendientes}
\label{todo}
\hypertarget{todo}{}
\label{todo__todo000001}
\hypertarget{todo__todo000001}{}
 
\begin{DoxyDescription}
\item[page \hyperlink{index}{Adquisicion de datos para Pioneer P3-\/DX} ]Presentar borrador 
\end{DoxyDescription}
\chapter{Lista de bugs}
\label{bug}
\hypertarget{bug}{}
\label{bug__bug000001}
\hypertarget{bug__bug000001}{}
 
\begin{DoxyDescription}
\item[Namespace \hyperlink{namespacecliente__lib}{cliente\_\-lib} ]myTemperature tiene falla por el tipo de dato en formato a python '' toca pasar a '0x81' 
\end{DoxyDescription}

\label{bug__bug000002}
\hypertarget{bug__bug000002}{}
 
\begin{DoxyDescription}
\item[Namespace \hyperlink{namespaceservidor}{servidor} ]Nada 
\end{DoxyDescription}
\chapter{Indice de namespaces}
\section{Paquetes}
Aquí van los paquetes con una breve descripción (si etá disponible):\begin{DoxyCompactList}
\item\contentsline{section}{\hyperlink{namespacecliente__lib}{cliente\_\-lib} (Libreria para realizar el cliente )}{\pageref{namespacecliente__lib}}{}
\item\contentsline{section}{\hyperlink{namespacegamepad}{gamepad} }{\pageref{namespacegamepad}}{}
\item\contentsline{section}{\hyperlink{namespaceinicio}{inicio} }{\pageref{namespaceinicio}}{}
\item\contentsline{section}{\hyperlink{namespacerenderizado}{renderizado} (Programa que utiliza la libreria openCV para la generacion del lienzo )}{\pageref{namespacerenderizado}}{}
\item\contentsline{section}{\hyperlink{namespaceservidor__novo}{servidor\_\-novo} (Servidor para el Pioneer P3-\/DX )}{\pageref{namespaceservidor__novo}}{}
\end{DoxyCompactList}

\chapter{Índice de clases}
\section{Lista de clases}
Lista de las clases, estructuras, uniones e interfaces con una breve descripci�n:\begin{DoxyCompactList}
\item\contentsline{section}{\hyperlink{classcliente__lib_1_1cliente__lib}{cliente\_\-lib.cliente\_\-lib} (No dispone ninguna utilidad )}{\pageref{classcliente__lib_1_1cliente__lib}}{}
\item\contentsline{section}{\hyperlink{classinicio_1_1prueba__teleoperacion}{inicio.prueba\_\-teleoperacion} (Se lo crea como objeto para poder trabajar con las senales de la interfaz grafica )}{\pageref{classinicio_1_1prueba__teleoperacion}}{}
\end{DoxyCompactList}

\chapter{Indice de archivos}
\section{Lista de archivos}
Lista de todos los archivos con descripciones breves:\begin{DoxyCompactList}
\item\contentsline{section}{\hyperlink{cliente__lib_8py}{cliente\_\-lib.py} }{\pageref{cliente__lib_8py}}{}
\item\contentsline{section}{\hyperlink{cliente__lib__original_8py}{cliente\_\-lib\_\-original.py} }{\pageref{cliente__lib__original_8py}}{}
\item\contentsline{section}{\hyperlink{inicio_8py}{inicio.py} }{\pageref{inicio_8py}}{}
\item\contentsline{section}{\hyperlink{servidor_8py}{servidor.py} }{\pageref{servidor_8py}}{}
\end{DoxyCompactList}

\chapter{Documentación de namespaces}
\hypertarget{namespacecliente__lib}{
\section{Paquetes cliente\_\-lib}
\label{namespacecliente__lib}\index{cliente\_\-lib@{cliente\_\-lib}}
}


libreria para realizar el cliente  


\subsection*{Clases}
\begin{DoxyCompactItemize}
\item 
class \hyperlink{classcliente__lib_1_1cliente__lib}{cliente\_\-lib}
\begin{DoxyCompactList}\small\item\em es la clase encargada del cliente \item\end{DoxyCompactList}\end{DoxyCompactItemize}
\subsection*{Funciones}
\begin{DoxyCompactItemize}
\item 
def \hyperlink{namespacecliente__lib_afb746084e43cb9c21db470d7b4990cae}{main}
\begin{DoxyCompactList}\small\item\em Sirve para realizar pruebas de conexion. \item\end{DoxyCompactList}\end{DoxyCompactItemize}


\subsection{Descripci�n detallada}
libreria para realizar el cliente Se debe especificar cual es el IP del servidor \begin{DoxyAuthor}{Autores}
Danny Vasconez 

Daniel Granda 
\end{DoxyAuthor}
\begin{DoxyVersion}{Versi�n}
0.0.2 
\end{DoxyVersion}
\begin{DoxyDate}{Fecha}
2012 
\end{DoxyDate}
\begin{DoxyPrecond}{Precondici�n}
Tener funcionando el servidor 
\end{DoxyPrecond}
\begin{Desc}
\item[\hyperlink{bug__bug000001}{Bug}]myTemperature tiene falla por el tipo de dato en formato a python '' toca pasar a '0x81' \end{Desc}
\begin{DoxyWarning}{Atenci�n}
uso inapropiado puede hacer que la aplicacion falle
\end{DoxyWarning}
\hypertarget{index_intro}{}\subsection{Ejemplo de uso}\label{index_intro}
En el ejemplo se muestra tres maneras de enviar comandos la general que es requestOnce y las otras que son la misma pero modificada para trabajar con comandos especificos 
\begin{DoxyVerbInclude}
a=cliente_lib() #Instancia a la clase cliete_lib
a.ip="192.168.1.10"
CLIENTE=a.cliente_inicio() 
# para realizar movimiento 
TransRatio,RotRatio,LatRatio = [-50,0,0]
CLIENTE=a.envio_ratioDrive(CLIENTE,TransRatio,RotRatio,LatRatio) #fijar los valores para mover
#
#para conocer valores como fisicos de la plataforma movil
CLIENTE=a.envio_consulta_fisica(CLIENTE,"updateNumbers")
valor=a.devuelve_valorf()
print valor
#
#para conocer los valores de los sonares
CLIENTE.requestOnce("pose")
valor=a.devuelve_valors()
print valor
#
ArUtil.sleep(1000)
a.cliente_apaga(CLIENTE)

\end{DoxyVerbInclude}
 

\subsection{Documentaci�n de las funciones}
\hypertarget{namespacecliente__lib_afb746084e43cb9c21db470d7b4990cae}{
\index{cliente\_\-lib@{cliente\_\-lib}!main@{main}}
\index{main@{main}!cliente_lib@{cliente\_\-lib}}
\subsubsection[{main}]{\setlength{\rightskip}{0pt plus 5cm}def cliente\_\-lib.main (
\begin{DoxyParamCaption}
{}
\end{DoxyParamCaption}
)}}
\label{namespacecliente__lib_afb746084e43cb9c21db470d7b4990cae}


Sirve para realizar pruebas de conexion. 

sin tener que ejecutar la aplicacion completa; de la siguiente forma \char`\"{}python2.5 cliente\_\-lib.py\char`\"{} \begin{DoxyReturn}{Devuelve}
0 
\end{DoxyReturn}


Definici�n en la l�nea 236 del archivo cliente\_\-lib.py.




\begin{DoxyCode}
237           :
238         #prueba de la libreria
239         a=cliente_lib()
240         a.ip="192.168.1.10" #Si el servidor esta en otra maquina 
241         CLIENTE=a.cliente_inicio()
242         TransRatio,RotRatio,LatRatio = [-50,0,0]
243         CLIENTE=a.envio_ratioDrive(CLIENTE,TransRatio,RotRatio,LatRatio) #fijar l
      os valores para mover
244         CLIENTE=a.envio_consulta_fisica(CLIENTE,"updateNumbers")
245         valor=a.devuelve_valorf()
246         print valor
247         CLIENTE.requestOnce("pose")
248         ArUtil.sleep(1000)
249         a.cliente_apaga(CLIENTE)
        return 0
\end{DoxyCode}



\hypertarget{namespacecliente__lib__original}{
\section{Paquetes cliente\_\-lib\_\-original}
\label{namespacecliente__lib__original}\index{cliente\_\-lib\_\-original@{cliente\_\-lib\_\-original}}
}
\subsection*{Clases}
\begin{DoxyCompactItemize}
\item 
class \hyperlink{classcliente__lib__original_1_1cliente__lib}{cliente\_\-lib}
\end{DoxyCompactItemize}
\subsection*{Funciones}
\begin{DoxyCompactItemize}
\item 
def \hyperlink{namespacecliente__lib__original_a41a934cc972855341f15854b1daff426}{main}
\begin{DoxyCompactList}\small\item\em Sirve para realizar pruebas de conexion. \end{DoxyCompactList}\end{DoxyCompactItemize}


\subsection{Documentación de las funciones}
\hypertarget{namespacecliente__lib__original_a41a934cc972855341f15854b1daff426}{
\index{cliente\_\-lib\_\-original@{cliente\_\-lib\_\-original}!main@{main}}
\index{main@{main}!cliente_lib_original@{cliente\_\-lib\_\-original}}
\subsubsection[{main}]{\setlength{\rightskip}{0pt plus 5cm}def cliente\_\-lib\_\-original.main (
\begin{DoxyParamCaption}
{}
\end{DoxyParamCaption}
)}}
\label{namespacecliente__lib__original_a41a934cc972855341f15854b1daff426}


Sirve para realizar pruebas de conexion. 

sin tener que ejecutar la aplicacion completa; de la siguiente forma \char`\"{}python2.5 cliente\_\-lib.py\char`\"{} \begin{DoxyReturn}{Devuelve}
0 
\end{DoxyReturn}


Definición en la línea 273 del archivo cliente\_\-lib\_\-original.py.


\begin{DoxyCode}
274           :
275         #prueba de la libreria
276         a=cliente_lib()
277         a.ip="192.168.0.124" #Si el servidor esta en otra maquina 
278         CLIENTE=a.cliente_inicio()
279         TransRatio,RotRatio,LatRatio = [-50,0,0]
280         CLIENTE=a.envio_ratioDrive(CLIENTE,TransRatio,RotRatio,LatRatio) #fijar l
      os valores para mover
281         CLIENTE=a.envio_consulta_fisica(CLIENTE,"updateNumbers")
282         valor=a.devuelve_valorf()
283         print valor
284         CLIENTE.requestOnce("pose")
285         ArUtil.sleep(1000)
286         a.cliente_apaga(CLIENTE)
        return 0
\end{DoxyCode}

\hypertarget{namespaceinicio}{
\section{Paquetes inicio}
\label{namespaceinicio}\index{inicio@{inicio}}
}
\subsection*{Clases}
\begin{DoxyCompactItemize}
\item 
class \hyperlink{classinicio_1_1prueba__adqui}{prueba\_\-adqui}
\begin{DoxyCompactList}\small\item\em es la clase encargada del entorno grafico y enlace con \hyperlink{namespacecliente__lib}{cliente\_\-lib} \end{DoxyCompactList}\end{DoxyCompactItemize}
\subsection*{Funciones}
\begin{DoxyCompactItemize}
\item 
def \hyperlink{namespaceinicio_a518864d4ff815064f5de420ab3996d94}{main}
\begin{DoxyCompactList}\small\item\em El encargado al momento de ejcutar la aplicacion de instanciar el objeto prueba\_\-teleoperacion. \end{DoxyCompactList}\end{DoxyCompactItemize}


\subsection{Documentación de las funciones}
\hypertarget{namespaceinicio_a518864d4ff815064f5de420ab3996d94}{
\index{inicio@{inicio}!main@{main}}
\index{main@{main}!inicio@{inicio}}
\subsubsection[{main}]{\setlength{\rightskip}{0pt plus 5cm}def inicio.main (
\begin{DoxyParamCaption}
{}
\end{DoxyParamCaption}
)}}
\label{namespaceinicio_a518864d4ff815064f5de420ab3996d94}


El encargado al momento de ejcutar la aplicacion de instanciar el objeto prueba\_\-teleoperacion. 



Definición en la línea 267 del archivo inicio.py.


\begin{DoxyCode}
268           :
269         if os.name=="posix": #verifica que sea un entorno Linux
270                 dimensiones=(600,400)
271                 nombre_archivo="filenamea.jpg"
272                 dimensiones_robot=(13,13)
273                 app = prueba_adqui(dimensiones,nombre_archivo,dimensiones_robot) 
      #instancia el objeto GUI con las senales enlazadas
274                 gtk.main() #levanta el motor GTK para poder visualizar
275   

\end{DoxyCode}

\hypertarget{namespaceservidor}{
\section{Paquetes servidor}
\label{namespaceservidor}\index{servidor@{servidor}}
}


Servidor para el Pioneer P3-\/DX.  




\subsection{Descripción detallada}
Servidor para el Pioneer P3-\/DX. Se lo puede utilizar con multiples conexiones de clientes trabaja en el puerto 7272 La lista de comandos para el paquete ArNetPacket se encuentra en el anexo 1 \begin{DoxyAuthor}{Autores}
Danny Vasconez 

Daniel Granda 
\end{DoxyAuthor}
\begin{DoxyVersion}{Versión}
0.0.2 
\end{DoxyVersion}
\begin{DoxyDate}{Fecha}
2012 
\end{DoxyDate}
\begin{DoxyPrecond}{Precondición}
Tener conectada la plataforma Pioneer P3-\/DX. 
\end{DoxyPrecond}
\begin{Desc}
\item[\hyperlink{bug__bug000003}{Bug}]Nada \end{Desc}
\begin{DoxyWarning}{Atención}
uso inapropiado puede hacer que la aplicacion falle 
\end{DoxyWarning}

\chapter{Documentación de las clases}
\hypertarget{classcliente__lib_1_1cliente__lib}{
\section{Referencia de la Clase cliente\_\-lib.cliente\_\-lib}
\label{classcliente__lib_1_1cliente__lib}\index{cliente\_\-lib::cliente\_\-lib@{cliente\_\-lib::cliente\_\-lib}}
}


es la clase encargada del cliente  


\subsection*{M�todos p�blicos}
\begin{DoxyCompactItemize}
\item 
def \hyperlink{classcliente__lib_1_1cliente__lib_ac5e4490f412835d35481f58d1ae503f9}{\_\-\_\-init\_\-\_\-}
\begin{DoxyCompactList}\small\item\em Carga valores a las variables necesarias para funcionar el cliente. \item\end{DoxyCompactList}\item 
def \hyperlink{classcliente__lib_1_1cliente__lib_af7b751bcf94c96150b23bacb5e477956}{valores}
\begin{DoxyCompactList}\small\item\em Sirve cuando se manda el comando \char`\"{}updateNumbers\char`\"{} en el paquete. \item\end{DoxyCompactList}\item 
def \hyperlink{classcliente__lib_1_1cliente__lib_ac0a4410b48b4c759028bec6ae1c641e8}{lista\_\-sonares}
\begin{DoxyCompactList}\small\item\em Sirve para leer el paquete arNetPacket con la lista del sonar. \item\end{DoxyCompactList}\item 
def \hyperlink{classcliente__lib_1_1cliente__lib_abcf28c2207cb5519090654484137db23}{valores\_\-sonares}
\begin{DoxyCompactList}\small\item\em Sirve para leer el paquete arNetPacket con los valores del sonar. \item\end{DoxyCompactList}\item 
def \hyperlink{classcliente__lib_1_1cliente__lib_acfc22af72a1668db28d18ab4ff40909e}{envio\_\-ratioDrive}
\begin{DoxyCompactList}\small\item\em Sirve para realizar la teleoperacion, mandando los parametros. \item\end{DoxyCompactList}\item 
def \hyperlink{classcliente__lib_1_1cliente__lib_ade1f44e9270c8835c284832a72b96b6c}{uC\_\-comandos\_\-movi}
\begin{DoxyCompactList}\small\item\em Sirve para mandar ordenes de movimiento directamente al controlador de la plataforma movil. \item\end{DoxyCompactList}\item 
def \hyperlink{classcliente__lib_1_1cliente__lib_ac3e89d3066207b05b217a50f549c239a}{envio\_\-consulta\_\-fisica}
\begin{DoxyCompactList}\small\item\em Sirve para mandar ordenes al servidor utilizando paquetes ArNetPacket con comandos {\bfseries pose} y {\bfseries updateNumbers} \item\end{DoxyCompactList}\item 
def \hyperlink{classcliente__lib_1_1cliente__lib_a1a7b5475a98772f0e48a4e1fd76e8d47}{cliente\_\-inicio}
\begin{DoxyCompactList}\small\item\em Sirve para iniciar la conexion con el servidor. \item\end{DoxyCompactList}\item 
def \hyperlink{classcliente__lib_1_1cliente__lib_a14a49495fd71fab84d36060e604415a5}{cliente\_\-apaga}
\begin{DoxyCompactList}\small\item\em Sirve para realizar la desconexion con el servidor. \item\end{DoxyCompactList}\item 
def \hyperlink{classcliente__lib_1_1cliente__lib_a3ca67c0c9d7f0a622abd740c780f64d1}{devuelve\_\-valorf}
\begin{DoxyCompactList}\small\item\em Devuelve el variable valor\_\-fisico, con usa espera de 100ms. \item\end{DoxyCompactList}\item 
def \hyperlink{classcliente__lib_1_1cliente__lib_a127d026872fbd11f4f5bbe4a73424b77}{devuelve\_\-valors}
\begin{DoxyCompactList}\small\item\em Devuelve el variable valor\_\-sonares, con usa espera de 100ms. \item\end{DoxyCompactList}\end{DoxyCompactItemize}
\subsection*{Atributos p�blicos}
\begin{DoxyCompactItemize}
\item 
\hyperlink{classcliente__lib_1_1cliente__lib_abba3409f89ee8dcec8b180c90aa5d77c}{valor\_\-fisico}
\item 
\hyperlink{classcliente__lib_1_1cliente__lib_aadea6e24bd3a01b0500fcc67543a97e9}{valor\_\-sonares}
\item 
\hyperlink{classcliente__lib_1_1cliente__lib_a675dd8430aa2eeb33240b8b07ed61543}{ip}
\end{DoxyCompactItemize}


\subsection{Descripci�n detallada}
es la clase encargada del cliente se lo utiliza de esta manera par poder trabajar con la informacion tanto leyendo las variables o utilizando los comandos 

Definici�n en la l�nea 50 del archivo cliente\_\-lib.py.



\subsection{Documentaci�n de las funciones miembro}
\hypertarget{classcliente__lib_1_1cliente__lib_ac5e4490f412835d35481f58d1ae503f9}{
\index{cliente\_\-lib::cliente\_\-lib@{cliente\_\-lib::cliente\_\-lib}!\_\-\_\-init\_\-\_\-@{\_\-\_\-init\_\-\_\-}}
\index{\_\-\_\-init\_\-\_\-@{\_\-\_\-init\_\-\_\-}!cliente_lib::cliente_lib@{cliente\_\-lib::cliente\_\-lib}}
\subsubsection[{\_\-\_\-init\_\-\_\-}]{\setlength{\rightskip}{0pt plus 5cm}def cliente\_\-lib.cliente\_\-lib.\_\-\_\-init\_\-\_\- (
\begin{DoxyParamCaption}
\item[{}]{ self}
\end{DoxyParamCaption}
)}}
\label{classcliente__lib_1_1cliente__lib_ac5e4490f412835d35481f58d1ae503f9}


Carga valores a las variables necesarias para funcionar el cliente. 

este comando no es necesario utilizarlo es usado al instanciar la clase 
\begin{DoxyParams}{Par�metros}
\item[{\em self}]este parametro no es necesario escribir \end{DoxyParams}


Definici�n en la l�nea 57 del archivo cliente\_\-lib.py.




\begin{DoxyCode}
58                           :
59                 print "Cargo modulo para cliente_lib"
60                 self.valor_fisico=[]
61                 self.valor_sonares=[] 
62                 self.ip="localhost"

\end{DoxyCode}


\hypertarget{classcliente__lib_1_1cliente__lib_a14a49495fd71fab84d36060e604415a5}{
\index{cliente\_\-lib::cliente\_\-lib@{cliente\_\-lib::cliente\_\-lib}!cliente\_\-apaga@{cliente\_\-apaga}}
\index{cliente\_\-apaga@{cliente\_\-apaga}!cliente_lib::cliente_lib@{cliente\_\-lib::cliente\_\-lib}}
\subsubsection[{cliente\_\-apaga}]{\setlength{\rightskip}{0pt plus 5cm}def cliente\_\-lib.cliente\_\-lib.cliente\_\-apaga (
\begin{DoxyParamCaption}
\item[{}]{ self, }
\item[{}]{ client}
\end{DoxyParamCaption}
)}}
\label{classcliente__lib_1_1cliente__lib_a14a49495fd71fab84d36060e604415a5}


Sirve para realizar la desconexion con el servidor. 


\begin{DoxyParams}{Par�metros}
\item[{\em self}]este parametro no es necesario escribir \item[{\em client}]para poder desconectar el cliente \end{DoxyParams}


Definici�n en la l�nea 208 del archivo cliente\_\-lib.py.




\begin{DoxyCode}
209                                       :
210                 ArUtil.sleep(1000)
211                 client.disconnect()
212                 ArUtil.sleep(50)
                return 0
\end{DoxyCode}


\hypertarget{classcliente__lib_1_1cliente__lib_a1a7b5475a98772f0e48a4e1fd76e8d47}{
\index{cliente\_\-lib::cliente\_\-lib@{cliente\_\-lib::cliente\_\-lib}!cliente\_\-inicio@{cliente\_\-inicio}}
\index{cliente\_\-inicio@{cliente\_\-inicio}!cliente_lib::cliente_lib@{cliente\_\-lib::cliente\_\-lib}}
\subsubsection[{cliente\_\-inicio}]{\setlength{\rightskip}{0pt plus 5cm}def cliente\_\-lib.cliente\_\-lib.cliente\_\-inicio (
\begin{DoxyParamCaption}
\item[{}]{ self}
\end{DoxyParamCaption}
)}}
\label{classcliente__lib_1_1cliente__lib_a1a7b5475a98772f0e48a4e1fd76e8d47}


Sirve para iniciar la conexion con el servidor. 


\begin{DoxyParams}{Par�metros}
\item[{\em self}]este parametro no es necesario escribir \end{DoxyParams}
\begin{DoxyReturn}{Devuelve}
client 
\end{DoxyReturn}


Definici�n en la l�nea 178 del archivo cliente\_\-lib.py.




\begin{DoxyCode}
179                                 :
180                 client = ArClientBase()
181                 Aria.init()
182                 
183                 startTime = ArTime()
184                 startTime.setToNow()
185                 if not client.blockingConnect(self.ip, 7272): #ip y puerto del se
      rvidor
186                         print "Could not connect to server at localhost port 7272
      , exiting"
187                         Aria.exit(1);
188                 print "cliente: Se tardo %ld msec en connectarse\n" % (startTime.
      mSecSince())
189                 
190                 client.runAsync()
191                 client.addHandler("updateNumbers",self.valores)
192                 client.addHandler("getSensorList",self.lista_sonares)
193                 client.addHandler("pose",self.valores_sonares)
194                 if client.dataExists("ratioDrive"): #supuestamente devuelve la in
      fo del robot con odometria
195                         print "ratioDrive si existe"
196                 else:
197                         Aria.exit(1);
198                 #client=envio_ratioDrive(client,TransRatio,RotRatio,LatRatio) #fi
      jar los valores para mover
199                 #client=uC_comandos_movi(client,comando,parametro) #Lo hace de un
      a manera directa anulando las demas operaciones
200                 #client.requestOnce("updateNumbers")
201                 #client.requestOnce("stop") #parada de emergencia
                return client
\end{DoxyCode}


\hypertarget{classcliente__lib_1_1cliente__lib_a3ca67c0c9d7f0a622abd740c780f64d1}{
\index{cliente\_\-lib::cliente\_\-lib@{cliente\_\-lib::cliente\_\-lib}!devuelve\_\-valorf@{devuelve\_\-valorf}}
\index{devuelve\_\-valorf@{devuelve\_\-valorf}!cliente_lib::cliente_lib@{cliente\_\-lib::cliente\_\-lib}}
\subsubsection[{devuelve\_\-valorf}]{\setlength{\rightskip}{0pt plus 5cm}def cliente\_\-lib.cliente\_\-lib.devuelve\_\-valorf (
\begin{DoxyParamCaption}
\item[{}]{ self}
\end{DoxyParamCaption}
)}}
\label{classcliente__lib_1_1cliente__lib_a3ca67c0c9d7f0a622abd740c780f64d1}


Devuelve el variable valor\_\-fisico, con usa espera de 100ms. 


\begin{DoxyParams}{Par�metros}
\item[{\em self}]este parametro no es necesario escribir \end{DoxyParams}


Definici�n en la l�nea 218 del archivo cliente\_\-lib.py.




\begin{DoxyCode}
219                                  :
220                 ArUtil.sleep(100)
                return self.valor_fisico
\end{DoxyCode}


\hypertarget{classcliente__lib_1_1cliente__lib_a127d026872fbd11f4f5bbe4a73424b77}{
\index{cliente\_\-lib::cliente\_\-lib@{cliente\_\-lib::cliente\_\-lib}!devuelve\_\-valors@{devuelve\_\-valors}}
\index{devuelve\_\-valors@{devuelve\_\-valors}!cliente_lib::cliente_lib@{cliente\_\-lib::cliente\_\-lib}}
\subsubsection[{devuelve\_\-valors}]{\setlength{\rightskip}{0pt plus 5cm}def cliente\_\-lib.cliente\_\-lib.devuelve\_\-valors (
\begin{DoxyParamCaption}
\item[{}]{ self}
\end{DoxyParamCaption}
)}}
\label{classcliente__lib_1_1cliente__lib_a127d026872fbd11f4f5bbe4a73424b77}


Devuelve el variable valor\_\-sonares, con usa espera de 100ms. 


\begin{DoxyParams}{Par�metros}
\item[{\em self}]este parametro no es necesario escribir \end{DoxyParams}


Definici�n en la l�nea 226 del archivo cliente\_\-lib.py.




\begin{DoxyCode}
227                                  :
228                 ArUtil.sleep(100)
229                 return self.valor_sonares

\end{DoxyCode}


\hypertarget{classcliente__lib_1_1cliente__lib_ac3e89d3066207b05b217a50f549c239a}{
\index{cliente\_\-lib::cliente\_\-lib@{cliente\_\-lib::cliente\_\-lib}!envio\_\-consulta\_\-fisica@{envio\_\-consulta\_\-fisica}}
\index{envio\_\-consulta\_\-fisica@{envio\_\-consulta\_\-fisica}!cliente_lib::cliente_lib@{cliente\_\-lib::cliente\_\-lib}}
\subsubsection[{envio\_\-consulta\_\-fisica}]{\setlength{\rightskip}{0pt plus 5cm}def cliente\_\-lib.cliente\_\-lib.envio\_\-consulta\_\-fisica (
\begin{DoxyParamCaption}
\item[{}]{ self, }
\item[{}]{ client, }
\item[{}]{ mensaje}
\end{DoxyParamCaption}
)}}
\label{classcliente__lib_1_1cliente__lib_ac3e89d3066207b05b217a50f549c239a}


Sirve para mandar ordenes al servidor utilizando paquetes ArNetPacket con comandos {\bfseries pose} y {\bfseries updateNumbers} 


\begin{DoxyParams}{Par�metros}
\item[{\em self}]este parametro no es necesario escribir \item[{\em client}]Se debe trar el objeto cliente a la definicion para poder utilizar el enlace del cliente para enviar el paquete al servidor \item[{\em mensaje}]puede ser cualquier comando del servidor que no devuelva informacion a exepcion de pose y updatenumbers \end{DoxyParams}
\begin{DoxyReturn}{Devuelve}
client 
\end{DoxyReturn}


Definici�n en la l�nea 167 del archivo cliente\_\-lib.py.




\begin{DoxyCode}
168                                                       :
169                 ## se puede usar pose y updateNumbers
170                 client.requestOnce(mensaje)
171                 return client
                
\end{DoxyCode}


\hypertarget{classcliente__lib_1_1cliente__lib_acfc22af72a1668db28d18ab4ff40909e}{
\index{cliente\_\-lib::cliente\_\-lib@{cliente\_\-lib::cliente\_\-lib}!envio\_\-ratioDrive@{envio\_\-ratioDrive}}
\index{envio\_\-ratioDrive@{envio\_\-ratioDrive}!cliente_lib::cliente_lib@{cliente\_\-lib::cliente\_\-lib}}
\subsubsection[{envio\_\-ratioDrive}]{\setlength{\rightskip}{0pt plus 5cm}def cliente\_\-lib.cliente\_\-lib.envio\_\-ratioDrive (
\begin{DoxyParamCaption}
\item[{}]{ self, }
\item[{}]{ client, }
\item[{}]{ TransRatio, }
\item[{}]{ RotRatio, }
\item[{}]{ LatRatio}
\end{DoxyParamCaption}
)}}
\label{classcliente__lib_1_1cliente__lib_acfc22af72a1668db28d18ab4ff40909e}


Sirve para realizar la teleoperacion, mandando los parametros. 


\begin{DoxyParams}{Par�metros}
\item[{\em self}]este parametro no es necesario escribir \item[{\em client}]Se debe trar el objeto cliente a la definicion para poder utilizar el enlace del cliente para enviar el paquete al servidor \item[{\em TransRatio}]Velocidad de traslacion \item[{\em RotRatio}]Velocidad de rotacion \item[{\em LatRatio}]velocidad lateral para el modelo Pioneer P3-\/DX no se necesario puede ser 0 \end{DoxyParams}
\begin{DoxyReturn}{Devuelve}
client 
\end{DoxyReturn}


Definici�n en la l�nea 130 del archivo cliente\_\-lib.py.




\begin{DoxyCode}
131                                                                       :
132                 myTransRatio=TransRatio
133                 myRotRatio=RotRatio
134                 myLatRatio=LatRatio
135                 packet=ArNetPacket()
136                 packet.doubleToBuf(myTransRatio)
137                 packet.doubleToBuf(myRotRatio)
138                 packet.doubleToBuf(50) # use half of the robot's maximum.
139                 packet.doubleToBuf(myLatRatio)
140                 client.requestOnce("ratioDrive", packet)
141                 return client
          
\end{DoxyCode}


\hypertarget{classcliente__lib_1_1cliente__lib_ac0a4410b48b4c759028bec6ae1c641e8}{
\index{cliente\_\-lib::cliente\_\-lib@{cliente\_\-lib::cliente\_\-lib}!lista\_\-sonares@{lista\_\-sonares}}
\index{lista\_\-sonares@{lista\_\-sonares}!cliente_lib::cliente_lib@{cliente\_\-lib::cliente\_\-lib}}
\subsubsection[{lista\_\-sonares}]{\setlength{\rightskip}{0pt plus 5cm}def cliente\_\-lib.cliente\_\-lib.lista\_\-sonares (
\begin{DoxyParamCaption}
\item[{}]{ self, }
\item[{}]{ packet}
\end{DoxyParamCaption}
)}}
\label{classcliente__lib_1_1cliente__lib_ac0a4410b48b4c759028bec6ae1c641e8}


Sirve para leer el paquete arNetPacket con la lista del sonar. 

este comando no es necesario utilizarlo es usado solo por el cliente para procesar el paquete 
\begin{DoxyParams}{Par�metros}
\item[{\em self}]este parametro no es necesario escribir \item[{\em packet}]este parametro no es necesario escribir \end{DoxyParams}
\begin{DoxyReturn}{Devuelve}
nada 
\end{DoxyReturn}


Definici�n en la l�nea 94 del archivo cliente\_\-lib.py.




\begin{DoxyCode}
95                                       :
96                 c="                                   "
97                 numSensor=packet.bufToByte2()
98                 numSensor2=packet.bufToStr(c,15)
99                 print str(numSensor)+" "+str(c.strip())

\end{DoxyCode}


\hypertarget{classcliente__lib_1_1cliente__lib_ade1f44e9270c8835c284832a72b96b6c}{
\index{cliente\_\-lib::cliente\_\-lib@{cliente\_\-lib::cliente\_\-lib}!uC\_\-comandos\_\-movi@{uC\_\-comandos\_\-movi}}
\index{uC\_\-comandos\_\-movi@{uC\_\-comandos\_\-movi}!cliente_lib::cliente_lib@{cliente\_\-lib::cliente\_\-lib}}
\subsubsection[{uC\_\-comandos\_\-movi}]{\setlength{\rightskip}{0pt plus 5cm}def cliente\_\-lib.cliente\_\-lib.uC\_\-comandos\_\-movi (
\begin{DoxyParamCaption}
\item[{}]{ self, }
\item[{}]{ client, }
\item[{}]{ comando, }
\item[{}]{ parametro}
\end{DoxyParamCaption}
)}}
\label{classcliente__lib_1_1cliente__lib_ade1f44e9270c8835c284832a72b96b6c}


Sirve para mandar ordenes de movimiento directamente al controlador de la plataforma movil. 


\begin{DoxyParams}{Par�metros}
\item[{\em self}]este parametro no es necesario escribir \item[{\em client}]Se debe trar el objeto cliente a la definicion para poder utilizar el enlace del cliente para enviar el paquete al servidor \item[{\em comando}]es un numero de 1-\/255 que representa una funcion esta informacion se puede encontrar en el API de ARIA \item[{\em parametro}]el parametro de la funcion en caso de no tener se deja el valor en blanco \end{DoxyParams}
\begin{DoxyReturn}{Devuelve}
client 
\end{DoxyReturn}


Definici�n en la l�nea 151 del archivo cliente\_\-lib.py.




\begin{DoxyCode}
152                                                            :
153                 mi_comando=comando #comando 8 es MOVE parametro un valor de 5000 
      a -4999 es en mm, 11 LEV y su parametro es velocidad +o- mm/s
154                 mi_parametro=parametro #parametro
155                 packet=ArNetPacket()
156                 packet.strToBuf(mi_comando+" "+mi_parametro)
157                 client.requestOnce("MicroControllerMotionCommand", packet) #Micro
      ControllerMotionCommand
158                 return client

\end{DoxyCode}


\hypertarget{classcliente__lib_1_1cliente__lib_af7b751bcf94c96150b23bacb5e477956}{
\index{cliente\_\-lib::cliente\_\-lib@{cliente\_\-lib::cliente\_\-lib}!valores@{valores}}
\index{valores@{valores}!cliente_lib::cliente_lib@{cliente\_\-lib::cliente\_\-lib}}
\subsubsection[{valores}]{\setlength{\rightskip}{0pt plus 5cm}def cliente\_\-lib.cliente\_\-lib.valores (
\begin{DoxyParamCaption}
\item[{}]{ self, }
\item[{}]{ packet}
\end{DoxyParamCaption}
)}}
\label{classcliente__lib_1_1cliente__lib_af7b751bcf94c96150b23bacb5e477956}


Sirve cuando se manda el comando \char`\"{}updateNumbers\char`\"{} en el paquete. 

este comando no es necesario utilizarlo es usado solo por el cliente para procesar el paquete 
\begin{DoxyParams}{Par�metros}
\item[{\em self}]este parametro no es necesario escribir \item[{\em packet}]el paquete que recibe el cliente del servidor, no es necesario escribir \end{DoxyParams}
\begin{DoxyReturn}{Devuelve}
Nada, pero guarda en self.valores\_\-fisico \mbox{[}voltaje\_\-bateria,myX,myY,myTh,myVel,myRotVel,myLatVel,myTemperature\mbox{]} 
\end{DoxyReturn}


Definici�n en la l�nea 71 del archivo cliente\_\-lib.py.




\begin{DoxyCode}
72                                 :
73                 #devuelve los valores voltaje_bateria,myX,myY,myTh,myVel,myRotVel
      ,myLatVel,myTemperature
74                 voltaje_bateria=packet.bufToByte2()/10
75                 myX = packet.bufToByte4()#
76                 myY = packet.bufToByte4()
77                 myTh = packet.bufToByte2()
78                 myVel = packet.bufToByte2()
79                 myRotVel = packet.bufToByte2()
80                 myLatVel = packet.bufToByte2()
81                 myTemperature = packet.bufToByte()
82                 #print "X= "+str(myX)+" y="+str(myY)+" th="+str(myTh)
83                 self.valor_fisico=(voltaje_bateria,myX,myY,myTh,myVel,myRotVel,my
      LatVel,myTemperature)
84                 self.valor_fisico=(voltaje_bateria,myX,myY,myTh,myVel,myRotVel,my
      LatVel,0)
85                 #print valor

\end{DoxyCode}


\hypertarget{classcliente__lib_1_1cliente__lib_abcf28c2207cb5519090654484137db23}{
\index{cliente\_\-lib::cliente\_\-lib@{cliente\_\-lib::cliente\_\-lib}!valores\_\-sonares@{valores\_\-sonares}}
\index{valores\_\-sonares@{valores\_\-sonares}!cliente_lib::cliente_lib@{cliente\_\-lib::cliente\_\-lib}}
\subsubsection[{valores\_\-sonares}]{\setlength{\rightskip}{0pt plus 5cm}def cliente\_\-lib.cliente\_\-lib.valores\_\-sonares (
\begin{DoxyParamCaption}
\item[{}]{ self, }
\item[{}]{ packet}
\end{DoxyParamCaption}
)}}
\label{classcliente__lib_1_1cliente__lib_abcf28c2207cb5519090654484137db23}


Sirve para leer el paquete arNetPacket con los valores del sonar. 

este comando no es necesario utilizarlo es usado solo por el cliente para procesar el paquete 
\begin{DoxyParams}{Par�metros}
\item[{\em self}]este parametro no es necesario escribir \item[{\em packet}]este parametro no es necesario escribir \end{DoxyParams}
\begin{DoxyReturn}{Devuelve}
nada 
\end{DoxyReturn}


Definici�n en la l�nea 108 del archivo cliente\_\-lib.py.




\begin{DoxyCode}
109                                         :
110                 c="                                                              
                                                          "
111                 cantidad=packet.bufToDouble()
112                 dato=range(int(cantidad))
113                 for j in range(int(cantidad)):
114                         dato[j]=c #ya que pasa el buffer a la variable esta debe 
      tener la longitud necesaria y deben ser guardada en cada lectura
115                 for i in range(int(cantidad)):
116                         packet.bufToStr(dato[i],50)
117                         dato[i]=dato[i].replace("\x00"," ") #Porque al final anad
      e un NONE del string
118                         dato[i]=dato[i].strip() 
119                 self.valor_sonares=[cantidad,dato]

\end{DoxyCode}




\subsection{Documentaci�n de los datos miembro}
\hypertarget{classcliente__lib_1_1cliente__lib_a675dd8430aa2eeb33240b8b07ed61543}{
\index{cliente\_\-lib::cliente\_\-lib@{cliente\_\-lib::cliente\_\-lib}!ip@{ip}}
\index{ip@{ip}!cliente_lib::cliente_lib@{cliente\_\-lib::cliente\_\-lib}}
\subsubsection[{ip}]{\setlength{\rightskip}{0pt plus 5cm}{\bf cliente\_\-lib.cliente\_\-lib.ip}}}
\label{classcliente__lib_1_1cliente__lib_a675dd8430aa2eeb33240b8b07ed61543}


Definici�n en la l�nea 61 del archivo cliente\_\-lib.py.

\hypertarget{classcliente__lib_1_1cliente__lib_abba3409f89ee8dcec8b180c90aa5d77c}{
\index{cliente\_\-lib::cliente\_\-lib@{cliente\_\-lib::cliente\_\-lib}!valor\_\-fisico@{valor\_\-fisico}}
\index{valor\_\-fisico@{valor\_\-fisico}!cliente_lib::cliente_lib@{cliente\_\-lib::cliente\_\-lib}}
\subsubsection[{valor\_\-fisico}]{\setlength{\rightskip}{0pt plus 5cm}{\bf cliente\_\-lib.cliente\_\-lib.valor\_\-fisico}}}
\label{classcliente__lib_1_1cliente__lib_abba3409f89ee8dcec8b180c90aa5d77c}


Definici�n en la l�nea 59 del archivo cliente\_\-lib.py.

\hypertarget{classcliente__lib_1_1cliente__lib_aadea6e24bd3a01b0500fcc67543a97e9}{
\index{cliente\_\-lib::cliente\_\-lib@{cliente\_\-lib::cliente\_\-lib}!valor\_\-sonares@{valor\_\-sonares}}
\index{valor\_\-sonares@{valor\_\-sonares}!cliente_lib::cliente_lib@{cliente\_\-lib::cliente\_\-lib}}
\subsubsection[{valor\_\-sonares}]{\setlength{\rightskip}{0pt plus 5cm}{\bf cliente\_\-lib.cliente\_\-lib.valor\_\-sonares}}}
\label{classcliente__lib_1_1cliente__lib_aadea6e24bd3a01b0500fcc67543a97e9}


Definici�n en la l�nea 60 del archivo cliente\_\-lib.py.



La documentaci�n para esta clase fue generada a partir del siguiente fichero:\begin{DoxyCompactItemize}
\item 
\hyperlink{cliente__lib_8py}{cliente\_\-lib.py}\end{DoxyCompactItemize}

\hypertarget{classcliente__lib_1_1cliente__lib}{
\section{Referencia de la Clase cliente\_\-lib.cliente\_\-lib}
\label{classcliente__lib_1_1cliente__lib}\index{cliente\_\-lib::cliente\_\-lib@{cliente\_\-lib::cliente\_\-lib}}
}


es la clase encargada del cliente  


\subsection*{M�todos p�blicos}
\begin{DoxyCompactItemize}
\item 
def \hyperlink{classcliente__lib_1_1cliente__lib_ac5e4490f412835d35481f58d1ae503f9}{\_\-\_\-init\_\-\_\-}
\begin{DoxyCompactList}\small\item\em Carga valores a las variables necesarias para funcionar el cliente. \item\end{DoxyCompactList}\item 
def \hyperlink{classcliente__lib_1_1cliente__lib_af7b751bcf94c96150b23bacb5e477956}{valores}
\begin{DoxyCompactList}\small\item\em Sirve cuando se manda el comando \char`\"{}updateNumbers\char`\"{} en el paquete. \item\end{DoxyCompactList}\item 
def \hyperlink{classcliente__lib_1_1cliente__lib_ac0a4410b48b4c759028bec6ae1c641e8}{lista\_\-sonares}
\begin{DoxyCompactList}\small\item\em Sirve para leer el paquete arNetPacket con la lista del sonar. \item\end{DoxyCompactList}\item 
def \hyperlink{classcliente__lib_1_1cliente__lib_abcf28c2207cb5519090654484137db23}{valores\_\-sonares}
\begin{DoxyCompactList}\small\item\em Sirve para leer el paquete arNetPacket con los valores del sonar. \item\end{DoxyCompactList}\item 
def \hyperlink{classcliente__lib_1_1cliente__lib_acfc22af72a1668db28d18ab4ff40909e}{envio\_\-ratioDrive}
\begin{DoxyCompactList}\small\item\em Sirve para realizar la teleoperacion, mandando los parametros. \item\end{DoxyCompactList}\item 
def \hyperlink{classcliente__lib_1_1cliente__lib_ade1f44e9270c8835c284832a72b96b6c}{uC\_\-comandos\_\-movi}
\begin{DoxyCompactList}\small\item\em Sirve para mandar ordenes de movimiento directamente al controlador de la plataforma movil. \item\end{DoxyCompactList}\item 
def \hyperlink{classcliente__lib_1_1cliente__lib_ac3e89d3066207b05b217a50f549c239a}{envio\_\-consulta\_\-fisica}
\begin{DoxyCompactList}\small\item\em Sirve para mandar ordenes al servidor utilizando paquetes ArNetPacket con comandos {\bfseries pose} y {\bfseries updateNumbers} \item\end{DoxyCompactList}\item 
def \hyperlink{classcliente__lib_1_1cliente__lib_a1a7b5475a98772f0e48a4e1fd76e8d47}{cliente\_\-inicio}
\begin{DoxyCompactList}\small\item\em Sirve para iniciar la conexion con el servidor. \item\end{DoxyCompactList}\item 
def \hyperlink{classcliente__lib_1_1cliente__lib_a14a49495fd71fab84d36060e604415a5}{cliente\_\-apaga}
\begin{DoxyCompactList}\small\item\em Sirve para realizar la desconexion con el servidor. \item\end{DoxyCompactList}\item 
def \hyperlink{classcliente__lib_1_1cliente__lib_a3ca67c0c9d7f0a622abd740c780f64d1}{devuelve\_\-valorf}
\begin{DoxyCompactList}\small\item\em Devuelve el variable valor\_\-fisico, con usa espera de 100ms. \item\end{DoxyCompactList}\item 
def \hyperlink{classcliente__lib_1_1cliente__lib_a127d026872fbd11f4f5bbe4a73424b77}{devuelve\_\-valors}
\begin{DoxyCompactList}\small\item\em Devuelve el variable valor\_\-sonares, con usa espera de 100ms. \item\end{DoxyCompactList}\end{DoxyCompactItemize}
\subsection*{Atributos p�blicos}
\begin{DoxyCompactItemize}
\item 
\hyperlink{classcliente__lib_1_1cliente__lib_abba3409f89ee8dcec8b180c90aa5d77c}{valor\_\-fisico}
\item 
\hyperlink{classcliente__lib_1_1cliente__lib_aadea6e24bd3a01b0500fcc67543a97e9}{valor\_\-sonares}
\item 
\hyperlink{classcliente__lib_1_1cliente__lib_a675dd8430aa2eeb33240b8b07ed61543}{ip}
\end{DoxyCompactItemize}


\subsection{Descripci�n detallada}
es la clase encargada del cliente se lo utiliza de esta manera par poder trabajar con la informacion tanto leyendo las variables o utilizando los comandos 

Definici�n en la l�nea 50 del archivo cliente\_\-lib.py.



\subsection{Documentaci�n de las funciones miembro}
\hypertarget{classcliente__lib_1_1cliente__lib_ac5e4490f412835d35481f58d1ae503f9}{
\index{cliente\_\-lib::cliente\_\-lib@{cliente\_\-lib::cliente\_\-lib}!\_\-\_\-init\_\-\_\-@{\_\-\_\-init\_\-\_\-}}
\index{\_\-\_\-init\_\-\_\-@{\_\-\_\-init\_\-\_\-}!cliente_lib::cliente_lib@{cliente\_\-lib::cliente\_\-lib}}
\subsubsection[{\_\-\_\-init\_\-\_\-}]{\setlength{\rightskip}{0pt plus 5cm}def cliente\_\-lib.cliente\_\-lib.\_\-\_\-init\_\-\_\- (
\begin{DoxyParamCaption}
\item[{}]{ self}
\end{DoxyParamCaption}
)}}
\label{classcliente__lib_1_1cliente__lib_ac5e4490f412835d35481f58d1ae503f9}


Carga valores a las variables necesarias para funcionar el cliente. 

este comando no es necesario utilizarlo es usado al instanciar la clase 
\begin{DoxyParams}{Par�metros}
\item[{\em self}]este parametro no es necesario escribir \end{DoxyParams}


Definici�n en la l�nea 57 del archivo cliente\_\-lib.py.




\begin{DoxyCode}
58                           :
59                 print "Cargo modulo para cliente_lib"
60                 self.valor_fisico=[]
61                 self.valor_sonares=[] 
62                 self.ip="localhost"

\end{DoxyCode}


\hypertarget{classcliente__lib_1_1cliente__lib_a14a49495fd71fab84d36060e604415a5}{
\index{cliente\_\-lib::cliente\_\-lib@{cliente\_\-lib::cliente\_\-lib}!cliente\_\-apaga@{cliente\_\-apaga}}
\index{cliente\_\-apaga@{cliente\_\-apaga}!cliente_lib::cliente_lib@{cliente\_\-lib::cliente\_\-lib}}
\subsubsection[{cliente\_\-apaga}]{\setlength{\rightskip}{0pt plus 5cm}def cliente\_\-lib.cliente\_\-lib.cliente\_\-apaga (
\begin{DoxyParamCaption}
\item[{}]{ self, }
\item[{}]{ client}
\end{DoxyParamCaption}
)}}
\label{classcliente__lib_1_1cliente__lib_a14a49495fd71fab84d36060e604415a5}


Sirve para realizar la desconexion con el servidor. 


\begin{DoxyParams}{Par�metros}
\item[{\em self}]este parametro no es necesario escribir \item[{\em client}]para poder desconectar el cliente \end{DoxyParams}


Definici�n en la l�nea 208 del archivo cliente\_\-lib.py.




\begin{DoxyCode}
209                                       :
210                 ArUtil.sleep(1000)
211                 client.disconnect()
212                 ArUtil.sleep(50)
                return 0
\end{DoxyCode}


\hypertarget{classcliente__lib_1_1cliente__lib_a1a7b5475a98772f0e48a4e1fd76e8d47}{
\index{cliente\_\-lib::cliente\_\-lib@{cliente\_\-lib::cliente\_\-lib}!cliente\_\-inicio@{cliente\_\-inicio}}
\index{cliente\_\-inicio@{cliente\_\-inicio}!cliente_lib::cliente_lib@{cliente\_\-lib::cliente\_\-lib}}
\subsubsection[{cliente\_\-inicio}]{\setlength{\rightskip}{0pt plus 5cm}def cliente\_\-lib.cliente\_\-lib.cliente\_\-inicio (
\begin{DoxyParamCaption}
\item[{}]{ self}
\end{DoxyParamCaption}
)}}
\label{classcliente__lib_1_1cliente__lib_a1a7b5475a98772f0e48a4e1fd76e8d47}


Sirve para iniciar la conexion con el servidor. 


\begin{DoxyParams}{Par�metros}
\item[{\em self}]este parametro no es necesario escribir \end{DoxyParams}
\begin{DoxyReturn}{Devuelve}
client 
\end{DoxyReturn}


Definici�n en la l�nea 178 del archivo cliente\_\-lib.py.




\begin{DoxyCode}
179                                 :
180                 client = ArClientBase()
181                 Aria.init()
182                 
183                 startTime = ArTime()
184                 startTime.setToNow()
185                 if not client.blockingConnect(self.ip, 7272): #ip y puerto del se
      rvidor
186                         print "Could not connect to server at localhost port 7272
      , exiting"
187                         Aria.exit(1);
188                 print "cliente: Se tardo %ld msec en connectarse\n" % (startTime.
      mSecSince())
189                 
190                 client.runAsync()
191                 client.addHandler("updateNumbers",self.valores)
192                 client.addHandler("getSensorList",self.lista_sonares)
193                 client.addHandler("pose",self.valores_sonares)
194                 if client.dataExists("ratioDrive"): #supuestamente devuelve la in
      fo del robot con odometria
195                         print "ratioDrive si existe"
196                 else:
197                         Aria.exit(1);
198                 #client=envio_ratioDrive(client,TransRatio,RotRatio,LatRatio) #fi
      jar los valores para mover
199                 #client=uC_comandos_movi(client,comando,parametro) #Lo hace de un
      a manera directa anulando las demas operaciones
200                 #client.requestOnce("updateNumbers")
201                 #client.requestOnce("stop") #parada de emergencia
                return client
\end{DoxyCode}


\hypertarget{classcliente__lib_1_1cliente__lib_a3ca67c0c9d7f0a622abd740c780f64d1}{
\index{cliente\_\-lib::cliente\_\-lib@{cliente\_\-lib::cliente\_\-lib}!devuelve\_\-valorf@{devuelve\_\-valorf}}
\index{devuelve\_\-valorf@{devuelve\_\-valorf}!cliente_lib::cliente_lib@{cliente\_\-lib::cliente\_\-lib}}
\subsubsection[{devuelve\_\-valorf}]{\setlength{\rightskip}{0pt plus 5cm}def cliente\_\-lib.cliente\_\-lib.devuelve\_\-valorf (
\begin{DoxyParamCaption}
\item[{}]{ self}
\end{DoxyParamCaption}
)}}
\label{classcliente__lib_1_1cliente__lib_a3ca67c0c9d7f0a622abd740c780f64d1}


Devuelve el variable valor\_\-fisico, con usa espera de 100ms. 


\begin{DoxyParams}{Par�metros}
\item[{\em self}]este parametro no es necesario escribir \end{DoxyParams}


Definici�n en la l�nea 218 del archivo cliente\_\-lib.py.




\begin{DoxyCode}
219                                  :
220                 ArUtil.sleep(100)
                return self.valor_fisico
\end{DoxyCode}


\hypertarget{classcliente__lib_1_1cliente__lib_a127d026872fbd11f4f5bbe4a73424b77}{
\index{cliente\_\-lib::cliente\_\-lib@{cliente\_\-lib::cliente\_\-lib}!devuelve\_\-valors@{devuelve\_\-valors}}
\index{devuelve\_\-valors@{devuelve\_\-valors}!cliente_lib::cliente_lib@{cliente\_\-lib::cliente\_\-lib}}
\subsubsection[{devuelve\_\-valors}]{\setlength{\rightskip}{0pt plus 5cm}def cliente\_\-lib.cliente\_\-lib.devuelve\_\-valors (
\begin{DoxyParamCaption}
\item[{}]{ self}
\end{DoxyParamCaption}
)}}
\label{classcliente__lib_1_1cliente__lib_a127d026872fbd11f4f5bbe4a73424b77}


Devuelve el variable valor\_\-sonares, con usa espera de 100ms. 


\begin{DoxyParams}{Par�metros}
\item[{\em self}]este parametro no es necesario escribir \end{DoxyParams}


Definici�n en la l�nea 226 del archivo cliente\_\-lib.py.




\begin{DoxyCode}
227                                  :
228                 ArUtil.sleep(100)
229                 return self.valor_sonares

\end{DoxyCode}


\hypertarget{classcliente__lib_1_1cliente__lib_ac3e89d3066207b05b217a50f549c239a}{
\index{cliente\_\-lib::cliente\_\-lib@{cliente\_\-lib::cliente\_\-lib}!envio\_\-consulta\_\-fisica@{envio\_\-consulta\_\-fisica}}
\index{envio\_\-consulta\_\-fisica@{envio\_\-consulta\_\-fisica}!cliente_lib::cliente_lib@{cliente\_\-lib::cliente\_\-lib}}
\subsubsection[{envio\_\-consulta\_\-fisica}]{\setlength{\rightskip}{0pt plus 5cm}def cliente\_\-lib.cliente\_\-lib.envio\_\-consulta\_\-fisica (
\begin{DoxyParamCaption}
\item[{}]{ self, }
\item[{}]{ client, }
\item[{}]{ mensaje}
\end{DoxyParamCaption}
)}}
\label{classcliente__lib_1_1cliente__lib_ac3e89d3066207b05b217a50f549c239a}


Sirve para mandar ordenes al servidor utilizando paquetes ArNetPacket con comandos {\bfseries pose} y {\bfseries updateNumbers} 


\begin{DoxyParams}{Par�metros}
\item[{\em self}]este parametro no es necesario escribir \item[{\em client}]Se debe trar el objeto cliente a la definicion para poder utilizar el enlace del cliente para enviar el paquete al servidor \item[{\em mensaje}]puede ser cualquier comando del servidor que no devuelva informacion a exepcion de pose y updatenumbers \end{DoxyParams}
\begin{DoxyReturn}{Devuelve}
client 
\end{DoxyReturn}


Definici�n en la l�nea 167 del archivo cliente\_\-lib.py.




\begin{DoxyCode}
168                                                       :
169                 ## se puede usar pose y updateNumbers
170                 client.requestOnce(mensaje)
171                 return client
                
\end{DoxyCode}


\hypertarget{classcliente__lib_1_1cliente__lib_acfc22af72a1668db28d18ab4ff40909e}{
\index{cliente\_\-lib::cliente\_\-lib@{cliente\_\-lib::cliente\_\-lib}!envio\_\-ratioDrive@{envio\_\-ratioDrive}}
\index{envio\_\-ratioDrive@{envio\_\-ratioDrive}!cliente_lib::cliente_lib@{cliente\_\-lib::cliente\_\-lib}}
\subsubsection[{envio\_\-ratioDrive}]{\setlength{\rightskip}{0pt plus 5cm}def cliente\_\-lib.cliente\_\-lib.envio\_\-ratioDrive (
\begin{DoxyParamCaption}
\item[{}]{ self, }
\item[{}]{ client, }
\item[{}]{ TransRatio, }
\item[{}]{ RotRatio, }
\item[{}]{ LatRatio}
\end{DoxyParamCaption}
)}}
\label{classcliente__lib_1_1cliente__lib_acfc22af72a1668db28d18ab4ff40909e}


Sirve para realizar la teleoperacion, mandando los parametros. 


\begin{DoxyParams}{Par�metros}
\item[{\em self}]este parametro no es necesario escribir \item[{\em client}]Se debe trar el objeto cliente a la definicion para poder utilizar el enlace del cliente para enviar el paquete al servidor \item[{\em TransRatio}]Velocidad de traslacion \item[{\em RotRatio}]Velocidad de rotacion \item[{\em LatRatio}]velocidad lateral para el modelo Pioneer P3-\/DX no se necesario puede ser 0 \end{DoxyParams}
\begin{DoxyReturn}{Devuelve}
client 
\end{DoxyReturn}


Definici�n en la l�nea 130 del archivo cliente\_\-lib.py.




\begin{DoxyCode}
131                                                                       :
132                 myTransRatio=TransRatio
133                 myRotRatio=RotRatio
134                 myLatRatio=LatRatio
135                 packet=ArNetPacket()
136                 packet.doubleToBuf(myTransRatio)
137                 packet.doubleToBuf(myRotRatio)
138                 packet.doubleToBuf(50) # use half of the robot's maximum.
139                 packet.doubleToBuf(myLatRatio)
140                 client.requestOnce("ratioDrive", packet)
141                 return client
          
\end{DoxyCode}


\hypertarget{classcliente__lib_1_1cliente__lib_ac0a4410b48b4c759028bec6ae1c641e8}{
\index{cliente\_\-lib::cliente\_\-lib@{cliente\_\-lib::cliente\_\-lib}!lista\_\-sonares@{lista\_\-sonares}}
\index{lista\_\-sonares@{lista\_\-sonares}!cliente_lib::cliente_lib@{cliente\_\-lib::cliente\_\-lib}}
\subsubsection[{lista\_\-sonares}]{\setlength{\rightskip}{0pt plus 5cm}def cliente\_\-lib.cliente\_\-lib.lista\_\-sonares (
\begin{DoxyParamCaption}
\item[{}]{ self, }
\item[{}]{ packet}
\end{DoxyParamCaption}
)}}
\label{classcliente__lib_1_1cliente__lib_ac0a4410b48b4c759028bec6ae1c641e8}


Sirve para leer el paquete arNetPacket con la lista del sonar. 

este comando no es necesario utilizarlo es usado solo por el cliente para procesar el paquete 
\begin{DoxyParams}{Par�metros}
\item[{\em self}]este parametro no es necesario escribir \item[{\em packet}]este parametro no es necesario escribir \end{DoxyParams}
\begin{DoxyReturn}{Devuelve}
nada 
\end{DoxyReturn}


Definici�n en la l�nea 94 del archivo cliente\_\-lib.py.




\begin{DoxyCode}
95                                       :
96                 c="                                   "
97                 numSensor=packet.bufToByte2()
98                 numSensor2=packet.bufToStr(c,15)
99                 print str(numSensor)+" "+str(c.strip())

\end{DoxyCode}


\hypertarget{classcliente__lib_1_1cliente__lib_ade1f44e9270c8835c284832a72b96b6c}{
\index{cliente\_\-lib::cliente\_\-lib@{cliente\_\-lib::cliente\_\-lib}!uC\_\-comandos\_\-movi@{uC\_\-comandos\_\-movi}}
\index{uC\_\-comandos\_\-movi@{uC\_\-comandos\_\-movi}!cliente_lib::cliente_lib@{cliente\_\-lib::cliente\_\-lib}}
\subsubsection[{uC\_\-comandos\_\-movi}]{\setlength{\rightskip}{0pt plus 5cm}def cliente\_\-lib.cliente\_\-lib.uC\_\-comandos\_\-movi (
\begin{DoxyParamCaption}
\item[{}]{ self, }
\item[{}]{ client, }
\item[{}]{ comando, }
\item[{}]{ parametro}
\end{DoxyParamCaption}
)}}
\label{classcliente__lib_1_1cliente__lib_ade1f44e9270c8835c284832a72b96b6c}


Sirve para mandar ordenes de movimiento directamente al controlador de la plataforma movil. 


\begin{DoxyParams}{Par�metros}
\item[{\em self}]este parametro no es necesario escribir \item[{\em client}]Se debe trar el objeto cliente a la definicion para poder utilizar el enlace del cliente para enviar el paquete al servidor \item[{\em comando}]es un numero de 1-\/255 que representa una funcion esta informacion se puede encontrar en el API de ARIA \item[{\em parametro}]el parametro de la funcion en caso de no tener se deja el valor en blanco \end{DoxyParams}
\begin{DoxyReturn}{Devuelve}
client 
\end{DoxyReturn}


Definici�n en la l�nea 151 del archivo cliente\_\-lib.py.




\begin{DoxyCode}
152                                                            :
153                 mi_comando=comando #comando 8 es MOVE parametro un valor de 5000 
      a -4999 es en mm, 11 LEV y su parametro es velocidad +o- mm/s
154                 mi_parametro=parametro #parametro
155                 packet=ArNetPacket()
156                 packet.strToBuf(mi_comando+" "+mi_parametro)
157                 client.requestOnce("MicroControllerMotionCommand", packet) #Micro
      ControllerMotionCommand
158                 return client

\end{DoxyCode}


\hypertarget{classcliente__lib_1_1cliente__lib_af7b751bcf94c96150b23bacb5e477956}{
\index{cliente\_\-lib::cliente\_\-lib@{cliente\_\-lib::cliente\_\-lib}!valores@{valores}}
\index{valores@{valores}!cliente_lib::cliente_lib@{cliente\_\-lib::cliente\_\-lib}}
\subsubsection[{valores}]{\setlength{\rightskip}{0pt plus 5cm}def cliente\_\-lib.cliente\_\-lib.valores (
\begin{DoxyParamCaption}
\item[{}]{ self, }
\item[{}]{ packet}
\end{DoxyParamCaption}
)}}
\label{classcliente__lib_1_1cliente__lib_af7b751bcf94c96150b23bacb5e477956}


Sirve cuando se manda el comando \char`\"{}updateNumbers\char`\"{} en el paquete. 

este comando no es necesario utilizarlo es usado solo por el cliente para procesar el paquete 
\begin{DoxyParams}{Par�metros}
\item[{\em self}]este parametro no es necesario escribir \item[{\em packet}]el paquete que recibe el cliente del servidor, no es necesario escribir \end{DoxyParams}
\begin{DoxyReturn}{Devuelve}
Nada, pero guarda en self.valores\_\-fisico \mbox{[}voltaje\_\-bateria,myX,myY,myTh,myVel,myRotVel,myLatVel,myTemperature\mbox{]} 
\end{DoxyReturn}


Definici�n en la l�nea 71 del archivo cliente\_\-lib.py.




\begin{DoxyCode}
72                                 :
73                 #devuelve los valores voltaje_bateria,myX,myY,myTh,myVel,myRotVel
      ,myLatVel,myTemperature
74                 voltaje_bateria=packet.bufToByte2()/10
75                 myX = packet.bufToByte4()#
76                 myY = packet.bufToByte4()
77                 myTh = packet.bufToByte2()
78                 myVel = packet.bufToByte2()
79                 myRotVel = packet.bufToByte2()
80                 myLatVel = packet.bufToByte2()
81                 myTemperature = packet.bufToByte()
82                 #print "X= "+str(myX)+" y="+str(myY)+" th="+str(myTh)
83                 self.valor_fisico=(voltaje_bateria,myX,myY,myTh,myVel,myRotVel,my
      LatVel,myTemperature)
84                 self.valor_fisico=(voltaje_bateria,myX,myY,myTh,myVel,myRotVel,my
      LatVel,0)
85                 #print valor

\end{DoxyCode}


\hypertarget{classcliente__lib_1_1cliente__lib_abcf28c2207cb5519090654484137db23}{
\index{cliente\_\-lib::cliente\_\-lib@{cliente\_\-lib::cliente\_\-lib}!valores\_\-sonares@{valores\_\-sonares}}
\index{valores\_\-sonares@{valores\_\-sonares}!cliente_lib::cliente_lib@{cliente\_\-lib::cliente\_\-lib}}
\subsubsection[{valores\_\-sonares}]{\setlength{\rightskip}{0pt plus 5cm}def cliente\_\-lib.cliente\_\-lib.valores\_\-sonares (
\begin{DoxyParamCaption}
\item[{}]{ self, }
\item[{}]{ packet}
\end{DoxyParamCaption}
)}}
\label{classcliente__lib_1_1cliente__lib_abcf28c2207cb5519090654484137db23}


Sirve para leer el paquete arNetPacket con los valores del sonar. 

este comando no es necesario utilizarlo es usado solo por el cliente para procesar el paquete 
\begin{DoxyParams}{Par�metros}
\item[{\em self}]este parametro no es necesario escribir \item[{\em packet}]este parametro no es necesario escribir \end{DoxyParams}
\begin{DoxyReturn}{Devuelve}
nada 
\end{DoxyReturn}


Definici�n en la l�nea 108 del archivo cliente\_\-lib.py.




\begin{DoxyCode}
109                                         :
110                 c="                                                              
                                                          "
111                 cantidad=packet.bufToDouble()
112                 dato=range(int(cantidad))
113                 for j in range(int(cantidad)):
114                         dato[j]=c #ya que pasa el buffer a la variable esta debe 
      tener la longitud necesaria y deben ser guardada en cada lectura
115                 for i in range(int(cantidad)):
116                         packet.bufToStr(dato[i],50)
117                         dato[i]=dato[i].replace("\x00"," ") #Porque al final anad
      e un NONE del string
118                         dato[i]=dato[i].strip() 
119                 self.valor_sonares=[cantidad,dato]

\end{DoxyCode}




\subsection{Documentaci�n de los datos miembro}
\hypertarget{classcliente__lib_1_1cliente__lib_a675dd8430aa2eeb33240b8b07ed61543}{
\index{cliente\_\-lib::cliente\_\-lib@{cliente\_\-lib::cliente\_\-lib}!ip@{ip}}
\index{ip@{ip}!cliente_lib::cliente_lib@{cliente\_\-lib::cliente\_\-lib}}
\subsubsection[{ip}]{\setlength{\rightskip}{0pt plus 5cm}{\bf cliente\_\-lib.cliente\_\-lib.ip}}}
\label{classcliente__lib_1_1cliente__lib_a675dd8430aa2eeb33240b8b07ed61543}


Definici�n en la l�nea 61 del archivo cliente\_\-lib.py.

\hypertarget{classcliente__lib_1_1cliente__lib_abba3409f89ee8dcec8b180c90aa5d77c}{
\index{cliente\_\-lib::cliente\_\-lib@{cliente\_\-lib::cliente\_\-lib}!valor\_\-fisico@{valor\_\-fisico}}
\index{valor\_\-fisico@{valor\_\-fisico}!cliente_lib::cliente_lib@{cliente\_\-lib::cliente\_\-lib}}
\subsubsection[{valor\_\-fisico}]{\setlength{\rightskip}{0pt plus 5cm}{\bf cliente\_\-lib.cliente\_\-lib.valor\_\-fisico}}}
\label{classcliente__lib_1_1cliente__lib_abba3409f89ee8dcec8b180c90aa5d77c}


Definici�n en la l�nea 59 del archivo cliente\_\-lib.py.

\hypertarget{classcliente__lib_1_1cliente__lib_aadea6e24bd3a01b0500fcc67543a97e9}{
\index{cliente\_\-lib::cliente\_\-lib@{cliente\_\-lib::cliente\_\-lib}!valor\_\-sonares@{valor\_\-sonares}}
\index{valor\_\-sonares@{valor\_\-sonares}!cliente_lib::cliente_lib@{cliente\_\-lib::cliente\_\-lib}}
\subsubsection[{valor\_\-sonares}]{\setlength{\rightskip}{0pt plus 5cm}{\bf cliente\_\-lib.cliente\_\-lib.valor\_\-sonares}}}
\label{classcliente__lib_1_1cliente__lib_aadea6e24bd3a01b0500fcc67543a97e9}


Definici�n en la l�nea 60 del archivo cliente\_\-lib.py.



La documentaci�n para esta clase fue generada a partir del siguiente fichero:\begin{DoxyCompactItemize}
\item 
\hyperlink{cliente__lib_8py}{cliente\_\-lib.py}\end{DoxyCompactItemize}

\hypertarget{classcliente__lib__original_1_1cliente__lib}{
\section{Referencia de la Clase cliente\_\-lib\_\-original.cliente\_\-lib}
\label{classcliente__lib__original_1_1cliente__lib}\index{cliente\_\-lib\_\-original::cliente\_\-lib@{cliente\_\-lib\_\-original::cliente\_\-lib}}
}
\subsection*{Métodos públicos}
\begin{DoxyCompactItemize}
\item 
def \hyperlink{classcliente__lib__original_1_1cliente__lib_a1edb12a09794f57a5c681bc5aac49650}{\_\-\_\-init\_\-\_\-}
\begin{DoxyCompactList}\small\item\em Carga valores a las variables necesarias para funcionar el cliente. \end{DoxyCompactList}\item 
def \hyperlink{classcliente__lib__original_1_1cliente__lib_a5f97cbead3de2cb78534afec8343c13e}{valores}
\begin{DoxyCompactList}\small\item\em Sirve cuando se manda el comando \char`\"{}updateNumbers\char`\"{} en el paquete. \end{DoxyCompactList}\item 
def \hyperlink{classcliente__lib__original_1_1cliente__lib_a01b5aae2c3ce57fca590b9d06e767f23}{lista\_\-sonares}
\begin{DoxyCompactList}\small\item\em Sirve para leer el paquete arNetPacket con la lista del sonar. \end{DoxyCompactList}\item 
def \hyperlink{classcliente__lib__original_1_1cliente__lib_a2ab0872984bef5af4bfbb7d2ec7f40c5}{valores\_\-sonares}
\begin{DoxyCompactList}\small\item\em Sirve para leer el paquete arNetPacket con los valores del sonar. \end{DoxyCompactList}\item 
def \hyperlink{classcliente__lib__original_1_1cliente__lib_aac50c9462dfe46d1b618692d7206295b}{envio\_\-ratioDrive}
\begin{DoxyCompactList}\small\item\em Sirve para realizar la teleoperacion, mandando los parametros. \end{DoxyCompactList}\item 
def \hyperlink{classcliente__lib__original_1_1cliente__lib_a16a11ee4bc738ae83b652343583ad556}{uC\_\-comandos\_\-movi}
\begin{DoxyCompactList}\small\item\em Sirve para mandar ordenes de movimiento directamente al controlador de la plataforma movil. \end{DoxyCompactList}\item 
def \hyperlink{classcliente__lib__original_1_1cliente__lib_a10ab9f40fbd7244e96c2b2493e9a9e86}{envio\_\-consulta\_\-fisica}
\begin{DoxyCompactList}\small\item\em Sirve para mandar ordenes al servidor utilizando paquetes ArNetPacket con comandos {\bfseries pose} y {\bfseries updateNumbers} \end{DoxyCompactList}\item 
def \hyperlink{classcliente__lib__original_1_1cliente__lib_a52e3e1ca7b1935b7fb7e6f0a093918e9}{cliente\_\-inicio}
\begin{DoxyCompactList}\small\item\em Sirve para iniciar la conexion con el servidor. \end{DoxyCompactList}\item 
def \hyperlink{classcliente__lib__original_1_1cliente__lib_a9ea49590b5ca6de4f6368f2209a9ac0e}{cliente\_\-apaga}
\begin{DoxyCompactList}\small\item\em Sirve para realizar la desconexion con el servidor. \end{DoxyCompactList}\item 
def \hyperlink{classcliente__lib__original_1_1cliente__lib_ae6a834b4525e77e3f88f6eaa68ae97eb}{devuelve\_\-valorf}
\begin{DoxyCompactList}\small\item\em Devuelve el variable valor\_\-fisico, con usa espera de 100ms. \end{DoxyCompactList}\item 
def \hyperlink{classcliente__lib__original_1_1cliente__lib_a5766042b7c5c2bd3b291c02141af8824}{devuelve\_\-valors}
\begin{DoxyCompactList}\small\item\em Devuelve el variable valor\_\-sonares, con usa espera de 100ms. \end{DoxyCompactList}\end{DoxyCompactItemize}
\subsection*{Atributos públicos}
\begin{DoxyCompactItemize}
\item 
\hyperlink{classcliente__lib__original_1_1cliente__lib_a74cb35b9f3246db8b6b16ac59ae8f320}{valor\_\-fisico}
\item 
\hyperlink{classcliente__lib__original_1_1cliente__lib_a4172f5914b673eb44e168b4afcbae651}{valor\_\-sonares}
\item 
\hyperlink{classcliente__lib__original_1_1cliente__lib_a030e232b37138f0a2ba15a3ef5f1fe21}{ip}
\item 
\hyperlink{classcliente__lib__original_1_1cliente__lib_aa2933583abd7844c57b2ec553cbb46ee}{x}
\item 
\hyperlink{classcliente__lib__original_1_1cliente__lib_a6050f7c724f8ca063505e8b80a1cdecf}{y}
\item 
\hyperlink{classcliente__lib__original_1_1cliente__lib_ae6a2ed4ac198fc965fbe3ee9a01f9401}{t}
\item 
\hyperlink{classcliente__lib__original_1_1cliente__lib_a049a973bc28127e24c2c717de93daf2f}{acu}
\end{DoxyCompactItemize}


\subsection{Descripción detallada}


Definición en la línea 51 del archivo cliente\_\-lib\_\-original.py.



\subsection{Documentación del constructor y destructor}
\hypertarget{classcliente__lib__original_1_1cliente__lib_a1edb12a09794f57a5c681bc5aac49650}{
\index{cliente\_\-lib\_\-original::cliente\_\-lib@{cliente\_\-lib\_\-original::cliente\_\-lib}!\_\-\_\-init\_\-\_\-@{\_\-\_\-init\_\-\_\-}}
\index{\_\-\_\-init\_\-\_\-@{\_\-\_\-init\_\-\_\-}!cliente_lib_original::cliente_lib@{cliente\_\-lib\_\-original::cliente\_\-lib}}
\subsubsection[{\_\-\_\-init\_\-\_\-}]{\setlength{\rightskip}{0pt plus 5cm}def cliente\_\-lib\_\-original.cliente\_\-lib.\_\-\_\-init\_\-\_\- (
\begin{DoxyParamCaption}
\item[{}]{self}
\end{DoxyParamCaption}
)}}
\label{classcliente__lib__original_1_1cliente__lib_a1edb12a09794f57a5c681bc5aac49650}


Carga valores a las variables necesarias para funcionar el cliente. 

este comando no es necesario utilizarlo es usado al instanciar la clase 
\begin{DoxyParams}{Parámetros}
{\em self} & este parametro no es necesario escribir \\
\hline
\end{DoxyParams}


Definición en la línea 58 del archivo cliente\_\-lib\_\-original.py.


\begin{DoxyCode}
59                           :
60                 print "Cargo modulo para cliente_lib"
61                 self.valor_fisico=[]
62                 self.valor_sonares=[] 
63                 self.ip="localhost"
64                 self.x={}
65                 self.y={}
66                 self.t={}
67                 self.acu=0

\end{DoxyCode}


\subsection{Documentación de las funciones miembro}
\hypertarget{classcliente__lib__original_1_1cliente__lib_a9ea49590b5ca6de4f6368f2209a9ac0e}{
\index{cliente\_\-lib\_\-original::cliente\_\-lib@{cliente\_\-lib\_\-original::cliente\_\-lib}!cliente\_\-apaga@{cliente\_\-apaga}}
\index{cliente\_\-apaga@{cliente\_\-apaga}!cliente_lib_original::cliente_lib@{cliente\_\-lib\_\-original::cliente\_\-lib}}
\subsubsection[{cliente\_\-apaga}]{\setlength{\rightskip}{0pt plus 5cm}def cliente\_\-lib\_\-original.cliente\_\-lib.cliente\_\-apaga (
\begin{DoxyParamCaption}
\item[{}]{self, }
\item[{}]{client}
\end{DoxyParamCaption}
)}}
\label{classcliente__lib__original_1_1cliente__lib_a9ea49590b5ca6de4f6368f2209a9ac0e}


Sirve para realizar la desconexion con el servidor. 


\begin{DoxyParams}{Parámetros}
{\em self} & este parametro no es necesario escribir \\
\hline
{\em client} & para poder desconectar el cliente \\
\hline
\end{DoxyParams}


Definición en la línea 213 del archivo cliente\_\-lib\_\-original.py.


\begin{DoxyCode}
214                                       :
215                 ArUtil.sleep(1000)
216                 client.disconnect()
217                 ArUtil.sleep(50)
                return 0
\end{DoxyCode}
\hypertarget{classcliente__lib__original_1_1cliente__lib_a52e3e1ca7b1935b7fb7e6f0a093918e9}{
\index{cliente\_\-lib\_\-original::cliente\_\-lib@{cliente\_\-lib\_\-original::cliente\_\-lib}!cliente\_\-inicio@{cliente\_\-inicio}}
\index{cliente\_\-inicio@{cliente\_\-inicio}!cliente_lib_original::cliente_lib@{cliente\_\-lib\_\-original::cliente\_\-lib}}
\subsubsection[{cliente\_\-inicio}]{\setlength{\rightskip}{0pt plus 5cm}def cliente\_\-lib\_\-original.cliente\_\-lib.cliente\_\-inicio (
\begin{DoxyParamCaption}
\item[{}]{self}
\end{DoxyParamCaption}
)}}
\label{classcliente__lib__original_1_1cliente__lib_a52e3e1ca7b1935b7fb7e6f0a093918e9}


Sirve para iniciar la conexion con el servidor. 


\begin{DoxyParams}{Parámetros}
{\em self} & este parametro no es necesario escribir \\
\hline
\end{DoxyParams}
\begin{DoxyReturn}{Devuelve}
client 
\end{DoxyReturn}


Definición en la línea 183 del archivo cliente\_\-lib\_\-original.py.


\begin{DoxyCode}
184                                 :
185                 client = ArClientBase()
186                 Aria.init()
187                 
188                 startTime = ArTime()
189                 startTime.setToNow()
190                 if not client.blockingConnect(self.ip, 7272): #ip y puerto del se
      rvidor
191                         print "Could not connect to server at localhost port 7272
      , exiting"
192                         Aria.exit(1);
193                 print "cliente: Se tardo %ld msec en connectarse\n" % (startTime.
      mSecSince())
194                 
195                 client.runAsync()
196                 client.addHandler("updateNumbers",self.valores)
197                 client.addHandler("getSensorList",self.lista_sonares)
198                 client.addHandler("pose",self.valores_sonares)
199                 if client.dataExists("ratioDrive"): #supuestamente devuelve la in
      fo del robot con odometria
200                         print "ratioDrive si existe"
201                 else:
202                         Aria.exit(1);
203                 #client=envio_ratioDrive(client,TransRatio,RotRatio,LatRatio) #fi
      jar los valores para mover
204                 #client=uC_comandos_movi(client,comando,parametro) #Lo hace de un
      a manera directa anulando las demas operaciones
205                 #client.requestOnce("updateNumbers")
206                 #client.requestOnce("stop") #parada de emergencia
                return client
\end{DoxyCode}
\hypertarget{classcliente__lib__original_1_1cliente__lib_ae6a834b4525e77e3f88f6eaa68ae97eb}{
\index{cliente\_\-lib\_\-original::cliente\_\-lib@{cliente\_\-lib\_\-original::cliente\_\-lib}!devuelve\_\-valorf@{devuelve\_\-valorf}}
\index{devuelve\_\-valorf@{devuelve\_\-valorf}!cliente_lib_original::cliente_lib@{cliente\_\-lib\_\-original::cliente\_\-lib}}
\subsubsection[{devuelve\_\-valorf}]{\setlength{\rightskip}{0pt plus 5cm}def cliente\_\-lib\_\-original.cliente\_\-lib.devuelve\_\-valorf (
\begin{DoxyParamCaption}
\item[{}]{self}
\end{DoxyParamCaption}
)}}
\label{classcliente__lib__original_1_1cliente__lib_ae6a834b4525e77e3f88f6eaa68ae97eb}


Devuelve el variable valor\_\-fisico, con usa espera de 100ms. 


\begin{DoxyParams}{Parámetros}
{\em self} & este parametro no es necesario escribir \\
\hline
\end{DoxyParams}


Definición en la línea 223 del archivo cliente\_\-lib\_\-original.py.


\begin{DoxyCode}
224                                  :
225                 ArUtil.sleep(100)
                return self.valor_fisico
\end{DoxyCode}
\hypertarget{classcliente__lib__original_1_1cliente__lib_a5766042b7c5c2bd3b291c02141af8824}{
\index{cliente\_\-lib\_\-original::cliente\_\-lib@{cliente\_\-lib\_\-original::cliente\_\-lib}!devuelve\_\-valors@{devuelve\_\-valors}}
\index{devuelve\_\-valors@{devuelve\_\-valors}!cliente_lib_original::cliente_lib@{cliente\_\-lib\_\-original::cliente\_\-lib}}
\subsubsection[{devuelve\_\-valors}]{\setlength{\rightskip}{0pt plus 5cm}def cliente\_\-lib\_\-original.cliente\_\-lib.devuelve\_\-valors (
\begin{DoxyParamCaption}
\item[{}]{self}
\end{DoxyParamCaption}
)}}
\label{classcliente__lib__original_1_1cliente__lib_a5766042b7c5c2bd3b291c02141af8824}


Devuelve el variable valor\_\-sonares, con usa espera de 100ms. 


\begin{DoxyParams}{Parámetros}
{\em self} & este parametro no es necesario escribir \\
\hline
\end{DoxyParams}


Definición en la línea 231 del archivo cliente\_\-lib\_\-original.py.


\begin{DoxyCode}
232                                  :
233                 
234                 try:
235                         ArUtil.sleep(100)
236                         valor=self.valor_sonares
237                         x_br=[]
238                         y_br=[]
239                         t_br=[]
240                         for a in range(int(valor[0])): #el primer valor indica cu
      antos elementos se tiene de sensores
241                                 valor[1][a]=valor[1][a].strip('(')
242                                 valor[1][a]=valor[1][a].strip(')')
243                                 valor[1][a]=valor[1][a].split(",") #comienza a se
      parar por que viene en forma (X:valor,Y:valor,T:valor)
244                                 valor[1][a][0]=valor[1][a][0].strip() #borrar los
       espacios en blanco de los elementos
245                                 valor[1][a][1]=valor[1][a][1].strip()
246                                 valor[1][a][2]=valor[1][a][2].strip()
247                                 x_br+=[valor[1][a][0].strip("X:")]
248                                 y_br+=[valor[1][a][1].strip("Y:")]
249                                 t_br+=[valor[1][a][2].strip("T:")]
250                         print len(x_br)
251                         print x_br
252                         print y_br
253                         for a in range(len(x_br)):
254                                 self.x.update({'x%d' % self.acu:float(x_br[a])}) 
      #Ordena los datos en forma de diccionario
255                                 self.y.update({'y%d' % self.acu:float(y_br[a])})
256                                 self.t.update({'t%d' % self.acu:0})
257                                 #self.t.update({'t%d' % self.acu:float(t_br[a])})
      
258                                 self.acu+=1
259                                 if self.acu==30:
260                                         self.acu=0
261                         #print self.x
262                         #print self.y
263                         #print self.t
264                         return self.x,self.y,self.t
265                 except:
266                         return {'x0':-9999,'x1':-9999},{'y0':-9999,'y1':-9999},{'
      t0':0,'t1':0}

\end{DoxyCode}
\hypertarget{classcliente__lib__original_1_1cliente__lib_a10ab9f40fbd7244e96c2b2493e9a9e86}{
\index{cliente\_\-lib\_\-original::cliente\_\-lib@{cliente\_\-lib\_\-original::cliente\_\-lib}!envio\_\-consulta\_\-fisica@{envio\_\-consulta\_\-fisica}}
\index{envio\_\-consulta\_\-fisica@{envio\_\-consulta\_\-fisica}!cliente_lib_original::cliente_lib@{cliente\_\-lib\_\-original::cliente\_\-lib}}
\subsubsection[{envio\_\-consulta\_\-fisica}]{\setlength{\rightskip}{0pt plus 5cm}def cliente\_\-lib\_\-original.cliente\_\-lib.envio\_\-consulta\_\-fisica (
\begin{DoxyParamCaption}
\item[{}]{self, }
\item[{}]{client, }
\item[{}]{mensaje}
\end{DoxyParamCaption}
)}}
\label{classcliente__lib__original_1_1cliente__lib_a10ab9f40fbd7244e96c2b2493e9a9e86}


Sirve para mandar ordenes al servidor utilizando paquetes ArNetPacket con comandos {\bfseries pose} y {\bfseries updateNumbers} 


\begin{DoxyParams}{Parámetros}
{\em self} & este parametro no es necesario escribir \\
\hline
{\em client} & Se debe trar el objeto cliente a la definicion para poder utilizar el enlace del cliente para enviar el paquete al servidor \\
\hline
{\em mensaje} & puede ser cualquier comando del servidor que no devuelva informacion a exepcion de pose y updatenumbers \\
\hline
\end{DoxyParams}
\begin{DoxyReturn}{Devuelve}
client 
\end{DoxyReturn}


Definición en la línea 172 del archivo cliente\_\-lib\_\-original.py.


\begin{DoxyCode}
173                                                       :
174                 ## se puede usar pose y updateNumbers
175                 client.requestOnce(mensaje)
176                 return client
                
\end{DoxyCode}
\hypertarget{classcliente__lib__original_1_1cliente__lib_aac50c9462dfe46d1b618692d7206295b}{
\index{cliente\_\-lib\_\-original::cliente\_\-lib@{cliente\_\-lib\_\-original::cliente\_\-lib}!envio\_\-ratioDrive@{envio\_\-ratioDrive}}
\index{envio\_\-ratioDrive@{envio\_\-ratioDrive}!cliente_lib_original::cliente_lib@{cliente\_\-lib\_\-original::cliente\_\-lib}}
\subsubsection[{envio\_\-ratioDrive}]{\setlength{\rightskip}{0pt plus 5cm}def cliente\_\-lib\_\-original.cliente\_\-lib.envio\_\-ratioDrive (
\begin{DoxyParamCaption}
\item[{}]{self, }
\item[{}]{client, }
\item[{}]{TransRatio, }
\item[{}]{RotRatio, }
\item[{}]{LatRatio}
\end{DoxyParamCaption}
)}}
\label{classcliente__lib__original_1_1cliente__lib_aac50c9462dfe46d1b618692d7206295b}


Sirve para realizar la teleoperacion, mandando los parametros. 


\begin{DoxyParams}{Parámetros}
{\em self} & este parametro no es necesario escribir \\
\hline
{\em client} & Se debe trar el objeto cliente a la definicion para poder utilizar el enlace del cliente para enviar el paquete al servidor \\
\hline
{\em TransRatio} & Velocidad de traslacion \\
\hline
{\em RotRatio} & Velocidad de rotacion \\
\hline
{\em LatRatio} & velocidad lateral para el modelo Pioneer P3-\/DX no se necesario puede ser 0 \\
\hline
\end{DoxyParams}
\begin{DoxyReturn}{Devuelve}
client 
\end{DoxyReturn}


Definición en la línea 135 del archivo cliente\_\-lib\_\-original.py.


\begin{DoxyCode}
136                                                                       :
137                 myTransRatio=TransRatio
138                 myRotRatio=RotRatio
139                 myLatRatio=LatRatio
140                 packet=ArNetPacket()
141                 packet.doubleToBuf(myTransRatio)
142                 packet.doubleToBuf(myRotRatio)
143                 packet.doubleToBuf(50) # use half of the robot's maximum.
144                 packet.doubleToBuf(myLatRatio)
145                 client.requestOnce("ratioDrive", packet)
146                 return client
          
\end{DoxyCode}
\hypertarget{classcliente__lib__original_1_1cliente__lib_a01b5aae2c3ce57fca590b9d06e767f23}{
\index{cliente\_\-lib\_\-original::cliente\_\-lib@{cliente\_\-lib\_\-original::cliente\_\-lib}!lista\_\-sonares@{lista\_\-sonares}}
\index{lista\_\-sonares@{lista\_\-sonares}!cliente_lib_original::cliente_lib@{cliente\_\-lib\_\-original::cliente\_\-lib}}
\subsubsection[{lista\_\-sonares}]{\setlength{\rightskip}{0pt plus 5cm}def cliente\_\-lib\_\-original.cliente\_\-lib.lista\_\-sonares (
\begin{DoxyParamCaption}
\item[{}]{self, }
\item[{}]{packet}
\end{DoxyParamCaption}
)}}
\label{classcliente__lib__original_1_1cliente__lib_a01b5aae2c3ce57fca590b9d06e767f23}


Sirve para leer el paquete arNetPacket con la lista del sonar. 

este comando no es necesario utilizarlo es usado solo por el cliente para procesar el paquete 
\begin{DoxyParams}{Parámetros}
{\em self} & este parametro no es necesario escribir \\
\hline
{\em packet} & este parametro no es necesario escribir \\
\hline
\end{DoxyParams}
\begin{DoxyReturn}{Devuelve}
nada 
\end{DoxyReturn}


Definición en la línea 99 del archivo cliente\_\-lib\_\-original.py.


\begin{DoxyCode}
100                                       :
101                 c="                                   "
102                 numSensor=packet.bufToByte2()
103                 numSensor2=packet.bufToStr(c,15)
104                 print str(numSensor)+" "+str(c.strip())

\end{DoxyCode}
\hypertarget{classcliente__lib__original_1_1cliente__lib_a16a11ee4bc738ae83b652343583ad556}{
\index{cliente\_\-lib\_\-original::cliente\_\-lib@{cliente\_\-lib\_\-original::cliente\_\-lib}!uC\_\-comandos\_\-movi@{uC\_\-comandos\_\-movi}}
\index{uC\_\-comandos\_\-movi@{uC\_\-comandos\_\-movi}!cliente_lib_original::cliente_lib@{cliente\_\-lib\_\-original::cliente\_\-lib}}
\subsubsection[{uC\_\-comandos\_\-movi}]{\setlength{\rightskip}{0pt plus 5cm}def cliente\_\-lib\_\-original.cliente\_\-lib.uC\_\-comandos\_\-movi (
\begin{DoxyParamCaption}
\item[{}]{self, }
\item[{}]{client, }
\item[{}]{comando, }
\item[{}]{parametro}
\end{DoxyParamCaption}
)}}
\label{classcliente__lib__original_1_1cliente__lib_a16a11ee4bc738ae83b652343583ad556}


Sirve para mandar ordenes de movimiento directamente al controlador de la plataforma movil. 


\begin{DoxyParams}{Parámetros}
{\em self} & este parametro no es necesario escribir \\
\hline
{\em client} & Se debe trar el objeto cliente a la definicion para poder utilizar el enlace del cliente para enviar el paquete al servidor \\
\hline
{\em comando} & es un numero de 1-\/255 que representa una funcion esta informacion se puede encontrar en el API de ARIA \\
\hline
{\em parametro} & el parametro de la funcion en caso de no tener se deja el valor en blanco \\
\hline
\end{DoxyParams}
\begin{DoxyReturn}{Devuelve}
client 
\end{DoxyReturn}


Definición en la línea 156 del archivo cliente\_\-lib\_\-original.py.


\begin{DoxyCode}
157                                                            :
158                 mi_comando=comando #comando 8 es MOVE parametro un valor de 5000 
      a -4999 es en mm, 11 LEV y su parametro es velocidad +o- mm/s
159                 mi_parametro=parametro #parametro
160                 packet=ArNetPacket()
161                 packet.strToBuf(mi_comando+" "+mi_parametro)
162                 client.requestOnce("MicroControllerMotionCommand", packet) #Micro
      ControllerMotionCommand
163                 return client

\end{DoxyCode}
\hypertarget{classcliente__lib__original_1_1cliente__lib_a5f97cbead3de2cb78534afec8343c13e}{
\index{cliente\_\-lib\_\-original::cliente\_\-lib@{cliente\_\-lib\_\-original::cliente\_\-lib}!valores@{valores}}
\index{valores@{valores}!cliente_lib_original::cliente_lib@{cliente\_\-lib\_\-original::cliente\_\-lib}}
\subsubsection[{valores}]{\setlength{\rightskip}{0pt plus 5cm}def cliente\_\-lib\_\-original.cliente\_\-lib.valores (
\begin{DoxyParamCaption}
\item[{}]{self, }
\item[{}]{packet}
\end{DoxyParamCaption}
)}}
\label{classcliente__lib__original_1_1cliente__lib_a5f97cbead3de2cb78534afec8343c13e}


Sirve cuando se manda el comando \char`\"{}updateNumbers\char`\"{} en el paquete. 

este comando no es necesario utilizarlo es usado solo por el cliente para procesar el paquete 
\begin{DoxyParams}{Parámetros}
{\em self} & este parametro no es necesario escribir \\
\hline
{\em packet} & el paquete que recibe el cliente del servidor, no es necesario escribir \\
\hline
\end{DoxyParams}
\begin{DoxyReturn}{Devuelve}
Nada, pero guarda en self.valores\_\-fisico \mbox{[}voltaje\_\-bateria,myX,myY,myTh,myVel,myRotVel,myLatVel,myTemperature\mbox{]} 
\end{DoxyReturn}


Definición en la línea 76 del archivo cliente\_\-lib\_\-original.py.


\begin{DoxyCode}
77                                 :
78                 #devuelve los valores voltaje_bateria,myX,myY,myTh,myVel,myRotVel
      ,myLatVel,myTemperature
79                 voltaje_bateria=packet.bufToByte2()/10
80                 myX = packet.bufToByte4()#
81                 myY = packet.bufToByte4()
82                 myTh = packet.bufToByte2()
83                 myVel = packet.bufToByte2()
84                 myRotVel = packet.bufToByte2()
85                 myLatVel = packet.bufToByte2()
86                 myTemperature = packet.bufToByte()
87                 #print "X= "+str(myX)+" y="+str(myY)+" th="+str(myTh)
88                 self.valor_fisico=(voltaje_bateria,myX,myY,myTh,myVel,myRotVel,my
      LatVel,myTemperature)
89                 self.valor_fisico=(voltaje_bateria,myX,myY,myTh,myVel,myRotVel,my
      LatVel,0)
90                 #print valor

\end{DoxyCode}
\hypertarget{classcliente__lib__original_1_1cliente__lib_a2ab0872984bef5af4bfbb7d2ec7f40c5}{
\index{cliente\_\-lib\_\-original::cliente\_\-lib@{cliente\_\-lib\_\-original::cliente\_\-lib}!valores\_\-sonares@{valores\_\-sonares}}
\index{valores\_\-sonares@{valores\_\-sonares}!cliente_lib_original::cliente_lib@{cliente\_\-lib\_\-original::cliente\_\-lib}}
\subsubsection[{valores\_\-sonares}]{\setlength{\rightskip}{0pt plus 5cm}def cliente\_\-lib\_\-original.cliente\_\-lib.valores\_\-sonares (
\begin{DoxyParamCaption}
\item[{}]{self, }
\item[{}]{packet}
\end{DoxyParamCaption}
)}}
\label{classcliente__lib__original_1_1cliente__lib_a2ab0872984bef5af4bfbb7d2ec7f40c5}


Sirve para leer el paquete arNetPacket con los valores del sonar. 

este comando no es necesario utilizarlo es usado solo por el cliente para procesar el paquete 
\begin{DoxyParams}{Parámetros}
{\em self} & este parametro no es necesario escribir \\
\hline
{\em packet} & este parametro no es necesario escribir \\
\hline
\end{DoxyParams}
\begin{DoxyReturn}{Devuelve}
nada 
\end{DoxyReturn}


Definición en la línea 113 del archivo cliente\_\-lib\_\-original.py.


\begin{DoxyCode}
114                                         :
115                 c="                                                              
                                                          "
116                 cantidad=packet.bufToDouble()
117                 dato=range(int(cantidad))
118                 for j in range(int(cantidad)):
119                         dato[j]=c #ya que pasa el buffer a la variable esta debe 
      tener la longitud necesaria y deben ser guardada en cada lectura
120                 for i in range(int(cantidad)):
121                         packet.bufToStr(dato[i],50)
122                         dato[i]=dato[i].replace("\x00"," ") #Porque al final anad
      e un NONE del string
123                         dato[i]=dato[i].strip() 
124                 self.valor_sonares=[cantidad,dato]

\end{DoxyCode}


\subsection{Documentación de los datos miembro}
\hypertarget{classcliente__lib__original_1_1cliente__lib_a049a973bc28127e24c2c717de93daf2f}{
\index{cliente\_\-lib\_\-original::cliente\_\-lib@{cliente\_\-lib\_\-original::cliente\_\-lib}!acu@{acu}}
\index{acu@{acu}!cliente_lib_original::cliente_lib@{cliente\_\-lib\_\-original::cliente\_\-lib}}
\subsubsection[{acu}]{\setlength{\rightskip}{0pt plus 5cm}{\bf cliente\_\-lib\_\-original.cliente\_\-lib.acu}}}
\label{classcliente__lib__original_1_1cliente__lib_a049a973bc28127e24c2c717de93daf2f}


Definición en la línea 58 del archivo cliente\_\-lib\_\-original.py.

\hypertarget{classcliente__lib__original_1_1cliente__lib_a030e232b37138f0a2ba15a3ef5f1fe21}{
\index{cliente\_\-lib\_\-original::cliente\_\-lib@{cliente\_\-lib\_\-original::cliente\_\-lib}!ip@{ip}}
\index{ip@{ip}!cliente_lib_original::cliente_lib@{cliente\_\-lib\_\-original::cliente\_\-lib}}
\subsubsection[{ip}]{\setlength{\rightskip}{0pt plus 5cm}{\bf cliente\_\-lib\_\-original.cliente\_\-lib.ip}}}
\label{classcliente__lib__original_1_1cliente__lib_a030e232b37138f0a2ba15a3ef5f1fe21}


Definición en la línea 58 del archivo cliente\_\-lib\_\-original.py.

\hypertarget{classcliente__lib__original_1_1cliente__lib_ae6a2ed4ac198fc965fbe3ee9a01f9401}{
\index{cliente\_\-lib\_\-original::cliente\_\-lib@{cliente\_\-lib\_\-original::cliente\_\-lib}!t@{t}}
\index{t@{t}!cliente_lib_original::cliente_lib@{cliente\_\-lib\_\-original::cliente\_\-lib}}
\subsubsection[{t}]{\setlength{\rightskip}{0pt plus 5cm}{\bf cliente\_\-lib\_\-original.cliente\_\-lib.t}}}
\label{classcliente__lib__original_1_1cliente__lib_ae6a2ed4ac198fc965fbe3ee9a01f9401}


Definición en la línea 58 del archivo cliente\_\-lib\_\-original.py.

\hypertarget{classcliente__lib__original_1_1cliente__lib_a74cb35b9f3246db8b6b16ac59ae8f320}{
\index{cliente\_\-lib\_\-original::cliente\_\-lib@{cliente\_\-lib\_\-original::cliente\_\-lib}!valor\_\-fisico@{valor\_\-fisico}}
\index{valor\_\-fisico@{valor\_\-fisico}!cliente_lib_original::cliente_lib@{cliente\_\-lib\_\-original::cliente\_\-lib}}
\subsubsection[{valor\_\-fisico}]{\setlength{\rightskip}{0pt plus 5cm}{\bf cliente\_\-lib\_\-original.cliente\_\-lib.valor\_\-fisico}}}
\label{classcliente__lib__original_1_1cliente__lib_a74cb35b9f3246db8b6b16ac59ae8f320}


Definición en la línea 58 del archivo cliente\_\-lib\_\-original.py.

\hypertarget{classcliente__lib__original_1_1cliente__lib_a4172f5914b673eb44e168b4afcbae651}{
\index{cliente\_\-lib\_\-original::cliente\_\-lib@{cliente\_\-lib\_\-original::cliente\_\-lib}!valor\_\-sonares@{valor\_\-sonares}}
\index{valor\_\-sonares@{valor\_\-sonares}!cliente_lib_original::cliente_lib@{cliente\_\-lib\_\-original::cliente\_\-lib}}
\subsubsection[{valor\_\-sonares}]{\setlength{\rightskip}{0pt plus 5cm}{\bf cliente\_\-lib\_\-original.cliente\_\-lib.valor\_\-sonares}}}
\label{classcliente__lib__original_1_1cliente__lib_a4172f5914b673eb44e168b4afcbae651}


Definición en la línea 58 del archivo cliente\_\-lib\_\-original.py.

\hypertarget{classcliente__lib__original_1_1cliente__lib_aa2933583abd7844c57b2ec553cbb46ee}{
\index{cliente\_\-lib\_\-original::cliente\_\-lib@{cliente\_\-lib\_\-original::cliente\_\-lib}!x@{x}}
\index{x@{x}!cliente_lib_original::cliente_lib@{cliente\_\-lib\_\-original::cliente\_\-lib}}
\subsubsection[{x}]{\setlength{\rightskip}{0pt plus 5cm}{\bf cliente\_\-lib\_\-original.cliente\_\-lib.x}}}
\label{classcliente__lib__original_1_1cliente__lib_aa2933583abd7844c57b2ec553cbb46ee}


Definición en la línea 58 del archivo cliente\_\-lib\_\-original.py.

\hypertarget{classcliente__lib__original_1_1cliente__lib_a6050f7c724f8ca063505e8b80a1cdecf}{
\index{cliente\_\-lib\_\-original::cliente\_\-lib@{cliente\_\-lib\_\-original::cliente\_\-lib}!y@{y}}
\index{y@{y}!cliente_lib_original::cliente_lib@{cliente\_\-lib\_\-original::cliente\_\-lib}}
\subsubsection[{y}]{\setlength{\rightskip}{0pt plus 5cm}{\bf cliente\_\-lib\_\-original.cliente\_\-lib.y}}}
\label{classcliente__lib__original_1_1cliente__lib_a6050f7c724f8ca063505e8b80a1cdecf}


Definición en la línea 58 del archivo cliente\_\-lib\_\-original.py.



La documentación para esta clase fue generada a partir del siguiente fichero:\begin{DoxyCompactItemize}
\item 
\hyperlink{cliente__lib__original_8py}{cliente\_\-lib\_\-original.py}\end{DoxyCompactItemize}

\hypertarget{classinicio_1_1prueba__adqui}{
\section{Referencia de la Clase inicio.prueba\_\-adqui}
\label{classinicio_1_1prueba__adqui}\index{inicio::prueba\_\-adqui@{inicio::prueba\_\-adqui}}
}


es la clase encargada del entorno grafico y enlace con \hyperlink{namespacecliente__lib}{cliente\_\-lib}  


\subsection*{Métodos públicos}
\begin{DoxyCompactItemize}
\item 
def \hyperlink{classinicio_1_1prueba__adqui_adab3c8bf1d5be6b8523065ccd474765a}{\_\-\_\-init\_\-\_\-}
\begin{DoxyCompactList}\small\item\em para cargar el XML de gtk+ y sus senales \end{DoxyCompactList}\item 
def \hyperlink{classinicio_1_1prueba__adqui_af626267088da2ac8ad96d85b91850c05}{izq\_\-clicked}
\item 
def \hyperlink{classinicio_1_1prueba__adqui_ad5348d604e42fe8466b4c1f7338068bf}{der\_\-clicked}
\item 
def \hyperlink{classinicio_1_1prueba__adqui_add1484bca105dc4a14533c487e6f7a78}{arriba\_\-clicked}
\item 
def \hyperlink{classinicio_1_1prueba__adqui_abac92d1dcf8c253686ddbf0ef741f692}{abajo\_\-clicked}
\item 
def \hyperlink{classinicio_1_1prueba__adqui_a133dddf01f8261f6cca38064b8ae006f}{parar\_\-clicked}
\item 
def \hyperlink{classinicio_1_1prueba__adqui_a903f8caf83ea99a6a35a31089cee5944}{button1\_\-clicked}
\item 
def \hyperlink{classinicio_1_1prueba__adqui_a24ae8befd66a50575811e300464e7f1a}{generador\_\-mapa}
\item 
def \hyperlink{classinicio_1_1prueba__adqui_a448fd223767febb13d3f929fcddc85cf}{on\_\-maps\_\-destroy}
\end{DoxyCompactItemize}
\subsection*{Atributos públicos}
\begin{DoxyCompactItemize}
\item 
\hyperlink{classinicio_1_1prueba__adqui_a7c099b095d3893076bcd6dcc22aa5ce4}{aument}
\item 
\hyperlink{classinicio_1_1prueba__adqui_a32103b146bd571e752923638e680db8b}{teleoper}
\item 
\hyperlink{classinicio_1_1prueba__adqui_abc25d678f19639848ca3b7509a842566}{boton}
\item 
\hyperlink{classinicio_1_1prueba__adqui_a1f0a7213982dc7228773c19258e425f4}{area}
\item 
\hyperlink{classinicio_1_1prueba__adqui_a1f0c7dfba4bed8128426c5058c255dbe}{imagen}
\item 
\hyperlink{classinicio_1_1prueba__adqui_a073ec339511e7a2660c9ae92c613f293}{nombre\_\-archivo}
\item 
\hyperlink{classinicio_1_1prueba__adqui_a5a6710e93f733c84b360e42513fdd4a9}{a}
\item 
\hyperlink{classinicio_1_1prueba__adqui_a3519a8ae5deb71530289fc1274500b7e}{cliente}
\item 
\hyperlink{classinicio_1_1prueba__adqui_aafd8544e61c02137d45d1202e4330da5}{valores\_\-f}
\end{DoxyCompactItemize}


\subsection{Descripción detallada}
es la clase encargada del entorno grafico y enlace con \hyperlink{namespacecliente__lib}{cliente\_\-lib} 

Definición en la línea 55 del archivo inicio.py.



\subsection{Documentación del constructor y destructor}
\hypertarget{classinicio_1_1prueba__adqui_adab3c8bf1d5be6b8523065ccd474765a}{
\index{inicio::prueba\_\-adqui@{inicio::prueba\_\-adqui}!\_\-\_\-init\_\-\_\-@{\_\-\_\-init\_\-\_\-}}
\index{\_\-\_\-init\_\-\_\-@{\_\-\_\-init\_\-\_\-}!inicio::prueba_adqui@{inicio::prueba\_\-adqui}}
\subsubsection[{\_\-\_\-init\_\-\_\-}]{\setlength{\rightskip}{0pt plus 5cm}def inicio.prueba\_\-adqui.\_\-\_\-init\_\-\_\- (
\begin{DoxyParamCaption}
\item[{}]{self, }
\item[{}]{dimensiones, }
\item[{}]{nombre\_\-archivo, }
\item[{}]{dimensiones\_\-robot}
\end{DoxyParamCaption}
)}}
\label{classinicio_1_1prueba__adqui_adab3c8bf1d5be6b8523065ccd474765a}


para cargar el XML de gtk+ y sus senales 


\begin{DoxyParams}{Parámetros}
{\em self} & no se necesita incluirlo al utilizar la funcion ya que se lo pone solo por ser la definicion de una clase \\
\hline
\end{DoxyParams}
\begin{DoxyReturn}{Devuelve}
self 
\end{DoxyReturn}


Definición en la línea 62 del archivo inicio.py.


\begin{DoxyCode}
63                                                                        :
64                 self.aument=0
65                 builder = gtk.Builder() #El archivo de glade debe estar en gtkbui
      lder
66                 builder.add_from_file("glade/GUI.glade") #Carga el archivo glade
67                 builder.connect_signals(self) #Toma todas las senales de glade
68                 self.teleoper = builder.get_object("map") #Ventana principal
69                 self.boton = builder.get_object("button1")
70                 self.area=builder.get_object("mapa")
71                 self.teleoper.show()
72                 #Inicializa cliente d-bus
73                 #self.client_dbus = DBusClient()
74                 #
75                 #Inicializar visor
76                 #self.client_dbus.visor()
77                 #
78                 #imnicio opencv
79                 self.imagen=mapa(dimensiones,nombre_archivo,dimensiones_robot)
80                 self.nombre_archivo=nombre_archivo
81                 self.imagen.cargar()
82                 #
83                 self.a=cliente_lib();
84                 self.a.ip="192.168.1.102"
85                 self.cliente=self.a.cliente_inicio()
86                 #self.a.cliente_apaga(self.cliente)
87                 try:
88                         self.cliente=self.a.envio_consulta_fisica(self.cliente,"p
      ose")
89                         #valor_x,valor_y,valor_t=self.a.devuelve_valors()
90                         valor_x=self.a.x
91                         valor_y=self.a.y
92                         ArUtil.sleep(100) #cambiar estos comandos por temporisado
      res de python
93                         #sleep(0.1)
94                         self.cliente=self.a.envio_consulta_fisica(self.cliente,"u
      pdateNumbers")
95                         self.valores_f=self.a.devuelve_valorf()
96                         mi_x=self.valores_f[1]
97                         mi_y=self.valores_f[2]
98                         mi_th=self.valores_f[3]
99                         #print valor_x
100                         #print valor_y
101                         print len(valor_x)
102                         self.generador_mapa(valor_x,valor_y,mi_x,mi_y,mi_th)
103                 except:
104                         print "Fallo inicio"
                        pass
\end{DoxyCode}


\subsection{Documentación de las funciones miembro}
\hypertarget{classinicio_1_1prueba__adqui_abac92d1dcf8c253686ddbf0ef741f692}{
\index{inicio::prueba\_\-adqui@{inicio::prueba\_\-adqui}!abajo\_\-clicked@{abajo\_\-clicked}}
\index{abajo\_\-clicked@{abajo\_\-clicked}!inicio::prueba_adqui@{inicio::prueba\_\-adqui}}
\subsubsection[{abajo\_\-clicked}]{\setlength{\rightskip}{0pt plus 5cm}def inicio.prueba\_\-adqui.abajo\_\-clicked (
\begin{DoxyParamCaption}
\item[{}]{self, }
\item[{}]{widget, }
\item[{}]{data = {\ttfamily None}}
\end{DoxyParamCaption}
)}}
\label{classinicio_1_1prueba__adqui_abac92d1dcf8c253686ddbf0ef741f692}


Definición en la línea 177 del archivo inicio.py.


\begin{DoxyCode}
178                                                   :
179                 try:
180                         print "Presiono abajo"
181                         TransRatio,RotRatio,LatRatio = [-50,0,0]
182                         self.cliente=self.a.envio_ratioDrive(self.cliente,TransRa
      tio,RotRatio,LatRatio) #fijar los valores para mover
183                         ArUtil.sleep(100)
184                         #sleep(0.1)
185                         self.cliente=self.a.envio_consulta_fisica(self.cliente,"p
      ose")
186                         #valor_x,valor_y,valor_t=self.a.devuelve_valors()
187                         valor_x=self.a.x
188                         valor_y=self.a.y
189                         self.cliente=self.a.envio_consulta_fisica(self.cliente,"u
      pdateNumbers")
190                         self.valores_f=self.a.devuelve_valorf()
191                         mi_x=self.valores_f[1]
192                         mi_y=self.valores_f[2]
193                         mi_th=self.valores_f[3]
194                         #print self.valores_f
195                         #print valor_x
196                         #print valor_y
197                         print len(valor_x)
198                         self.generador_mapa(valor_x,valor_y,mi_x,mi_y,mi_th)
199                 except:
200                         print "Fallo"
                        pass
\end{DoxyCode}
\hypertarget{classinicio_1_1prueba__adqui_add1484bca105dc4a14533c487e6f7a78}{
\index{inicio::prueba\_\-adqui@{inicio::prueba\_\-adqui}!arriba\_\-clicked@{arriba\_\-clicked}}
\index{arriba\_\-clicked@{arriba\_\-clicked}!inicio::prueba_adqui@{inicio::prueba\_\-adqui}}
\subsubsection[{arriba\_\-clicked}]{\setlength{\rightskip}{0pt plus 5cm}def inicio.prueba\_\-adqui.arriba\_\-clicked (
\begin{DoxyParamCaption}
\item[{}]{self, }
\item[{}]{widget, }
\item[{}]{data = {\ttfamily None}}
\end{DoxyParamCaption}
)}}
\label{classinicio_1_1prueba__adqui_add1484bca105dc4a14533c487e6f7a78}


Definición en la línea 153 del archivo inicio.py.


\begin{DoxyCode}
154                                                    :
155                 try:
156                         print "Presiono arriba"
157                         TransRatio,RotRatio,LatRatio = [50,0,0]
158                         self.cliente=self.a.envio_ratioDrive(self.cliente,TransRa
      tio,RotRatio,LatRatio) #fijar los valores para mover
159                         ArUtil.sleep(100)
160                         #sleep(0.1)
161                         self.cliente=self.a.envio_consulta_fisica(self.cliente,"p
      ose")
162                         #valor_x,valor_y,valor_t=self.a.devuelve_valors()
163                         valor_x=self.a.x
164                         valor_y=self.a.y
165                         self.cliente=self.a.envio_consulta_fisica(self.cliente,"u
      pdateNumbers")
166                         self.valores_f=self.a.devuelve_valorf()
167                         mi_x=self.valores_f[1]
168                         mi_y=self.valores_f[2]
169                         mi_th=self.valores_f[3]
170                         #print self.valores_f
171                         #print valor_x
172                         #print valor_y
173                         print len(valor_x)
174                         self.generador_mapa(valor_x,valor_y,mi_x,mi_y,mi_th)
175                 except:
176                         print "Fallo"
                        pass
\end{DoxyCode}
\hypertarget{classinicio_1_1prueba__adqui_a903f8caf83ea99a6a35a31089cee5944}{
\index{inicio::prueba\_\-adqui@{inicio::prueba\_\-adqui}!button1\_\-clicked@{button1\_\-clicked}}
\index{button1\_\-clicked@{button1\_\-clicked}!inicio::prueba_adqui@{inicio::prueba\_\-adqui}}
\subsubsection[{button1\_\-clicked}]{\setlength{\rightskip}{0pt plus 5cm}def inicio.prueba\_\-adqui.button1\_\-clicked (
\begin{DoxyParamCaption}
\item[{}]{self, }
\item[{}]{widget, }
\item[{}]{data = {\ttfamily None}}
\end{DoxyParamCaption}
)}}
\label{classinicio_1_1prueba__adqui_a903f8caf83ea99a6a35a31089cee5944}


Definición en la línea 224 del archivo inicio.py.


\begin{DoxyCode}
225                                                     :
226                 try:
227                         self.cliente=self.a.envio_consulta_fisica(self.cliente,"p
      ose")
228                         ArUtil.sleep(100)
229                         #sleep(0.1)
230                         #valor_x,valor_y,valor_t=self.a.devuelve_valors()
231                         valor_x=self.a.x
232                         valor_y=self.a.y
233                         self.cliente=self.a.envio_consulta_fisica(self.cliente,"u
      pdateNumbers")
234                         self.valores_f=self.a.devuelve_valorf()
235                         mi_x=self.valores_f[1]
236                         mi_y=self.valores_f[2]
237                         mi_th=self.valores_f[3]
238                         #print self.valores_f
239                         #print valor_x
240                         #print valor_y
241                         print len(valor_x)
242                         self.generador_mapa(valor_x,valor_y,mi_x,mi_y,mi_th)
243                 except:
244                         print "Fallo paro"
245                         pass

\end{DoxyCode}
\hypertarget{classinicio_1_1prueba__adqui_ad5348d604e42fe8466b4c1f7338068bf}{
\index{inicio::prueba\_\-adqui@{inicio::prueba\_\-adqui}!der\_\-clicked@{der\_\-clicked}}
\index{der\_\-clicked@{der\_\-clicked}!inicio::prueba_adqui@{inicio::prueba\_\-adqui}}
\subsubsection[{der\_\-clicked}]{\setlength{\rightskip}{0pt plus 5cm}def inicio.prueba\_\-adqui.der\_\-clicked (
\begin{DoxyParamCaption}
\item[{}]{self, }
\item[{}]{widget, }
\item[{}]{data = {\ttfamily None}}
\end{DoxyParamCaption}
)}}
\label{classinicio_1_1prueba__adqui_ad5348d604e42fe8466b4c1f7338068bf}


Definición en la línea 129 del archivo inicio.py.


\begin{DoxyCode}
130                                                 :
131                 try:
132                         print "presiono derecha"
133                         TransRatio,RotRatio,LatRatio = [0,-90,-90]
134                         self.cliente=self.a.envio_ratioDrive(self.cliente,TransRa
      tio,RotRatio,LatRatio) #fijar los valores para mover
135                         ArUtil.sleep(100)
136                         #sleep(0.1)
137                         self.cliente=self.a.envio_consulta_fisica(self.cliente,"p
      ose")
138                         #valor_x,valor_y,valor_t=self.a.devuelve_valors()
139                         valor_x=self.a.x
140                         valor_y=self.a.y
141                         self.cliente=self.a.envio_consulta_fisica(self.cliente,"u
      pdateNumbers")
142                         self.valores_f=self.a.devuelve_valorf()
143                         mi_x=self.valores_f[1]
144                         mi_y=self.valores_f[2]
145                         mi_th=self.valores_f[3]
146                         #print self.valores_f
147                         #print valor_x
148                         #print valor_y
149                         print len(valor_x)
150                         self.generador_mapa(valor_x,valor_y,mi_x,mi_y,mi_th)
151                 except:
152                         print "Fallo"
                        pass
\end{DoxyCode}
\hypertarget{classinicio_1_1prueba__adqui_a24ae8befd66a50575811e300464e7f1a}{
\index{inicio::prueba\_\-adqui@{inicio::prueba\_\-adqui}!generador\_\-mapa@{generador\_\-mapa}}
\index{generador\_\-mapa@{generador\_\-mapa}!inicio::prueba_adqui@{inicio::prueba\_\-adqui}}
\subsubsection[{generador\_\-mapa}]{\setlength{\rightskip}{0pt plus 5cm}def inicio.prueba\_\-adqui.generador\_\-mapa (
\begin{DoxyParamCaption}
\item[{}]{self, }
\item[{}]{valor\_\-x, }
\item[{}]{valor\_\-y, }
\item[{}]{mi\_\-x, }
\item[{}]{mi\_\-y, }
\item[{}]{mi\_\-th}
\end{DoxyParamCaption}
)}}
\label{classinicio_1_1prueba__adqui_a24ae8befd66a50575811e300464e7f1a}


Definición en la línea 246 del archivo inicio.py.


\begin{DoxyCode}
247                                                                 :
248                 try:
249                         print valor_x
250                         if (len(valor_x)>0):
251                                 for a in range(len(valor_x)): #SIP en c. 
252                                         self.imagen.anadir_punto((valor_x['x%d' %
       a]/100+300,valor_y['y%d' % a]/100+200),radio=2)
253                         self.imagen.crear_imagen()
254                         self.imagen.rotacion_y_posicion_robot(mi_x/100+300,mi_y/1
      00+200,mi_th)
255                         self.area.set_from_file(self.nombre_archivo)
256                 except:
257                         print "Fallo mapa"
258                         pass
        
\end{DoxyCode}
\hypertarget{classinicio_1_1prueba__adqui_af626267088da2ac8ad96d85b91850c05}{
\index{inicio::prueba\_\-adqui@{inicio::prueba\_\-adqui}!izq\_\-clicked@{izq\_\-clicked}}
\index{izq\_\-clicked@{izq\_\-clicked}!inicio::prueba_adqui@{inicio::prueba\_\-adqui}}
\subsubsection[{izq\_\-clicked}]{\setlength{\rightskip}{0pt plus 5cm}def inicio.prueba\_\-adqui.izq\_\-clicked (
\begin{DoxyParamCaption}
\item[{}]{self, }
\item[{}]{widget, }
\item[{}]{data = {\ttfamily None}}
\end{DoxyParamCaption}
)}}
\label{classinicio_1_1prueba__adqui_af626267088da2ac8ad96d85b91850c05}


Definición en la línea 105 del archivo inicio.py.


\begin{DoxyCode}
106                                                 :
107                 try:
108                         print "presiono izquierda"
109                         TransRatio,RotRatio,LatRatio = [0,90,90]
110                         self.cliente=self.a.envio_ratioDrive(self.cliente,TransRa
      tio,RotRatio,LatRatio) #fijar los valores para mover
111                         ArUtil.sleep(100)
112                         #sleep(0.1)
113                         self.cliente=self.a.envio_consulta_fisica(self.cliente,"p
      ose")
114                         #valor_x,valor_y,valor_t=self.a.devuelve_valors()
115                         valor_x=self.a.x
116                         valor_y=self.a.y
117                         self.cliente=self.a.envio_consulta_fisica(self.cliente,"u
      pdateNumbers")
118                         self.valores_f=self.a.devuelve_valorf()
119                         mi_x=self.valores_f[1]
120                         mi_y=self.valores_f[2]
121                         mi_th=self.valores_f[3]
122                         #print self.valores_f
123                         #print valor_x
124                         #print valor_y
125                         print len(valor_x)
126                         self.generador_mapa(valor_x,valor_y,mi_x,mi_y,mi_th)
127                 except:
128                         print "Fallo init"
                        pass
\end{DoxyCode}
\hypertarget{classinicio_1_1prueba__adqui_a448fd223767febb13d3f929fcddc85cf}{
\index{inicio::prueba\_\-adqui@{inicio::prueba\_\-adqui}!on\_\-maps\_\-destroy@{on\_\-maps\_\-destroy}}
\index{on\_\-maps\_\-destroy@{on\_\-maps\_\-destroy}!inicio::prueba_adqui@{inicio::prueba\_\-adqui}}
\subsubsection[{on\_\-maps\_\-destroy}]{\setlength{\rightskip}{0pt plus 5cm}def inicio.prueba\_\-adqui.on\_\-maps\_\-destroy (
\begin{DoxyParamCaption}
\item[{}]{self, }
\item[{}]{widget, }
\item[{}]{data = {\ttfamily None}}
\end{DoxyParamCaption}
)}}
\label{classinicio_1_1prueba__adqui_a448fd223767febb13d3f929fcddc85cf}


Definición en la línea 259 del archivo inicio.py.


\begin{DoxyCode}
260                                                     :
261                 print self.a.cliente_apaga(self.cliente)
                gtk.main_quit()
\end{DoxyCode}
\hypertarget{classinicio_1_1prueba__adqui_a133dddf01f8261f6cca38064b8ae006f}{
\index{inicio::prueba\_\-adqui@{inicio::prueba\_\-adqui}!parar\_\-clicked@{parar\_\-clicked}}
\index{parar\_\-clicked@{parar\_\-clicked}!inicio::prueba_adqui@{inicio::prueba\_\-adqui}}
\subsubsection[{parar\_\-clicked}]{\setlength{\rightskip}{0pt plus 5cm}def inicio.prueba\_\-adqui.parar\_\-clicked (
\begin{DoxyParamCaption}
\item[{}]{self, }
\item[{}]{widget, }
\item[{}]{data = {\ttfamily None}}
\end{DoxyParamCaption}
)}}
\label{classinicio_1_1prueba__adqui_a133dddf01f8261f6cca38064b8ae006f}


Definición en la línea 201 del archivo inicio.py.


\begin{DoxyCode}
202                                                   :
203                 try:
204                         print "Presiono alto"
205                         self.cliente.requestOnce("stop") #parada de emergencia
206                         ArUtil.sleep(100)
207                         #sleep(0.1)
208                         self.cliente=self.a.envio_consulta_fisica(self.cliente,"p
      ose")
209                         #valor_x,valor_y,valor_t=self.a.devuelve_valors()
210                         valor_x=self.a.x
211                         valor_y=self.a.y
212                         self.cliente=self.a.envio_consulta_fisica(self.cliente,"u
      pdateNumbers")
213                         self.valores_f=self.a.devuelve_valorf()
214                         mi_x=self.valores_f[1]
215                         mi_y=self.valores_f[2]
216                         mi_th=self.valores_f[3]
217                         #print self.valores_f
218                         #print valor_x
219                         #print valor_y
220                         print len(valor_x)
221                         self.generador_mapa(valor_x,valor_y,mi_x,mi_y,mi_th)
222                 except:
223                         print "Fallo"
                        pass
\end{DoxyCode}


\subsection{Documentación de los datos miembro}
\hypertarget{classinicio_1_1prueba__adqui_a5a6710e93f733c84b360e42513fdd4a9}{
\index{inicio::prueba\_\-adqui@{inicio::prueba\_\-adqui}!a@{a}}
\index{a@{a}!inicio::prueba_adqui@{inicio::prueba\_\-adqui}}
\subsubsection[{a}]{\setlength{\rightskip}{0pt plus 5cm}{\bf inicio.prueba\_\-adqui.a}}}
\label{classinicio_1_1prueba__adqui_a5a6710e93f733c84b360e42513fdd4a9}


Definición en la línea 62 del archivo inicio.py.

\hypertarget{classinicio_1_1prueba__adqui_a1f0a7213982dc7228773c19258e425f4}{
\index{inicio::prueba\_\-adqui@{inicio::prueba\_\-adqui}!area@{area}}
\index{area@{area}!inicio::prueba_adqui@{inicio::prueba\_\-adqui}}
\subsubsection[{area}]{\setlength{\rightskip}{0pt plus 5cm}{\bf inicio.prueba\_\-adqui.area}}}
\label{classinicio_1_1prueba__adqui_a1f0a7213982dc7228773c19258e425f4}


Definición en la línea 62 del archivo inicio.py.

\hypertarget{classinicio_1_1prueba__adqui_a7c099b095d3893076bcd6dcc22aa5ce4}{
\index{inicio::prueba\_\-adqui@{inicio::prueba\_\-adqui}!aument@{aument}}
\index{aument@{aument}!inicio::prueba_adqui@{inicio::prueba\_\-adqui}}
\subsubsection[{aument}]{\setlength{\rightskip}{0pt plus 5cm}{\bf inicio.prueba\_\-adqui.aument}}}
\label{classinicio_1_1prueba__adqui_a7c099b095d3893076bcd6dcc22aa5ce4}


Definición en la línea 62 del archivo inicio.py.

\hypertarget{classinicio_1_1prueba__adqui_abc25d678f19639848ca3b7509a842566}{
\index{inicio::prueba\_\-adqui@{inicio::prueba\_\-adqui}!boton@{boton}}
\index{boton@{boton}!inicio::prueba_adqui@{inicio::prueba\_\-adqui}}
\subsubsection[{boton}]{\setlength{\rightskip}{0pt plus 5cm}{\bf inicio.prueba\_\-adqui.boton}}}
\label{classinicio_1_1prueba__adqui_abc25d678f19639848ca3b7509a842566}


Definición en la línea 62 del archivo inicio.py.

\hypertarget{classinicio_1_1prueba__adqui_a3519a8ae5deb71530289fc1274500b7e}{
\index{inicio::prueba\_\-adqui@{inicio::prueba\_\-adqui}!cliente@{cliente}}
\index{cliente@{cliente}!inicio::prueba_adqui@{inicio::prueba\_\-adqui}}
\subsubsection[{cliente}]{\setlength{\rightskip}{0pt plus 5cm}{\bf inicio.prueba\_\-adqui.cliente}}}
\label{classinicio_1_1prueba__adqui_a3519a8ae5deb71530289fc1274500b7e}


Definición en la línea 62 del archivo inicio.py.

\hypertarget{classinicio_1_1prueba__adqui_a1f0c7dfba4bed8128426c5058c255dbe}{
\index{inicio::prueba\_\-adqui@{inicio::prueba\_\-adqui}!imagen@{imagen}}
\index{imagen@{imagen}!inicio::prueba_adqui@{inicio::prueba\_\-adqui}}
\subsubsection[{imagen}]{\setlength{\rightskip}{0pt plus 5cm}{\bf inicio.prueba\_\-adqui.imagen}}}
\label{classinicio_1_1prueba__adqui_a1f0c7dfba4bed8128426c5058c255dbe}


Definición en la línea 62 del archivo inicio.py.

\hypertarget{classinicio_1_1prueba__adqui_a073ec339511e7a2660c9ae92c613f293}{
\index{inicio::prueba\_\-adqui@{inicio::prueba\_\-adqui}!nombre\_\-archivo@{nombre\_\-archivo}}
\index{nombre\_\-archivo@{nombre\_\-archivo}!inicio::prueba_adqui@{inicio::prueba\_\-adqui}}
\subsubsection[{nombre\_\-archivo}]{\setlength{\rightskip}{0pt plus 5cm}{\bf inicio.prueba\_\-adqui.nombre\_\-archivo}}}
\label{classinicio_1_1prueba__adqui_a073ec339511e7a2660c9ae92c613f293}


Definición en la línea 62 del archivo inicio.py.

\hypertarget{classinicio_1_1prueba__adqui_a32103b146bd571e752923638e680db8b}{
\index{inicio::prueba\_\-adqui@{inicio::prueba\_\-adqui}!teleoper@{teleoper}}
\index{teleoper@{teleoper}!inicio::prueba_adqui@{inicio::prueba\_\-adqui}}
\subsubsection[{teleoper}]{\setlength{\rightskip}{0pt plus 5cm}{\bf inicio.prueba\_\-adqui.teleoper}}}
\label{classinicio_1_1prueba__adqui_a32103b146bd571e752923638e680db8b}


Definición en la línea 62 del archivo inicio.py.

\hypertarget{classinicio_1_1prueba__adqui_aafd8544e61c02137d45d1202e4330da5}{
\index{inicio::prueba\_\-adqui@{inicio::prueba\_\-adqui}!valores\_\-f@{valores\_\-f}}
\index{valores\_\-f@{valores\_\-f}!inicio::prueba_adqui@{inicio::prueba\_\-adqui}}
\subsubsection[{valores\_\-f}]{\setlength{\rightskip}{0pt plus 5cm}{\bf inicio.prueba\_\-adqui.valores\_\-f}}}
\label{classinicio_1_1prueba__adqui_aafd8544e61c02137d45d1202e4330da5}


Definición en la línea 62 del archivo inicio.py.



La documentación para esta clase fue generada a partir del siguiente fichero:\begin{DoxyCompactItemize}
\item 
\hyperlink{inicio_8py}{inicio.py}\end{DoxyCompactItemize}

\chapter{Documentación de archivos}
\hypertarget{cliente__lib_8py}{
\section{Referencia del Archivo cliente\_\-lib.py}
\label{cliente__lib_8py}\index{cliente\_\-lib.py@{cliente\_\-lib.py}}
}
\subsection*{Clases}
\begin{DoxyCompactItemize}
\item 
class \hyperlink{classcliente__lib_1_1cliente__lib}{cliente\_\-lib.cliente\_\-lib}
\begin{DoxyCompactList}\small\item\em es la clase encargada del cliente \end{DoxyCompactList}\end{DoxyCompactItemize}
\subsection*{Paquetes}
\begin{DoxyCompactItemize}
\item 
package \hyperlink{namespacecliente__lib}{cliente\_\-lib}


\begin{DoxyCompactList}\small\item\em libreria para realizar el cliente \end{DoxyCompactList}

\end{DoxyCompactItemize}
\subsection*{Funciones}
\begin{DoxyCompactItemize}
\item 
def \hyperlink{namespacecliente__lib_afb746084e43cb9c21db470d7b4990cae}{cliente\_\-lib.main}
\begin{DoxyCompactList}\small\item\em Sirve para realizar pruebas de conexion. \end{DoxyCompactList}\end{DoxyCompactItemize}

\hypertarget{cliente__lib__original_8py}{
\section{Referencia del Archivo cliente\_\-lib\_\-original.py}
\label{cliente__lib__original_8py}\index{cliente\_\-lib\_\-original.py@{cliente\_\-lib\_\-original.py}}
}
\subsection*{Clases}
\begin{DoxyCompactItemize}
\item 
class \hyperlink{classcliente__lib__original_1_1cliente__lib}{cliente\_\-lib\_\-original.cliente\_\-lib}
\end{DoxyCompactItemize}
\subsection*{Paquetes}
\begin{DoxyCompactItemize}
\item 
package \hyperlink{namespacecliente__lib__original}{cliente\_\-lib\_\-original}
\item 
package \hyperlink{namespacecliente__lib}{cliente\_\-lib}


\begin{DoxyCompactList}\small\item\em libreria para realizar el cliente \end{DoxyCompactList}

\end{DoxyCompactItemize}
\subsection*{Funciones}
\begin{DoxyCompactItemize}
\item 
def \hyperlink{namespacecliente__lib__original_a41a934cc972855341f15854b1daff426}{cliente\_\-lib\_\-original.main}
\begin{DoxyCompactList}\small\item\em Sirve para realizar pruebas de conexion. \end{DoxyCompactList}\end{DoxyCompactItemize}

\hypertarget{inicio_8py}{
\section{Referencia del Archivo inicio.py}
\label{inicio_8py}\index{inicio.py@{inicio.py}}
}
\subsection*{Clases}
\begin{DoxyCompactItemize}
\item 
class \hyperlink{classinicio_1_1prueba__teleoperacion}{inicio.prueba\_\-teleoperacion}
\begin{DoxyCompactList}\small\item\em Se lo crea como objeto para poder trabajar con las senales de la interfaz grafica. \end{DoxyCompactList}\end{DoxyCompactItemize}
\subsection*{Paquetes}
\begin{DoxyCompactItemize}
\item 
package \hyperlink{namespaceinicio}{inicio}
\end{DoxyCompactItemize}
\subsection*{Variables}
\begin{DoxyCompactItemize}
\item 
tuple \hyperlink{namespaceinicio_a3d259595825f914437642eb35265f3ad}{inicio.app} = prueba\_\-teleoperacion()
\end{DoxyCompactItemize}

\hypertarget{mainpage_8dox}{
\section{Referencia del Archivo mainpage.dox}
\label{mainpage_8dox}\index{mainpage.dox@{mainpage.dox}}
}

\hypertarget{servidor_8py}{
\section{Referencia del Archivo servidor.py}
\label{servidor_8py}\index{servidor.py@{servidor.py}}
}
\subsection*{Paquetes}
\begin{DoxyCompactItemize}
\item 
package \hyperlink{namespaceservidor}{servidor}


\begin{DoxyCompactList}\small\item\em Servidor para el Pioneer P3-\/DX. \item\end{DoxyCompactList}

\end{DoxyCompactItemize}
\subsection*{Funciones}
\begin{DoxyCompactItemize}
\item 
def \hyperlink{namespaceservidor_a06ea535cfe56429259d8de76298416cb}{servidor.requestCallback}
\begin{DoxyCompactList}\small\item\em Sirve cuando se manda el comando \char`\"{}test\char`\"{} en el paquete. \item\end{DoxyCompactList}\item 
def \hyperlink{namespaceservidor_aef5180c02cf1d163167be16732d7250e}{servidor.movimiento}
\begin{DoxyCompactList}\small\item\em Sirve cuando se manda el comando \char`\"{}mover\char`\"{} en el paquete. \item\end{DoxyCompactList}\item 
def \hyperlink{namespaceservidor_a66205c06d52988bbeb961068e554edfc}{servidor.rotar}
\begin{DoxyCompactList}\small\item\em Sirve cuando se manda el comando \char`\"{}rotar\char`\"{} en el paquete. \item\end{DoxyCompactList}\item 
def \hyperlink{namespaceservidor_a4f30fc83b9ff1f43b6dde147ed4c31fc}{servidor.posicion}
\begin{DoxyCompactList}\small\item\em Sirve cuando se manda el comando \char`\"{}pose\char`\"{} en el paquete. \item\end{DoxyCompactList}\end{DoxyCompactItemize}
\subsection*{Variables}
\begin{DoxyCompactItemize}
\item 
tuple \hyperlink{namespaceservidor_a20c40528942a814c3ba639d6fdf80c34}{servidor.robot} = ArRobot()
\item 
tuple \hyperlink{namespaceservidor_a73a7e14af1c6774da1d3c5580e02573c}{servidor.gyro} = ArAnalogGyro(robot)
\item 
tuple \hyperlink{namespaceservidor_a509c8beb9fe73900b5ab2a0fc5f793b2}{servidor.sonarDev} = ArSonarDevice()
\item 
tuple \hyperlink{namespaceservidor_adfced13b57fb69c12f8ab5a84d1a2356}{servidor.server} = ArServerBase()
\item 
tuple \hyperlink{namespaceservidor_a159e1bd236b55d72f3fd3cb94afc7650}{servidor.packet} = ArNetPacket()
\item 
tuple \hyperlink{namespaceservidor_ae78d17158a0f45369b3621468e736bf6}{servidor.con} = ArSimpleConnector(sys.argv)
\item 
tuple \hyperlink{namespaceservidor_a214f853047fdc2f26cfac0db1a9e87b8}{servidor.serverInfoRobot} = ArServerInfoRobot(server, robot)
\item 
tuple \hyperlink{namespaceservidor_ae93d5f481f48442f0db6959867fbf8c4}{servidor.serverInfoSensor} = ArServerInfoSensor(server, robot)
\item 
tuple \hyperlink{namespaceservidor_a691fe74e57a668507d56d3acd8e19ba2}{servidor.drawings} = ArServerInfoDrawings(server)
\item 
tuple \hyperlink{namespaceservidor_ae7b0d696eea14b76aba1e014dbfe70f4}{servidor.modeStop} = ArServerModeStop(server, robot)
\item 
tuple \hyperlink{namespaceservidor_ac7cf650e754b329444bd902af26c1527}{servidor.modeRatioDrive} = ArServerModeRatioDrive(server, robot)
\item 
tuple \hyperlink{namespaceservidor_ae22efed138dafc7284a3f71676a6757f}{servidor.modeWander} = ArServerModeWander(server, robot)
\item 
tuple \hyperlink{namespaceservidor_a4bc8244e73099a1e67418ea064c30097}{servidor.commands} = ArServerHandlerCommands(server)
\item 
tuple \hyperlink{namespaceservidor_aa877c09d4a72ca2434441ff2dc406a29}{servidor.uCCommands} = ArServerSimpleComUC(commands, robot)
\item 
tuple \hyperlink{namespaceservidor_a8f73ad33af2f2a4f3bc9209b9390cfe8}{servidor.loggingCommands} = ArServerSimpleComMovementLogging(commands, robot)
\item 
tuple \hyperlink{namespaceservidor_a9bcfd4f54186b9c69cea8188548af8c8}{servidor.gyroCommands} = ArServerSimpleComGyro(commands, robot, gyro)
\item 
tuple \hyperlink{namespaceservidor_aa210848229916c300b66b7f7958baa24}{servidor.configCommands} = ArServerSimpleComLogRobotConfig(commands, robot)
\end{DoxyCompactItemize}

\printindex
\end{document}
