\documentclass[a4paper]{book}
\usepackage{makeidx}
\usepackage{graphicx}
\usepackage{multicol}
\usepackage{float}
\usepackage{listings}
\usepackage{color}
\usepackage{ifthen}
\usepackage[table]{xcolor}
\usepackage{textcomp}
\usepackage{alltt}
\usepackage{ifpdf}
\ifpdf
\usepackage[pdftex,
            pagebackref=true,
            colorlinks=true,
            linkcolor=blue,
            unicode
           ]{hyperref}
\else
\usepackage[ps2pdf,
            pagebackref=true,
            colorlinks=true,
            linkcolor=blue,
            unicode
           ]{hyperref}
\usepackage{pspicture}
\fi
\usepackage[utf8]{inputenc}
\usepackage[spanish]{babel}
\usepackage{mathptmx}
\usepackage[scaled=.90]{helvet}
\usepackage{courier}
\usepackage{sectsty}
\usepackage[titles]{tocloft}
\usepackage{doxygen}
\lstset{language=C++,inputencoding=utf8,basicstyle=\footnotesize,breaklines=true,breakatwhitespace=true,tabsize=8,numbers=left }
\makeindex
\setcounter{tocdepth}{3}
\renewcommand{\footrulewidth}{0.4pt}
\renewcommand{\familydefault}{\sfdefault}
\begin{document}
\hypersetup{pageanchor=false}
\begin{titlepage}
\vspace*{7cm}
\begin{center}
{\Large Adquisicion de datos \\[1ex]\large 0.1 }\\
\vspace*{1cm}
{\large Generado por Doxygen 1.7.4}\\
\vspace*{0.5cm}
{\small Martes, 7 de Agosto de 2012 10:48:37}\\
\end{center}
\end{titlepage}
\clearemptydoublepage
\pagenumbering{roman}
\tableofcontents
\clearemptydoublepage
\pagenumbering{arabic}
\hypersetup{pageanchor=true}
\chapter{Adquisicion de datos para Pioneer P3-\/DX}
\label{index}\hypertarget{index}{}\begin{center}\section*{Mapas de entornos mediante navegacion difusa y sistema de teleoperacion de una plataforma Pioneer P3-\/DX}\end{center} 

\begin{center}\end{center}   \begin{DoxyAuthor}{Autores}
Danny Vasconez 

Daniel granda
\end{DoxyAuthor}
\hypertarget{index_intro}{}\section{Ejemplo de uso}\label{index_intro}
Esta aplicacion fue disenada para realizar la generacion de mapas de entorno para la plataforma Pioneer P3-\/DX pero por utilizar ArNetPacket puede ser utilizada sin ningun problema por otras plataformas moviles que utilizen ARCOS; la aplicacion cuenta con dos programas el \hyperlink{namespaceservidor__novo}{servidor\_\-novo} y inicio; mas una libreria cliente.


\begin{DoxyItemize}
\item Programas
\end{DoxyItemize}
\begin{DoxyEnumerate}
\item \hyperlink{servidor__novo_8py}{servidor\_\-novo.py}
\item \hyperlink{inicio_8py}{inicio.py}
\end{DoxyEnumerate}
\begin{DoxyItemize}
\item Libreria
\end{DoxyItemize}
\begin{DoxyEnumerate}
\item \hyperlink{cliente__lib_8py}{cliente\_\-lib.py}
\item \hyperlink{renderizado_8py}{renderizado.py}
\end{DoxyEnumerate}
\begin{DoxyItemize}
\item Dependdencias de Servidor
\end{DoxyItemize}
\begin{DoxyEnumerate}
\item AriaPy
\item NetWorkingPy
\item sys
\end{DoxyEnumerate}
\begin{DoxyItemize}
\item Dependencias de \hyperlink{namespacecliente__lib}{cliente\_\-lib}
\end{DoxyItemize}
\begin{DoxyEnumerate}
\item AriaPy
\item NetWorkingPy
\item sys
\end{DoxyEnumerate}
\begin{DoxyItemize}
\item Dependencias de inicio
\end{DoxyItemize}
\begin{DoxyEnumerate}
\item \hyperlink{namespacecliente__lib}{cliente\_\-lib}
\item pygtk
\item gtk
\item os
\item sys
\end{DoxyEnumerate}
\begin{DoxyItemize}
\item Dependencias de renderizado
\end{DoxyItemize}
\begin{DoxyEnumerate}
\item opencv
\end{DoxyEnumerate}

Tener en cuenta al trabajar los siguientes archivos
\begin{DoxyItemize}
\item glade
\end{DoxyItemize}
\begin{DoxyEnumerate}
\item Es el archivo xml de la interfaz grafica de la aplicacion
\end{DoxyEnumerate}



 \hypertarget{index_faqs}{}\section{Problemas frecuentes}\label{index_faqs}
Al utilizar un clable USB-\/Serial no encuentra a la plataforma movil, se debe ingresar los siguientes comandos en consola 
\begin{DoxyVerbInclude}
rm /dev/ttyS0
ln -s /dev/ttyUSB0 /dev/ttyS0

\end{DoxyVerbInclude}
 El primero solo en caso de que se tenga que trabajar necesariamente con el serial 0, de esta manera se realiza un enlace simbolico del ttyUSB0 a ttyS0. Este problema no se puede corregir ya que es de ARIA 

 \hypertarget{index_notes}{}\section{Licencia}\label{index_notes}

\begin{DoxyVerbInclude}
GNU GENERAL PUBLIC LICENSE

Version 2, June 1991

Copyright (C) 1989, 1991 Free Software Foundation, Inc.  
51 Franklin Street, Fifth Floor, Boston, MA  02110-1301, USA

Everyone is permitted to copy and distribute verbatim copies
of this license document, but changing it is not allowed.
Preamble

The licenses for most software are designed to take away your freedom to share and change it. By contrast, the GNU General Public License is intended to guarantee your freedom to share and change free software--to make sure the software is free for all its users. This General Public License applies to most of the Free Software Foundation's software and to any other program whose authors commit to using it. (Some other Free Software Foundation software is covered by the GNU Lesser General Public License instead.) You can apply it to your programs, too.

When we speak of free software, we are referring to freedom, not price. Our General Public Licenses are designed to make sure that you have the freedom to distribute copies of free software (and charge for this service if you wish), that you receive source code or can get it if you want it, that you can change the software or use pieces of it in new free programs; and that you know you can do these things.

To protect your rights, we need to make restrictions that forbid anyone to deny you these rights or to ask you to surrender the rights. These restrictions translate to certain responsibilities for you if you distribute copies of the software, or if you modify it.

For example, if you distribute copies of such a program, whether gratis or for a fee, you must give the recipients all the rights that you have. You must make sure that they, too, receive or can get the source code. And you must show them these terms so they know their rights.

We protect your rights with two steps: (1) copyright the software, and (2) offer you this license which gives you legal permission to copy, distribute and/or modify the software.

Also, for each author's protection and ours, we want to make certain that everyone understands that there is no warranty for this free software. If the software is modified by someone else and passed on, we want its recipients to know that what they have is not the original, so that any problems introduced by others will not reflect on the original authors' reputations.

Finally, any free program is threatened constantly by software patents. We wish to avoid the danger that redistributors of a free program will individually obtain patent licenses, in effect making the program proprietary. To prevent this, we have made it clear that any patent must be licensed for everyone's free use or not licensed at all.

The precise terms and conditions for copying, distribution and modification follow.

TERMS AND CONDITIONS FOR COPYING, DISTRIBUTION AND MODIFICATION

0. This License applies to any program or other work which contains a notice placed by the copyright holder saying it may be distributed under the terms of this General Public License. The "Program", below, refers to any such program or work, and a "work based on the Program" means either the Program or any derivative work under copyright law: that is to say, a work containing the Program or a portion of it, either verbatim or with modifications and/or translated into another language. (Hereinafter, translation is included without limitation in the term "modification".) Each licensee is addressed as "you".

Activities other than copying, distribution and modification are not covered by this License; they are outside its scope. The act of running the Program is not restricted, and the output from the Program is covered only if its contents constitute a work based on the Program (independent of having been made by running the Program). Whether that is true depends on what the Program does.

1. You may copy and distribute verbatim copies of the Program's source code as you receive it, in any medium, provided that you conspicuously and appropriately publish on each copy an appropriate copyright notice and disclaimer of warranty; keep intact all the notices that refer to this License and to the absence of any warranty; and give any other recipients of the Program a copy of this License along with the Program.

You may charge a fee for the physical act of transferring a copy, and you may at your option offer warranty protection in exchange for a fee.

2. You may modify your copy or copies of the Program or any portion of it, thus forming a work based on the Program, and copy and distribute such modifications or work under the terms of Section 1 above, provided that you also meet all of these conditions:

a) You must cause the modified files to carry prominent notices stating that you changed the files and the date of any change.
b) You must cause any work that you distribute or publish, that in whole or in part contains or is derived from the Program or any part thereof, to be licensed as a whole at no charge to all third parties under the terms of this License.
c) If the modified program normally reads commands interactively when run, you must cause it, when started running for such interactive use in the most ordinary way, to print or display an announcement including an appropriate copyright notice and a notice that there is no warranty (or else, saying that you provide a warranty) and that users may redistribute the program under these conditions, and telling the user how to view a copy of this License. (Exception: if the Program itself is interactive but does not normally print such an announcement, your work based on the Program is not required to print an announcement.)
These requirements apply to the modified work as a whole. If identifiable sections of that work are not derived from the Program, and can be reasonably considered independent and separate works in themselves, then this License, and its terms, do not apply to those sections when you distribute them as separate works. But when you distribute the same sections as part of a whole which is a work based on the Program, the distribution of the whole must be on the terms of this License, whose permissions for other licensees extend to the entire whole, and thus to each and every part regardless of who wrote it.

Thus, it is not the intent of this section to claim rights or contest your rights to work written entirely by you; rather, the intent is to exercise the right to control the distribution of derivative or collective works based on the Program.

In addition, mere aggregation of another work not based on the Program with the Program (or with a work based on the Program) on a volume of a storage or distribution medium does not bring the other work under the scope of this License.

3. You may copy and distribute the Program (or a work based on it, under Section 2) in object code or executable form under the terms of Sections 1 and 2 above provided that you also do one of the following:

a) Accompany it with the complete corresponding machine-readable source code, which must be distributed under the terms of Sections 1 and 2 above on a medium customarily used for software interchange; or,
b) Accompany it with a written offer, valid for at least three years, to give any third party, for a charge no more than your cost of physically performing source distribution, a complete machine-readable copy of the corresponding source code, to be distributed under the terms of Sections 1 and 2 above on a medium customarily used for software interchange; or,
c) Accompany it with the information you received as to the offer to distribute corresponding source code. (This alternative is allowed only for noncommercial distribution and only if you received the program in object code or executable form with such an offer, in accord with Subsection b above.)
The source code for a work means the preferred form of the work for making modifications to it. For an executable work, complete source code means all the source code for all modules it contains, plus any associated interface definition files, plus the scripts used to control compilation and installation of the executable. However, as a special exception, the source code distributed need not include anything that is normally distributed (in either source or binary form) with the major components (compiler, kernel, and so on) of the operating system on which the executable runs, unless that component itself accompanies the executable.

If distribution of executable or object code is made by offering access to copy from a designated place, then offering equivalent access to copy the source code from the same place counts as distribution of the source code, even though third parties are not compelled to copy the source along with the object code.

4. You may not copy, modify, sublicense, or distribute the Program except as expressly provided under this License. Any attempt otherwise to copy, modify, sublicense or distribute the Program is void, and will automatically terminate your rights under this License. However, parties who have received copies, or rights, from you under this License will not have their licenses terminated so long as such parties remain in full compliance.

5. You are not required to accept this License, since you have not signed it. However, nothing else grants you permission to modify or distribute the Program or its derivative works. These actions are prohibited by law if you do not accept this License. Therefore, by modifying or distributing the Program (or any work based on the Program), you indicate your acceptance of this License to do so, and all its terms and conditions for copying, distributing or modifying the Program or works based on it.

6. Each time you redistribute the Program (or any work based on the Program), the recipient automatically receives a license from the original licensor to copy, distribute or modify the Program subject to these terms and conditions. You may not impose any further restrictions on the recipients' exercise of the rights granted herein. You are not responsible for enforcing compliance by third parties to this License.

7. If, as a consequence of a court judgment or allegation of patent infringement or for any other reason (not limited to patent issues), conditions are imposed on you (whether by court order, agreement or otherwise) that contradict the conditions of this License, they do not excuse you from the conditions of this License. If you cannot distribute so as to satisfy simultaneously your obligations under this License and any other pertinent obligations, then as a consequence you may not distribute the Program at all. For example, if a patent license would not permit royalty-free redistribution of the Program by all those who receive copies directly or indirectly through you, then the only way you could satisfy both it and this License would be to refrain entirely from distribution of the Program.

If any portion of this section is held invalid or unenforceable under any particular circumstance, the balance of the section is intended to apply and the section as a whole is intended to apply in other circumstances.

It is not the purpose of this section to induce you to infringe any patents or other property right claims or to contest validity of any such claims; this section has the sole purpose of protecting the integrity of the free software distribution system, which is implemented by public license practices. Many people have made generous contributions to the wide range of software distributed through that system in reliance on consistent application of that system; it is up to the author/donor to decide if he or she is willing to distribute software through any other system and a licensee cannot impose that choice.

This section is intended to make thoroughly clear what is believed to be a consequence of the rest of this License.

8. If the distribution and/or use of the Program is restricted in certain countries either by patents or by copyrighted interfaces, the original copyright holder who places the Program under this License may add an explicit geographical distribution limitation excluding those countries, so that distribution is permitted only in or among countries not thus excluded. In such case, this License incorporates the limitation as if written in the body of this License.

9. The Free Software Foundation may publish revised and/or new versions of the General Public License from time to time. Such new versions will be similar in spirit to the present version, but may differ in detail to address new problems or concerns.

Each version is given a distinguishing version number. If the Program specifies a version number of this License which applies to it and "any later version", you have the option of following the terms and conditions either of that version or of any later version published by the Free Software Foundation. If the Program does not specify a version number of this License, you may choose any version ever published by the Free Software Foundation.

10. If you wish to incorporate parts of the Program into other free programs whose distribution conditions are different, write to the author to ask for permission. For software which is copyrighted by the Free Software Foundation, write to the Free Software Foundation; we sometimes make exceptions for this. Our decision will be guided by the two goals of preserving the free status of all derivatives of our free software and of promoting the sharing and reuse of software generally.

NO WARRANTY

11. BECAUSE THE PROGRAM IS LICENSED FREE OF CHARGE, THERE IS NO WARRANTY FOR THE PROGRAM, TO THE EXTENT PERMITTED BY APPLICABLE LAW. EXCEPT WHEN OTHERWISE STATED IN WRITING THE COPYRIGHT HOLDERS AND/OR OTHER PARTIES PROVIDE THE PROGRAM "AS IS" WITHOUT WARRANTY OF ANY KIND, EITHER EXPRESSED OR IMPLIED, INCLUDING, BUT NOT LIMITED TO, THE IMPLIED WARRANTIES OF MERCHANTABILITY AND FITNESS FOR A PARTICULAR PURPOSE. THE ENTIRE RISK AS TO THE QUALITY AND PERFORMANCE OF THE PROGRAM IS WITH YOU. SHOULD THE PROGRAM PROVE DEFECTIVE, YOU ASSUME THE COST OF ALL NECESSARY SERVICING, REPAIR OR CORRECTION.

12. IN NO EVENT UNLESS REQUIRED BY APPLICABLE LAW OR AGREED TO IN WRITING WILL ANY COPYRIGHT HOLDER, OR ANY OTHER PARTY WHO MAY MODIFY AND/OR REDISTRIBUTE THE PROGRAM AS PERMITTED ABOVE, BE LIABLE TO YOU FOR DAMAGES, INCLUDING ANY GENERAL, SPECIAL, INCIDENTAL OR CONSEQUENTIAL DAMAGES ARISING OUT OF THE USE OR INABILITY TO USE THE PROGRAM (INCLUDING BUT NOT LIMITED TO LOSS OF DATA OR DATA BEING RENDERED INACCURATE OR LOSSES SUSTAINED BY YOU OR THIRD PARTIES OR A FAILURE OF THE PROGRAM TO OPERATE WITH ANY OTHER PROGRAMS), EVEN IF SUCH HOLDER OR OTHER PARTY HAS BEEN ADVISED OF THE POSSIBILITY OF SUCH DAMAGES.

END OF TERMS AND CONDITIONS

\end{DoxyVerbInclude}
 
\chapter{Lista de tareas pendientes}
\label{todo}
\hypertarget{todo}{}
\label{todo__todo000001}
\hypertarget{todo__todo000001}{}
 
\begin{DoxyDescription}
\item[page \hyperlink{index}{Adquisicion de datos para Pioneer P3-\/DX} ]Presentar borrador 
\end{DoxyDescription}
\chapter{Lista de bugs}
\label{bug}
\hypertarget{bug}{}
\label{bug__bug000001}
\hypertarget{bug__bug000001}{}
 
\begin{DoxyDescription}
\item[Namespace \hyperlink{namespacecliente__lib}{cliente\_\-lib} ]myTemperature tiene falla por el tipo de dato en formato a python '' toca pasar a '0x81' 
\end{DoxyDescription}

\label{bug__bug000002}
\hypertarget{bug__bug000002}{}
 
\begin{DoxyDescription}
\item[Namespace \hyperlink{namespaceservidor}{servidor} ]Nada 
\end{DoxyDescription}
\chapter{Indice de namespaces}
\section{Paquetes}
Aquí van los paquetes con una breve descripción (si etá disponible):\begin{DoxyCompactList}
\item\contentsline{section}{\hyperlink{namespacecliente__lib}{cliente\_\-lib} (Libreria para realizar el cliente )}{\pageref{namespacecliente__lib}}{}
\item\contentsline{section}{\hyperlink{namespacecliente__lib__original}{cliente\_\-lib\_\-original} }{\pageref{namespacecliente__lib__original}}{}
\item\contentsline{section}{\hyperlink{namespaceinicio}{inicio} }{\pageref{namespaceinicio}}{}
\item\contentsline{section}{\hyperlink{namespaceservidor}{servidor} (Servidor para el Pioneer P3-\/DX )}{\pageref{namespaceservidor}}{}
\end{DoxyCompactList}

\chapter{Índice de clases}
\section{Lista de clases}
Lista de las clases, estructuras, uniones e interfaces con una breve descripción:\begin{DoxyCompactList}
\item\contentsline{section}{\hyperlink{classcliente__lib_1_1cliente__lib}{cliente\_\-lib.cliente\_\-lib} (No dispone ninguna utilidad )}{\pageref{classcliente__lib_1_1cliente__lib}}{}
\item\contentsline{section}{\hyperlink{classinicio_1_1prueba__teleoperacion}{inicio.prueba\_\-teleoperacion} (Se lo crea como objeto para poder trabajar con las senales de la interfaz grafica )}{\pageref{classinicio_1_1prueba__teleoperacion}}{}
\end{DoxyCompactList}

\chapter{Indice de archivos}
\section{Lista de archivos}
Lista de todos los archivos con descripciones breves:\begin{DoxyCompactList}
\item\contentsline{section}{\hyperlink{cliente__lib_8py}{cliente\_\-lib.py} }{\pageref{cliente__lib_8py}}{}
\item\contentsline{section}{\hyperlink{inicio_8py}{inicio.py} }{\pageref{inicio_8py}}{}
\item\contentsline{section}{\hyperlink{servidor_8py}{servidor.py} }{\pageref{servidor_8py}}{}
\end{DoxyCompactList}

\chapter{Documentación de namespaces}
\hypertarget{namespacecliente__lib}{
\section{Paquetes cliente\_\-lib}
\label{namespacecliente__lib}\index{cliente\_\-lib@{cliente\_\-lib}}
}


libreria para realizar el cliente  


\subsection*{Clases}
\begin{DoxyCompactItemize}
\item 
class \hyperlink{classcliente__lib_1_1cliente__lib}{cliente\_\-lib}
\begin{DoxyCompactList}\small\item\em es la clase encargada del cliente \item\end{DoxyCompactList}\end{DoxyCompactItemize}
\subsection*{Funciones}
\begin{DoxyCompactItemize}
\item 
def \hyperlink{namespacecliente__lib_afb746084e43cb9c21db470d7b4990cae}{main}
\begin{DoxyCompactList}\small\item\em Sirve para realizar pruebas de conexion. \item\end{DoxyCompactList}\end{DoxyCompactItemize}


\subsection{Descripci�n detallada}
libreria para realizar el cliente Se debe especificar cual es el IP del servidor \begin{DoxyAuthor}{Autores}
Danny Vasconez 

Daniel Granda 
\end{DoxyAuthor}
\begin{DoxyVersion}{Versi�n}
0.0.2 
\end{DoxyVersion}
\begin{DoxyDate}{Fecha}
2012 
\end{DoxyDate}
\begin{DoxyPrecond}{Precondici�n}
Tener funcionando el servidor 
\end{DoxyPrecond}
\begin{Desc}
\item[\hyperlink{bug__bug000001}{Bug}]myTemperature tiene falla por el tipo de dato en formato a python '' toca pasar a '0x81' \end{Desc}
\begin{DoxyWarning}{Atenci�n}
uso inapropiado puede hacer que la aplicacion falle
\end{DoxyWarning}
\hypertarget{index_intro}{}\subsection{Ejemplo de uso}\label{index_intro}
En el ejemplo se muestra tres maneras de enviar comandos la general que es requestOnce y las otras que son la misma pero modificada para trabajar con comandos especificos 
\begin{DoxyVerbInclude}
a=cliente_lib() #Instancia a la clase cliete_lib
a.ip="192.168.1.10"
CLIENTE=a.cliente_inicio() 
# para realizar movimiento 
TransRatio,RotRatio,LatRatio = [-50,0,0]
CLIENTE=a.envio_ratioDrive(CLIENTE,TransRatio,RotRatio,LatRatio) #fijar los valores para mover
#
#para conocer valores como fisicos de la plataforma movil
CLIENTE=a.envio_consulta_fisica(CLIENTE,"updateNumbers")
valor=a.devuelve_valorf()
print valor
#
#para conocer los valores de los sonares
CLIENTE.requestOnce("pose")
valor=a.devuelve_valors()
print valor
#
ArUtil.sleep(1000)
a.cliente_apaga(CLIENTE)

\end{DoxyVerbInclude}
 

\subsection{Documentaci�n de las funciones}
\hypertarget{namespacecliente__lib_afb746084e43cb9c21db470d7b4990cae}{
\index{cliente\_\-lib@{cliente\_\-lib}!main@{main}}
\index{main@{main}!cliente_lib@{cliente\_\-lib}}
\subsubsection[{main}]{\setlength{\rightskip}{0pt plus 5cm}def cliente\_\-lib.main (
\begin{DoxyParamCaption}
{}
\end{DoxyParamCaption}
)}}
\label{namespacecliente__lib_afb746084e43cb9c21db470d7b4990cae}


Sirve para realizar pruebas de conexion. 

sin tener que ejecutar la aplicacion completa; de la siguiente forma \char`\"{}python2.5 cliente\_\-lib.py\char`\"{} \begin{DoxyReturn}{Devuelve}
0 
\end{DoxyReturn}


Definici�n en la l�nea 236 del archivo cliente\_\-lib.py.




\begin{DoxyCode}
237           :
238         #prueba de la libreria
239         a=cliente_lib()
240         a.ip="192.168.1.10" #Si el servidor esta en otra maquina 
241         CLIENTE=a.cliente_inicio()
242         TransRatio,RotRatio,LatRatio = [-50,0,0]
243         CLIENTE=a.envio_ratioDrive(CLIENTE,TransRatio,RotRatio,LatRatio) #fijar l
      os valores para mover
244         CLIENTE=a.envio_consulta_fisica(CLIENTE,"updateNumbers")
245         valor=a.devuelve_valorf()
246         print valor
247         CLIENTE.requestOnce("pose")
248         ArUtil.sleep(1000)
249         a.cliente_apaga(CLIENTE)
        return 0
\end{DoxyCode}



\hypertarget{namespacecliente__lib__original}{
\section{Paquetes cliente\_\-lib\_\-original}
\label{namespacecliente__lib__original}\index{cliente\_\-lib\_\-original@{cliente\_\-lib\_\-original}}
}
\subsection*{Clases}
\begin{DoxyCompactItemize}
\item 
class \hyperlink{classcliente__lib__original_1_1cliente__lib}{cliente\_\-lib}
\end{DoxyCompactItemize}
\subsection*{Funciones}
\begin{DoxyCompactItemize}
\item 
def \hyperlink{namespacecliente__lib__original_a41a934cc972855341f15854b1daff426}{main}
\begin{DoxyCompactList}\small\item\em Sirve para realizar pruebas de conexion. \end{DoxyCompactList}\end{DoxyCompactItemize}


\subsection{Documentación de las funciones}
\hypertarget{namespacecliente__lib__original_a41a934cc972855341f15854b1daff426}{
\index{cliente\_\-lib\_\-original@{cliente\_\-lib\_\-original}!main@{main}}
\index{main@{main}!cliente_lib_original@{cliente\_\-lib\_\-original}}
\subsubsection[{main}]{\setlength{\rightskip}{0pt plus 5cm}def cliente\_\-lib\_\-original.main (
\begin{DoxyParamCaption}
{}
\end{DoxyParamCaption}
)}}
\label{namespacecliente__lib__original_a41a934cc972855341f15854b1daff426}


Sirve para realizar pruebas de conexion. 

sin tener que ejecutar la aplicacion completa; de la siguiente forma \char`\"{}python2.5 cliente\_\-lib.py\char`\"{} \begin{DoxyReturn}{Devuelve}
0 
\end{DoxyReturn}


Definición en la línea 273 del archivo cliente\_\-lib\_\-original.py.


\begin{DoxyCode}
274           :
275         #prueba de la libreria
276         a=cliente_lib()
277         a.ip="192.168.0.124" #Si el servidor esta en otra maquina 
278         CLIENTE=a.cliente_inicio()
279         TransRatio,RotRatio,LatRatio = [-50,0,0]
280         CLIENTE=a.envio_ratioDrive(CLIENTE,TransRatio,RotRatio,LatRatio) #fijar l
      os valores para mover
281         CLIENTE=a.envio_consulta_fisica(CLIENTE,"updateNumbers")
282         valor=a.devuelve_valorf()
283         print valor
284         CLIENTE.requestOnce("pose")
285         ArUtil.sleep(1000)
286         a.cliente_apaga(CLIENTE)
        return 0
\end{DoxyCode}

\hypertarget{namespaceinicio}{
\section{Paquetes inicio}
\label{namespaceinicio}\index{inicio@{inicio}}
}
\subsection*{Clases}
\begin{DoxyCompactItemize}
\item 
class \hyperlink{classinicio_1_1prueba__teleoperacion}{prueba\_\-teleoperacion}
\begin{DoxyCompactList}\small\item\em Se lo crea como objeto para poder trabajar con las senales de la interfaz grafica. \end{DoxyCompactList}\end{DoxyCompactItemize}
\subsection*{Variables}
\begin{DoxyCompactItemize}
\item 
tuple \hyperlink{namespaceinicio_a3d259595825f914437642eb35265f3ad}{app} = \hyperlink{classinicio_1_1prueba__teleoperacion}{prueba\_\-teleoperacion}()
\end{DoxyCompactItemize}


\subsection{Documentación de las variables}
\hypertarget{namespaceinicio_a3d259595825f914437642eb35265f3ad}{
\index{inicio@{inicio}!app@{app}}
\index{app@{app}!inicio@{inicio}}
\subsubsection[{app}]{\setlength{\rightskip}{0pt plus 5cm}tuple {\bf inicio.app} = {\bf prueba\_\-teleoperacion}()}}
\label{namespaceinicio_a3d259595825f914437642eb35265f3ad}


Definición en la línea 188 del archivo inicio.py.


\hypertarget{namespaceservidor}{
\section{Paquetes servidor}
\label{namespaceservidor}\index{servidor@{servidor}}
}


Servidor para el Pioneer P3-\/DX.  


\subsection*{Funciones}
\begin{DoxyCompactItemize}
\item 
def \hyperlink{namespaceservidor_a06ea535cfe56429259d8de76298416cb}{requestCallback}
\begin{DoxyCompactList}\small\item\em Sirve cuando se manda el comando \char`\"{}test\char`\"{} en el paquete. \end{DoxyCompactList}\item 
def \hyperlink{namespaceservidor_aef5180c02cf1d163167be16732d7250e}{movimiento}
\begin{DoxyCompactList}\small\item\em Sirve cuando se manda el comando \char`\"{}mover\char`\"{} en el paquete. \end{DoxyCompactList}\item 
def \hyperlink{namespaceservidor_a66205c06d52988bbeb961068e554edfc}{rotar}
\begin{DoxyCompactList}\small\item\em Sirve cuando se manda el comando \char`\"{}rotar\char`\"{} en el paquete. \end{DoxyCompactList}\item 
def \hyperlink{namespaceservidor_a4f30fc83b9ff1f43b6dde147ed4c31fc}{posicion}
\begin{DoxyCompactList}\small\item\em Sirve cuando se manda el comando \char`\"{}pose\char`\"{} en el paquete. \end{DoxyCompactList}\end{DoxyCompactItemize}
\subsection*{Variables}
\begin{DoxyCompactItemize}
\item 
tuple \hyperlink{namespaceservidor_a20c40528942a814c3ba639d6fdf80c34}{robot} = ArRobot()
\item 
tuple \hyperlink{namespaceservidor_a73a7e14af1c6774da1d3c5580e02573c}{gyro} = ArAnalogGyro(\hyperlink{namespaceservidor_a20c40528942a814c3ba639d6fdf80c34}{robot})
\item 
tuple \hyperlink{namespaceservidor_a509c8beb9fe73900b5ab2a0fc5f793b2}{sonarDev} = ArSonarDevice(2)
\item 
tuple \hyperlink{namespaceservidor_adfced13b57fb69c12f8ab5a84d1a2356}{server} = ArServerBase()
\item 
tuple \hyperlink{namespaceservidor_a159e1bd236b55d72f3fd3cb94afc7650}{packet} = ArNetPacket()
\item 
tuple \hyperlink{namespaceservidor_ae78d17158a0f45369b3621468e736bf6}{con} = ArSimpleConnector(sys.argv)
\item 
tuple \hyperlink{namespaceservidor_a214f853047fdc2f26cfac0db1a9e87b8}{serverInfoRobot} = ArServerInfoRobot(\hyperlink{namespaceservidor_adfced13b57fb69c12f8ab5a84d1a2356}{server}, \hyperlink{namespaceservidor_a20c40528942a814c3ba639d6fdf80c34}{robot})
\item 
tuple \hyperlink{namespaceservidor_ae93d5f481f48442f0db6959867fbf8c4}{serverInfoSensor} = ArServerInfoSensor(\hyperlink{namespaceservidor_adfced13b57fb69c12f8ab5a84d1a2356}{server}, \hyperlink{namespaceservidor_a20c40528942a814c3ba639d6fdf80c34}{robot})
\item 
tuple \hyperlink{namespaceservidor_a691fe74e57a668507d56d3acd8e19ba2}{drawings} = ArServerInfoDrawings(\hyperlink{namespaceservidor_adfced13b57fb69c12f8ab5a84d1a2356}{server})
\item 
tuple \hyperlink{namespaceservidor_ae7b0d696eea14b76aba1e014dbfe70f4}{modeStop} = ArServerModeStop(\hyperlink{namespaceservidor_adfced13b57fb69c12f8ab5a84d1a2356}{server}, \hyperlink{namespaceservidor_a20c40528942a814c3ba639d6fdf80c34}{robot})
\item 
tuple \hyperlink{namespaceservidor_ac7cf650e754b329444bd902af26c1527}{modeRatioDrive} = ArServerModeRatioDrive(\hyperlink{namespaceservidor_adfced13b57fb69c12f8ab5a84d1a2356}{server}, \hyperlink{namespaceservidor_a20c40528942a814c3ba639d6fdf80c34}{robot})
\item 
tuple \hyperlink{namespaceservidor_ae22efed138dafc7284a3f71676a6757f}{modeWander} = ArServerModeWander(\hyperlink{namespaceservidor_adfced13b57fb69c12f8ab5a84d1a2356}{server}, \hyperlink{namespaceservidor_a20c40528942a814c3ba639d6fdf80c34}{robot})
\item 
tuple \hyperlink{namespaceservidor_a4bc8244e73099a1e67418ea064c30097}{commands} = ArServerHandlerCommands(\hyperlink{namespaceservidor_adfced13b57fb69c12f8ab5a84d1a2356}{server})
\item 
tuple \hyperlink{namespaceservidor_aa877c09d4a72ca2434441ff2dc406a29}{uCCommands} = ArServerSimpleComUC(\hyperlink{namespaceservidor_a4bc8244e73099a1e67418ea064c30097}{commands}, \hyperlink{namespaceservidor_a20c40528942a814c3ba639d6fdf80c34}{robot})
\item 
tuple \hyperlink{namespaceservidor_a8f73ad33af2f2a4f3bc9209b9390cfe8}{loggingCommands} = ArServerSimpleComMovementLogging(\hyperlink{namespaceservidor_a4bc8244e73099a1e67418ea064c30097}{commands}, \hyperlink{namespaceservidor_a20c40528942a814c3ba639d6fdf80c34}{robot})
\item 
tuple \hyperlink{namespaceservidor_a9bcfd4f54186b9c69cea8188548af8c8}{gyroCommands} = ArServerSimpleComGyro(\hyperlink{namespaceservidor_a4bc8244e73099a1e67418ea064c30097}{commands}, \hyperlink{namespaceservidor_a20c40528942a814c3ba639d6fdf80c34}{robot}, \hyperlink{namespaceservidor_a73a7e14af1c6774da1d3c5580e02573c}{gyro})
\item 
tuple \hyperlink{namespaceservidor_aa210848229916c300b66b7f7958baa24}{configCommands} = ArServerSimpleComLogRobotConfig(\hyperlink{namespaceservidor_a4bc8244e73099a1e67418ea064c30097}{commands}, \hyperlink{namespaceservidor_a20c40528942a814c3ba639d6fdf80c34}{robot})
\end{DoxyCompactItemize}


\subsection{Descripción detallada}
Servidor para el Pioneer P3-\/DX. Se lo puede utilizar con multiples conexiones de clientes trabaja en el puerto 7272 La lista de comandos para el paquete ArNetPacket se encuentra en el anexo 1 \begin{DoxyAuthor}{Autores}
Danny Vasconez 

Daniel Granda 
\end{DoxyAuthor}
\begin{DoxyVersion}{Versión}
0.0.1b 
\end{DoxyVersion}
\begin{DoxyDate}{Fecha}
2012 
\end{DoxyDate}
\begin{DoxyPrecond}{Precondición}
Tener conectada la plataforma Pioneer P3-\/DX. 
\end{DoxyPrecond}
\begin{Desc}
\item[\hyperlink{bug__bug000003}{Bug}]Nada \end{Desc}
\begin{DoxyWarning}{Atención}
uso inapropiado puede hacer que la aplicacion falle 
\end{DoxyWarning}


\subsection{Documentación de las funciones}
\hypertarget{namespaceservidor_aef5180c02cf1d163167be16732d7250e}{
\index{servidor@{servidor}!movimiento@{movimiento}}
\index{movimiento@{movimiento}!servidor@{servidor}}
\subsubsection[{movimiento}]{\setlength{\rightskip}{0pt plus 5cm}def servidor.movimiento (
\begin{DoxyParamCaption}
\item[{}]{client, }
\item[{}]{packet}
\end{DoxyParamCaption}
)}}
\label{namespaceservidor_aef5180c02cf1d163167be16732d7250e}


Sirve cuando se manda el comando \char`\"{}mover\char`\"{} en el paquete. 


\begin{DoxyParams}{Parámetros}
{\em client} & El cliente que manda el paquete \\
\hline
{\em packet} & el paquete que recibe el servidor para el comando \\
\hline
\end{DoxyParams}
\begin{DoxyReturn}{Devuelve}
Nada 
\end{DoxyReturn}


Definición en la línea 62 del archivo servidor.py.


\begin{DoxyCode}
63                              :
64   robot.lock()
65   robot.comInt(8,5000) #move(5000) para atras move(-4999)
  robot.unlock()
\end{DoxyCode}
\hypertarget{namespaceservidor_a4f30fc83b9ff1f43b6dde147ed4c31fc}{
\index{servidor@{servidor}!posicion@{posicion}}
\index{posicion@{posicion}!servidor@{servidor}}
\subsubsection[{posicion}]{\setlength{\rightskip}{0pt plus 5cm}def servidor.posicion (
\begin{DoxyParamCaption}
\item[{}]{client, }
\item[{}]{packet}
\end{DoxyParamCaption}
)}}
\label{namespaceservidor_a4f30fc83b9ff1f43b6dde147ed4c31fc}


Sirve cuando se manda el comando \char`\"{}pose\char`\"{} en el paquete. 


\begin{DoxyParams}{Parámetros}
{\em client} & El cliente que manda el paquete \\
\hline
{\em packet} & el paquete que recibe el servidor para el comando \\
\hline
\end{DoxyParams}
\begin{DoxyReturn}{Devuelve}
ArNetPacket con la informacion en coordenadas X,Y,T al cliente 
\end{DoxyReturn}


Definición en la línea 84 del archivo servidor.py.


\begin{DoxyCode}
85                            :
86   poses = sonarDev.getCurrentBufferAsVector()
87   packet=ArNetPacket()
88   packet.doubleToBuf(len(poses))
89   for p in poses:
90     packet.strToBuf(str(p))
91   packet.finalizePacket()
92   print str(packet.verifyCheckSum()) + " paquete de sonar correcto checksum"
93   client.sendPacketTcp(packet)
94 
95 # This example demonstrates how to use ArNetworking in Python. 
96 
97 # Global library initialization, just like the C++ API:
98 Aria.init()
99 
# Create a robot object:
\end{DoxyCode}
\hypertarget{namespaceservidor_a06ea535cfe56429259d8de76298416cb}{
\index{servidor@{servidor}!requestCallback@{requestCallback}}
\index{requestCallback@{requestCallback}!servidor@{servidor}}
\subsubsection[{requestCallback}]{\setlength{\rightskip}{0pt plus 5cm}def servidor.requestCallback (
\begin{DoxyParamCaption}
\item[{}]{client, }
\item[{}]{packet}
\end{DoxyParamCaption}
)}}
\label{namespaceservidor_a06ea535cfe56429259d8de76298416cb}


Sirve cuando se manda el comando \char`\"{}test\char`\"{} en el paquete. 


\begin{DoxyParams}{Parámetros}
{\em client} & El cliente que manda el paquete \\
\hline
{\em packet} & el paquete que recibe el servidor para el comando \\
\hline
\end{DoxyParams}
\begin{DoxyReturn}{Devuelve}
Nada 
\end{DoxyReturn}


Definición en la línea 50 del archivo servidor.py.


\begin{DoxyCode}
51                                    :
52   replyPacket = ArNetPacket()
53   replyPacket.strToBuf(str(robot.getPose().x));
54   print "requestCallback received a packet with command #%d. Sending a reply...\n
      " % (packet.getCommand())
  client.sendPacketTcp(replyPacket)
\end{DoxyCode}
\hypertarget{namespaceservidor_a66205c06d52988bbeb961068e554edfc}{
\index{servidor@{servidor}!rotar@{rotar}}
\index{rotar@{rotar}!servidor@{servidor}}
\subsubsection[{rotar}]{\setlength{\rightskip}{0pt plus 5cm}def servidor.rotar (
\begin{DoxyParamCaption}
\item[{}]{client, }
\item[{}]{packet}
\end{DoxyParamCaption}
)}}
\label{namespaceservidor_a66205c06d52988bbeb961068e554edfc}


Sirve cuando se manda el comando \char`\"{}rotar\char`\"{} en el paquete. 


\begin{DoxyParams}{Parámetros}
{\em client} & El cliente que manda el paquete \\
\hline
{\em packet} & el paquete que recibe el servidor para el comando \\
\hline
\end{DoxyParams}
\begin{DoxyReturn}{Devuelve}
Nada 
\end{DoxyReturn}


Definición en la línea 73 del archivo servidor.py.


\begin{DoxyCode}
74                         :
75   robot.lock()
76   robot.comInt(12,50) 
  robot.unlock()
\end{DoxyCode}


\subsection{Documentación de las variables}
\hypertarget{namespaceservidor_a4bc8244e73099a1e67418ea064c30097}{
\index{servidor@{servidor}!commands@{commands}}
\index{commands@{commands}!servidor@{servidor}}
\subsubsection[{commands}]{\setlength{\rightskip}{0pt plus 5cm}tuple {\bf servidor.commands} = ArServerHandlerCommands({\bf server})}}
\label{namespaceservidor_a4bc8244e73099a1e67418ea064c30097}


Definición en la línea 151 del archivo servidor.py.

\hypertarget{namespaceservidor_ae78d17158a0f45369b3621468e736bf6}{
\index{servidor@{servidor}!con@{con}}
\index{con@{con}!servidor@{servidor}}
\subsubsection[{con}]{\setlength{\rightskip}{0pt plus 5cm}tuple {\bf servidor.con} = ArSimpleConnector(sys.argv)}}
\label{namespaceservidor_ae78d17158a0f45369b3621468e736bf6}


Definición en la línea 126 del archivo servidor.py.

\hypertarget{namespaceservidor_aa210848229916c300b66b7f7958baa24}{
\index{servidor@{servidor}!configCommands@{configCommands}}
\index{configCommands@{configCommands}!servidor@{servidor}}
\subsubsection[{configCommands}]{\setlength{\rightskip}{0pt plus 5cm}tuple {\bf servidor.configCommands} = ArServerSimpleComLogRobotConfig({\bf commands}, {\bf robot})}}
\label{namespaceservidor_aa210848229916c300b66b7f7958baa24}


Definición en la línea 159 del archivo servidor.py.

\hypertarget{namespaceservidor_a691fe74e57a668507d56d3acd8e19ba2}{
\index{servidor@{servidor}!drawings@{drawings}}
\index{drawings@{drawings}!servidor@{servidor}}
\subsubsection[{drawings}]{\setlength{\rightskip}{0pt plus 5cm}tuple {\bf servidor.drawings} = ArServerInfoDrawings({\bf server})}}
\label{namespaceservidor_a691fe74e57a668507d56d3acd8e19ba2}


Definición en la línea 140 del archivo servidor.py.

\hypertarget{namespaceservidor_a73a7e14af1c6774da1d3c5580e02573c}{
\index{servidor@{servidor}!gyro@{gyro}}
\index{gyro@{gyro}!servidor@{servidor}}
\subsubsection[{gyro}]{\setlength{\rightskip}{0pt plus 5cm}tuple {\bf servidor.gyro} = ArAnalogGyro({\bf robot})}}
\label{namespaceservidor_a73a7e14af1c6774da1d3c5580e02573c}


Definición en la línea 103 del archivo servidor.py.

\hypertarget{namespaceservidor_a9bcfd4f54186b9c69cea8188548af8c8}{
\index{servidor@{servidor}!gyroCommands@{gyroCommands}}
\index{gyroCommands@{gyroCommands}!servidor@{servidor}}
\subsubsection[{gyroCommands}]{\setlength{\rightskip}{0pt plus 5cm}tuple {\bf servidor.gyroCommands} = ArServerSimpleComGyro({\bf commands}, {\bf robot}, {\bf gyro})}}
\label{namespaceservidor_a9bcfd4f54186b9c69cea8188548af8c8}


Definición en la línea 157 del archivo servidor.py.

\hypertarget{namespaceservidor_a8f73ad33af2f2a4f3bc9209b9390cfe8}{
\index{servidor@{servidor}!loggingCommands@{loggingCommands}}
\index{loggingCommands@{loggingCommands}!servidor@{servidor}}
\subsubsection[{loggingCommands}]{\setlength{\rightskip}{0pt plus 5cm}tuple {\bf servidor.loggingCommands} = ArServerSimpleComMovementLogging({\bf commands}, {\bf robot})}}
\label{namespaceservidor_a8f73ad33af2f2a4f3bc9209b9390cfe8}


Definición en la línea 155 del archivo servidor.py.

\hypertarget{namespaceservidor_ac7cf650e754b329444bd902af26c1527}{
\index{servidor@{servidor}!modeRatioDrive@{modeRatioDrive}}
\index{modeRatioDrive@{modeRatioDrive}!servidor@{servidor}}
\subsubsection[{modeRatioDrive}]{\setlength{\rightskip}{0pt plus 5cm}tuple {\bf servidor.modeRatioDrive} = ArServerModeRatioDrive({\bf server}, {\bf robot})}}
\label{namespaceservidor_ac7cf650e754b329444bd902af26c1527}


Definición en la línea 145 del archivo servidor.py.

\hypertarget{namespaceservidor_ae7b0d696eea14b76aba1e014dbfe70f4}{
\index{servidor@{servidor}!modeStop@{modeStop}}
\index{modeStop@{modeStop}!servidor@{servidor}}
\subsubsection[{modeStop}]{\setlength{\rightskip}{0pt plus 5cm}tuple {\bf servidor.modeStop} = ArServerModeStop({\bf server}, {\bf robot})}}
\label{namespaceservidor_ae7b0d696eea14b76aba1e014dbfe70f4}


Definición en la línea 144 del archivo servidor.py.

\hypertarget{namespaceservidor_ae22efed138dafc7284a3f71676a6757f}{
\index{servidor@{servidor}!modeWander@{modeWander}}
\index{modeWander@{modeWander}!servidor@{servidor}}
\subsubsection[{modeWander}]{\setlength{\rightskip}{0pt plus 5cm}tuple {\bf servidor.modeWander} = ArServerModeWander({\bf server}, {\bf robot})}}
\label{namespaceservidor_ae22efed138dafc7284a3f71676a6757f}


Definición en la línea 146 del archivo servidor.py.

\hypertarget{namespaceservidor_a159e1bd236b55d72f3fd3cb94afc7650}{
\index{servidor@{servidor}!packet@{packet}}
\index{packet@{packet}!servidor@{servidor}}
\subsubsection[{packet}]{\setlength{\rightskip}{0pt plus 5cm}tuple {\bf servidor.packet} = ArNetPacket()}}
\label{namespaceservidor_a159e1bd236b55d72f3fd3cb94afc7650}


Definición en la línea 113 del archivo servidor.py.

\hypertarget{namespaceservidor_a20c40528942a814c3ba639d6fdf80c34}{
\index{servidor@{servidor}!robot@{robot}}
\index{robot@{robot}!servidor@{servidor}}
\subsubsection[{robot}]{\setlength{\rightskip}{0pt plus 5cm}tuple {\bf servidor.robot} = ArRobot()}}
\label{namespaceservidor_a20c40528942a814c3ba639d6fdf80c34}


Definición en la línea 100 del archivo servidor.py.

\hypertarget{namespaceservidor_adfced13b57fb69c12f8ab5a84d1a2356}{
\index{servidor@{servidor}!server@{server}}
\index{server@{server}!servidor@{servidor}}
\subsubsection[{server}]{\setlength{\rightskip}{0pt plus 5cm}tuple {\bf servidor.server} = ArServerBase()}}
\label{namespaceservidor_adfced13b57fb69c12f8ab5a84d1a2356}


Definición en la línea 111 del archivo servidor.py.

\hypertarget{namespaceservidor_a214f853047fdc2f26cfac0db1a9e87b8}{
\index{servidor@{servidor}!serverInfoRobot@{serverInfoRobot}}
\index{serverInfoRobot@{serverInfoRobot}!servidor@{servidor}}
\subsubsection[{serverInfoRobot}]{\setlength{\rightskip}{0pt plus 5cm}tuple {\bf servidor.serverInfoRobot} = ArServerInfoRobot({\bf server}, {\bf robot})}}
\label{namespaceservidor_a214f853047fdc2f26cfac0db1a9e87b8}


Definición en la línea 138 del archivo servidor.py.

\hypertarget{namespaceservidor_ae93d5f481f48442f0db6959867fbf8c4}{
\index{servidor@{servidor}!serverInfoSensor@{serverInfoSensor}}
\index{serverInfoSensor@{serverInfoSensor}!servidor@{servidor}}
\subsubsection[{serverInfoSensor}]{\setlength{\rightskip}{0pt plus 5cm}tuple {\bf servidor.serverInfoSensor} = ArServerInfoSensor({\bf server}, {\bf robot})}}
\label{namespaceservidor_ae93d5f481f48442f0db6959867fbf8c4}


Definición en la línea 139 del archivo servidor.py.

\hypertarget{namespaceservidor_a509c8beb9fe73900b5ab2a0fc5f793b2}{
\index{servidor@{servidor}!sonarDev@{sonarDev}}
\index{sonarDev@{sonarDev}!servidor@{servidor}}
\subsubsection[{sonarDev}]{\setlength{\rightskip}{0pt plus 5cm}tuple {\bf servidor.sonarDev} = ArSonarDevice(2)}}
\label{namespaceservidor_a509c8beb9fe73900b5ab2a0fc5f793b2}


Definición en la línea 106 del archivo servidor.py.

\hypertarget{namespaceservidor_aa877c09d4a72ca2434441ff2dc406a29}{
\index{servidor@{servidor}!uCCommands@{uCCommands}}
\index{uCCommands@{uCCommands}!servidor@{servidor}}
\subsubsection[{uCCommands}]{\setlength{\rightskip}{0pt plus 5cm}tuple {\bf servidor.uCCommands} = ArServerSimpleComUC({\bf commands}, {\bf robot})}}
\label{namespaceservidor_aa877c09d4a72ca2434441ff2dc406a29}


Definición en la línea 153 del archivo servidor.py.


\chapter{Documentación de las clases}
\hypertarget{classcliente__lib_1_1cliente__lib}{
\section{Referencia de la Clase cliente\_\-lib.cliente\_\-lib}
\label{classcliente__lib_1_1cliente__lib}\index{cliente\_\-lib::cliente\_\-lib@{cliente\_\-lib::cliente\_\-lib}}
}


es la clase encargada del cliente  


\subsection*{Métodos públicos}
\begin{DoxyCompactItemize}
\item 
def \hyperlink{classcliente__lib_1_1cliente__lib_ac5e4490f412835d35481f58d1ae503f9}{\_\-\_\-init\_\-\_\-}
\begin{DoxyCompactList}\small\item\em Carga valores a las variables necesarias para funcionar el cliente. \end{DoxyCompactList}\item 
def \hyperlink{classcliente__lib_1_1cliente__lib_af7b751bcf94c96150b23bacb5e477956}{valores}
\begin{DoxyCompactList}\small\item\em Sirve cuando se manda el comando \char`\"{}updateNumbers\char`\"{} en el paquete. \end{DoxyCompactList}\item 
def \hyperlink{classcliente__lib_1_1cliente__lib_ac0a4410b48b4c759028bec6ae1c641e8}{lista\_\-sonares}
\begin{DoxyCompactList}\small\item\em Sirve para leer el paquete arNetPacket con la lista del sonar. \end{DoxyCompactList}\item 
def \hyperlink{classcliente__lib_1_1cliente__lib_abcf28c2207cb5519090654484137db23}{valores\_\-sonares}
\begin{DoxyCompactList}\small\item\em Sirve para leer el paquete arNetPacket con los valores del sonar. \end{DoxyCompactList}\item 
def \hyperlink{classcliente__lib_1_1cliente__lib_acfc22af72a1668db28d18ab4ff40909e}{envio\_\-ratioDrive}
\begin{DoxyCompactList}\small\item\em Sirve para realizar la teleoperacion, mandando los parametros. \end{DoxyCompactList}\item 
def \hyperlink{classcliente__lib_1_1cliente__lib_a3c43af1448ad0fa10cfa6f398b7029db}{envio\_\-griper}
\begin{DoxyCompactList}\small\item\em Sirve para realizar la teleoperacion, mandando los parametros. \end{DoxyCompactList}\item 
def \hyperlink{classcliente__lib_1_1cliente__lib_ade1f44e9270c8835c284832a72b96b6c}{uC\_\-comandos\_\-movi}
\begin{DoxyCompactList}\small\item\em Sirve para mandar ordenes de movimiento directamente al controlador de la plataforma movil. \end{DoxyCompactList}\item 
def \hyperlink{classcliente__lib_1_1cliente__lib_ac3e89d3066207b05b217a50f549c239a}{envio\_\-consulta\_\-fisica}
\begin{DoxyCompactList}\small\item\em Sirve para mandar ordenes al servidor utilizando paquetes ArNetPacket con comandos {\bfseries pose} y {\bfseries updateNumbers} \end{DoxyCompactList}\item 
def \hyperlink{classcliente__lib_1_1cliente__lib_a1a7b5475a98772f0e48a4e1fd76e8d47}{cliente\_\-inicio}
\begin{DoxyCompactList}\small\item\em Sirve para iniciar la conexion con el servidor. \end{DoxyCompactList}\item 
def \hyperlink{classcliente__lib_1_1cliente__lib_a14a49495fd71fab84d36060e604415a5}{cliente\_\-apaga}
\begin{DoxyCompactList}\small\item\em Sirve para realizar la desconexion con el servidor. \end{DoxyCompactList}\item 
def \hyperlink{classcliente__lib_1_1cliente__lib_a3ca67c0c9d7f0a622abd740c780f64d1}{devuelve\_\-valorf}
\begin{DoxyCompactList}\small\item\em Devuelve el variable valor\_\-fisico, con usa espera de 100ms. \end{DoxyCompactList}\item 
def \hyperlink{classcliente__lib_1_1cliente__lib_a127d026872fbd11f4f5bbe4a73424b77}{devuelve\_\-valors}
\begin{DoxyCompactList}\small\item\em Devuelve el variable valor\_\-sonares, con usa espera de 100ms. \end{DoxyCompactList}\end{DoxyCompactItemize}
\subsection*{Atributos públicos}
\begin{DoxyCompactItemize}
\item 
\hyperlink{classcliente__lib_1_1cliente__lib_abba3409f89ee8dcec8b180c90aa5d77c}{valor\_\-fisico}
\item 
\hyperlink{classcliente__lib_1_1cliente__lib_aadea6e24bd3a01b0500fcc67543a97e9}{valor\_\-sonares}
\item 
\hyperlink{classcliente__lib_1_1cliente__lib_a675dd8430aa2eeb33240b8b07ed61543}{ip}
\item 
\hyperlink{classcliente__lib_1_1cliente__lib_a7bbce98e978c840ab3ef04e2e3715b56}{x}
\item 
\hyperlink{classcliente__lib_1_1cliente__lib_aa6b4463261d78e82876d5f01ab3800bb}{y}
\item 
\hyperlink{classcliente__lib_1_1cliente__lib_adf30f5e41e7c96565a7588e808d7543a}{t}
\item 
\hyperlink{classcliente__lib_1_1cliente__lib_a5b15a0be5d30abf1f71d776ccead9d9d}{acu}
\end{DoxyCompactItemize}


\subsection{Descripción detallada}
es la clase encargada del cliente 

se lo utiliza de esta manera par poder trabajar con la informacion tanto leyendo las variables o utilizando los comandos 

Definición en la línea 50 del archivo cliente\_\-lib.py.



\subsection{Documentación del constructor y destructor}
\hypertarget{classcliente__lib_1_1cliente__lib_ac5e4490f412835d35481f58d1ae503f9}{
\index{cliente\_\-lib::cliente\_\-lib@{cliente\_\-lib::cliente\_\-lib}!\_\-\_\-init\_\-\_\-@{\_\-\_\-init\_\-\_\-}}
\index{\_\-\_\-init\_\-\_\-@{\_\-\_\-init\_\-\_\-}!cliente_lib::cliente_lib@{cliente\_\-lib::cliente\_\-lib}}
\subsubsection[{\_\-\_\-init\_\-\_\-}]{\setlength{\rightskip}{0pt plus 5cm}def cliente\_\-lib.cliente\_\-lib.\_\-\_\-init\_\-\_\- (
\begin{DoxyParamCaption}
\item[{}]{self}
\end{DoxyParamCaption}
)}}
\label{classcliente__lib_1_1cliente__lib_ac5e4490f412835d35481f58d1ae503f9}


Carga valores a las variables necesarias para funcionar el cliente. 

este comando no es necesario utilizarlo es usado al instanciar la clase 
\begin{DoxyParams}{Parámetros}
{\em self} & este parametro no es necesario escribir \\
\hline
\end{DoxyParams}


Definición en la línea 57 del archivo cliente\_\-lib.py.


\begin{DoxyCode}
58                           :
59                 print "Cargo modulo para cliente_lib"
60                 self.valor_fisico=[]
61                 self.valor_sonares=[] 
62                 self.ip="localhost"
63                 self.x={}
64                 self.y={}
65                 self.t={}
66                 self.acu=0

\end{DoxyCode}


\subsection{Documentación de las funciones miembro}
\hypertarget{classcliente__lib_1_1cliente__lib_a14a49495fd71fab84d36060e604415a5}{
\index{cliente\_\-lib::cliente\_\-lib@{cliente\_\-lib::cliente\_\-lib}!cliente\_\-apaga@{cliente\_\-apaga}}
\index{cliente\_\-apaga@{cliente\_\-apaga}!cliente_lib::cliente_lib@{cliente\_\-lib::cliente\_\-lib}}
\subsubsection[{cliente\_\-apaga}]{\setlength{\rightskip}{0pt plus 5cm}def cliente\_\-lib.cliente\_\-lib.cliente\_\-apaga (
\begin{DoxyParamCaption}
\item[{}]{self, }
\item[{}]{client}
\end{DoxyParamCaption}
)}}
\label{classcliente__lib_1_1cliente__lib_a14a49495fd71fab84d36060e604415a5}


Sirve para realizar la desconexion con el servidor. 


\begin{DoxyParams}{Parámetros}
{\em self} & este parametro no es necesario escribir \\
\hline
{\em client} & para poder desconectar el cliente \\
\hline
\end{DoxyParams}


Definición en la línea 252 del archivo cliente\_\-lib.py.


\begin{DoxyCode}
253                                       :
254                 ArUtil.sleep(1000)
255                 client.disconnect()
256                 ArUtil.sleep(50)
                return 0
\end{DoxyCode}
\hypertarget{classcliente__lib_1_1cliente__lib_a1a7b5475a98772f0e48a4e1fd76e8d47}{
\index{cliente\_\-lib::cliente\_\-lib@{cliente\_\-lib::cliente\_\-lib}!cliente\_\-inicio@{cliente\_\-inicio}}
\index{cliente\_\-inicio@{cliente\_\-inicio}!cliente_lib::cliente_lib@{cliente\_\-lib::cliente\_\-lib}}
\subsubsection[{cliente\_\-inicio}]{\setlength{\rightskip}{0pt plus 5cm}def cliente\_\-lib.cliente\_\-lib.cliente\_\-inicio (
\begin{DoxyParamCaption}
\item[{}]{self}
\end{DoxyParamCaption}
)}}
\label{classcliente__lib_1_1cliente__lib_a1a7b5475a98772f0e48a4e1fd76e8d47}


Sirve para iniciar la conexion con el servidor. 


\begin{DoxyParams}{Parámetros}
{\em self} & este parametro no es necesario escribir \\
\hline
\end{DoxyParams}
\begin{DoxyReturn}{Devuelve}
client 
\end{DoxyReturn}


Definición en la línea 208 del archivo cliente\_\-lib.py.


\begin{DoxyCode}
209                                 :
210                 client = ArClientBase()
211                 #Solo funcionar el lectura de datos por TCP
212                 client.setTcpOnlyFromServer()
213                 client.setTcpOnlyToServer()
214                 #
215                 Aria.init()
216                 
217                 startTime = ArTime()
218                 startTime.setToNow()
219                 if not client.blockingConnect(self.ip, 7272): #ip y puerto del se
      rvidor
220                         print "Could not connect to server at %s port 7272, exiti
      ng" % self.ip
221                         
222                         Aria.exit(1);
223                 print "cliente: Se tardo %ld msec en connectarse\n" % (startTime.
      mSecSince())
224                 
225                 client.runAsync()
226                 client.addHandler("updateNumbers",self.valores)
227                 #
228                 #client.request("updateNumbers",100)
229                 #
230                 client.addHandler("getSensorList",self.lista_sonares)
231                 client.addHandler("pose",self.valores_sonares)
232                 #
233                 #client.lock()
234                 #client.request("pose",100)
235                 #client.unlock()
236                 #
237                 
238                 if client.dataExists("ratioDrive"): #supuestamente devuelve la in
      fo del robot con odometria
239                         print "ratioDrive si existe"
240                 else:
241                         Aria.exit(1);
242                 #client=envio_ratioDrive(client,TransRatio,RotRatio,LatRatio) #fi
      jar los valores para mover
243                 #client=uC_comandos_movi(client,comando,parametro) #Lo hace de un
      a manera directa anulando las demas operaciones
244                 #client.requestOnce("updateNumbers")
245                 #client.requestOnce("stop") #parada de emergencia
                return client
\end{DoxyCode}
\hypertarget{classcliente__lib_1_1cliente__lib_a3ca67c0c9d7f0a622abd740c780f64d1}{
\index{cliente\_\-lib::cliente\_\-lib@{cliente\_\-lib::cliente\_\-lib}!devuelve\_\-valorf@{devuelve\_\-valorf}}
\index{devuelve\_\-valorf@{devuelve\_\-valorf}!cliente_lib::cliente_lib@{cliente\_\-lib::cliente\_\-lib}}
\subsubsection[{devuelve\_\-valorf}]{\setlength{\rightskip}{0pt plus 5cm}def cliente\_\-lib.cliente\_\-lib.devuelve\_\-valorf (
\begin{DoxyParamCaption}
\item[{}]{self}
\end{DoxyParamCaption}
)}}
\label{classcliente__lib_1_1cliente__lib_a3ca67c0c9d7f0a622abd740c780f64d1}


Devuelve el variable valor\_\-fisico, con usa espera de 100ms. 


\begin{DoxyParams}{Parámetros}
{\em self} & este parametro no es necesario escribir \\
\hline
\end{DoxyParams}


Definición en la línea 262 del archivo cliente\_\-lib.py.


\begin{DoxyCode}
263                                  :
264                 ArUtil.sleep(100)
                return self.valor_fisico
\end{DoxyCode}
\hypertarget{classcliente__lib_1_1cliente__lib_a127d026872fbd11f4f5bbe4a73424b77}{
\index{cliente\_\-lib::cliente\_\-lib@{cliente\_\-lib::cliente\_\-lib}!devuelve\_\-valors@{devuelve\_\-valors}}
\index{devuelve\_\-valors@{devuelve\_\-valors}!cliente_lib::cliente_lib@{cliente\_\-lib::cliente\_\-lib}}
\subsubsection[{devuelve\_\-valors}]{\setlength{\rightskip}{0pt plus 5cm}def cliente\_\-lib.cliente\_\-lib.devuelve\_\-valors (
\begin{DoxyParamCaption}
\item[{}]{self}
\end{DoxyParamCaption}
)}}
\label{classcliente__lib_1_1cliente__lib_a127d026872fbd11f4f5bbe4a73424b77}


Devuelve el variable valor\_\-sonares, con usa espera de 100ms. 


\begin{DoxyParams}{Parámetros}
{\em self} & este parametro no es necesario escribir \\
\hline
\end{DoxyParams}


Definición en la línea 270 del archivo cliente\_\-lib.py.


\begin{DoxyCode}
271                                  :
272                 ArUtil.sleep(100)
273                 valor=self.valor_sonares
274                 if valor==[]:
275                         print valor
276                         return {'x0':-9999,'x1':-9999},{'y0':-9999,'y1':-9999},{'
      t0':0,'t1':0} #valores quitar despues
277                 else:
278                         try:
279                                 x_br=[]
280                                 y_br=[]
281                                 t_br=[]
282                                 for a in range(int(valor[0])): #el primer valor i
      ndica cuantos elementos se tiene de sensores
283                                         #print valor[1][a]
284                                         valor[1][a]=valor[1][a].strip('(')
285                                         valor[1][a]=valor[1][a].strip(')')
286                                         valor[1][a]=valor[1][a].split(",") #comie
      nza a separar por que viene en forma (X:valor,Y:valor,T:valor)
287                                         valor[1][a][0]=valor[1][a][0].strip() #bo
      rrar los espacios en blanco de los elementos
288                                         valor[1][a][1]=valor[1][a][1].strip()
289                                         valor[1][a][2]=valor[1][a][2].strip()
290                                         x_br+=[valor[1][a][0].strip("X:")]
291                                         #print x_br
292                                         y_br+=[valor[1][a][1].strip("Y:")]
293                                         t_br+=[valor[1][a][2].strip("T:")]
294                                         
295                                 #print len(x_br)
296                                 for a in range(len(x_br)):
297                                         self.x.update({'x%d' % self.acu:float(x_b
      r[a])}) #Ordena los datos en forma de diccionario, float tiene problemas con 0.00
      0
298                                         self.y.update({'y%d' % self.acu:float(y_b
      r[a])})
299                                         #self.t.update({'t%d' % self.acu:float(t_
      br[a].strip(")"))})
300                                         self.t.update({'t%d' % self.acu:0})
301                                         self.acu+=1
302                                         if self.acu==5000:
303                                                 self.acu=0
304                                                 #pass
305                                 #print self.x
306                                 #print self.y
307                                 #print self.t
308                                 return self.x,self.y,self.t
309                         except:
310                                 pass
311 

\end{DoxyCode}
\hypertarget{classcliente__lib_1_1cliente__lib_ac3e89d3066207b05b217a50f549c239a}{
\index{cliente\_\-lib::cliente\_\-lib@{cliente\_\-lib::cliente\_\-lib}!envio\_\-consulta\_\-fisica@{envio\_\-consulta\_\-fisica}}
\index{envio\_\-consulta\_\-fisica@{envio\_\-consulta\_\-fisica}!cliente_lib::cliente_lib@{cliente\_\-lib::cliente\_\-lib}}
\subsubsection[{envio\_\-consulta\_\-fisica}]{\setlength{\rightskip}{0pt plus 5cm}def cliente\_\-lib.cliente\_\-lib.envio\_\-consulta\_\-fisica (
\begin{DoxyParamCaption}
\item[{}]{self, }
\item[{}]{client, }
\item[{}]{mensaje}
\end{DoxyParamCaption}
)}}
\label{classcliente__lib_1_1cliente__lib_ac3e89d3066207b05b217a50f549c239a}


Sirve para mandar ordenes al servidor utilizando paquetes ArNetPacket con comandos {\bfseries pose} y {\bfseries updateNumbers} 


\begin{DoxyParams}{Parámetros}
{\em self} & este parametro no es necesario escribir \\
\hline
{\em client} & Se debe trar el objeto cliente a la definicion para poder utilizar el enlace del cliente para enviar el paquete al servidor \\
\hline
{\em mensaje} & puede ser cualquier comando del servidor que no devuelva informacion a exepcion de pose y updatenumbers \\
\hline
\end{DoxyParams}
\begin{DoxyReturn}{Devuelve}
client 
\end{DoxyReturn}


Definición en la línea 197 del archivo cliente\_\-lib.py.


\begin{DoxyCode}
198                                                       :
199                 ## se puede usar pose y updateNumbers
200                 client.requestOnce(mensaje)
201                 return client
                
\end{DoxyCode}
\hypertarget{classcliente__lib_1_1cliente__lib_a3c43af1448ad0fa10cfa6f398b7029db}{
\index{cliente\_\-lib::cliente\_\-lib@{cliente\_\-lib::cliente\_\-lib}!envio\_\-griper@{envio\_\-griper}}
\index{envio\_\-griper@{envio\_\-griper}!cliente_lib::cliente_lib@{cliente\_\-lib::cliente\_\-lib}}
\subsubsection[{envio\_\-griper}]{\setlength{\rightskip}{0pt plus 5cm}def cliente\_\-lib.cliente\_\-lib.envio\_\-griper (
\begin{DoxyParamCaption}
\item[{}]{self, }
\item[{}]{client, }
\item[{}]{TransRatio}
\end{DoxyParamCaption}
)}}
\label{classcliente__lib_1_1cliente__lib_a3c43af1448ad0fa10cfa6f398b7029db}


Sirve para realizar la teleoperacion, mandando los parametros. 


\begin{DoxyParams}{Parámetros}
{\em self} & este parametro no es necesario escribir \\
\hline
{\em client} & Se debe trar el objeto cliente a la definicion para poder utilizar el enlace del cliente para enviar el paquete al servidor \\
\hline
{\em TransRatio} & Velocidad de traslacion \\
\hline
{\em RotRatio} & Velocidad de rotacion \\
\hline
{\em LatRatio} & velocidad lateral para el modelo Pioneer P3-\/DX no se necesario puede ser 0 \\
\hline
\end{DoxyParams}
\begin{DoxyReturn}{Devuelve}
client 
\end{DoxyReturn}


Definición en la línea 165 del archivo cliente\_\-lib.py.


\begin{DoxyCode}
166                                                 :
167                 myTransRatio=TransRatio
168                 packet=ArNetPacket()
169                 packet.doubleToBuf(myTransRatio)
170                 client.requestOnce("ComandoGripper", packet)
171                 return client
          
\end{DoxyCode}
\hypertarget{classcliente__lib_1_1cliente__lib_acfc22af72a1668db28d18ab4ff40909e}{
\index{cliente\_\-lib::cliente\_\-lib@{cliente\_\-lib::cliente\_\-lib}!envio\_\-ratioDrive@{envio\_\-ratioDrive}}
\index{envio\_\-ratioDrive@{envio\_\-ratioDrive}!cliente_lib::cliente_lib@{cliente\_\-lib::cliente\_\-lib}}
\subsubsection[{envio\_\-ratioDrive}]{\setlength{\rightskip}{0pt plus 5cm}def cliente\_\-lib.cliente\_\-lib.envio\_\-ratioDrive (
\begin{DoxyParamCaption}
\item[{}]{self, }
\item[{}]{client, }
\item[{}]{TransRatio, }
\item[{}]{RotRatio, }
\item[{}]{LatRatio}
\end{DoxyParamCaption}
)}}
\label{classcliente__lib_1_1cliente__lib_acfc22af72a1668db28d18ab4ff40909e}


Sirve para realizar la teleoperacion, mandando los parametros. 


\begin{DoxyParams}{Parámetros}
{\em self} & este parametro no es necesario escribir \\
\hline
{\em client} & Se debe trar el objeto cliente a la definicion para poder utilizar el enlace del cliente para enviar el paquete al servidor \\
\hline
{\em TransRatio} & Velocidad de traslacion \\
\hline
{\em RotRatio} & Velocidad de rotacion \\
\hline
{\em LatRatio} & velocidad lateral para el modelo Pioneer P3-\/DX no se necesario puede ser 0 \\
\hline
\end{DoxyParams}
\begin{DoxyReturn}{Devuelve}
client 
\end{DoxyReturn}


Definición en la línea 143 del archivo cliente\_\-lib.py.


\begin{DoxyCode}
144                                                                       :
145                 myTransRatio=TransRatio
146                 myRotRatio=RotRatio
147                 myLatRatio=LatRatio
148                 packet=ArNetPacket()
149                 packet.doubleToBuf(myTransRatio)
150                 packet.doubleToBuf(myRotRatio)
151                 packet.doubleToBuf(50) # use half of the robot's maximum.
152                 packet.doubleToBuf(myLatRatio)
153                 client.requestOnce("ratioDrive", packet)
154                 return client
        
\end{DoxyCode}
\hypertarget{classcliente__lib_1_1cliente__lib_ac0a4410b48b4c759028bec6ae1c641e8}{
\index{cliente\_\-lib::cliente\_\-lib@{cliente\_\-lib::cliente\_\-lib}!lista\_\-sonares@{lista\_\-sonares}}
\index{lista\_\-sonares@{lista\_\-sonares}!cliente_lib::cliente_lib@{cliente\_\-lib::cliente\_\-lib}}
\subsubsection[{lista\_\-sonares}]{\setlength{\rightskip}{0pt plus 5cm}def cliente\_\-lib.cliente\_\-lib.lista\_\-sonares (
\begin{DoxyParamCaption}
\item[{}]{self, }
\item[{}]{packet}
\end{DoxyParamCaption}
)}}
\label{classcliente__lib_1_1cliente__lib_ac0a4410b48b4c759028bec6ae1c641e8}


Sirve para leer el paquete arNetPacket con la lista del sonar. 

este comando no es necesario utilizarlo es usado solo por el cliente para procesar el paquete 
\begin{DoxyParams}{Parámetros}
{\em self} & este parametro no es necesario escribir \\
\hline
{\em packet} & este parametro no es necesario escribir \\
\hline
\end{DoxyParams}
\begin{DoxyReturn}{Devuelve}
nada 
\end{DoxyReturn}


Definición en la línea 98 del archivo cliente\_\-lib.py.


\begin{DoxyCode}
99                                       :
100                 c="                                   "
101                 numSensor=packet.bufToByte2()
102                 numSensor2=packet.bufToStr(c,15)
103                 print str(numSensor)+" "+str(c.strip())

\end{DoxyCode}
\hypertarget{classcliente__lib_1_1cliente__lib_ade1f44e9270c8835c284832a72b96b6c}{
\index{cliente\_\-lib::cliente\_\-lib@{cliente\_\-lib::cliente\_\-lib}!uC\_\-comandos\_\-movi@{uC\_\-comandos\_\-movi}}
\index{uC\_\-comandos\_\-movi@{uC\_\-comandos\_\-movi}!cliente_lib::cliente_lib@{cliente\_\-lib::cliente\_\-lib}}
\subsubsection[{uC\_\-comandos\_\-movi}]{\setlength{\rightskip}{0pt plus 5cm}def cliente\_\-lib.cliente\_\-lib.uC\_\-comandos\_\-movi (
\begin{DoxyParamCaption}
\item[{}]{self, }
\item[{}]{client, }
\item[{}]{comando, }
\item[{}]{parametro}
\end{DoxyParamCaption}
)}}
\label{classcliente__lib_1_1cliente__lib_ade1f44e9270c8835c284832a72b96b6c}


Sirve para mandar ordenes de movimiento directamente al controlador de la plataforma movil. 


\begin{DoxyParams}{Parámetros}
{\em self} & este parametro no es necesario escribir \\
\hline
{\em client} & Se debe trar el objeto cliente a la definicion para poder utilizar el enlace del cliente para enviar el paquete al servidor \\
\hline
{\em comando} & es un numero de 1-\/255 que representa una funcion esta informacion se puede encontrar en el API de ARIA \\
\hline
{\em parametro} & el parametro de la funcion en caso de no tener se deja el valor en blanco \\
\hline
\end{DoxyParams}
\begin{DoxyReturn}{Devuelve}
client 
\end{DoxyReturn}


Definición en la línea 181 del archivo cliente\_\-lib.py.


\begin{DoxyCode}
182                                                            :
183                 mi_comando=comando #comando 8 es MOVE parametro un valor de 5000 
      a -4999 es en mm, 11 LEV y su parametro es velocidad +o- mm/s
184                 mi_parametro=parametro #parametro
185                 packet=ArNetPacket()
186                 packet.strToBuf(mi_comando+" "+mi_parametro)
187                 client.requestOnce("MicroControllerMotionCommand", packet) #Micro
      ControllerMotionCommand
188                 return client

\end{DoxyCode}
\hypertarget{classcliente__lib_1_1cliente__lib_af7b751bcf94c96150b23bacb5e477956}{
\index{cliente\_\-lib::cliente\_\-lib@{cliente\_\-lib::cliente\_\-lib}!valores@{valores}}
\index{valores@{valores}!cliente_lib::cliente_lib@{cliente\_\-lib::cliente\_\-lib}}
\subsubsection[{valores}]{\setlength{\rightskip}{0pt plus 5cm}def cliente\_\-lib.cliente\_\-lib.valores (
\begin{DoxyParamCaption}
\item[{}]{self, }
\item[{}]{packet}
\end{DoxyParamCaption}
)}}
\label{classcliente__lib_1_1cliente__lib_af7b751bcf94c96150b23bacb5e477956}


Sirve cuando se manda el comando \char`\"{}updateNumbers\char`\"{} en el paquete. 

este comando no es necesario utilizarlo es usado solo por el cliente para procesar el paquete 
\begin{DoxyParams}{Parámetros}
{\em self} & este parametro no es necesario escribir \\
\hline
{\em packet} & el paquete que recibe el cliente del servidor, no es necesario escribir \\
\hline
\end{DoxyParams}
\begin{DoxyReturn}{Devuelve}
Nada, pero guarda en self.valores\_\-fisico \mbox{[}voltaje\_\-bateria,myX,myY,myTh,myVel,myRotVel,myLatVel,myTemperature\mbox{]} 
\end{DoxyReturn}


Definición en la línea 75 del archivo cliente\_\-lib.py.


\begin{DoxyCode}
76                                 :
77                 #devuelve los valores voltaje_bateria,myX,myY,myTh,myVel,myRotVel
      ,myLatVel,myTemperature
78                 voltaje_bateria=packet.bufToByte2()/10
79                 myX = packet.bufToByte4()#
80                 myY = packet.bufToByte4()
81                 myTh = packet.bufToByte2()
82                 myVel = packet.bufToByte2()
83                 myRotVel = packet.bufToByte2()
84                 myLatVel = packet.bufToByte2()
85                 myTemperature = packet.bufToByte()
86                 #print "X= "+str(myX)+" y="+str(myY)+" th="+str(myTh)
87                 self.valor_fisico=(voltaje_bateria,myX,myY,myTh,myVel,myRotVel,my
      LatVel,myTemperature)
88                 self.valor_fisico=(voltaje_bateria,myX,myY,myTh,myVel,myRotVel,my
      LatVel,0)
89                 #print valor

\end{DoxyCode}
\hypertarget{classcliente__lib_1_1cliente__lib_abcf28c2207cb5519090654484137db23}{
\index{cliente\_\-lib::cliente\_\-lib@{cliente\_\-lib::cliente\_\-lib}!valores\_\-sonares@{valores\_\-sonares}}
\index{valores\_\-sonares@{valores\_\-sonares}!cliente_lib::cliente_lib@{cliente\_\-lib::cliente\_\-lib}}
\subsubsection[{valores\_\-sonares}]{\setlength{\rightskip}{0pt plus 5cm}def cliente\_\-lib.cliente\_\-lib.valores\_\-sonares (
\begin{DoxyParamCaption}
\item[{}]{self, }
\item[{}]{packet}
\end{DoxyParamCaption}
)}}
\label{classcliente__lib_1_1cliente__lib_abcf28c2207cb5519090654484137db23}


Sirve para leer el paquete arNetPacket con los valores del sonar. 

este comando no es necesario utilizarlo es usado solo por el cliente para procesar el paquete 
\begin{DoxyParams}{Parámetros}
{\em self} & este parametro no es necesario escribir \\
\hline
{\em packet} & este parametro no es necesario escribir \\
\hline
\end{DoxyParams}
\begin{DoxyReturn}{Devuelve}
nada 
\end{DoxyReturn}


Definición en la línea 112 del archivo cliente\_\-lib.py.


\begin{DoxyCode}
113                                         :
114                 try:
115                         bandera1=0
116                         cantidad=packet.bufToDouble()
117                         print int(cantidad)
118                         for i in range(2*int(cantidad)):
119                                 if bandera1==0:
120                                         self.x.update({'x%d' % (self.acu):int(pac
      ket.bufToByte4())})
121                                         bandera1=1
122                                 else:
123                                         self.y.update({'y%d' % (self.acu):int(pac
      ket.bufToByte4())})
124                                         bandera1=0
125                                         self.acu=self.acu+1
126                                 if self.acu==200:
127                                         self.acu=0
128                                         self.x={}
129                                         self.y={}
130                 except:
131                         print "Fallo lectura sensores"
132                         pass

\end{DoxyCode}


\subsection{Documentación de los datos miembro}
\hypertarget{classcliente__lib_1_1cliente__lib_a5b15a0be5d30abf1f71d776ccead9d9d}{
\index{cliente\_\-lib::cliente\_\-lib@{cliente\_\-lib::cliente\_\-lib}!acu@{acu}}
\index{acu@{acu}!cliente_lib::cliente_lib@{cliente\_\-lib::cliente\_\-lib}}
\subsubsection[{acu}]{\setlength{\rightskip}{0pt plus 5cm}{\bf cliente\_\-lib.cliente\_\-lib.acu}}}
\label{classcliente__lib_1_1cliente__lib_a5b15a0be5d30abf1f71d776ccead9d9d}


Definición en la línea 57 del archivo cliente\_\-lib.py.

\hypertarget{classcliente__lib_1_1cliente__lib_a675dd8430aa2eeb33240b8b07ed61543}{
\index{cliente\_\-lib::cliente\_\-lib@{cliente\_\-lib::cliente\_\-lib}!ip@{ip}}
\index{ip@{ip}!cliente_lib::cliente_lib@{cliente\_\-lib::cliente\_\-lib}}
\subsubsection[{ip}]{\setlength{\rightskip}{0pt plus 5cm}{\bf cliente\_\-lib.cliente\_\-lib.ip}}}
\label{classcliente__lib_1_1cliente__lib_a675dd8430aa2eeb33240b8b07ed61543}


Definición en la línea 57 del archivo cliente\_\-lib.py.

\hypertarget{classcliente__lib_1_1cliente__lib_adf30f5e41e7c96565a7588e808d7543a}{
\index{cliente\_\-lib::cliente\_\-lib@{cliente\_\-lib::cliente\_\-lib}!t@{t}}
\index{t@{t}!cliente_lib::cliente_lib@{cliente\_\-lib::cliente\_\-lib}}
\subsubsection[{t}]{\setlength{\rightskip}{0pt plus 5cm}{\bf cliente\_\-lib.cliente\_\-lib.t}}}
\label{classcliente__lib_1_1cliente__lib_adf30f5e41e7c96565a7588e808d7543a}


Definición en la línea 57 del archivo cliente\_\-lib.py.

\hypertarget{classcliente__lib_1_1cliente__lib_abba3409f89ee8dcec8b180c90aa5d77c}{
\index{cliente\_\-lib::cliente\_\-lib@{cliente\_\-lib::cliente\_\-lib}!valor\_\-fisico@{valor\_\-fisico}}
\index{valor\_\-fisico@{valor\_\-fisico}!cliente_lib::cliente_lib@{cliente\_\-lib::cliente\_\-lib}}
\subsubsection[{valor\_\-fisico}]{\setlength{\rightskip}{0pt plus 5cm}{\bf cliente\_\-lib.cliente\_\-lib.valor\_\-fisico}}}
\label{classcliente__lib_1_1cliente__lib_abba3409f89ee8dcec8b180c90aa5d77c}


Definición en la línea 57 del archivo cliente\_\-lib.py.

\hypertarget{classcliente__lib_1_1cliente__lib_aadea6e24bd3a01b0500fcc67543a97e9}{
\index{cliente\_\-lib::cliente\_\-lib@{cliente\_\-lib::cliente\_\-lib}!valor\_\-sonares@{valor\_\-sonares}}
\index{valor\_\-sonares@{valor\_\-sonares}!cliente_lib::cliente_lib@{cliente\_\-lib::cliente\_\-lib}}
\subsubsection[{valor\_\-sonares}]{\setlength{\rightskip}{0pt plus 5cm}{\bf cliente\_\-lib.cliente\_\-lib.valor\_\-sonares}}}
\label{classcliente__lib_1_1cliente__lib_aadea6e24bd3a01b0500fcc67543a97e9}


Definición en la línea 57 del archivo cliente\_\-lib.py.

\hypertarget{classcliente__lib_1_1cliente__lib_a7bbce98e978c840ab3ef04e2e3715b56}{
\index{cliente\_\-lib::cliente\_\-lib@{cliente\_\-lib::cliente\_\-lib}!x@{x}}
\index{x@{x}!cliente_lib::cliente_lib@{cliente\_\-lib::cliente\_\-lib}}
\subsubsection[{x}]{\setlength{\rightskip}{0pt plus 5cm}{\bf cliente\_\-lib.cliente\_\-lib.x}}}
\label{classcliente__lib_1_1cliente__lib_a7bbce98e978c840ab3ef04e2e3715b56}


Definición en la línea 57 del archivo cliente\_\-lib.py.

\hypertarget{classcliente__lib_1_1cliente__lib_aa6b4463261d78e82876d5f01ab3800bb}{
\index{cliente\_\-lib::cliente\_\-lib@{cliente\_\-lib::cliente\_\-lib}!y@{y}}
\index{y@{y}!cliente_lib::cliente_lib@{cliente\_\-lib::cliente\_\-lib}}
\subsubsection[{y}]{\setlength{\rightskip}{0pt plus 5cm}{\bf cliente\_\-lib.cliente\_\-lib.y}}}
\label{classcliente__lib_1_1cliente__lib_aa6b4463261d78e82876d5f01ab3800bb}


Definición en la línea 57 del archivo cliente\_\-lib.py.



La documentación para esta clase fue generada a partir del siguiente fichero:\begin{DoxyCompactItemize}
\item 
\hyperlink{cliente__lib_8py}{cliente\_\-lib.py}\end{DoxyCompactItemize}

\hypertarget{classcliente__lib_1_1cliente__lib}{
\section{Referencia de la Clase cliente\_\-lib.cliente\_\-lib}
\label{classcliente__lib_1_1cliente__lib}\index{cliente\_\-lib::cliente\_\-lib@{cliente\_\-lib::cliente\_\-lib}}
}


es la clase encargada del cliente  


\subsection*{Métodos públicos}
\begin{DoxyCompactItemize}
\item 
def \hyperlink{classcliente__lib_1_1cliente__lib_ac5e4490f412835d35481f58d1ae503f9}{\_\-\_\-init\_\-\_\-}
\begin{DoxyCompactList}\small\item\em Carga valores a las variables necesarias para funcionar el cliente. \end{DoxyCompactList}\item 
def \hyperlink{classcliente__lib_1_1cliente__lib_af7b751bcf94c96150b23bacb5e477956}{valores}
\begin{DoxyCompactList}\small\item\em Sirve cuando se manda el comando \char`\"{}updateNumbers\char`\"{} en el paquete. \end{DoxyCompactList}\item 
def \hyperlink{classcliente__lib_1_1cliente__lib_ac0a4410b48b4c759028bec6ae1c641e8}{lista\_\-sonares}
\begin{DoxyCompactList}\small\item\em Sirve para leer el paquete arNetPacket con la lista del sonar. \end{DoxyCompactList}\item 
def \hyperlink{classcliente__lib_1_1cliente__lib_abcf28c2207cb5519090654484137db23}{valores\_\-sonares}
\begin{DoxyCompactList}\small\item\em Sirve para leer el paquete arNetPacket con los valores del sonar. \end{DoxyCompactList}\item 
def \hyperlink{classcliente__lib_1_1cliente__lib_acfc22af72a1668db28d18ab4ff40909e}{envio\_\-ratioDrive}
\begin{DoxyCompactList}\small\item\em Sirve para realizar la teleoperacion, mandando los parametros. \end{DoxyCompactList}\item 
def \hyperlink{classcliente__lib_1_1cliente__lib_a3c43af1448ad0fa10cfa6f398b7029db}{envio\_\-griper}
\begin{DoxyCompactList}\small\item\em Sirve para realizar la teleoperacion, mandando los parametros. \end{DoxyCompactList}\item 
def \hyperlink{classcliente__lib_1_1cliente__lib_ade1f44e9270c8835c284832a72b96b6c}{uC\_\-comandos\_\-movi}
\begin{DoxyCompactList}\small\item\em Sirve para mandar ordenes de movimiento directamente al controlador de la plataforma movil. \end{DoxyCompactList}\item 
def \hyperlink{classcliente__lib_1_1cliente__lib_ac3e89d3066207b05b217a50f549c239a}{envio\_\-consulta\_\-fisica}
\begin{DoxyCompactList}\small\item\em Sirve para mandar ordenes al servidor utilizando paquetes ArNetPacket con comandos {\bfseries pose} y {\bfseries updateNumbers} \end{DoxyCompactList}\item 
def \hyperlink{classcliente__lib_1_1cliente__lib_a1a7b5475a98772f0e48a4e1fd76e8d47}{cliente\_\-inicio}
\begin{DoxyCompactList}\small\item\em Sirve para iniciar la conexion con el servidor. \end{DoxyCompactList}\item 
def \hyperlink{classcliente__lib_1_1cliente__lib_a14a49495fd71fab84d36060e604415a5}{cliente\_\-apaga}
\begin{DoxyCompactList}\small\item\em Sirve para realizar la desconexion con el servidor. \end{DoxyCompactList}\item 
def \hyperlink{classcliente__lib_1_1cliente__lib_a3ca67c0c9d7f0a622abd740c780f64d1}{devuelve\_\-valorf}
\begin{DoxyCompactList}\small\item\em Devuelve el variable valor\_\-fisico, con usa espera de 100ms. \end{DoxyCompactList}\item 
def \hyperlink{classcliente__lib_1_1cliente__lib_a127d026872fbd11f4f5bbe4a73424b77}{devuelve\_\-valors}
\begin{DoxyCompactList}\small\item\em Devuelve el variable valor\_\-sonares, con usa espera de 100ms. \end{DoxyCompactList}\end{DoxyCompactItemize}
\subsection*{Atributos públicos}
\begin{DoxyCompactItemize}
\item 
\hyperlink{classcliente__lib_1_1cliente__lib_abba3409f89ee8dcec8b180c90aa5d77c}{valor\_\-fisico}
\item 
\hyperlink{classcliente__lib_1_1cliente__lib_aadea6e24bd3a01b0500fcc67543a97e9}{valor\_\-sonares}
\item 
\hyperlink{classcliente__lib_1_1cliente__lib_a675dd8430aa2eeb33240b8b07ed61543}{ip}
\item 
\hyperlink{classcliente__lib_1_1cliente__lib_a7bbce98e978c840ab3ef04e2e3715b56}{x}
\item 
\hyperlink{classcliente__lib_1_1cliente__lib_aa6b4463261d78e82876d5f01ab3800bb}{y}
\item 
\hyperlink{classcliente__lib_1_1cliente__lib_adf30f5e41e7c96565a7588e808d7543a}{t}
\item 
\hyperlink{classcliente__lib_1_1cliente__lib_a5b15a0be5d30abf1f71d776ccead9d9d}{acu}
\end{DoxyCompactItemize}


\subsection{Descripción detallada}
es la clase encargada del cliente 

se lo utiliza de esta manera par poder trabajar con la informacion tanto leyendo las variables o utilizando los comandos 

Definición en la línea 50 del archivo cliente\_\-lib.py.



\subsection{Documentación del constructor y destructor}
\hypertarget{classcliente__lib_1_1cliente__lib_ac5e4490f412835d35481f58d1ae503f9}{
\index{cliente\_\-lib::cliente\_\-lib@{cliente\_\-lib::cliente\_\-lib}!\_\-\_\-init\_\-\_\-@{\_\-\_\-init\_\-\_\-}}
\index{\_\-\_\-init\_\-\_\-@{\_\-\_\-init\_\-\_\-}!cliente_lib::cliente_lib@{cliente\_\-lib::cliente\_\-lib}}
\subsubsection[{\_\-\_\-init\_\-\_\-}]{\setlength{\rightskip}{0pt plus 5cm}def cliente\_\-lib.cliente\_\-lib.\_\-\_\-init\_\-\_\- (
\begin{DoxyParamCaption}
\item[{}]{self}
\end{DoxyParamCaption}
)}}
\label{classcliente__lib_1_1cliente__lib_ac5e4490f412835d35481f58d1ae503f9}


Carga valores a las variables necesarias para funcionar el cliente. 

este comando no es necesario utilizarlo es usado al instanciar la clase 
\begin{DoxyParams}{Parámetros}
{\em self} & este parametro no es necesario escribir \\
\hline
\end{DoxyParams}


Definición en la línea 57 del archivo cliente\_\-lib.py.


\begin{DoxyCode}
58                           :
59                 print "Cargo modulo para cliente_lib"
60                 self.valor_fisico=[]
61                 self.valor_sonares=[] 
62                 self.ip="localhost"
63                 self.x={}
64                 self.y={}
65                 self.t={}
66                 self.acu=0

\end{DoxyCode}


\subsection{Documentación de las funciones miembro}
\hypertarget{classcliente__lib_1_1cliente__lib_a14a49495fd71fab84d36060e604415a5}{
\index{cliente\_\-lib::cliente\_\-lib@{cliente\_\-lib::cliente\_\-lib}!cliente\_\-apaga@{cliente\_\-apaga}}
\index{cliente\_\-apaga@{cliente\_\-apaga}!cliente_lib::cliente_lib@{cliente\_\-lib::cliente\_\-lib}}
\subsubsection[{cliente\_\-apaga}]{\setlength{\rightskip}{0pt plus 5cm}def cliente\_\-lib.cliente\_\-lib.cliente\_\-apaga (
\begin{DoxyParamCaption}
\item[{}]{self, }
\item[{}]{client}
\end{DoxyParamCaption}
)}}
\label{classcliente__lib_1_1cliente__lib_a14a49495fd71fab84d36060e604415a5}


Sirve para realizar la desconexion con el servidor. 


\begin{DoxyParams}{Parámetros}
{\em self} & este parametro no es necesario escribir \\
\hline
{\em client} & para poder desconectar el cliente \\
\hline
\end{DoxyParams}


Definición en la línea 252 del archivo cliente\_\-lib.py.


\begin{DoxyCode}
253                                       :
254                 ArUtil.sleep(1000)
255                 client.disconnect()
256                 ArUtil.sleep(50)
                return 0
\end{DoxyCode}
\hypertarget{classcliente__lib_1_1cliente__lib_a1a7b5475a98772f0e48a4e1fd76e8d47}{
\index{cliente\_\-lib::cliente\_\-lib@{cliente\_\-lib::cliente\_\-lib}!cliente\_\-inicio@{cliente\_\-inicio}}
\index{cliente\_\-inicio@{cliente\_\-inicio}!cliente_lib::cliente_lib@{cliente\_\-lib::cliente\_\-lib}}
\subsubsection[{cliente\_\-inicio}]{\setlength{\rightskip}{0pt plus 5cm}def cliente\_\-lib.cliente\_\-lib.cliente\_\-inicio (
\begin{DoxyParamCaption}
\item[{}]{self}
\end{DoxyParamCaption}
)}}
\label{classcliente__lib_1_1cliente__lib_a1a7b5475a98772f0e48a4e1fd76e8d47}


Sirve para iniciar la conexion con el servidor. 


\begin{DoxyParams}{Parámetros}
{\em self} & este parametro no es necesario escribir \\
\hline
\end{DoxyParams}
\begin{DoxyReturn}{Devuelve}
client 
\end{DoxyReturn}


Definición en la línea 208 del archivo cliente\_\-lib.py.


\begin{DoxyCode}
209                                 :
210                 client = ArClientBase()
211                 #Solo funcionar el lectura de datos por TCP
212                 client.setTcpOnlyFromServer()
213                 client.setTcpOnlyToServer()
214                 #
215                 Aria.init()
216                 
217                 startTime = ArTime()
218                 startTime.setToNow()
219                 if not client.blockingConnect(self.ip, 7272): #ip y puerto del se
      rvidor
220                         print "Could not connect to server at %s port 7272, exiti
      ng" % self.ip
221                         
222                         Aria.exit(1);
223                 print "cliente: Se tardo %ld msec en connectarse\n" % (startTime.
      mSecSince())
224                 
225                 client.runAsync()
226                 client.addHandler("updateNumbers",self.valores)
227                 #
228                 #client.request("updateNumbers",100)
229                 #
230                 client.addHandler("getSensorList",self.lista_sonares)
231                 client.addHandler("pose",self.valores_sonares)
232                 #
233                 #client.lock()
234                 #client.request("pose",100)
235                 #client.unlock()
236                 #
237                 
238                 if client.dataExists("ratioDrive"): #supuestamente devuelve la in
      fo del robot con odometria
239                         print "ratioDrive si existe"
240                 else:
241                         Aria.exit(1);
242                 #client=envio_ratioDrive(client,TransRatio,RotRatio,LatRatio) #fi
      jar los valores para mover
243                 #client=uC_comandos_movi(client,comando,parametro) #Lo hace de un
      a manera directa anulando las demas operaciones
244                 #client.requestOnce("updateNumbers")
245                 #client.requestOnce("stop") #parada de emergencia
                return client
\end{DoxyCode}
\hypertarget{classcliente__lib_1_1cliente__lib_a3ca67c0c9d7f0a622abd740c780f64d1}{
\index{cliente\_\-lib::cliente\_\-lib@{cliente\_\-lib::cliente\_\-lib}!devuelve\_\-valorf@{devuelve\_\-valorf}}
\index{devuelve\_\-valorf@{devuelve\_\-valorf}!cliente_lib::cliente_lib@{cliente\_\-lib::cliente\_\-lib}}
\subsubsection[{devuelve\_\-valorf}]{\setlength{\rightskip}{0pt plus 5cm}def cliente\_\-lib.cliente\_\-lib.devuelve\_\-valorf (
\begin{DoxyParamCaption}
\item[{}]{self}
\end{DoxyParamCaption}
)}}
\label{classcliente__lib_1_1cliente__lib_a3ca67c0c9d7f0a622abd740c780f64d1}


Devuelve el variable valor\_\-fisico, con usa espera de 100ms. 


\begin{DoxyParams}{Parámetros}
{\em self} & este parametro no es necesario escribir \\
\hline
\end{DoxyParams}


Definición en la línea 262 del archivo cliente\_\-lib.py.


\begin{DoxyCode}
263                                  :
264                 ArUtil.sleep(100)
                return self.valor_fisico
\end{DoxyCode}
\hypertarget{classcliente__lib_1_1cliente__lib_a127d026872fbd11f4f5bbe4a73424b77}{
\index{cliente\_\-lib::cliente\_\-lib@{cliente\_\-lib::cliente\_\-lib}!devuelve\_\-valors@{devuelve\_\-valors}}
\index{devuelve\_\-valors@{devuelve\_\-valors}!cliente_lib::cliente_lib@{cliente\_\-lib::cliente\_\-lib}}
\subsubsection[{devuelve\_\-valors}]{\setlength{\rightskip}{0pt plus 5cm}def cliente\_\-lib.cliente\_\-lib.devuelve\_\-valors (
\begin{DoxyParamCaption}
\item[{}]{self}
\end{DoxyParamCaption}
)}}
\label{classcliente__lib_1_1cliente__lib_a127d026872fbd11f4f5bbe4a73424b77}


Devuelve el variable valor\_\-sonares, con usa espera de 100ms. 


\begin{DoxyParams}{Parámetros}
{\em self} & este parametro no es necesario escribir \\
\hline
\end{DoxyParams}


Definición en la línea 270 del archivo cliente\_\-lib.py.


\begin{DoxyCode}
271                                  :
272                 ArUtil.sleep(100)
273                 valor=self.valor_sonares
274                 if valor==[]:
275                         print valor
276                         return {'x0':-9999,'x1':-9999},{'y0':-9999,'y1':-9999},{'
      t0':0,'t1':0} #valores quitar despues
277                 else:
278                         try:
279                                 x_br=[]
280                                 y_br=[]
281                                 t_br=[]
282                                 for a in range(int(valor[0])): #el primer valor i
      ndica cuantos elementos se tiene de sensores
283                                         #print valor[1][a]
284                                         valor[1][a]=valor[1][a].strip('(')
285                                         valor[1][a]=valor[1][a].strip(')')
286                                         valor[1][a]=valor[1][a].split(",") #comie
      nza a separar por que viene en forma (X:valor,Y:valor,T:valor)
287                                         valor[1][a][0]=valor[1][a][0].strip() #bo
      rrar los espacios en blanco de los elementos
288                                         valor[1][a][1]=valor[1][a][1].strip()
289                                         valor[1][a][2]=valor[1][a][2].strip()
290                                         x_br+=[valor[1][a][0].strip("X:")]
291                                         #print x_br
292                                         y_br+=[valor[1][a][1].strip("Y:")]
293                                         t_br+=[valor[1][a][2].strip("T:")]
294                                         
295                                 #print len(x_br)
296                                 for a in range(len(x_br)):
297                                         self.x.update({'x%d' % self.acu:float(x_b
      r[a])}) #Ordena los datos en forma de diccionario, float tiene problemas con 0.00
      0
298                                         self.y.update({'y%d' % self.acu:float(y_b
      r[a])})
299                                         #self.t.update({'t%d' % self.acu:float(t_
      br[a].strip(")"))})
300                                         self.t.update({'t%d' % self.acu:0})
301                                         self.acu+=1
302                                         if self.acu==5000:
303                                                 self.acu=0
304                                                 #pass
305                                 #print self.x
306                                 #print self.y
307                                 #print self.t
308                                 return self.x,self.y,self.t
309                         except:
310                                 pass
311 

\end{DoxyCode}
\hypertarget{classcliente__lib_1_1cliente__lib_ac3e89d3066207b05b217a50f549c239a}{
\index{cliente\_\-lib::cliente\_\-lib@{cliente\_\-lib::cliente\_\-lib}!envio\_\-consulta\_\-fisica@{envio\_\-consulta\_\-fisica}}
\index{envio\_\-consulta\_\-fisica@{envio\_\-consulta\_\-fisica}!cliente_lib::cliente_lib@{cliente\_\-lib::cliente\_\-lib}}
\subsubsection[{envio\_\-consulta\_\-fisica}]{\setlength{\rightskip}{0pt plus 5cm}def cliente\_\-lib.cliente\_\-lib.envio\_\-consulta\_\-fisica (
\begin{DoxyParamCaption}
\item[{}]{self, }
\item[{}]{client, }
\item[{}]{mensaje}
\end{DoxyParamCaption}
)}}
\label{classcliente__lib_1_1cliente__lib_ac3e89d3066207b05b217a50f549c239a}


Sirve para mandar ordenes al servidor utilizando paquetes ArNetPacket con comandos {\bfseries pose} y {\bfseries updateNumbers} 


\begin{DoxyParams}{Parámetros}
{\em self} & este parametro no es necesario escribir \\
\hline
{\em client} & Se debe trar el objeto cliente a la definicion para poder utilizar el enlace del cliente para enviar el paquete al servidor \\
\hline
{\em mensaje} & puede ser cualquier comando del servidor que no devuelva informacion a exepcion de pose y updatenumbers \\
\hline
\end{DoxyParams}
\begin{DoxyReturn}{Devuelve}
client 
\end{DoxyReturn}


Definición en la línea 197 del archivo cliente\_\-lib.py.


\begin{DoxyCode}
198                                                       :
199                 ## se puede usar pose y updateNumbers
200                 client.requestOnce(mensaje)
201                 return client
                
\end{DoxyCode}
\hypertarget{classcliente__lib_1_1cliente__lib_a3c43af1448ad0fa10cfa6f398b7029db}{
\index{cliente\_\-lib::cliente\_\-lib@{cliente\_\-lib::cliente\_\-lib}!envio\_\-griper@{envio\_\-griper}}
\index{envio\_\-griper@{envio\_\-griper}!cliente_lib::cliente_lib@{cliente\_\-lib::cliente\_\-lib}}
\subsubsection[{envio\_\-griper}]{\setlength{\rightskip}{0pt plus 5cm}def cliente\_\-lib.cliente\_\-lib.envio\_\-griper (
\begin{DoxyParamCaption}
\item[{}]{self, }
\item[{}]{client, }
\item[{}]{TransRatio}
\end{DoxyParamCaption}
)}}
\label{classcliente__lib_1_1cliente__lib_a3c43af1448ad0fa10cfa6f398b7029db}


Sirve para realizar la teleoperacion, mandando los parametros. 


\begin{DoxyParams}{Parámetros}
{\em self} & este parametro no es necesario escribir \\
\hline
{\em client} & Se debe trar el objeto cliente a la definicion para poder utilizar el enlace del cliente para enviar el paquete al servidor \\
\hline
{\em TransRatio} & Velocidad de traslacion \\
\hline
{\em RotRatio} & Velocidad de rotacion \\
\hline
{\em LatRatio} & velocidad lateral para el modelo Pioneer P3-\/DX no se necesario puede ser 0 \\
\hline
\end{DoxyParams}
\begin{DoxyReturn}{Devuelve}
client 
\end{DoxyReturn}


Definición en la línea 165 del archivo cliente\_\-lib.py.


\begin{DoxyCode}
166                                                 :
167                 myTransRatio=TransRatio
168                 packet=ArNetPacket()
169                 packet.doubleToBuf(myTransRatio)
170                 client.requestOnce("ComandoGripper", packet)
171                 return client
          
\end{DoxyCode}
\hypertarget{classcliente__lib_1_1cliente__lib_acfc22af72a1668db28d18ab4ff40909e}{
\index{cliente\_\-lib::cliente\_\-lib@{cliente\_\-lib::cliente\_\-lib}!envio\_\-ratioDrive@{envio\_\-ratioDrive}}
\index{envio\_\-ratioDrive@{envio\_\-ratioDrive}!cliente_lib::cliente_lib@{cliente\_\-lib::cliente\_\-lib}}
\subsubsection[{envio\_\-ratioDrive}]{\setlength{\rightskip}{0pt plus 5cm}def cliente\_\-lib.cliente\_\-lib.envio\_\-ratioDrive (
\begin{DoxyParamCaption}
\item[{}]{self, }
\item[{}]{client, }
\item[{}]{TransRatio, }
\item[{}]{RotRatio, }
\item[{}]{LatRatio}
\end{DoxyParamCaption}
)}}
\label{classcliente__lib_1_1cliente__lib_acfc22af72a1668db28d18ab4ff40909e}


Sirve para realizar la teleoperacion, mandando los parametros. 


\begin{DoxyParams}{Parámetros}
{\em self} & este parametro no es necesario escribir \\
\hline
{\em client} & Se debe trar el objeto cliente a la definicion para poder utilizar el enlace del cliente para enviar el paquete al servidor \\
\hline
{\em TransRatio} & Velocidad de traslacion \\
\hline
{\em RotRatio} & Velocidad de rotacion \\
\hline
{\em LatRatio} & velocidad lateral para el modelo Pioneer P3-\/DX no se necesario puede ser 0 \\
\hline
\end{DoxyParams}
\begin{DoxyReturn}{Devuelve}
client 
\end{DoxyReturn}


Definición en la línea 143 del archivo cliente\_\-lib.py.


\begin{DoxyCode}
144                                                                       :
145                 myTransRatio=TransRatio
146                 myRotRatio=RotRatio
147                 myLatRatio=LatRatio
148                 packet=ArNetPacket()
149                 packet.doubleToBuf(myTransRatio)
150                 packet.doubleToBuf(myRotRatio)
151                 packet.doubleToBuf(50) # use half of the robot's maximum.
152                 packet.doubleToBuf(myLatRatio)
153                 client.requestOnce("ratioDrive", packet)
154                 return client
        
\end{DoxyCode}
\hypertarget{classcliente__lib_1_1cliente__lib_ac0a4410b48b4c759028bec6ae1c641e8}{
\index{cliente\_\-lib::cliente\_\-lib@{cliente\_\-lib::cliente\_\-lib}!lista\_\-sonares@{lista\_\-sonares}}
\index{lista\_\-sonares@{lista\_\-sonares}!cliente_lib::cliente_lib@{cliente\_\-lib::cliente\_\-lib}}
\subsubsection[{lista\_\-sonares}]{\setlength{\rightskip}{0pt plus 5cm}def cliente\_\-lib.cliente\_\-lib.lista\_\-sonares (
\begin{DoxyParamCaption}
\item[{}]{self, }
\item[{}]{packet}
\end{DoxyParamCaption}
)}}
\label{classcliente__lib_1_1cliente__lib_ac0a4410b48b4c759028bec6ae1c641e8}


Sirve para leer el paquete arNetPacket con la lista del sonar. 

este comando no es necesario utilizarlo es usado solo por el cliente para procesar el paquete 
\begin{DoxyParams}{Parámetros}
{\em self} & este parametro no es necesario escribir \\
\hline
{\em packet} & este parametro no es necesario escribir \\
\hline
\end{DoxyParams}
\begin{DoxyReturn}{Devuelve}
nada 
\end{DoxyReturn}


Definición en la línea 98 del archivo cliente\_\-lib.py.


\begin{DoxyCode}
99                                       :
100                 c="                                   "
101                 numSensor=packet.bufToByte2()
102                 numSensor2=packet.bufToStr(c,15)
103                 print str(numSensor)+" "+str(c.strip())

\end{DoxyCode}
\hypertarget{classcliente__lib_1_1cliente__lib_ade1f44e9270c8835c284832a72b96b6c}{
\index{cliente\_\-lib::cliente\_\-lib@{cliente\_\-lib::cliente\_\-lib}!uC\_\-comandos\_\-movi@{uC\_\-comandos\_\-movi}}
\index{uC\_\-comandos\_\-movi@{uC\_\-comandos\_\-movi}!cliente_lib::cliente_lib@{cliente\_\-lib::cliente\_\-lib}}
\subsubsection[{uC\_\-comandos\_\-movi}]{\setlength{\rightskip}{0pt plus 5cm}def cliente\_\-lib.cliente\_\-lib.uC\_\-comandos\_\-movi (
\begin{DoxyParamCaption}
\item[{}]{self, }
\item[{}]{client, }
\item[{}]{comando, }
\item[{}]{parametro}
\end{DoxyParamCaption}
)}}
\label{classcliente__lib_1_1cliente__lib_ade1f44e9270c8835c284832a72b96b6c}


Sirve para mandar ordenes de movimiento directamente al controlador de la plataforma movil. 


\begin{DoxyParams}{Parámetros}
{\em self} & este parametro no es necesario escribir \\
\hline
{\em client} & Se debe trar el objeto cliente a la definicion para poder utilizar el enlace del cliente para enviar el paquete al servidor \\
\hline
{\em comando} & es un numero de 1-\/255 que representa una funcion esta informacion se puede encontrar en el API de ARIA \\
\hline
{\em parametro} & el parametro de la funcion en caso de no tener se deja el valor en blanco \\
\hline
\end{DoxyParams}
\begin{DoxyReturn}{Devuelve}
client 
\end{DoxyReturn}


Definición en la línea 181 del archivo cliente\_\-lib.py.


\begin{DoxyCode}
182                                                            :
183                 mi_comando=comando #comando 8 es MOVE parametro un valor de 5000 
      a -4999 es en mm, 11 LEV y su parametro es velocidad +o- mm/s
184                 mi_parametro=parametro #parametro
185                 packet=ArNetPacket()
186                 packet.strToBuf(mi_comando+" "+mi_parametro)
187                 client.requestOnce("MicroControllerMotionCommand", packet) #Micro
      ControllerMotionCommand
188                 return client

\end{DoxyCode}
\hypertarget{classcliente__lib_1_1cliente__lib_af7b751bcf94c96150b23bacb5e477956}{
\index{cliente\_\-lib::cliente\_\-lib@{cliente\_\-lib::cliente\_\-lib}!valores@{valores}}
\index{valores@{valores}!cliente_lib::cliente_lib@{cliente\_\-lib::cliente\_\-lib}}
\subsubsection[{valores}]{\setlength{\rightskip}{0pt plus 5cm}def cliente\_\-lib.cliente\_\-lib.valores (
\begin{DoxyParamCaption}
\item[{}]{self, }
\item[{}]{packet}
\end{DoxyParamCaption}
)}}
\label{classcliente__lib_1_1cliente__lib_af7b751bcf94c96150b23bacb5e477956}


Sirve cuando se manda el comando \char`\"{}updateNumbers\char`\"{} en el paquete. 

este comando no es necesario utilizarlo es usado solo por el cliente para procesar el paquete 
\begin{DoxyParams}{Parámetros}
{\em self} & este parametro no es necesario escribir \\
\hline
{\em packet} & el paquete que recibe el cliente del servidor, no es necesario escribir \\
\hline
\end{DoxyParams}
\begin{DoxyReturn}{Devuelve}
Nada, pero guarda en self.valores\_\-fisico \mbox{[}voltaje\_\-bateria,myX,myY,myTh,myVel,myRotVel,myLatVel,myTemperature\mbox{]} 
\end{DoxyReturn}


Definición en la línea 75 del archivo cliente\_\-lib.py.


\begin{DoxyCode}
76                                 :
77                 #devuelve los valores voltaje_bateria,myX,myY,myTh,myVel,myRotVel
      ,myLatVel,myTemperature
78                 voltaje_bateria=packet.bufToByte2()/10
79                 myX = packet.bufToByte4()#
80                 myY = packet.bufToByte4()
81                 myTh = packet.bufToByte2()
82                 myVel = packet.bufToByte2()
83                 myRotVel = packet.bufToByte2()
84                 myLatVel = packet.bufToByte2()
85                 myTemperature = packet.bufToByte()
86                 #print "X= "+str(myX)+" y="+str(myY)+" th="+str(myTh)
87                 self.valor_fisico=(voltaje_bateria,myX,myY,myTh,myVel,myRotVel,my
      LatVel,myTemperature)
88                 self.valor_fisico=(voltaje_bateria,myX,myY,myTh,myVel,myRotVel,my
      LatVel,0)
89                 #print valor

\end{DoxyCode}
\hypertarget{classcliente__lib_1_1cliente__lib_abcf28c2207cb5519090654484137db23}{
\index{cliente\_\-lib::cliente\_\-lib@{cliente\_\-lib::cliente\_\-lib}!valores\_\-sonares@{valores\_\-sonares}}
\index{valores\_\-sonares@{valores\_\-sonares}!cliente_lib::cliente_lib@{cliente\_\-lib::cliente\_\-lib}}
\subsubsection[{valores\_\-sonares}]{\setlength{\rightskip}{0pt plus 5cm}def cliente\_\-lib.cliente\_\-lib.valores\_\-sonares (
\begin{DoxyParamCaption}
\item[{}]{self, }
\item[{}]{packet}
\end{DoxyParamCaption}
)}}
\label{classcliente__lib_1_1cliente__lib_abcf28c2207cb5519090654484137db23}


Sirve para leer el paquete arNetPacket con los valores del sonar. 

este comando no es necesario utilizarlo es usado solo por el cliente para procesar el paquete 
\begin{DoxyParams}{Parámetros}
{\em self} & este parametro no es necesario escribir \\
\hline
{\em packet} & este parametro no es necesario escribir \\
\hline
\end{DoxyParams}
\begin{DoxyReturn}{Devuelve}
nada 
\end{DoxyReturn}


Definición en la línea 112 del archivo cliente\_\-lib.py.


\begin{DoxyCode}
113                                         :
114                 try:
115                         bandera1=0
116                         cantidad=packet.bufToDouble()
117                         print int(cantidad)
118                         for i in range(2*int(cantidad)):
119                                 if bandera1==0:
120                                         self.x.update({'x%d' % (self.acu):int(pac
      ket.bufToByte4())})
121                                         bandera1=1
122                                 else:
123                                         self.y.update({'y%d' % (self.acu):int(pac
      ket.bufToByte4())})
124                                         bandera1=0
125                                         self.acu=self.acu+1
126                                 if self.acu==200:
127                                         self.acu=0
128                                         self.x={}
129                                         self.y={}
130                 except:
131                         print "Fallo lectura sensores"
132                         pass

\end{DoxyCode}


\subsection{Documentación de los datos miembro}
\hypertarget{classcliente__lib_1_1cliente__lib_a5b15a0be5d30abf1f71d776ccead9d9d}{
\index{cliente\_\-lib::cliente\_\-lib@{cliente\_\-lib::cliente\_\-lib}!acu@{acu}}
\index{acu@{acu}!cliente_lib::cliente_lib@{cliente\_\-lib::cliente\_\-lib}}
\subsubsection[{acu}]{\setlength{\rightskip}{0pt plus 5cm}{\bf cliente\_\-lib.cliente\_\-lib.acu}}}
\label{classcliente__lib_1_1cliente__lib_a5b15a0be5d30abf1f71d776ccead9d9d}


Definición en la línea 57 del archivo cliente\_\-lib.py.

\hypertarget{classcliente__lib_1_1cliente__lib_a675dd8430aa2eeb33240b8b07ed61543}{
\index{cliente\_\-lib::cliente\_\-lib@{cliente\_\-lib::cliente\_\-lib}!ip@{ip}}
\index{ip@{ip}!cliente_lib::cliente_lib@{cliente\_\-lib::cliente\_\-lib}}
\subsubsection[{ip}]{\setlength{\rightskip}{0pt plus 5cm}{\bf cliente\_\-lib.cliente\_\-lib.ip}}}
\label{classcliente__lib_1_1cliente__lib_a675dd8430aa2eeb33240b8b07ed61543}


Definición en la línea 57 del archivo cliente\_\-lib.py.

\hypertarget{classcliente__lib_1_1cliente__lib_adf30f5e41e7c96565a7588e808d7543a}{
\index{cliente\_\-lib::cliente\_\-lib@{cliente\_\-lib::cliente\_\-lib}!t@{t}}
\index{t@{t}!cliente_lib::cliente_lib@{cliente\_\-lib::cliente\_\-lib}}
\subsubsection[{t}]{\setlength{\rightskip}{0pt plus 5cm}{\bf cliente\_\-lib.cliente\_\-lib.t}}}
\label{classcliente__lib_1_1cliente__lib_adf30f5e41e7c96565a7588e808d7543a}


Definición en la línea 57 del archivo cliente\_\-lib.py.

\hypertarget{classcliente__lib_1_1cliente__lib_abba3409f89ee8dcec8b180c90aa5d77c}{
\index{cliente\_\-lib::cliente\_\-lib@{cliente\_\-lib::cliente\_\-lib}!valor\_\-fisico@{valor\_\-fisico}}
\index{valor\_\-fisico@{valor\_\-fisico}!cliente_lib::cliente_lib@{cliente\_\-lib::cliente\_\-lib}}
\subsubsection[{valor\_\-fisico}]{\setlength{\rightskip}{0pt plus 5cm}{\bf cliente\_\-lib.cliente\_\-lib.valor\_\-fisico}}}
\label{classcliente__lib_1_1cliente__lib_abba3409f89ee8dcec8b180c90aa5d77c}


Definición en la línea 57 del archivo cliente\_\-lib.py.

\hypertarget{classcliente__lib_1_1cliente__lib_aadea6e24bd3a01b0500fcc67543a97e9}{
\index{cliente\_\-lib::cliente\_\-lib@{cliente\_\-lib::cliente\_\-lib}!valor\_\-sonares@{valor\_\-sonares}}
\index{valor\_\-sonares@{valor\_\-sonares}!cliente_lib::cliente_lib@{cliente\_\-lib::cliente\_\-lib}}
\subsubsection[{valor\_\-sonares}]{\setlength{\rightskip}{0pt plus 5cm}{\bf cliente\_\-lib.cliente\_\-lib.valor\_\-sonares}}}
\label{classcliente__lib_1_1cliente__lib_aadea6e24bd3a01b0500fcc67543a97e9}


Definición en la línea 57 del archivo cliente\_\-lib.py.

\hypertarget{classcliente__lib_1_1cliente__lib_a7bbce98e978c840ab3ef04e2e3715b56}{
\index{cliente\_\-lib::cliente\_\-lib@{cliente\_\-lib::cliente\_\-lib}!x@{x}}
\index{x@{x}!cliente_lib::cliente_lib@{cliente\_\-lib::cliente\_\-lib}}
\subsubsection[{x}]{\setlength{\rightskip}{0pt plus 5cm}{\bf cliente\_\-lib.cliente\_\-lib.x}}}
\label{classcliente__lib_1_1cliente__lib_a7bbce98e978c840ab3ef04e2e3715b56}


Definición en la línea 57 del archivo cliente\_\-lib.py.

\hypertarget{classcliente__lib_1_1cliente__lib_aa6b4463261d78e82876d5f01ab3800bb}{
\index{cliente\_\-lib::cliente\_\-lib@{cliente\_\-lib::cliente\_\-lib}!y@{y}}
\index{y@{y}!cliente_lib::cliente_lib@{cliente\_\-lib::cliente\_\-lib}}
\subsubsection[{y}]{\setlength{\rightskip}{0pt plus 5cm}{\bf cliente\_\-lib.cliente\_\-lib.y}}}
\label{classcliente__lib_1_1cliente__lib_aa6b4463261d78e82876d5f01ab3800bb}


Definición en la línea 57 del archivo cliente\_\-lib.py.



La documentación para esta clase fue generada a partir del siguiente fichero:\begin{DoxyCompactItemize}
\item 
\hyperlink{cliente__lib_8py}{cliente\_\-lib.py}\end{DoxyCompactItemize}

\hypertarget{classcliente__lib__original_1_1cliente__lib}{
\section{Referencia de la Clase cliente\_\-lib\_\-original.cliente\_\-lib}
\label{classcliente__lib__original_1_1cliente__lib}\index{cliente\_\-lib\_\-original::cliente\_\-lib@{cliente\_\-lib\_\-original::cliente\_\-lib}}
}
\subsection*{Métodos públicos}
\begin{DoxyCompactItemize}
\item 
def \hyperlink{classcliente__lib__original_1_1cliente__lib_a1edb12a09794f57a5c681bc5aac49650}{\_\-\_\-init\_\-\_\-}
\begin{DoxyCompactList}\small\item\em Carga valores a las variables necesarias para funcionar el cliente. \end{DoxyCompactList}\item 
def \hyperlink{classcliente__lib__original_1_1cliente__lib_a5f97cbead3de2cb78534afec8343c13e}{valores}
\begin{DoxyCompactList}\small\item\em Sirve cuando se manda el comando \char`\"{}updateNumbers\char`\"{} en el paquete. \end{DoxyCompactList}\item 
def \hyperlink{classcliente__lib__original_1_1cliente__lib_a01b5aae2c3ce57fca590b9d06e767f23}{lista\_\-sonares}
\begin{DoxyCompactList}\small\item\em Sirve para leer el paquete arNetPacket con la lista del sonar. \end{DoxyCompactList}\item 
def \hyperlink{classcliente__lib__original_1_1cliente__lib_a2ab0872984bef5af4bfbb7d2ec7f40c5}{valores\_\-sonares}
\begin{DoxyCompactList}\small\item\em Sirve para leer el paquete arNetPacket con los valores del sonar. \end{DoxyCompactList}\item 
def \hyperlink{classcliente__lib__original_1_1cliente__lib_aac50c9462dfe46d1b618692d7206295b}{envio\_\-ratioDrive}
\begin{DoxyCompactList}\small\item\em Sirve para realizar la teleoperacion, mandando los parametros. \end{DoxyCompactList}\item 
def \hyperlink{classcliente__lib__original_1_1cliente__lib_a16a11ee4bc738ae83b652343583ad556}{uC\_\-comandos\_\-movi}
\begin{DoxyCompactList}\small\item\em Sirve para mandar ordenes de movimiento directamente al controlador de la plataforma movil. \end{DoxyCompactList}\item 
def \hyperlink{classcliente__lib__original_1_1cliente__lib_a10ab9f40fbd7244e96c2b2493e9a9e86}{envio\_\-consulta\_\-fisica}
\begin{DoxyCompactList}\small\item\em Sirve para mandar ordenes al servidor utilizando paquetes ArNetPacket con comandos {\bfseries pose} y {\bfseries updateNumbers} \end{DoxyCompactList}\item 
def \hyperlink{classcliente__lib__original_1_1cliente__lib_a52e3e1ca7b1935b7fb7e6f0a093918e9}{cliente\_\-inicio}
\begin{DoxyCompactList}\small\item\em Sirve para iniciar la conexion con el servidor. \end{DoxyCompactList}\item 
def \hyperlink{classcliente__lib__original_1_1cliente__lib_a9ea49590b5ca6de4f6368f2209a9ac0e}{cliente\_\-apaga}
\begin{DoxyCompactList}\small\item\em Sirve para realizar la desconexion con el servidor. \end{DoxyCompactList}\item 
def \hyperlink{classcliente__lib__original_1_1cliente__lib_ae6a834b4525e77e3f88f6eaa68ae97eb}{devuelve\_\-valorf}
\begin{DoxyCompactList}\small\item\em Devuelve el variable valor\_\-fisico, con usa espera de 100ms. \end{DoxyCompactList}\item 
def \hyperlink{classcliente__lib__original_1_1cliente__lib_a5766042b7c5c2bd3b291c02141af8824}{devuelve\_\-valors}
\begin{DoxyCompactList}\small\item\em Devuelve el variable valor\_\-sonares, con usa espera de 100ms. \end{DoxyCompactList}\end{DoxyCompactItemize}
\subsection*{Atributos públicos}
\begin{DoxyCompactItemize}
\item 
\hyperlink{classcliente__lib__original_1_1cliente__lib_a74cb35b9f3246db8b6b16ac59ae8f320}{valor\_\-fisico}
\item 
\hyperlink{classcliente__lib__original_1_1cliente__lib_a4172f5914b673eb44e168b4afcbae651}{valor\_\-sonares}
\item 
\hyperlink{classcliente__lib__original_1_1cliente__lib_a030e232b37138f0a2ba15a3ef5f1fe21}{ip}
\item 
\hyperlink{classcliente__lib__original_1_1cliente__lib_aa2933583abd7844c57b2ec553cbb46ee}{x}
\item 
\hyperlink{classcliente__lib__original_1_1cliente__lib_a6050f7c724f8ca063505e8b80a1cdecf}{y}
\item 
\hyperlink{classcliente__lib__original_1_1cliente__lib_ae6a2ed4ac198fc965fbe3ee9a01f9401}{t}
\item 
\hyperlink{classcliente__lib__original_1_1cliente__lib_a049a973bc28127e24c2c717de93daf2f}{acu}
\end{DoxyCompactItemize}


\subsection{Descripción detallada}


Definición en la línea 51 del archivo cliente\_\-lib\_\-original.py.



\subsection{Documentación del constructor y destructor}
\hypertarget{classcliente__lib__original_1_1cliente__lib_a1edb12a09794f57a5c681bc5aac49650}{
\index{cliente\_\-lib\_\-original::cliente\_\-lib@{cliente\_\-lib\_\-original::cliente\_\-lib}!\_\-\_\-init\_\-\_\-@{\_\-\_\-init\_\-\_\-}}
\index{\_\-\_\-init\_\-\_\-@{\_\-\_\-init\_\-\_\-}!cliente_lib_original::cliente_lib@{cliente\_\-lib\_\-original::cliente\_\-lib}}
\subsubsection[{\_\-\_\-init\_\-\_\-}]{\setlength{\rightskip}{0pt plus 5cm}def cliente\_\-lib\_\-original.cliente\_\-lib.\_\-\_\-init\_\-\_\- (
\begin{DoxyParamCaption}
\item[{}]{self}
\end{DoxyParamCaption}
)}}
\label{classcliente__lib__original_1_1cliente__lib_a1edb12a09794f57a5c681bc5aac49650}


Carga valores a las variables necesarias para funcionar el cliente. 

este comando no es necesario utilizarlo es usado al instanciar la clase 
\begin{DoxyParams}{Parámetros}
{\em self} & este parametro no es necesario escribir \\
\hline
\end{DoxyParams}


Definición en la línea 58 del archivo cliente\_\-lib\_\-original.py.


\begin{DoxyCode}
59                           :
60                 print "Cargo modulo para cliente_lib"
61                 self.valor_fisico=[]
62                 self.valor_sonares=[] 
63                 self.ip="localhost"
64                 self.x={}
65                 self.y={}
66                 self.t={}
67                 self.acu=0

\end{DoxyCode}


\subsection{Documentación de las funciones miembro}
\hypertarget{classcliente__lib__original_1_1cliente__lib_a9ea49590b5ca6de4f6368f2209a9ac0e}{
\index{cliente\_\-lib\_\-original::cliente\_\-lib@{cliente\_\-lib\_\-original::cliente\_\-lib}!cliente\_\-apaga@{cliente\_\-apaga}}
\index{cliente\_\-apaga@{cliente\_\-apaga}!cliente_lib_original::cliente_lib@{cliente\_\-lib\_\-original::cliente\_\-lib}}
\subsubsection[{cliente\_\-apaga}]{\setlength{\rightskip}{0pt plus 5cm}def cliente\_\-lib\_\-original.cliente\_\-lib.cliente\_\-apaga (
\begin{DoxyParamCaption}
\item[{}]{self, }
\item[{}]{client}
\end{DoxyParamCaption}
)}}
\label{classcliente__lib__original_1_1cliente__lib_a9ea49590b5ca6de4f6368f2209a9ac0e}


Sirve para realizar la desconexion con el servidor. 


\begin{DoxyParams}{Parámetros}
{\em self} & este parametro no es necesario escribir \\
\hline
{\em client} & para poder desconectar el cliente \\
\hline
\end{DoxyParams}


Definición en la línea 213 del archivo cliente\_\-lib\_\-original.py.


\begin{DoxyCode}
214                                       :
215                 ArUtil.sleep(1000)
216                 client.disconnect()
217                 ArUtil.sleep(50)
                return 0
\end{DoxyCode}
\hypertarget{classcliente__lib__original_1_1cliente__lib_a52e3e1ca7b1935b7fb7e6f0a093918e9}{
\index{cliente\_\-lib\_\-original::cliente\_\-lib@{cliente\_\-lib\_\-original::cliente\_\-lib}!cliente\_\-inicio@{cliente\_\-inicio}}
\index{cliente\_\-inicio@{cliente\_\-inicio}!cliente_lib_original::cliente_lib@{cliente\_\-lib\_\-original::cliente\_\-lib}}
\subsubsection[{cliente\_\-inicio}]{\setlength{\rightskip}{0pt plus 5cm}def cliente\_\-lib\_\-original.cliente\_\-lib.cliente\_\-inicio (
\begin{DoxyParamCaption}
\item[{}]{self}
\end{DoxyParamCaption}
)}}
\label{classcliente__lib__original_1_1cliente__lib_a52e3e1ca7b1935b7fb7e6f0a093918e9}


Sirve para iniciar la conexion con el servidor. 


\begin{DoxyParams}{Parámetros}
{\em self} & este parametro no es necesario escribir \\
\hline
\end{DoxyParams}
\begin{DoxyReturn}{Devuelve}
client 
\end{DoxyReturn}


Definición en la línea 183 del archivo cliente\_\-lib\_\-original.py.


\begin{DoxyCode}
184                                 :
185                 client = ArClientBase()
186                 Aria.init()
187                 
188                 startTime = ArTime()
189                 startTime.setToNow()
190                 if not client.blockingConnect(self.ip, 7272): #ip y puerto del se
      rvidor
191                         print "Could not connect to server at localhost port 7272
      , exiting"
192                         Aria.exit(1);
193                 print "cliente: Se tardo %ld msec en connectarse\n" % (startTime.
      mSecSince())
194                 
195                 client.runAsync()
196                 client.addHandler("updateNumbers",self.valores)
197                 client.addHandler("getSensorList",self.lista_sonares)
198                 client.addHandler("pose",self.valores_sonares)
199                 if client.dataExists("ratioDrive"): #supuestamente devuelve la in
      fo del robot con odometria
200                         print "ratioDrive si existe"
201                 else:
202                         Aria.exit(1);
203                 #client=envio_ratioDrive(client,TransRatio,RotRatio,LatRatio) #fi
      jar los valores para mover
204                 #client=uC_comandos_movi(client,comando,parametro) #Lo hace de un
      a manera directa anulando las demas operaciones
205                 #client.requestOnce("updateNumbers")
206                 #client.requestOnce("stop") #parada de emergencia
                return client
\end{DoxyCode}
\hypertarget{classcliente__lib__original_1_1cliente__lib_ae6a834b4525e77e3f88f6eaa68ae97eb}{
\index{cliente\_\-lib\_\-original::cliente\_\-lib@{cliente\_\-lib\_\-original::cliente\_\-lib}!devuelve\_\-valorf@{devuelve\_\-valorf}}
\index{devuelve\_\-valorf@{devuelve\_\-valorf}!cliente_lib_original::cliente_lib@{cliente\_\-lib\_\-original::cliente\_\-lib}}
\subsubsection[{devuelve\_\-valorf}]{\setlength{\rightskip}{0pt plus 5cm}def cliente\_\-lib\_\-original.cliente\_\-lib.devuelve\_\-valorf (
\begin{DoxyParamCaption}
\item[{}]{self}
\end{DoxyParamCaption}
)}}
\label{classcliente__lib__original_1_1cliente__lib_ae6a834b4525e77e3f88f6eaa68ae97eb}


Devuelve el variable valor\_\-fisico, con usa espera de 100ms. 


\begin{DoxyParams}{Parámetros}
{\em self} & este parametro no es necesario escribir \\
\hline
\end{DoxyParams}


Definición en la línea 223 del archivo cliente\_\-lib\_\-original.py.


\begin{DoxyCode}
224                                  :
225                 ArUtil.sleep(100)
                return self.valor_fisico
\end{DoxyCode}
\hypertarget{classcliente__lib__original_1_1cliente__lib_a5766042b7c5c2bd3b291c02141af8824}{
\index{cliente\_\-lib\_\-original::cliente\_\-lib@{cliente\_\-lib\_\-original::cliente\_\-lib}!devuelve\_\-valors@{devuelve\_\-valors}}
\index{devuelve\_\-valors@{devuelve\_\-valors}!cliente_lib_original::cliente_lib@{cliente\_\-lib\_\-original::cliente\_\-lib}}
\subsubsection[{devuelve\_\-valors}]{\setlength{\rightskip}{0pt plus 5cm}def cliente\_\-lib\_\-original.cliente\_\-lib.devuelve\_\-valors (
\begin{DoxyParamCaption}
\item[{}]{self}
\end{DoxyParamCaption}
)}}
\label{classcliente__lib__original_1_1cliente__lib_a5766042b7c5c2bd3b291c02141af8824}


Devuelve el variable valor\_\-sonares, con usa espera de 100ms. 


\begin{DoxyParams}{Parámetros}
{\em self} & este parametro no es necesario escribir \\
\hline
\end{DoxyParams}


Definición en la línea 231 del archivo cliente\_\-lib\_\-original.py.


\begin{DoxyCode}
232                                  :
233                 
234                 try:
235                         ArUtil.sleep(100)
236                         valor=self.valor_sonares
237                         x_br=[]
238                         y_br=[]
239                         t_br=[]
240                         for a in range(int(valor[0])): #el primer valor indica cu
      antos elementos se tiene de sensores
241                                 valor[1][a]=valor[1][a].strip('(')
242                                 valor[1][a]=valor[1][a].strip(')')
243                                 valor[1][a]=valor[1][a].split(",") #comienza a se
      parar por que viene en forma (X:valor,Y:valor,T:valor)
244                                 valor[1][a][0]=valor[1][a][0].strip() #borrar los
       espacios en blanco de los elementos
245                                 valor[1][a][1]=valor[1][a][1].strip()
246                                 valor[1][a][2]=valor[1][a][2].strip()
247                                 x_br+=[valor[1][a][0].strip("X:")]
248                                 y_br+=[valor[1][a][1].strip("Y:")]
249                                 t_br+=[valor[1][a][2].strip("T:")]
250                         print len(x_br)
251                         print x_br
252                         print y_br
253                         for a in range(len(x_br)):
254                                 self.x.update({'x%d' % self.acu:float(x_br[a])}) 
      #Ordena los datos en forma de diccionario
255                                 self.y.update({'y%d' % self.acu:float(y_br[a])})
256                                 self.t.update({'t%d' % self.acu:0})
257                                 #self.t.update({'t%d' % self.acu:float(t_br[a])})
      
258                                 self.acu+=1
259                                 if self.acu==30:
260                                         self.acu=0
261                         #print self.x
262                         #print self.y
263                         #print self.t
264                         return self.x,self.y,self.t
265                 except:
266                         return {'x0':-9999,'x1':-9999},{'y0':-9999,'y1':-9999},{'
      t0':0,'t1':0}

\end{DoxyCode}
\hypertarget{classcliente__lib__original_1_1cliente__lib_a10ab9f40fbd7244e96c2b2493e9a9e86}{
\index{cliente\_\-lib\_\-original::cliente\_\-lib@{cliente\_\-lib\_\-original::cliente\_\-lib}!envio\_\-consulta\_\-fisica@{envio\_\-consulta\_\-fisica}}
\index{envio\_\-consulta\_\-fisica@{envio\_\-consulta\_\-fisica}!cliente_lib_original::cliente_lib@{cliente\_\-lib\_\-original::cliente\_\-lib}}
\subsubsection[{envio\_\-consulta\_\-fisica}]{\setlength{\rightskip}{0pt plus 5cm}def cliente\_\-lib\_\-original.cliente\_\-lib.envio\_\-consulta\_\-fisica (
\begin{DoxyParamCaption}
\item[{}]{self, }
\item[{}]{client, }
\item[{}]{mensaje}
\end{DoxyParamCaption}
)}}
\label{classcliente__lib__original_1_1cliente__lib_a10ab9f40fbd7244e96c2b2493e9a9e86}


Sirve para mandar ordenes al servidor utilizando paquetes ArNetPacket con comandos {\bfseries pose} y {\bfseries updateNumbers} 


\begin{DoxyParams}{Parámetros}
{\em self} & este parametro no es necesario escribir \\
\hline
{\em client} & Se debe trar el objeto cliente a la definicion para poder utilizar el enlace del cliente para enviar el paquete al servidor \\
\hline
{\em mensaje} & puede ser cualquier comando del servidor que no devuelva informacion a exepcion de pose y updatenumbers \\
\hline
\end{DoxyParams}
\begin{DoxyReturn}{Devuelve}
client 
\end{DoxyReturn}


Definición en la línea 172 del archivo cliente\_\-lib\_\-original.py.


\begin{DoxyCode}
173                                                       :
174                 ## se puede usar pose y updateNumbers
175                 client.requestOnce(mensaje)
176                 return client
                
\end{DoxyCode}
\hypertarget{classcliente__lib__original_1_1cliente__lib_aac50c9462dfe46d1b618692d7206295b}{
\index{cliente\_\-lib\_\-original::cliente\_\-lib@{cliente\_\-lib\_\-original::cliente\_\-lib}!envio\_\-ratioDrive@{envio\_\-ratioDrive}}
\index{envio\_\-ratioDrive@{envio\_\-ratioDrive}!cliente_lib_original::cliente_lib@{cliente\_\-lib\_\-original::cliente\_\-lib}}
\subsubsection[{envio\_\-ratioDrive}]{\setlength{\rightskip}{0pt plus 5cm}def cliente\_\-lib\_\-original.cliente\_\-lib.envio\_\-ratioDrive (
\begin{DoxyParamCaption}
\item[{}]{self, }
\item[{}]{client, }
\item[{}]{TransRatio, }
\item[{}]{RotRatio, }
\item[{}]{LatRatio}
\end{DoxyParamCaption}
)}}
\label{classcliente__lib__original_1_1cliente__lib_aac50c9462dfe46d1b618692d7206295b}


Sirve para realizar la teleoperacion, mandando los parametros. 


\begin{DoxyParams}{Parámetros}
{\em self} & este parametro no es necesario escribir \\
\hline
{\em client} & Se debe trar el objeto cliente a la definicion para poder utilizar el enlace del cliente para enviar el paquete al servidor \\
\hline
{\em TransRatio} & Velocidad de traslacion \\
\hline
{\em RotRatio} & Velocidad de rotacion \\
\hline
{\em LatRatio} & velocidad lateral para el modelo Pioneer P3-\/DX no se necesario puede ser 0 \\
\hline
\end{DoxyParams}
\begin{DoxyReturn}{Devuelve}
client 
\end{DoxyReturn}


Definición en la línea 135 del archivo cliente\_\-lib\_\-original.py.


\begin{DoxyCode}
136                                                                       :
137                 myTransRatio=TransRatio
138                 myRotRatio=RotRatio
139                 myLatRatio=LatRatio
140                 packet=ArNetPacket()
141                 packet.doubleToBuf(myTransRatio)
142                 packet.doubleToBuf(myRotRatio)
143                 packet.doubleToBuf(50) # use half of the robot's maximum.
144                 packet.doubleToBuf(myLatRatio)
145                 client.requestOnce("ratioDrive", packet)
146                 return client
          
\end{DoxyCode}
\hypertarget{classcliente__lib__original_1_1cliente__lib_a01b5aae2c3ce57fca590b9d06e767f23}{
\index{cliente\_\-lib\_\-original::cliente\_\-lib@{cliente\_\-lib\_\-original::cliente\_\-lib}!lista\_\-sonares@{lista\_\-sonares}}
\index{lista\_\-sonares@{lista\_\-sonares}!cliente_lib_original::cliente_lib@{cliente\_\-lib\_\-original::cliente\_\-lib}}
\subsubsection[{lista\_\-sonares}]{\setlength{\rightskip}{0pt plus 5cm}def cliente\_\-lib\_\-original.cliente\_\-lib.lista\_\-sonares (
\begin{DoxyParamCaption}
\item[{}]{self, }
\item[{}]{packet}
\end{DoxyParamCaption}
)}}
\label{classcliente__lib__original_1_1cliente__lib_a01b5aae2c3ce57fca590b9d06e767f23}


Sirve para leer el paquete arNetPacket con la lista del sonar. 

este comando no es necesario utilizarlo es usado solo por el cliente para procesar el paquete 
\begin{DoxyParams}{Parámetros}
{\em self} & este parametro no es necesario escribir \\
\hline
{\em packet} & este parametro no es necesario escribir \\
\hline
\end{DoxyParams}
\begin{DoxyReturn}{Devuelve}
nada 
\end{DoxyReturn}


Definición en la línea 99 del archivo cliente\_\-lib\_\-original.py.


\begin{DoxyCode}
100                                       :
101                 c="                                   "
102                 numSensor=packet.bufToByte2()
103                 numSensor2=packet.bufToStr(c,15)
104                 print str(numSensor)+" "+str(c.strip())

\end{DoxyCode}
\hypertarget{classcliente__lib__original_1_1cliente__lib_a16a11ee4bc738ae83b652343583ad556}{
\index{cliente\_\-lib\_\-original::cliente\_\-lib@{cliente\_\-lib\_\-original::cliente\_\-lib}!uC\_\-comandos\_\-movi@{uC\_\-comandos\_\-movi}}
\index{uC\_\-comandos\_\-movi@{uC\_\-comandos\_\-movi}!cliente_lib_original::cliente_lib@{cliente\_\-lib\_\-original::cliente\_\-lib}}
\subsubsection[{uC\_\-comandos\_\-movi}]{\setlength{\rightskip}{0pt plus 5cm}def cliente\_\-lib\_\-original.cliente\_\-lib.uC\_\-comandos\_\-movi (
\begin{DoxyParamCaption}
\item[{}]{self, }
\item[{}]{client, }
\item[{}]{comando, }
\item[{}]{parametro}
\end{DoxyParamCaption}
)}}
\label{classcliente__lib__original_1_1cliente__lib_a16a11ee4bc738ae83b652343583ad556}


Sirve para mandar ordenes de movimiento directamente al controlador de la plataforma movil. 


\begin{DoxyParams}{Parámetros}
{\em self} & este parametro no es necesario escribir \\
\hline
{\em client} & Se debe trar el objeto cliente a la definicion para poder utilizar el enlace del cliente para enviar el paquete al servidor \\
\hline
{\em comando} & es un numero de 1-\/255 que representa una funcion esta informacion se puede encontrar en el API de ARIA \\
\hline
{\em parametro} & el parametro de la funcion en caso de no tener se deja el valor en blanco \\
\hline
\end{DoxyParams}
\begin{DoxyReturn}{Devuelve}
client 
\end{DoxyReturn}


Definición en la línea 156 del archivo cliente\_\-lib\_\-original.py.


\begin{DoxyCode}
157                                                            :
158                 mi_comando=comando #comando 8 es MOVE parametro un valor de 5000 
      a -4999 es en mm, 11 LEV y su parametro es velocidad +o- mm/s
159                 mi_parametro=parametro #parametro
160                 packet=ArNetPacket()
161                 packet.strToBuf(mi_comando+" "+mi_parametro)
162                 client.requestOnce("MicroControllerMotionCommand", packet) #Micro
      ControllerMotionCommand
163                 return client

\end{DoxyCode}
\hypertarget{classcliente__lib__original_1_1cliente__lib_a5f97cbead3de2cb78534afec8343c13e}{
\index{cliente\_\-lib\_\-original::cliente\_\-lib@{cliente\_\-lib\_\-original::cliente\_\-lib}!valores@{valores}}
\index{valores@{valores}!cliente_lib_original::cliente_lib@{cliente\_\-lib\_\-original::cliente\_\-lib}}
\subsubsection[{valores}]{\setlength{\rightskip}{0pt plus 5cm}def cliente\_\-lib\_\-original.cliente\_\-lib.valores (
\begin{DoxyParamCaption}
\item[{}]{self, }
\item[{}]{packet}
\end{DoxyParamCaption}
)}}
\label{classcliente__lib__original_1_1cliente__lib_a5f97cbead3de2cb78534afec8343c13e}


Sirve cuando se manda el comando \char`\"{}updateNumbers\char`\"{} en el paquete. 

este comando no es necesario utilizarlo es usado solo por el cliente para procesar el paquete 
\begin{DoxyParams}{Parámetros}
{\em self} & este parametro no es necesario escribir \\
\hline
{\em packet} & el paquete que recibe el cliente del servidor, no es necesario escribir \\
\hline
\end{DoxyParams}
\begin{DoxyReturn}{Devuelve}
Nada, pero guarda en self.valores\_\-fisico \mbox{[}voltaje\_\-bateria,myX,myY,myTh,myVel,myRotVel,myLatVel,myTemperature\mbox{]} 
\end{DoxyReturn}


Definición en la línea 76 del archivo cliente\_\-lib\_\-original.py.


\begin{DoxyCode}
77                                 :
78                 #devuelve los valores voltaje_bateria,myX,myY,myTh,myVel,myRotVel
      ,myLatVel,myTemperature
79                 voltaje_bateria=packet.bufToByte2()/10
80                 myX = packet.bufToByte4()#
81                 myY = packet.bufToByte4()
82                 myTh = packet.bufToByte2()
83                 myVel = packet.bufToByte2()
84                 myRotVel = packet.bufToByte2()
85                 myLatVel = packet.bufToByte2()
86                 myTemperature = packet.bufToByte()
87                 #print "X= "+str(myX)+" y="+str(myY)+" th="+str(myTh)
88                 self.valor_fisico=(voltaje_bateria,myX,myY,myTh,myVel,myRotVel,my
      LatVel,myTemperature)
89                 self.valor_fisico=(voltaje_bateria,myX,myY,myTh,myVel,myRotVel,my
      LatVel,0)
90                 #print valor

\end{DoxyCode}
\hypertarget{classcliente__lib__original_1_1cliente__lib_a2ab0872984bef5af4bfbb7d2ec7f40c5}{
\index{cliente\_\-lib\_\-original::cliente\_\-lib@{cliente\_\-lib\_\-original::cliente\_\-lib}!valores\_\-sonares@{valores\_\-sonares}}
\index{valores\_\-sonares@{valores\_\-sonares}!cliente_lib_original::cliente_lib@{cliente\_\-lib\_\-original::cliente\_\-lib}}
\subsubsection[{valores\_\-sonares}]{\setlength{\rightskip}{0pt plus 5cm}def cliente\_\-lib\_\-original.cliente\_\-lib.valores\_\-sonares (
\begin{DoxyParamCaption}
\item[{}]{self, }
\item[{}]{packet}
\end{DoxyParamCaption}
)}}
\label{classcliente__lib__original_1_1cliente__lib_a2ab0872984bef5af4bfbb7d2ec7f40c5}


Sirve para leer el paquete arNetPacket con los valores del sonar. 

este comando no es necesario utilizarlo es usado solo por el cliente para procesar el paquete 
\begin{DoxyParams}{Parámetros}
{\em self} & este parametro no es necesario escribir \\
\hline
{\em packet} & este parametro no es necesario escribir \\
\hline
\end{DoxyParams}
\begin{DoxyReturn}{Devuelve}
nada 
\end{DoxyReturn}


Definición en la línea 113 del archivo cliente\_\-lib\_\-original.py.


\begin{DoxyCode}
114                                         :
115                 c="                                                              
                                                          "
116                 cantidad=packet.bufToDouble()
117                 dato=range(int(cantidad))
118                 for j in range(int(cantidad)):
119                         dato[j]=c #ya que pasa el buffer a la variable esta debe 
      tener la longitud necesaria y deben ser guardada en cada lectura
120                 for i in range(int(cantidad)):
121                         packet.bufToStr(dato[i],50)
122                         dato[i]=dato[i].replace("\x00"," ") #Porque al final anad
      e un NONE del string
123                         dato[i]=dato[i].strip() 
124                 self.valor_sonares=[cantidad,dato]

\end{DoxyCode}


\subsection{Documentación de los datos miembro}
\hypertarget{classcliente__lib__original_1_1cliente__lib_a049a973bc28127e24c2c717de93daf2f}{
\index{cliente\_\-lib\_\-original::cliente\_\-lib@{cliente\_\-lib\_\-original::cliente\_\-lib}!acu@{acu}}
\index{acu@{acu}!cliente_lib_original::cliente_lib@{cliente\_\-lib\_\-original::cliente\_\-lib}}
\subsubsection[{acu}]{\setlength{\rightskip}{0pt plus 5cm}{\bf cliente\_\-lib\_\-original.cliente\_\-lib.acu}}}
\label{classcliente__lib__original_1_1cliente__lib_a049a973bc28127e24c2c717de93daf2f}


Definición en la línea 58 del archivo cliente\_\-lib\_\-original.py.

\hypertarget{classcliente__lib__original_1_1cliente__lib_a030e232b37138f0a2ba15a3ef5f1fe21}{
\index{cliente\_\-lib\_\-original::cliente\_\-lib@{cliente\_\-lib\_\-original::cliente\_\-lib}!ip@{ip}}
\index{ip@{ip}!cliente_lib_original::cliente_lib@{cliente\_\-lib\_\-original::cliente\_\-lib}}
\subsubsection[{ip}]{\setlength{\rightskip}{0pt plus 5cm}{\bf cliente\_\-lib\_\-original.cliente\_\-lib.ip}}}
\label{classcliente__lib__original_1_1cliente__lib_a030e232b37138f0a2ba15a3ef5f1fe21}


Definición en la línea 58 del archivo cliente\_\-lib\_\-original.py.

\hypertarget{classcliente__lib__original_1_1cliente__lib_ae6a2ed4ac198fc965fbe3ee9a01f9401}{
\index{cliente\_\-lib\_\-original::cliente\_\-lib@{cliente\_\-lib\_\-original::cliente\_\-lib}!t@{t}}
\index{t@{t}!cliente_lib_original::cliente_lib@{cliente\_\-lib\_\-original::cliente\_\-lib}}
\subsubsection[{t}]{\setlength{\rightskip}{0pt plus 5cm}{\bf cliente\_\-lib\_\-original.cliente\_\-lib.t}}}
\label{classcliente__lib__original_1_1cliente__lib_ae6a2ed4ac198fc965fbe3ee9a01f9401}


Definición en la línea 58 del archivo cliente\_\-lib\_\-original.py.

\hypertarget{classcliente__lib__original_1_1cliente__lib_a74cb35b9f3246db8b6b16ac59ae8f320}{
\index{cliente\_\-lib\_\-original::cliente\_\-lib@{cliente\_\-lib\_\-original::cliente\_\-lib}!valor\_\-fisico@{valor\_\-fisico}}
\index{valor\_\-fisico@{valor\_\-fisico}!cliente_lib_original::cliente_lib@{cliente\_\-lib\_\-original::cliente\_\-lib}}
\subsubsection[{valor\_\-fisico}]{\setlength{\rightskip}{0pt plus 5cm}{\bf cliente\_\-lib\_\-original.cliente\_\-lib.valor\_\-fisico}}}
\label{classcliente__lib__original_1_1cliente__lib_a74cb35b9f3246db8b6b16ac59ae8f320}


Definición en la línea 58 del archivo cliente\_\-lib\_\-original.py.

\hypertarget{classcliente__lib__original_1_1cliente__lib_a4172f5914b673eb44e168b4afcbae651}{
\index{cliente\_\-lib\_\-original::cliente\_\-lib@{cliente\_\-lib\_\-original::cliente\_\-lib}!valor\_\-sonares@{valor\_\-sonares}}
\index{valor\_\-sonares@{valor\_\-sonares}!cliente_lib_original::cliente_lib@{cliente\_\-lib\_\-original::cliente\_\-lib}}
\subsubsection[{valor\_\-sonares}]{\setlength{\rightskip}{0pt plus 5cm}{\bf cliente\_\-lib\_\-original.cliente\_\-lib.valor\_\-sonares}}}
\label{classcliente__lib__original_1_1cliente__lib_a4172f5914b673eb44e168b4afcbae651}


Definición en la línea 58 del archivo cliente\_\-lib\_\-original.py.

\hypertarget{classcliente__lib__original_1_1cliente__lib_aa2933583abd7844c57b2ec553cbb46ee}{
\index{cliente\_\-lib\_\-original::cliente\_\-lib@{cliente\_\-lib\_\-original::cliente\_\-lib}!x@{x}}
\index{x@{x}!cliente_lib_original::cliente_lib@{cliente\_\-lib\_\-original::cliente\_\-lib}}
\subsubsection[{x}]{\setlength{\rightskip}{0pt plus 5cm}{\bf cliente\_\-lib\_\-original.cliente\_\-lib.x}}}
\label{classcliente__lib__original_1_1cliente__lib_aa2933583abd7844c57b2ec553cbb46ee}


Definición en la línea 58 del archivo cliente\_\-lib\_\-original.py.

\hypertarget{classcliente__lib__original_1_1cliente__lib_a6050f7c724f8ca063505e8b80a1cdecf}{
\index{cliente\_\-lib\_\-original::cliente\_\-lib@{cliente\_\-lib\_\-original::cliente\_\-lib}!y@{y}}
\index{y@{y}!cliente_lib_original::cliente_lib@{cliente\_\-lib\_\-original::cliente\_\-lib}}
\subsubsection[{y}]{\setlength{\rightskip}{0pt plus 5cm}{\bf cliente\_\-lib\_\-original.cliente\_\-lib.y}}}
\label{classcliente__lib__original_1_1cliente__lib_a6050f7c724f8ca063505e8b80a1cdecf}


Definición en la línea 58 del archivo cliente\_\-lib\_\-original.py.



La documentación para esta clase fue generada a partir del siguiente fichero:\begin{DoxyCompactItemize}
\item 
\hyperlink{cliente__lib__original_8py}{cliente\_\-lib\_\-original.py}\end{DoxyCompactItemize}

\hypertarget{classinicio_1_1prueba__adqui}{
\section{Referencia de la Clase inicio.prueba\_\-adqui}
\label{classinicio_1_1prueba__adqui}\index{inicio::prueba\_\-adqui@{inicio::prueba\_\-adqui}}
}


es la clase encargada del entorno grafico y enlace con \hyperlink{namespacecliente__lib}{cliente\_\-lib}  


\subsection*{M�todos p�blicos}
\begin{DoxyCompactItemize}
\item 
def \hyperlink{classinicio_1_1prueba__adqui_adab3c8bf1d5be6b8523065ccd474765a}{\_\-\_\-init\_\-\_\-}
\begin{DoxyCompactList}\small\item\em para cargar el XML de gtk+ y sus senales \item\end{DoxyCompactList}\item 
def \hyperlink{classinicio_1_1prueba__adqui_a8c6a3aea6c366d774c06bb10590997db}{on\_\-menuitem\_\-quit\_\-activate}
\begin{DoxyCompactList}\small\item\em senal de menu \char`\"{}Salir\char`\"{} y cierra la conexion al salir \item\end{DoxyCompactList}\item 
def \hyperlink{classinicio_1_1prueba__adqui_a7eab73096bb40a159687460826d5bfa7}{on\_\-adqui\_\-datos\_\-destroy}
\begin{DoxyCompactList}\small\item\em cierra la conexion al cerrar la ventana \item\end{DoxyCompactList}\item 
def \hyperlink{classinicio_1_1prueba__adqui_a29aff32a047ab1f8771d47f90fc8efb7}{on\_\-about\_\-response}
\begin{DoxyCompactList}\small\item\em cierra la ventana about \item\end{DoxyCompactList}\item 
def \hyperlink{classinicio_1_1prueba__adqui_ac5b1c3821f39cc144ef5b4f0236817a8}{on\_\-menuitem\_\-abuot\_\-activate}
\begin{DoxyCompactList}\small\item\em senal de menu about \item\end{DoxyCompactList}\item 
def \hyperlink{classinicio_1_1prueba__adqui_a3a6f8fec14c16dbb62dd3cbad2843669}{on\_\-button1\_\-clicked}
\begin{DoxyCompactList}\small\item\em senal de boton para leer datos de la plataforma movil \item\end{DoxyCompactList}\end{DoxyCompactItemize}
\subsection*{Atributos p�blicos}
\begin{DoxyCompactItemize}
\item 
\hyperlink{classinicio_1_1prueba__adqui_a51eaad41d48c56482b7e7dc2af0f352f}{about}
\item 
\hyperlink{classinicio_1_1prueba__adqui_a32103b146bd571e752923638e680db8b}{teleoper}
\item 
\hyperlink{classinicio_1_1prueba__adqui_ac00108c4677fe544c8a187c120369d27}{texto\_\-datos}
\item 
\hyperlink{classinicio_1_1prueba__adqui_a960901fb7d1086379f6c3de6076b7fab}{texto\_\-datos\_\-sonar}
\item 
\hyperlink{classinicio_1_1prueba__adqui_a44a9d5479ccbb7494838db31c4e07233}{boton\_\-foto}
\item 
\hyperlink{classinicio_1_1prueba__adqui_a5a6710e93f733c84b360e42513fdd4a9}{a}
\item 
\hyperlink{classinicio_1_1prueba__adqui_a3519a8ae5deb71530289fc1274500b7e}{cliente}
\end{DoxyCompactItemize}


\subsection{Descripci�n detallada}
es la clase encargada del entorno grafico y enlace con \hyperlink{namespacecliente__lib}{cliente\_\-lib} 

Definici�n en la l�nea 47 del archivo inicio.py.



\subsection{Documentaci�n de las funciones miembro}
\hypertarget{classinicio_1_1prueba__adqui_adab3c8bf1d5be6b8523065ccd474765a}{
\index{inicio::prueba\_\-adqui@{inicio::prueba\_\-adqui}!\_\-\_\-init\_\-\_\-@{\_\-\_\-init\_\-\_\-}}
\index{\_\-\_\-init\_\-\_\-@{\_\-\_\-init\_\-\_\-}!inicio::prueba_adqui@{inicio::prueba\_\-adqui}}
\subsubsection[{\_\-\_\-init\_\-\_\-}]{\setlength{\rightskip}{0pt plus 5cm}def inicio.prueba\_\-adqui.\_\-\_\-init\_\-\_\- (
\begin{DoxyParamCaption}
\item[{}]{ self}
\end{DoxyParamCaption}
)}}
\label{classinicio_1_1prueba__adqui_adab3c8bf1d5be6b8523065ccd474765a}


para cargar el XML de gtk+ y sus senales 


\begin{DoxyParams}{Par�metros}
\item[{\em self}]no se necesita incluirlo al utilizar la funcion ya que se lo pone solo por ser la definicion de una clase \end{DoxyParams}
\begin{DoxyReturn}{Devuelve}
self 
\end{DoxyReturn}


Definici�n en la l�nea 54 del archivo inicio.py.




\begin{DoxyCode}
55                           :
56                 builder = gtk.Builder() #El archivo de glade debe estar en gtkbui
      lder
57                 builder.add_from_file("glade/GUI.glade") #Carga el archivo glade
58                 builder.connect_signals(self) #Toma todas las senales de glade
59                 self.about = builder.get_object("about") #Ventana about
60                 self.teleoper = builder.get_object("adqui_datos") #Ventana princi
      pal
61                 self.texto_datos = builder.get_object("datos") #boton arriba
62                 self.texto_datos_sonar = builder.get_object("datos_sonar") #boton
       arriba
63                 self.boton_foto = builder.get_object("imagen0") #imagen para el d
      ibujo de la plataforma movil
64                 self.teleoper.show()
65                 datos_fisicos=(-11.90,500,4000,-34,15,5,3,14)
66                 mensaje="Voltaje bateria: %f [V] \nX: %d [mm]\nY: %d [mm]\nTh: %d
       [grad]\nvelocidad: %d [mm/s]\nVelocidad de giro: %d [rad/s]\nVelocidad lateral: 
      %d [mm/s]\ntemp: %d \n " % (datos_fisicos)
67                 #[voltaje_bateria,myX,myY,myTh,myVel,myRotVel,myLatVel,myTemperat
      ure]
68                 self.texto_datos.set_text(mensaje)
69                 #Arranca el cliente
70                 self.a=cliente_lib()
71                 #self.a.ip="192.168.1.10"
                self.cliente=self.a.cliente_inicio() 
\end{DoxyCode}


\hypertarget{classinicio_1_1prueba__adqui_a29aff32a047ab1f8771d47f90fc8efb7}{
\index{inicio::prueba\_\-adqui@{inicio::prueba\_\-adqui}!on\_\-about\_\-response@{on\_\-about\_\-response}}
\index{on\_\-about\_\-response@{on\_\-about\_\-response}!inicio::prueba_adqui@{inicio::prueba\_\-adqui}}
\subsubsection[{on\_\-about\_\-response}]{\setlength{\rightskip}{0pt plus 5cm}def inicio.prueba\_\-adqui.on\_\-about\_\-response (
\begin{DoxyParamCaption}
\item[{}]{ self, }
\item[{}]{ widget, }
\item[{}]{ data = {\ttfamily None}}
\end{DoxyParamCaption}
)}}
\label{classinicio_1_1prueba__adqui_a29aff32a047ab1f8771d47f90fc8efb7}


cierra la ventana about 


\begin{DoxyParams}{Par�metros}
\item[{\em self}]no se necesita incluirlo al utilizar la funcion ya que se lo pone solo por ser la definicion de una clase \item[{\em widget}]no se necesita incluirlo al utilizar la funcion ya que se lo pone solo por ser la senal de la interfaz \item[{\em data=None}]este no es necesario incluirlo ya que viene predefinido con valor None \end{DoxyParams}
\begin{DoxyReturn}{Devuelve}
self 
\end{DoxyReturn}


Definici�n en la l�nea 102 del archivo inicio.py.




\begin{DoxyCode}
103                                                       :
                self.about.hide()
\end{DoxyCode}


\hypertarget{classinicio_1_1prueba__adqui_a7eab73096bb40a159687460826d5bfa7}{
\index{inicio::prueba\_\-adqui@{inicio::prueba\_\-adqui}!on\_\-adqui\_\-datos\_\-destroy@{on\_\-adqui\_\-datos\_\-destroy}}
\index{on\_\-adqui\_\-datos\_\-destroy@{on\_\-adqui\_\-datos\_\-destroy}!inicio::prueba_adqui@{inicio::prueba\_\-adqui}}
\subsubsection[{on\_\-adqui\_\-datos\_\-destroy}]{\setlength{\rightskip}{0pt plus 5cm}def inicio.prueba\_\-adqui.on\_\-adqui\_\-datos\_\-destroy (
\begin{DoxyParamCaption}
\item[{}]{ self, }
\item[{}]{ widget, }
\item[{}]{ data = {\ttfamily None}}
\end{DoxyParamCaption}
)}}
\label{classinicio_1_1prueba__adqui_a7eab73096bb40a159687460826d5bfa7}


cierra la conexion al cerrar la ventana 


\begin{DoxyParams}{Par�metros}
\item[{\em self}]no se necesita incluirlo al utilizar la funcion ya que se lo pone solo por ser la definicion de una clase \item[{\em widget}]no se necesita incluirlo al utilizar la funcion ya que se lo pone solo por ser la senal de la interfaz param data=None este no es necesario incluirlo ya que viene predefinido con valor None \end{DoxyParams}
\begin{DoxyReturn}{Devuelve}
self 
\end{DoxyReturn}


Definici�n en la l�nea 91 del archivo inicio.py.




\begin{DoxyCode}
92                                                            :
93                 print self.a.cliente_apaga(self.cliente)
                gtk.main_quit()
\end{DoxyCode}


\hypertarget{classinicio_1_1prueba__adqui_a3a6f8fec14c16dbb62dd3cbad2843669}{
\index{inicio::prueba\_\-adqui@{inicio::prueba\_\-adqui}!on\_\-button1\_\-clicked@{on\_\-button1\_\-clicked}}
\index{on\_\-button1\_\-clicked@{on\_\-button1\_\-clicked}!inicio::prueba_adqui@{inicio::prueba\_\-adqui}}
\subsubsection[{on\_\-button1\_\-clicked}]{\setlength{\rightskip}{0pt plus 5cm}def inicio.prueba\_\-adqui.on\_\-button1\_\-clicked (
\begin{DoxyParamCaption}
\item[{}]{ self, }
\item[{}]{ widget, }
\item[{}]{ data = {\ttfamily None}}
\end{DoxyParamCaption}
)}}
\label{classinicio_1_1prueba__adqui_a3a6f8fec14c16dbb62dd3cbad2843669}


senal de boton para leer datos de la plataforma movil 


\begin{DoxyParams}{Par�metros}
\item[{\em self}]no se necesita incluirlo al utilizar la funcion ya que se lo pone solo por ser la definicion de una clase \item[{\em widget}]no se necesita incluirlo al utilizar la funcion ya que se lo pone solo por ser la senal de la interfaz param data=None este no es necesario incluirlo ya que viene predefinido con valor None \end{DoxyParams}
\begin{DoxyReturn}{Devuelve}
self 
\end{DoxyReturn}


Definici�n en la l�nea 122 del archivo inicio.py.




\begin{DoxyCode}
123                                                        :
124                 self.cliente=self.a.envio_consulta_fisica(self.cliente,"updateNum
      bers")
125                 valor=self.a.devuelve_valorf()
126                 #toca corregir el dato de temp en cliente_lib '\x81' a '0x81'
127                 mensaje="Voltaje bateria: %f [V] \nX: %d [mm]\nY: %d [mm]\nTh: %d
       [grad]\nvelocidad: %d [mm/s]\nVelocidad de giro: %d [rad/s]\nVelocidad lateral: 
      %d [mm/s]\ntemp: %d \n" % (valor)
128                 #[voltaje_bateria,myX,myY,myTh,myVel,myRotVel,myLatVel,myTemperat
      ure]
129                 self.texto_datos.set_text(mensaje)
130                 self.cliente=self.a.envio_consulta_fisica(self.cliente,"pose")
131                 valor=self.a.devuelve_valors()
132                 mensaje=''
133                 for a in range(int(valor[0])): #el primer valor indica cuantos el
      ementos se tiene de sensores
134                         mensaje+="sonar "+str(a)+" "+valor[1][a]+"\n"
135                 self.texto_datos_sonar.set_text(mensaje)
        
\end{DoxyCode}


\hypertarget{classinicio_1_1prueba__adqui_ac5b1c3821f39cc144ef5b4f0236817a8}{
\index{inicio::prueba\_\-adqui@{inicio::prueba\_\-adqui}!on\_\-menuitem\_\-abuot\_\-activate@{on\_\-menuitem\_\-abuot\_\-activate}}
\index{on\_\-menuitem\_\-abuot\_\-activate@{on\_\-menuitem\_\-abuot\_\-activate}!inicio::prueba_adqui@{inicio::prueba\_\-adqui}}
\subsubsection[{on\_\-menuitem\_\-abuot\_\-activate}]{\setlength{\rightskip}{0pt plus 5cm}def inicio.prueba\_\-adqui.on\_\-menuitem\_\-abuot\_\-activate (
\begin{DoxyParamCaption}
\item[{}]{ self, }
\item[{}]{ widget, }
\item[{}]{ data = {\ttfamily None}}
\end{DoxyParamCaption}
)}}
\label{classinicio_1_1prueba__adqui_ac5b1c3821f39cc144ef5b4f0236817a8}


senal de menu about 


\begin{DoxyParams}{Par�metros}
\item[{\em self}]no se necesita incluirlo al utilizar la funcion ya que se lo pone solo por ser la definicion de una clase \item[{\em widget}]no se necesita incluirlo al utilizar la funcion ya que se lo pone solo por ser la senal de la interfaz param data=None este no es necesario incluirlo ya que viene predefinido con valor None \end{DoxyParams}
\begin{DoxyReturn}{Devuelve}
self 
\end{DoxyReturn}


Definici�n en la l�nea 112 del archivo inicio.py.




\begin{DoxyCode}
113                                                                :
                self.about.show() #Muestra la ventana about
\end{DoxyCode}


\hypertarget{classinicio_1_1prueba__adqui_a8c6a3aea6c366d774c06bb10590997db}{
\index{inicio::prueba\_\-adqui@{inicio::prueba\_\-adqui}!on\_\-menuitem\_\-quit\_\-activate@{on\_\-menuitem\_\-quit\_\-activate}}
\index{on\_\-menuitem\_\-quit\_\-activate@{on\_\-menuitem\_\-quit\_\-activate}!inicio::prueba_adqui@{inicio::prueba\_\-adqui}}
\subsubsection[{on\_\-menuitem\_\-quit\_\-activate}]{\setlength{\rightskip}{0pt plus 5cm}def inicio.prueba\_\-adqui.on\_\-menuitem\_\-quit\_\-activate (
\begin{DoxyParamCaption}
\item[{}]{ self, }
\item[{}]{ widget, }
\item[{}]{ data = {\ttfamily None}}
\end{DoxyParamCaption}
)}}
\label{classinicio_1_1prueba__adqui_a8c6a3aea6c366d774c06bb10590997db}


senal de menu \char`\"{}Salir\char`\"{} y cierra la conexion al salir 


\begin{DoxyParams}{Par�metros}
\item[{\em self}]no se necesita incluirlo al utilizar la funcion ya que se lo pone solo por ser la definicion de una clase \item[{\em widget}]no se necesita incluirlo al utilizar la funcion ya que se lo pone solo por ser la senal de la interfaz param data=None este no es necesario incluirlo ya que viene predefinido con valor None \end{DoxyParams}
\begin{DoxyReturn}{Devuelve}
self 
\end{DoxyReturn}


Definici�n en la l�nea 80 del archivo inicio.py.




\begin{DoxyCode}
81                                                               :
82                 print self.a.cliente_apaga(self.cliente)
                gtk.main_quit()
\end{DoxyCode}




\subsection{Documentaci�n de los datos miembro}
\hypertarget{classinicio_1_1prueba__adqui_a5a6710e93f733c84b360e42513fdd4a9}{
\index{inicio::prueba\_\-adqui@{inicio::prueba\_\-adqui}!a@{a}}
\index{a@{a}!inicio::prueba_adqui@{inicio::prueba\_\-adqui}}
\subsubsection[{a}]{\setlength{\rightskip}{0pt plus 5cm}{\bf inicio.prueba\_\-adqui.a}}}
\label{classinicio_1_1prueba__adqui_a5a6710e93f733c84b360e42513fdd4a9}


Definici�n en la l�nea 69 del archivo inicio.py.

\hypertarget{classinicio_1_1prueba__adqui_a51eaad41d48c56482b7e7dc2af0f352f}{
\index{inicio::prueba\_\-adqui@{inicio::prueba\_\-adqui}!about@{about}}
\index{about@{about}!inicio::prueba_adqui@{inicio::prueba\_\-adqui}}
\subsubsection[{about}]{\setlength{\rightskip}{0pt plus 5cm}{\bf inicio.prueba\_\-adqui.about}}}
\label{classinicio_1_1prueba__adqui_a51eaad41d48c56482b7e7dc2af0f352f}


Definici�n en la l�nea 58 del archivo inicio.py.

\hypertarget{classinicio_1_1prueba__adqui_a44a9d5479ccbb7494838db31c4e07233}{
\index{inicio::prueba\_\-adqui@{inicio::prueba\_\-adqui}!boton\_\-foto@{boton\_\-foto}}
\index{boton\_\-foto@{boton\_\-foto}!inicio::prueba_adqui@{inicio::prueba\_\-adqui}}
\subsubsection[{boton\_\-foto}]{\setlength{\rightskip}{0pt plus 5cm}{\bf inicio.prueba\_\-adqui.boton\_\-foto}}}
\label{classinicio_1_1prueba__adqui_a44a9d5479ccbb7494838db31c4e07233}


Definici�n en la l�nea 62 del archivo inicio.py.

\hypertarget{classinicio_1_1prueba__adqui_a3519a8ae5deb71530289fc1274500b7e}{
\index{inicio::prueba\_\-adqui@{inicio::prueba\_\-adqui}!cliente@{cliente}}
\index{cliente@{cliente}!inicio::prueba_adqui@{inicio::prueba\_\-adqui}}
\subsubsection[{cliente}]{\setlength{\rightskip}{0pt plus 5cm}{\bf inicio.prueba\_\-adqui.cliente}}}
\label{classinicio_1_1prueba__adqui_a3519a8ae5deb71530289fc1274500b7e}


Definici�n en la l�nea 71 del archivo inicio.py.

\hypertarget{classinicio_1_1prueba__adqui_a32103b146bd571e752923638e680db8b}{
\index{inicio::prueba\_\-adqui@{inicio::prueba\_\-adqui}!teleoper@{teleoper}}
\index{teleoper@{teleoper}!inicio::prueba_adqui@{inicio::prueba\_\-adqui}}
\subsubsection[{teleoper}]{\setlength{\rightskip}{0pt plus 5cm}{\bf inicio.prueba\_\-adqui.teleoper}}}
\label{classinicio_1_1prueba__adqui_a32103b146bd571e752923638e680db8b}


Definici�n en la l�nea 59 del archivo inicio.py.

\hypertarget{classinicio_1_1prueba__adqui_ac00108c4677fe544c8a187c120369d27}{
\index{inicio::prueba\_\-adqui@{inicio::prueba\_\-adqui}!texto\_\-datos@{texto\_\-datos}}
\index{texto\_\-datos@{texto\_\-datos}!inicio::prueba_adqui@{inicio::prueba\_\-adqui}}
\subsubsection[{texto\_\-datos}]{\setlength{\rightskip}{0pt plus 5cm}{\bf inicio.prueba\_\-adqui.texto\_\-datos}}}
\label{classinicio_1_1prueba__adqui_ac00108c4677fe544c8a187c120369d27}


Definici�n en la l�nea 60 del archivo inicio.py.

\hypertarget{classinicio_1_1prueba__adqui_a960901fb7d1086379f6c3de6076b7fab}{
\index{inicio::prueba\_\-adqui@{inicio::prueba\_\-adqui}!texto\_\-datos\_\-sonar@{texto\_\-datos\_\-sonar}}
\index{texto\_\-datos\_\-sonar@{texto\_\-datos\_\-sonar}!inicio::prueba_adqui@{inicio::prueba\_\-adqui}}
\subsubsection[{texto\_\-datos\_\-sonar}]{\setlength{\rightskip}{0pt plus 5cm}{\bf inicio.prueba\_\-adqui.texto\_\-datos\_\-sonar}}}
\label{classinicio_1_1prueba__adqui_a960901fb7d1086379f6c3de6076b7fab}


Definici�n en la l�nea 61 del archivo inicio.py.



La documentaci�n para esta clase fue generada a partir del siguiente fichero:\begin{DoxyCompactItemize}
\item 
\hyperlink{inicio_8py}{inicio.py}\end{DoxyCompactItemize}

\chapter{Documentación de archivos}
\hypertarget{cliente__lib_8py}{
\section{Referencia del Archivo cliente\_\-lib.py}
\label{cliente__lib_8py}\index{cliente\_\-lib.py@{cliente\_\-lib.py}}
}
\subsection*{Clases}
\begin{DoxyCompactItemize}
\item 
class \hyperlink{classcliente__lib_1_1cliente__lib}{cliente\_\-lib.cliente\_\-lib}
\begin{DoxyCompactList}\small\item\em es la clase encargada del cliente \item\end{DoxyCompactList}\end{DoxyCompactItemize}
\subsection*{Paquetes}
\begin{DoxyCompactItemize}
\item 
package \hyperlink{namespacecliente__lib}{cliente\_\-lib}


\begin{DoxyCompactList}\small\item\em libreria para realizar el cliente \item\end{DoxyCompactList}

\end{DoxyCompactItemize}
\subsection*{Funciones}
\begin{DoxyCompactItemize}
\item 
def \hyperlink{namespacecliente__lib_afb746084e43cb9c21db470d7b4990cae}{cliente\_\-lib.main}
\begin{DoxyCompactList}\small\item\em Sirve para realizar pruebas de conexion. \item\end{DoxyCompactList}\end{DoxyCompactItemize}

\hypertarget{cliente__lib__original_8py}{
\section{Referencia del Archivo /home/badanni/tesis\_\-programas/adqui\_\-datos/cliente\_\-lib\_\-original.py}
\label{cliente__lib__original_8py}\index{/home/badanni/tesis\_\-programas/adqui\_\-datos/cliente\_\-lib\_\-original.py@{/home/badanni/tesis\_\-programas/adqui\_\-datos/cliente\_\-lib\_\-original.py}}
}
\subsection*{Paquetes}
\begin{DoxyCompactItemize}
\item 
package \hyperlink{namespacecliente__lib__original}{cliente\_\-lib\_\-original}
\end{DoxyCompactItemize}

\hypertarget{inicio_8py}{
\section{Referencia del Archivo inicio.py}
\label{inicio_8py}\index{inicio.py@{inicio.py}}
}
\subsection*{Clases}
\begin{DoxyCompactItemize}
\item 
class \hyperlink{classinicio_1_1prueba__adqui}{inicio.prueba\_\-adqui}
\begin{DoxyCompactList}\small\item\em es la clase encargada del entorno grafico y enlace con \hyperlink{namespacecliente__lib}{cliente\_\-lib} \item\end{DoxyCompactList}\end{DoxyCompactItemize}
\subsection*{Paquetes}
\begin{DoxyCompactItemize}
\item 
package \hyperlink{namespaceinicio}{inicio}
\end{DoxyCompactItemize}
\subsection*{Funciones}
\begin{DoxyCompactItemize}
\item 
def \hyperlink{namespaceinicio_a518864d4ff815064f5de420ab3996d94}{inicio.main}
\begin{DoxyCompactList}\small\item\em El encargado al momento de ejcutar la aplicacion de instanciar el objeto prueba\_\-teleoperacion. \item\end{DoxyCompactList}\end{DoxyCompactItemize}

\hypertarget{mainpage_8dox}{
\section{Referencia del Archivo mainpage.dox}
\label{mainpage_8dox}\index{mainpage.dox@{mainpage.dox}}
}

\hypertarget{servidor_8py}{
\section{Referencia del Archivo servidor.py}
\label{servidor_8py}\index{servidor.py@{servidor.py}}
}
\subsection*{Paquetes}
\begin{DoxyCompactItemize}
\item 
package \hyperlink{namespaceservidor}{servidor}


\begin{DoxyCompactList}\small\item\em Servidor para el Pioneer P3-\/DX. \end{DoxyCompactList}

\end{DoxyCompactItemize}
\subsection*{Funciones}
\begin{DoxyCompactItemize}
\item 
def \hyperlink{namespaceservidor_a06ea535cfe56429259d8de76298416cb}{servidor.requestCallback}
\begin{DoxyCompactList}\small\item\em Sirve cuando se manda el comando \char`\"{}test\char`\"{} en el paquete. \end{DoxyCompactList}\item 
def \hyperlink{namespaceservidor_aef5180c02cf1d163167be16732d7250e}{servidor.movimiento}
\begin{DoxyCompactList}\small\item\em Sirve cuando se manda el comando \char`\"{}mover\char`\"{} en el paquete. \end{DoxyCompactList}\item 
def \hyperlink{namespaceservidor_a66205c06d52988bbeb961068e554edfc}{servidor.rotar}
\begin{DoxyCompactList}\small\item\em Sirve cuando se manda el comando \char`\"{}rotar\char`\"{} en el paquete. \end{DoxyCompactList}\item 
def \hyperlink{namespaceservidor_a4f30fc83b9ff1f43b6dde147ed4c31fc}{servidor.posicion}
\begin{DoxyCompactList}\small\item\em Sirve cuando se manda el comando \char`\"{}pose\char`\"{} en el paquete. \end{DoxyCompactList}\end{DoxyCompactItemize}
\subsection*{Variables}
\begin{DoxyCompactItemize}
\item 
tuple \hyperlink{namespaceservidor_a20c40528942a814c3ba639d6fdf80c34}{servidor.robot} = ArRobot()
\item 
tuple \hyperlink{namespaceservidor_a73a7e14af1c6774da1d3c5580e02573c}{servidor.gyro} = ArAnalogGyro(robot)
\item 
tuple \hyperlink{namespaceservidor_a509c8beb9fe73900b5ab2a0fc5f793b2}{servidor.sonarDev} = ArSonarDevice(2)
\item 
tuple \hyperlink{namespaceservidor_adfced13b57fb69c12f8ab5a84d1a2356}{servidor.server} = ArServerBase()
\item 
tuple \hyperlink{namespaceservidor_a159e1bd236b55d72f3fd3cb94afc7650}{servidor.packet} = ArNetPacket()
\item 
tuple \hyperlink{namespaceservidor_ae78d17158a0f45369b3621468e736bf6}{servidor.con} = ArSimpleConnector(sys.argv)
\item 
tuple \hyperlink{namespaceservidor_a214f853047fdc2f26cfac0db1a9e87b8}{servidor.serverInfoRobot} = ArServerInfoRobot(server, robot)
\item 
tuple \hyperlink{namespaceservidor_ae93d5f481f48442f0db6959867fbf8c4}{servidor.serverInfoSensor} = ArServerInfoSensor(server, robot)
\item 
tuple \hyperlink{namespaceservidor_a691fe74e57a668507d56d3acd8e19ba2}{servidor.drawings} = ArServerInfoDrawings(server)
\item 
tuple \hyperlink{namespaceservidor_ae7b0d696eea14b76aba1e014dbfe70f4}{servidor.modeStop} = ArServerModeStop(server, robot)
\item 
tuple \hyperlink{namespaceservidor_ac7cf650e754b329444bd902af26c1527}{servidor.modeRatioDrive} = ArServerModeRatioDrive(server, robot)
\item 
tuple \hyperlink{namespaceservidor_ae22efed138dafc7284a3f71676a6757f}{servidor.modeWander} = ArServerModeWander(server, robot)
\item 
tuple \hyperlink{namespaceservidor_a4bc8244e73099a1e67418ea064c30097}{servidor.commands} = ArServerHandlerCommands(server)
\item 
tuple \hyperlink{namespaceservidor_aa877c09d4a72ca2434441ff2dc406a29}{servidor.uCCommands} = ArServerSimpleComUC(commands, robot)
\item 
tuple \hyperlink{namespaceservidor_a8f73ad33af2f2a4f3bc9209b9390cfe8}{servidor.loggingCommands} = ArServerSimpleComMovementLogging(commands, robot)
\item 
tuple \hyperlink{namespaceservidor_a9bcfd4f54186b9c69cea8188548af8c8}{servidor.gyroCommands} = ArServerSimpleComGyro(commands, robot, gyro)
\item 
tuple \hyperlink{namespaceservidor_aa210848229916c300b66b7f7958baa24}{servidor.configCommands} = ArServerSimpleComLogRobotConfig(commands, robot)
\end{DoxyCompactItemize}

\printindex
\end{document}
