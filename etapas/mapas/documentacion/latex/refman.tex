\documentclass[a4paper]{book}
\usepackage{makeidx}
\usepackage{graphicx}
\usepackage{multicol}
\usepackage{float}
\usepackage{listings}
\usepackage{color}
\usepackage{ifthen}
\usepackage[table]{xcolor}
\usepackage{textcomp}
\usepackage{alltt}
\usepackage{ifpdf}
\ifpdf
\usepackage[pdftex,
            pagebackref=true,
            colorlinks=true,
            linkcolor=blue,
            unicode
           ]{hyperref}
\else
\usepackage[ps2pdf,
            pagebackref=true,
            colorlinks=true,
            linkcolor=blue,
            unicode
           ]{hyperref}
\usepackage{pspicture}
\fi
\usepackage[utf8]{inputenc}
\usepackage[spanish]{babel}
\usepackage{mathptmx}
\usepackage[scaled=.90]{helvet}
\usepackage{courier}
\usepackage{sectsty}
\usepackage[titles]{tocloft}
\usepackage{doxygen}
\lstset{language=C++,inputencoding=utf8,basicstyle=\footnotesize,breaklines=true,breakatwhitespace=true,tabsize=8,numbers=left }
\makeindex
\setcounter{tocdepth}{3}
\renewcommand{\footrulewidth}{0.4pt}
\renewcommand{\familydefault}{\sfdefault}
\begin{document}
\hypersetup{pageanchor=false}
\begin{titlepage}
\vspace*{7cm}
\begin{center}
{\Large Generacion de Mapas \\[1ex]\large 0.1 }\\
\vspace*{1cm}
{\large Generado por Doxygen 1.7.4}\\
\vspace*{0.5cm}
{\small Martes, 7 de Agosto de 2012 11:49:15}\\
\end{center}
\end{titlepage}
\clearemptydoublepage
\pagenumbering{roman}
\tableofcontents
\clearemptydoublepage
\pagenumbering{arabic}
\hypersetup{pageanchor=true}
\chapter{Generacion de Mapas para Pioneer P3-\/DX}
\label{index}\hypertarget{index}{}\input{index}
\chapter{Lista de bugs}
\label{bug}
\hypertarget{bug}{}
\label{bug__bug000001}
\hypertarget{bug__bug000001}{}
 
\begin{DoxyDescription}
\item[Namespace \hyperlink{namespacecliente__lib}{cliente\_\-lib} ]myTemperature tiene falla por el tipo de dato en formato a python '' toca pasar a '0x81' 
\end{DoxyDescription}

\label{bug__bug000002}
\hypertarget{bug__bug000002}{}
 
\begin{DoxyDescription}
\item[Namespace \hyperlink{namespaceservidor}{servidor} ]Nada 
\end{DoxyDescription}
\chapter{Indice de namespaces}
\section{Paquetes}
Aquí van los paquetes con una breve descripción (si etá disponible):\begin{DoxyCompactList}
\item\contentsline{section}{\hyperlink{namespacecliente__lib}{cliente\_\-lib} (Libreria para realizar el cliente )}{\pageref{namespacecliente__lib}}{}
\item\contentsline{section}{\hyperlink{namespacegamepad}{gamepad} }{\pageref{namespacegamepad}}{}
\item\contentsline{section}{\hyperlink{namespaceinicio}{inicio} }{\pageref{namespaceinicio}}{}
\item\contentsline{section}{\hyperlink{namespacerenderizado}{renderizado} (Programa que utiliza la libreria openCV para la generacion del lienzo )}{\pageref{namespacerenderizado}}{}
\item\contentsline{section}{\hyperlink{namespaceservidor__novo}{servidor\_\-novo} (Servidor para el Pioneer P3-\/DX )}{\pageref{namespaceservidor__novo}}{}
\end{DoxyCompactList}

\chapter{Índice de clases}
\section{Lista de clases}
Lista de las clases, estructuras, uniones e interfaces con una breve descripci�n:\begin{DoxyCompactList}
\item\contentsline{section}{\hyperlink{classcliente__lib_1_1cliente__lib}{cliente\_\-lib.cliente\_\-lib} (No dispone ninguna utilidad )}{\pageref{classcliente__lib_1_1cliente__lib}}{}
\item\contentsline{section}{\hyperlink{classinicio_1_1prueba__teleoperacion}{inicio.prueba\_\-teleoperacion} (Se lo crea como objeto para poder trabajar con las senales de la interfaz grafica )}{\pageref{classinicio_1_1prueba__teleoperacion}}{}
\end{DoxyCompactList}

\chapter{Indice de archivos}
\section{Lista de archivos}
Lista de todos los archivos con descripciones breves:\begin{DoxyCompactList}
\item\contentsline{section}{\hyperlink{cliente__lib_8py}{cliente\_\-lib.py} }{\pageref{cliente__lib_8py}}{}
\item\contentsline{section}{\hyperlink{cliente__lib__original_8py}{cliente\_\-lib\_\-original.py} }{\pageref{cliente__lib__original_8py}}{}
\item\contentsline{section}{\hyperlink{inicio_8py}{inicio.py} }{\pageref{inicio_8py}}{}
\item\contentsline{section}{\hyperlink{servidor_8py}{servidor.py} }{\pageref{servidor_8py}}{}
\end{DoxyCompactList}

\chapter{Documentación de namespaces}
\hypertarget{namespacecliente__lib}{
\section{Paquetes cliente\_\-lib}
\label{namespacecliente__lib}\index{cliente\_\-lib@{cliente\_\-lib}}
}


libreria para realizar el cliente  


\subsection*{Clases}
\begin{DoxyCompactItemize}
\item 
class \hyperlink{classcliente__lib_1_1cliente__lib}{cliente\_\-lib}
\begin{DoxyCompactList}\small\item\em es la clase encargada del cliente \item\end{DoxyCompactList}\end{DoxyCompactItemize}
\subsection*{Funciones}
\begin{DoxyCompactItemize}
\item 
def \hyperlink{namespacecliente__lib_afb746084e43cb9c21db470d7b4990cae}{main}
\begin{DoxyCompactList}\small\item\em Sirve para realizar pruebas de conexion. \item\end{DoxyCompactList}\end{DoxyCompactItemize}


\subsection{Descripci�n detallada}
libreria para realizar el cliente Se debe especificar cual es el IP del servidor \begin{DoxyAuthor}{Autores}
Danny Vasconez 

Daniel Granda 
\end{DoxyAuthor}
\begin{DoxyVersion}{Versi�n}
0.0.2 
\end{DoxyVersion}
\begin{DoxyDate}{Fecha}
2012 
\end{DoxyDate}
\begin{DoxyPrecond}{Precondici�n}
Tener funcionando el servidor 
\end{DoxyPrecond}
\begin{Desc}
\item[\hyperlink{bug__bug000001}{Bug}]myTemperature tiene falla por el tipo de dato en formato a python '' toca pasar a '0x81' \end{Desc}
\begin{DoxyWarning}{Atenci�n}
uso inapropiado puede hacer que la aplicacion falle
\end{DoxyWarning}
\hypertarget{index_intro}{}\subsection{Ejemplo de uso}\label{index_intro}
En el ejemplo se muestra tres maneras de enviar comandos la general que es requestOnce y las otras que son la misma pero modificada para trabajar con comandos especificos 
\begin{DoxyVerbInclude}
a=cliente_lib() #Instancia a la clase cliete_lib
a.ip="192.168.1.10"
CLIENTE=a.cliente_inicio() 
# para realizar movimiento 
TransRatio,RotRatio,LatRatio = [-50,0,0]
CLIENTE=a.envio_ratioDrive(CLIENTE,TransRatio,RotRatio,LatRatio) #fijar los valores para mover
#
#para conocer valores como fisicos de la plataforma movil
CLIENTE=a.envio_consulta_fisica(CLIENTE,"updateNumbers")
valor=a.devuelve_valorf()
print valor
#
#para conocer los valores de los sonares
CLIENTE.requestOnce("pose")
valor=a.devuelve_valors()
print valor
#
ArUtil.sleep(1000)
a.cliente_apaga(CLIENTE)

\end{DoxyVerbInclude}
 

\subsection{Documentaci�n de las funciones}
\hypertarget{namespacecliente__lib_afb746084e43cb9c21db470d7b4990cae}{
\index{cliente\_\-lib@{cliente\_\-lib}!main@{main}}
\index{main@{main}!cliente_lib@{cliente\_\-lib}}
\subsubsection[{main}]{\setlength{\rightskip}{0pt plus 5cm}def cliente\_\-lib.main (
\begin{DoxyParamCaption}
{}
\end{DoxyParamCaption}
)}}
\label{namespacecliente__lib_afb746084e43cb9c21db470d7b4990cae}


Sirve para realizar pruebas de conexion. 

sin tener que ejecutar la aplicacion completa; de la siguiente forma \char`\"{}python2.5 cliente\_\-lib.py\char`\"{} \begin{DoxyReturn}{Devuelve}
0 
\end{DoxyReturn}


Definici�n en la l�nea 236 del archivo cliente\_\-lib.py.




\begin{DoxyCode}
237           :
238         #prueba de la libreria
239         a=cliente_lib()
240         a.ip="192.168.1.10" #Si el servidor esta en otra maquina 
241         CLIENTE=a.cliente_inicio()
242         TransRatio,RotRatio,LatRatio = [-50,0,0]
243         CLIENTE=a.envio_ratioDrive(CLIENTE,TransRatio,RotRatio,LatRatio) #fijar l
      os valores para mover
244         CLIENTE=a.envio_consulta_fisica(CLIENTE,"updateNumbers")
245         valor=a.devuelve_valorf()
246         print valor
247         CLIENTE.requestOnce("pose")
248         ArUtil.sleep(1000)
249         a.cliente_apaga(CLIENTE)
        return 0
\end{DoxyCode}



\hypertarget{namespacegamepad}{
\section{Paquetes gamepad}
\label{namespacegamepad}\index{gamepad@{gamepad}}
}
\subsection*{Clases}
\begin{DoxyCompactItemize}
\item 
class \hyperlink{classgamepad_1_1gamepad}{gamepad}
\end{DoxyCompactItemize}
\subsection*{Variables}
\begin{DoxyCompactItemize}
\item 
tuple \hyperlink{namespacegamepad_ac2d3a197e612c4d688cc70b717d040f4}{c} = \hyperlink{classgamepad_1_1gamepad}{gamepad}()
\item 
tuple \hyperlink{namespacegamepad_a71ea7805d13c1a442407730008cf5247}{a} = c.gamepad\_\-lectura(pipe,msg)
\end{DoxyCompactItemize}


\subsection{Documentación de las variables}
\hypertarget{namespacegamepad_a71ea7805d13c1a442407730008cf5247}{
\index{gamepad@{gamepad}!a@{a}}
\index{a@{a}!gamepad@{gamepad}}
\subsubsection[{a}]{\setlength{\rightskip}{0pt plus 5cm}tuple {\bf gamepad.a} = c.gamepad\_\-lectura(pipe,msg)}}
\label{namespacegamepad_a71ea7805d13c1a442407730008cf5247}


Definición en la línea 99 del archivo gamepad.py.

\hypertarget{namespacegamepad_ac2d3a197e612c4d688cc70b717d040f4}{
\index{gamepad@{gamepad}!c@{c}}
\index{c@{c}!gamepad@{gamepad}}
\subsubsection[{c}]{\setlength{\rightskip}{0pt plus 5cm}tuple {\bf gamepad.c} = {\bf gamepad}()}}
\label{namespacegamepad_ac2d3a197e612c4d688cc70b717d040f4}


Definición en la línea 96 del archivo gamepad.py.


\hypertarget{namespaceinicio}{
\section{Paquetes inicio}
\label{namespaceinicio}\index{inicio@{inicio}}
}
\subsection*{Clases}
\begin{DoxyCompactItemize}
\item 
class \hyperlink{classinicio_1_1prueba__adqui}{prueba\_\-adqui}
\begin{DoxyCompactList}\small\item\em es la clase encargada del entorno grafico y enlace con \hyperlink{namespacecliente__lib}{cliente\_\-lib} \end{DoxyCompactList}\end{DoxyCompactItemize}
\subsection*{Funciones}
\begin{DoxyCompactItemize}
\item 
def \hyperlink{namespaceinicio_a518864d4ff815064f5de420ab3996d94}{main}
\begin{DoxyCompactList}\small\item\em El encargado al momento de ejcutar la aplicacion de instanciar el objeto prueba\_\-teleoperacion. \end{DoxyCompactList}\end{DoxyCompactItemize}


\subsection{Documentación de las funciones}
\hypertarget{namespaceinicio_a518864d4ff815064f5de420ab3996d94}{
\index{inicio@{inicio}!main@{main}}
\index{main@{main}!inicio@{inicio}}
\subsubsection[{main}]{\setlength{\rightskip}{0pt plus 5cm}def inicio.main (
\begin{DoxyParamCaption}
{}
\end{DoxyParamCaption}
)}}
\label{namespaceinicio_a518864d4ff815064f5de420ab3996d94}


El encargado al momento de ejcutar la aplicacion de instanciar el objeto prueba\_\-teleoperacion. 



Definición en la línea 267 del archivo inicio.py.


\begin{DoxyCode}
268           :
269         if os.name=="posix": #verifica que sea un entorno Linux
270                 dimensiones=(600,400)
271                 nombre_archivo="filenamea.jpg"
272                 dimensiones_robot=(13,13)
273                 app = prueba_adqui(dimensiones,nombre_archivo,dimensiones_robot) 
      #instancia el objeto GUI con las senales enlazadas
274                 gtk.main() #levanta el motor GTK para poder visualizar
275   

\end{DoxyCode}

\hypertarget{namespacerenderizado}{
\section{Paquetes renderizado}
\label{namespacerenderizado}\index{renderizado@{renderizado}}
}


programa que utiliza la libreria openCV para la generacion del lienzo  


\subsection*{Clases}
\begin{DoxyCompactItemize}
\item 
class \hyperlink{classrenderizado_1_1renderizado}{renderizado}
\begin{DoxyCompactList}\small\item\em es la clase encargada de la generacion del mapa atravez de comandos. \end{DoxyCompactList}\end{DoxyCompactItemize}
\subsection*{Variables}
\begin{DoxyCompactItemize}
\item 
tuple \hyperlink{namespacerenderizado_a67d5237e7cc5eff92a6e3b9d4782723e}{a} = \hyperlink{classrenderizado_1_1renderizado}{renderizado}((600,400),\char`\"{}imagen.jpg\char`\"{},(13,13))
\end{DoxyCompactItemize}


\subsection{Descripción detallada}
programa que utiliza la libreria openCV para la generacion del lienzo Programa para realizar el mapa con la informacion del sonar y odometria en la plataforma movil. \begin{DoxyAuthor}{Autores}
Danny Vasconez 

Daniel Granda 
\end{DoxyAuthor}
\begin{DoxyVersion}{Versión}
0.0.1 
\end{DoxyVersion}
\begin{DoxyDate}{Fecha}
2012 
\end{DoxyDate}
\begin{DoxyPrecond}{Precondición}
Ningun requisito 
\end{DoxyPrecond}
\begin{Desc}
\item[\hyperlink{bug__bug000002}{Bug}]Si la posicion del robot se sale del lienzo proboca un error en la matriz de la imagen. \end{Desc}
\begin{DoxyWarning}{Atención}
uso inapropiado puede hacer que la aplicacion falle
\end{DoxyWarning}
\hypertarget{index_intro}{}\subsection{Ejemplo de uso}\label{index_intro}
En el ejemplo se muestra como se debe utilizar este paquete 
\begin{DoxyVerbInclude}
a=renderizado((600,400),"imagen.jpg",(13,13))
a.cargar()
a.anadir_punto((5,51),radio=13)
a.graficar(500)
a.crear_imagen()
a.rotacion_y_posicion_robot(200,100,0)
a.crear_imagen()
a.rotacion_y_posicion_robot(20,300,-40)


\end{DoxyVerbInclude}
 

\subsection{Documentación de las variables}
\hypertarget{namespacerenderizado_a67d5237e7cc5eff92a6e3b9d4782723e}{
\index{renderizado@{renderizado}!a@{a}}
\index{a@{a}!renderizado@{renderizado}}
\subsubsection[{a}]{\setlength{\rightskip}{0pt plus 5cm}tuple {\bf renderizado.a} = {\bf renderizado}((600,400),\char`\"{}imagen.jpg\char`\"{},(13,13))}}
\label{namespacerenderizado_a67d5237e7cc5eff92a6e3b9d4782723e}


Definición en la línea 154 del archivo renderizado.py.


\hypertarget{namespaceservidor__novo}{
\section{Paquetes servidor\_\-novo}
\label{namespaceservidor__novo}\index{servidor\_\-novo@{servidor\_\-novo}}
}


Servidor para el Pioneer P3-\/DX.  


\subsection*{Funciones}
\begin{DoxyCompactItemize}
\item 
def \hyperlink{namespaceservidor__novo_a994b4e4dc22448a609ecd86baa9ff6da}{requestCallback}
\begin{DoxyCompactList}\small\item\em Sirve cuando se manda el comando \char`\"{}test\char`\"{} en el paquete. \end{DoxyCompactList}\item 
def \hyperlink{namespaceservidor__novo_ae02aab4783a9544975ff25e7d2fde80a}{movimiento}
\begin{DoxyCompactList}\small\item\em Sirve cuando se manda el comando \char`\"{}mover\char`\"{} en el paquete. \end{DoxyCompactList}\item 
def \hyperlink{namespaceservidor__novo_a8af363dfe669e9277832c4491d253db3}{rotar}
\begin{DoxyCompactList}\small\item\em Sirve cuando se manda el comando \char`\"{}rotar\char`\"{} en el paquete. \end{DoxyCompactList}\item 
def \hyperlink{namespaceservidor__novo_a6bb7f16cf80ec7b45c715da84a62257d}{posicion}
\begin{DoxyCompactList}\small\item\em Sirve cuando se manda el comando \char`\"{}pose\char`\"{} en el paquete. \end{DoxyCompactList}\end{DoxyCompactItemize}
\subsection*{Variables}
\begin{DoxyCompactItemize}
\item 
tuple \hyperlink{namespaceservidor__novo_ae604d92f7f5c43d2a993ae9402534424}{robot} = ArRobot()
\item 
tuple \hyperlink{namespaceservidor__novo_a2b0e3a19479a3a456184ce1f64a648a0}{gyro} = ArAnalogGyro(\hyperlink{namespaceservidor__novo_ae604d92f7f5c43d2a993ae9402534424}{robot})
\item 
tuple \hyperlink{namespaceservidor__novo_a966ece9821b12a79fc5b83f628987f7d}{sonarDev} = ArSonarDevice(7)
\item 
tuple \hyperlink{namespaceservidor__novo_a538b68fbb948f9cf0573a119042c9e0d}{server} = ArServerBase()
\item 
tuple \hyperlink{namespaceservidor__novo_a4644b693a5931c2f1a9d68500534d955}{packet} = ArNetPacket()
\item 
tuple \hyperlink{namespaceservidor__novo_a201a5b59bc2f5a75d2a4c82d81f17180}{con} = ArSimpleConnector(sys.argv)
\item 
tuple \hyperlink{namespaceservidor__novo_aae2aad1feed83599e4783f4d39086ee6}{serverInfoRobot} = ArServerInfoRobot(\hyperlink{namespaceservidor__novo_a538b68fbb948f9cf0573a119042c9e0d}{server}, \hyperlink{namespaceservidor__novo_ae604d92f7f5c43d2a993ae9402534424}{robot})
\item 
tuple \hyperlink{namespaceservidor__novo_a31795a5817e65297017799e1fdbc6fdd}{serverInfoSensor} = ArServerInfoSensor(\hyperlink{namespaceservidor__novo_a538b68fbb948f9cf0573a119042c9e0d}{server}, \hyperlink{namespaceservidor__novo_ae604d92f7f5c43d2a993ae9402534424}{robot})
\item 
tuple \hyperlink{namespaceservidor__novo_aaddb28ac7dd14daa90eaadd8fde37bbd}{drawings} = ArServerInfoDrawings(\hyperlink{namespaceservidor__novo_a538b68fbb948f9cf0573a119042c9e0d}{server})
\item 
tuple \hyperlink{namespaceservidor__novo_a63f8dd02ccddbc7be4de022c23246578}{modeStop} = ArServerModeStop(\hyperlink{namespaceservidor__novo_a538b68fbb948f9cf0573a119042c9e0d}{server}, \hyperlink{namespaceservidor__novo_ae604d92f7f5c43d2a993ae9402534424}{robot})
\item 
tuple \hyperlink{namespaceservidor__novo_a17ac782014a1ecc8ae25509094a6812b}{modeRatioDrive} = ArServerModeRatioDrive(\hyperlink{namespaceservidor__novo_a538b68fbb948f9cf0573a119042c9e0d}{server}, \hyperlink{namespaceservidor__novo_ae604d92f7f5c43d2a993ae9402534424}{robot})
\item 
tuple \hyperlink{namespaceservidor__novo_a1f0f40c9f2b17d9579e189b4ec8f4ffe}{modeWander} = ArServerModeWander(\hyperlink{namespaceservidor__novo_a538b68fbb948f9cf0573a119042c9e0d}{server}, \hyperlink{namespaceservidor__novo_ae604d92f7f5c43d2a993ae9402534424}{robot})
\item 
tuple \hyperlink{namespaceservidor__novo_a1cd792305e65be0c6e06c0cd885b1593}{commands} = ArServerHandlerCommands(\hyperlink{namespaceservidor__novo_a538b68fbb948f9cf0573a119042c9e0d}{server})
\item 
tuple \hyperlink{namespaceservidor__novo_a9cc0dcde8396c1ff04833e8811a1df74}{uCCommands} = ArServerSimpleComUC(\hyperlink{namespaceservidor__novo_a1cd792305e65be0c6e06c0cd885b1593}{commands}, \hyperlink{namespaceservidor__novo_ae604d92f7f5c43d2a993ae9402534424}{robot})
\item 
tuple \hyperlink{namespaceservidor__novo_a0407c97de55cf4fa7aac903f2813aa50}{loggingCommands} = ArServerSimpleComMovementLogging(\hyperlink{namespaceservidor__novo_a1cd792305e65be0c6e06c0cd885b1593}{commands}, \hyperlink{namespaceservidor__novo_ae604d92f7f5c43d2a993ae9402534424}{robot})
\item 
tuple \hyperlink{namespaceservidor__novo_a20fff7251fe63ba8161a62b254023d04}{gyroCommands} = ArServerSimpleComGyro(\hyperlink{namespaceservidor__novo_a1cd792305e65be0c6e06c0cd885b1593}{commands}, \hyperlink{namespaceservidor__novo_ae604d92f7f5c43d2a993ae9402534424}{robot}, \hyperlink{namespaceservidor__novo_a2b0e3a19479a3a456184ce1f64a648a0}{gyro})
\item 
tuple \hyperlink{namespaceservidor__novo_a0b77e4ad4ff622636ae781f9e76c0bba}{configCommands} = ArServerSimpleComLogRobotConfig(\hyperlink{namespaceservidor__novo_a1cd792305e65be0c6e06c0cd885b1593}{commands}, \hyperlink{namespaceservidor__novo_ae604d92f7f5c43d2a993ae9402534424}{robot})
\end{DoxyCompactItemize}


\subsection{Descripción detallada}
Servidor para el Pioneer P3-\/DX. Se lo puede utilizar con multiples conexiones de clientes trabaja en el puerto 7272 La lista de comandos para el paquete ArNetPacket se encuentra en el anexo 1 \begin{DoxyAuthor}{Autores}
Danny Vasconez 

Daniel Granda 
\end{DoxyAuthor}
\begin{DoxyVersion}{Versión}
0.0.2 
\end{DoxyVersion}
\begin{DoxyDate}{Fecha}
2012 
\end{DoxyDate}
\begin{DoxyPrecond}{Precondición}
Tener conectada la plataforma Pioneer P3-\/DX. 
\end{DoxyPrecond}
\begin{Desc}
\item[\hyperlink{bug__bug000003}{Bug}]Nada \end{Desc}
\begin{DoxyWarning}{Atención}
uso inapropiado puede hacer que la aplicacion falle 
\end{DoxyWarning}


\subsection{Documentación de las funciones}
\hypertarget{namespaceservidor__novo_ae02aab4783a9544975ff25e7d2fde80a}{
\index{servidor\_\-novo@{servidor\_\-novo}!movimiento@{movimiento}}
\index{movimiento@{movimiento}!servidor_novo@{servidor\_\-novo}}
\subsubsection[{movimiento}]{\setlength{\rightskip}{0pt plus 5cm}def servidor\_\-novo.movimiento (
\begin{DoxyParamCaption}
\item[{}]{client, }
\item[{}]{packet}
\end{DoxyParamCaption}
)}}
\label{namespaceservidor__novo_ae02aab4783a9544975ff25e7d2fde80a}


Sirve cuando se manda el comando \char`\"{}mover\char`\"{} en el paquete. 


\begin{DoxyParams}{Parámetros}
{\em client} & El cliente que manda el paquete \\
\hline
{\em packet} & el paquete que recibe el servidor para el comando \\
\hline
\end{DoxyParams}
\begin{DoxyReturn}{Devuelve}
Nada 
\end{DoxyReturn}


Definición en la línea 62 del archivo servidor\_\-novo.py.


\begin{DoxyCode}
63                              :
64   robot.lock()
65   robot.comInt(8,5000) #move(5000) para atras move(-4999)
  robot.unlock()
\end{DoxyCode}
\hypertarget{namespaceservidor__novo_a6bb7f16cf80ec7b45c715da84a62257d}{
\index{servidor\_\-novo@{servidor\_\-novo}!posicion@{posicion}}
\index{posicion@{posicion}!servidor_novo@{servidor\_\-novo}}
\subsubsection[{posicion}]{\setlength{\rightskip}{0pt plus 5cm}def servidor\_\-novo.posicion (
\begin{DoxyParamCaption}
\item[{}]{client, }
\item[{}]{packet}
\end{DoxyParamCaption}
)}}
\label{namespaceservidor__novo_a6bb7f16cf80ec7b45c715da84a62257d}


Sirve cuando se manda el comando \char`\"{}pose\char`\"{} en el paquete. 


\begin{DoxyParams}{Parámetros}
{\em client} & El cliente que manda el paquete \\
\hline
{\em packet} & el paquete que recibe el servidor para el comando \\
\hline
\end{DoxyParams}
\begin{DoxyReturn}{Devuelve}
ArNetPacket con la informacion en coordenadas X,Y,T al cliente 
\end{DoxyReturn}


Definición en la línea 84 del archivo servidor\_\-novo.py.


\begin{DoxyCode}
85                            :
86   robot.lock()
87   poses = sonarDev.getCurrentBufferAsVector()
88   packet=ArNetPacket()
89   packet.doubleToBuf(len(poses))
90   for p in poses:
91     print p
92     valor=str(p).strip('(').strip(')').strip('X:')
93     valor=valor.split(",")
94     valor_x=valor[0]
95     valor_y=valor[1].strip().strip('Y:')
96     packet.byte4ToBuf(int(float(valor_x)))
97     packet.byte4ToBuf(int(float(valor_y)))
98   packet.finalizePacket()
99   print packet.verifyCheckSum()
100   client.sendPacketTcp(packet)
101   robot.unlock()
102 
103 # This example demonstrates how to use ArNetworking in Python. 
104 
105 # Global library initialization, just like the C++ API:
106 Aria.init()
107 
# Create a robot object:
\end{DoxyCode}
\hypertarget{namespaceservidor__novo_a994b4e4dc22448a609ecd86baa9ff6da}{
\index{servidor\_\-novo@{servidor\_\-novo}!requestCallback@{requestCallback}}
\index{requestCallback@{requestCallback}!servidor_novo@{servidor\_\-novo}}
\subsubsection[{requestCallback}]{\setlength{\rightskip}{0pt plus 5cm}def servidor\_\-novo.requestCallback (
\begin{DoxyParamCaption}
\item[{}]{client, }
\item[{}]{packet}
\end{DoxyParamCaption}
)}}
\label{namespaceservidor__novo_a994b4e4dc22448a609ecd86baa9ff6da}


Sirve cuando se manda el comando \char`\"{}test\char`\"{} en el paquete. 


\begin{DoxyParams}{Parámetros}
{\em client} & El cliente que manda el paquete \\
\hline
{\em packet} & el paquete que recibe el servidor para el comando \\
\hline
\end{DoxyParams}
\begin{DoxyReturn}{Devuelve}
Nada 
\end{DoxyReturn}


Definición en la línea 50 del archivo servidor\_\-novo.py.


\begin{DoxyCode}
51                                    :
52   replyPacket = ArNetPacket()
53   replyPacket.strToBuf(str(robot.getPose().x));
54   print "requestCallback received a packet with command #%d. Sending a reply...\n
      " % (packet.getCommand())
  client.sendPacketTcp(replyPacket)
\end{DoxyCode}
\hypertarget{namespaceservidor__novo_a8af363dfe669e9277832c4491d253db3}{
\index{servidor\_\-novo@{servidor\_\-novo}!rotar@{rotar}}
\index{rotar@{rotar}!servidor_novo@{servidor\_\-novo}}
\subsubsection[{rotar}]{\setlength{\rightskip}{0pt plus 5cm}def servidor\_\-novo.rotar (
\begin{DoxyParamCaption}
\item[{}]{client, }
\item[{}]{packet}
\end{DoxyParamCaption}
)}}
\label{namespaceservidor__novo_a8af363dfe669e9277832c4491d253db3}


Sirve cuando se manda el comando \char`\"{}rotar\char`\"{} en el paquete. 


\begin{DoxyParams}{Parámetros}
{\em client} & El cliente que manda el paquete \\
\hline
{\em packet} & el paquete que recibe el servidor para el comando \\
\hline
\end{DoxyParams}
\begin{DoxyReturn}{Devuelve}
Nada 
\end{DoxyReturn}


Definición en la línea 73 del archivo servidor\_\-novo.py.


\begin{DoxyCode}
74                         :
75   robot.lock()
76   robot.comInt(12,50) 
  robot.unlock()
\end{DoxyCode}


\subsection{Documentación de las variables}
\hypertarget{namespaceservidor__novo_a1cd792305e65be0c6e06c0cd885b1593}{
\index{servidor\_\-novo@{servidor\_\-novo}!commands@{commands}}
\index{commands@{commands}!servidor_novo@{servidor\_\-novo}}
\subsubsection[{commands}]{\setlength{\rightskip}{0pt plus 5cm}tuple {\bf servidor\_\-novo.commands} = ArServerHandlerCommands({\bf server})}}
\label{namespaceservidor__novo_a1cd792305e65be0c6e06c0cd885b1593}


Definición en la línea 162 del archivo servidor\_\-novo.py.

\hypertarget{namespaceservidor__novo_a201a5b59bc2f5a75d2a4c82d81f17180}{
\index{servidor\_\-novo@{servidor\_\-novo}!con@{con}}
\index{con@{con}!servidor_novo@{servidor\_\-novo}}
\subsubsection[{con}]{\setlength{\rightskip}{0pt plus 5cm}tuple {\bf servidor\_\-novo.con} = ArSimpleConnector(sys.argv)}}
\label{namespaceservidor__novo_a201a5b59bc2f5a75d2a4c82d81f17180}


Definición en la línea 134 del archivo servidor\_\-novo.py.

\hypertarget{namespaceservidor__novo_a0b77e4ad4ff622636ae781f9e76c0bba}{
\index{servidor\_\-novo@{servidor\_\-novo}!configCommands@{configCommands}}
\index{configCommands@{configCommands}!servidor_novo@{servidor\_\-novo}}
\subsubsection[{configCommands}]{\setlength{\rightskip}{0pt plus 5cm}tuple {\bf servidor\_\-novo.configCommands} = ArServerSimpleComLogRobotConfig({\bf commands}, {\bf robot})}}
\label{namespaceservidor__novo_a0b77e4ad4ff622636ae781f9e76c0bba}


Definición en la línea 170 del archivo servidor\_\-novo.py.

\hypertarget{namespaceservidor__novo_aaddb28ac7dd14daa90eaadd8fde37bbd}{
\index{servidor\_\-novo@{servidor\_\-novo}!drawings@{drawings}}
\index{drawings@{drawings}!servidor_novo@{servidor\_\-novo}}
\subsubsection[{drawings}]{\setlength{\rightskip}{0pt plus 5cm}tuple {\bf servidor\_\-novo.drawings} = ArServerInfoDrawings({\bf server})}}
\label{namespaceservidor__novo_aaddb28ac7dd14daa90eaadd8fde37bbd}


Definición en la línea 151 del archivo servidor\_\-novo.py.

\hypertarget{namespaceservidor__novo_a2b0e3a19479a3a456184ce1f64a648a0}{
\index{servidor\_\-novo@{servidor\_\-novo}!gyro@{gyro}}
\index{gyro@{gyro}!servidor_novo@{servidor\_\-novo}}
\subsubsection[{gyro}]{\setlength{\rightskip}{0pt plus 5cm}tuple {\bf servidor\_\-novo.gyro} = ArAnalogGyro({\bf robot})}}
\label{namespaceservidor__novo_a2b0e3a19479a3a456184ce1f64a648a0}


Definición en la línea 111 del archivo servidor\_\-novo.py.

\hypertarget{namespaceservidor__novo_a20fff7251fe63ba8161a62b254023d04}{
\index{servidor\_\-novo@{servidor\_\-novo}!gyroCommands@{gyroCommands}}
\index{gyroCommands@{gyroCommands}!servidor_novo@{servidor\_\-novo}}
\subsubsection[{gyroCommands}]{\setlength{\rightskip}{0pt plus 5cm}tuple {\bf servidor\_\-novo.gyroCommands} = ArServerSimpleComGyro({\bf commands}, {\bf robot}, {\bf gyro})}}
\label{namespaceservidor__novo_a20fff7251fe63ba8161a62b254023d04}


Definición en la línea 168 del archivo servidor\_\-novo.py.

\hypertarget{namespaceservidor__novo_a0407c97de55cf4fa7aac903f2813aa50}{
\index{servidor\_\-novo@{servidor\_\-novo}!loggingCommands@{loggingCommands}}
\index{loggingCommands@{loggingCommands}!servidor_novo@{servidor\_\-novo}}
\subsubsection[{loggingCommands}]{\setlength{\rightskip}{0pt plus 5cm}tuple {\bf servidor\_\-novo.loggingCommands} = ArServerSimpleComMovementLogging({\bf commands}, {\bf robot})}}
\label{namespaceservidor__novo_a0407c97de55cf4fa7aac903f2813aa50}


Definición en la línea 166 del archivo servidor\_\-novo.py.

\hypertarget{namespaceservidor__novo_a17ac782014a1ecc8ae25509094a6812b}{
\index{servidor\_\-novo@{servidor\_\-novo}!modeRatioDrive@{modeRatioDrive}}
\index{modeRatioDrive@{modeRatioDrive}!servidor_novo@{servidor\_\-novo}}
\subsubsection[{modeRatioDrive}]{\setlength{\rightskip}{0pt plus 5cm}tuple {\bf servidor\_\-novo.modeRatioDrive} = ArServerModeRatioDrive({\bf server}, {\bf robot})}}
\label{namespaceservidor__novo_a17ac782014a1ecc8ae25509094a6812b}


Definición en la línea 156 del archivo servidor\_\-novo.py.

\hypertarget{namespaceservidor__novo_a63f8dd02ccddbc7be4de022c23246578}{
\index{servidor\_\-novo@{servidor\_\-novo}!modeStop@{modeStop}}
\index{modeStop@{modeStop}!servidor_novo@{servidor\_\-novo}}
\subsubsection[{modeStop}]{\setlength{\rightskip}{0pt plus 5cm}tuple {\bf servidor\_\-novo.modeStop} = ArServerModeStop({\bf server}, {\bf robot})}}
\label{namespaceservidor__novo_a63f8dd02ccddbc7be4de022c23246578}


Definición en la línea 155 del archivo servidor\_\-novo.py.

\hypertarget{namespaceservidor__novo_a1f0f40c9f2b17d9579e189b4ec8f4ffe}{
\index{servidor\_\-novo@{servidor\_\-novo}!modeWander@{modeWander}}
\index{modeWander@{modeWander}!servidor_novo@{servidor\_\-novo}}
\subsubsection[{modeWander}]{\setlength{\rightskip}{0pt plus 5cm}tuple {\bf servidor\_\-novo.modeWander} = ArServerModeWander({\bf server}, {\bf robot})}}
\label{namespaceservidor__novo_a1f0f40c9f2b17d9579e189b4ec8f4ffe}


Definición en la línea 157 del archivo servidor\_\-novo.py.

\hypertarget{namespaceservidor__novo_a4644b693a5931c2f1a9d68500534d955}{
\index{servidor\_\-novo@{servidor\_\-novo}!packet@{packet}}
\index{packet@{packet}!servidor_novo@{servidor\_\-novo}}
\subsubsection[{packet}]{\setlength{\rightskip}{0pt plus 5cm}tuple {\bf servidor\_\-novo.packet} = ArNetPacket()}}
\label{namespaceservidor__novo_a4644b693a5931c2f1a9d68500534d955}


Definición en la línea 121 del archivo servidor\_\-novo.py.

\hypertarget{namespaceservidor__novo_ae604d92f7f5c43d2a993ae9402534424}{
\index{servidor\_\-novo@{servidor\_\-novo}!robot@{robot}}
\index{robot@{robot}!servidor_novo@{servidor\_\-novo}}
\subsubsection[{robot}]{\setlength{\rightskip}{0pt plus 5cm}tuple {\bf servidor\_\-novo.robot} = ArRobot()}}
\label{namespaceservidor__novo_ae604d92f7f5c43d2a993ae9402534424}


Definición en la línea 108 del archivo servidor\_\-novo.py.

\hypertarget{namespaceservidor__novo_a538b68fbb948f9cf0573a119042c9e0d}{
\index{servidor\_\-novo@{servidor\_\-novo}!server@{server}}
\index{server@{server}!servidor_novo@{servidor\_\-novo}}
\subsubsection[{server}]{\setlength{\rightskip}{0pt plus 5cm}tuple {\bf servidor\_\-novo.server} = ArServerBase()}}
\label{namespaceservidor__novo_a538b68fbb948f9cf0573a119042c9e0d}


Definición en la línea 119 del archivo servidor\_\-novo.py.

\hypertarget{namespaceservidor__novo_aae2aad1feed83599e4783f4d39086ee6}{
\index{servidor\_\-novo@{servidor\_\-novo}!serverInfoRobot@{serverInfoRobot}}
\index{serverInfoRobot@{serverInfoRobot}!servidor_novo@{servidor\_\-novo}}
\subsubsection[{serverInfoRobot}]{\setlength{\rightskip}{0pt plus 5cm}tuple {\bf servidor\_\-novo.serverInfoRobot} = ArServerInfoRobot({\bf server}, {\bf robot})}}
\label{namespaceservidor__novo_aae2aad1feed83599e4783f4d39086ee6}


Definición en la línea 149 del archivo servidor\_\-novo.py.

\hypertarget{namespaceservidor__novo_a31795a5817e65297017799e1fdbc6fdd}{
\index{servidor\_\-novo@{servidor\_\-novo}!serverInfoSensor@{serverInfoSensor}}
\index{serverInfoSensor@{serverInfoSensor}!servidor_novo@{servidor\_\-novo}}
\subsubsection[{serverInfoSensor}]{\setlength{\rightskip}{0pt plus 5cm}tuple {\bf servidor\_\-novo.serverInfoSensor} = ArServerInfoSensor({\bf server}, {\bf robot})}}
\label{namespaceservidor__novo_a31795a5817e65297017799e1fdbc6fdd}


Definición en la línea 150 del archivo servidor\_\-novo.py.

\hypertarget{namespaceservidor__novo_a966ece9821b12a79fc5b83f628987f7d}{
\index{servidor\_\-novo@{servidor\_\-novo}!sonarDev@{sonarDev}}
\index{sonarDev@{sonarDev}!servidor_novo@{servidor\_\-novo}}
\subsubsection[{sonarDev}]{\setlength{\rightskip}{0pt plus 5cm}tuple {\bf servidor\_\-novo.sonarDev} = ArSonarDevice(7)}}
\label{namespaceservidor__novo_a966ece9821b12a79fc5b83f628987f7d}


Definición en la línea 114 del archivo servidor\_\-novo.py.

\hypertarget{namespaceservidor__novo_a9cc0dcde8396c1ff04833e8811a1df74}{
\index{servidor\_\-novo@{servidor\_\-novo}!uCCommands@{uCCommands}}
\index{uCCommands@{uCCommands}!servidor_novo@{servidor\_\-novo}}
\subsubsection[{uCCommands}]{\setlength{\rightskip}{0pt plus 5cm}tuple {\bf servidor\_\-novo.uCCommands} = ArServerSimpleComUC({\bf commands}, {\bf robot})}}
\label{namespaceservidor__novo_a9cc0dcde8396c1ff04833e8811a1df74}


Definición en la línea 164 del archivo servidor\_\-novo.py.


\chapter{Documentación de las clases}
\hypertarget{classcliente__lib_1_1cliente__lib}{
\section{Referencia de la Clase cliente\_\-lib.cliente\_\-lib}
\label{classcliente__lib_1_1cliente__lib}\index{cliente\_\-lib::cliente\_\-lib@{cliente\_\-lib::cliente\_\-lib}}
}


es la clase encargada del cliente  


\subsection*{M�todos p�blicos}
\begin{DoxyCompactItemize}
\item 
def \hyperlink{classcliente__lib_1_1cliente__lib_ac5e4490f412835d35481f58d1ae503f9}{\_\-\_\-init\_\-\_\-}
\begin{DoxyCompactList}\small\item\em Carga valores a las variables necesarias para funcionar el cliente. \item\end{DoxyCompactList}\item 
def \hyperlink{classcliente__lib_1_1cliente__lib_af7b751bcf94c96150b23bacb5e477956}{valores}
\begin{DoxyCompactList}\small\item\em Sirve cuando se manda el comando \char`\"{}updateNumbers\char`\"{} en el paquete. \item\end{DoxyCompactList}\item 
def \hyperlink{classcliente__lib_1_1cliente__lib_ac0a4410b48b4c759028bec6ae1c641e8}{lista\_\-sonares}
\begin{DoxyCompactList}\small\item\em Sirve para leer el paquete arNetPacket con la lista del sonar. \item\end{DoxyCompactList}\item 
def \hyperlink{classcliente__lib_1_1cliente__lib_abcf28c2207cb5519090654484137db23}{valores\_\-sonares}
\begin{DoxyCompactList}\small\item\em Sirve para leer el paquete arNetPacket con los valores del sonar. \item\end{DoxyCompactList}\item 
def \hyperlink{classcliente__lib_1_1cliente__lib_acfc22af72a1668db28d18ab4ff40909e}{envio\_\-ratioDrive}
\begin{DoxyCompactList}\small\item\em Sirve para realizar la teleoperacion, mandando los parametros. \item\end{DoxyCompactList}\item 
def \hyperlink{classcliente__lib_1_1cliente__lib_ade1f44e9270c8835c284832a72b96b6c}{uC\_\-comandos\_\-movi}
\begin{DoxyCompactList}\small\item\em Sirve para mandar ordenes de movimiento directamente al controlador de la plataforma movil. \item\end{DoxyCompactList}\item 
def \hyperlink{classcliente__lib_1_1cliente__lib_ac3e89d3066207b05b217a50f549c239a}{envio\_\-consulta\_\-fisica}
\begin{DoxyCompactList}\small\item\em Sirve para mandar ordenes al servidor utilizando paquetes ArNetPacket con comandos {\bfseries pose} y {\bfseries updateNumbers} \item\end{DoxyCompactList}\item 
def \hyperlink{classcliente__lib_1_1cliente__lib_a1a7b5475a98772f0e48a4e1fd76e8d47}{cliente\_\-inicio}
\begin{DoxyCompactList}\small\item\em Sirve para iniciar la conexion con el servidor. \item\end{DoxyCompactList}\item 
def \hyperlink{classcliente__lib_1_1cliente__lib_a14a49495fd71fab84d36060e604415a5}{cliente\_\-apaga}
\begin{DoxyCompactList}\small\item\em Sirve para realizar la desconexion con el servidor. \item\end{DoxyCompactList}\item 
def \hyperlink{classcliente__lib_1_1cliente__lib_a3ca67c0c9d7f0a622abd740c780f64d1}{devuelve\_\-valorf}
\begin{DoxyCompactList}\small\item\em Devuelve el variable valor\_\-fisico, con usa espera de 100ms. \item\end{DoxyCompactList}\item 
def \hyperlink{classcliente__lib_1_1cliente__lib_a127d026872fbd11f4f5bbe4a73424b77}{devuelve\_\-valors}
\begin{DoxyCompactList}\small\item\em Devuelve el variable valor\_\-sonares, con usa espera de 100ms. \item\end{DoxyCompactList}\end{DoxyCompactItemize}
\subsection*{Atributos p�blicos}
\begin{DoxyCompactItemize}
\item 
\hyperlink{classcliente__lib_1_1cliente__lib_abba3409f89ee8dcec8b180c90aa5d77c}{valor\_\-fisico}
\item 
\hyperlink{classcliente__lib_1_1cliente__lib_aadea6e24bd3a01b0500fcc67543a97e9}{valor\_\-sonares}
\item 
\hyperlink{classcliente__lib_1_1cliente__lib_a675dd8430aa2eeb33240b8b07ed61543}{ip}
\end{DoxyCompactItemize}


\subsection{Descripci�n detallada}
es la clase encargada del cliente se lo utiliza de esta manera par poder trabajar con la informacion tanto leyendo las variables o utilizando los comandos 

Definici�n en la l�nea 50 del archivo cliente\_\-lib.py.



\subsection{Documentaci�n de las funciones miembro}
\hypertarget{classcliente__lib_1_1cliente__lib_ac5e4490f412835d35481f58d1ae503f9}{
\index{cliente\_\-lib::cliente\_\-lib@{cliente\_\-lib::cliente\_\-lib}!\_\-\_\-init\_\-\_\-@{\_\-\_\-init\_\-\_\-}}
\index{\_\-\_\-init\_\-\_\-@{\_\-\_\-init\_\-\_\-}!cliente_lib::cliente_lib@{cliente\_\-lib::cliente\_\-lib}}
\subsubsection[{\_\-\_\-init\_\-\_\-}]{\setlength{\rightskip}{0pt plus 5cm}def cliente\_\-lib.cliente\_\-lib.\_\-\_\-init\_\-\_\- (
\begin{DoxyParamCaption}
\item[{}]{ self}
\end{DoxyParamCaption}
)}}
\label{classcliente__lib_1_1cliente__lib_ac5e4490f412835d35481f58d1ae503f9}


Carga valores a las variables necesarias para funcionar el cliente. 

este comando no es necesario utilizarlo es usado al instanciar la clase 
\begin{DoxyParams}{Par�metros}
\item[{\em self}]este parametro no es necesario escribir \end{DoxyParams}


Definici�n en la l�nea 57 del archivo cliente\_\-lib.py.




\begin{DoxyCode}
58                           :
59                 print "Cargo modulo para cliente_lib"
60                 self.valor_fisico=[]
61                 self.valor_sonares=[] 
62                 self.ip="localhost"

\end{DoxyCode}


\hypertarget{classcliente__lib_1_1cliente__lib_a14a49495fd71fab84d36060e604415a5}{
\index{cliente\_\-lib::cliente\_\-lib@{cliente\_\-lib::cliente\_\-lib}!cliente\_\-apaga@{cliente\_\-apaga}}
\index{cliente\_\-apaga@{cliente\_\-apaga}!cliente_lib::cliente_lib@{cliente\_\-lib::cliente\_\-lib}}
\subsubsection[{cliente\_\-apaga}]{\setlength{\rightskip}{0pt plus 5cm}def cliente\_\-lib.cliente\_\-lib.cliente\_\-apaga (
\begin{DoxyParamCaption}
\item[{}]{ self, }
\item[{}]{ client}
\end{DoxyParamCaption}
)}}
\label{classcliente__lib_1_1cliente__lib_a14a49495fd71fab84d36060e604415a5}


Sirve para realizar la desconexion con el servidor. 


\begin{DoxyParams}{Par�metros}
\item[{\em self}]este parametro no es necesario escribir \item[{\em client}]para poder desconectar el cliente \end{DoxyParams}


Definici�n en la l�nea 208 del archivo cliente\_\-lib.py.




\begin{DoxyCode}
209                                       :
210                 ArUtil.sleep(1000)
211                 client.disconnect()
212                 ArUtil.sleep(50)
                return 0
\end{DoxyCode}


\hypertarget{classcliente__lib_1_1cliente__lib_a1a7b5475a98772f0e48a4e1fd76e8d47}{
\index{cliente\_\-lib::cliente\_\-lib@{cliente\_\-lib::cliente\_\-lib}!cliente\_\-inicio@{cliente\_\-inicio}}
\index{cliente\_\-inicio@{cliente\_\-inicio}!cliente_lib::cliente_lib@{cliente\_\-lib::cliente\_\-lib}}
\subsubsection[{cliente\_\-inicio}]{\setlength{\rightskip}{0pt plus 5cm}def cliente\_\-lib.cliente\_\-lib.cliente\_\-inicio (
\begin{DoxyParamCaption}
\item[{}]{ self}
\end{DoxyParamCaption}
)}}
\label{classcliente__lib_1_1cliente__lib_a1a7b5475a98772f0e48a4e1fd76e8d47}


Sirve para iniciar la conexion con el servidor. 


\begin{DoxyParams}{Par�metros}
\item[{\em self}]este parametro no es necesario escribir \end{DoxyParams}
\begin{DoxyReturn}{Devuelve}
client 
\end{DoxyReturn}


Definici�n en la l�nea 178 del archivo cliente\_\-lib.py.




\begin{DoxyCode}
179                                 :
180                 client = ArClientBase()
181                 Aria.init()
182                 
183                 startTime = ArTime()
184                 startTime.setToNow()
185                 if not client.blockingConnect(self.ip, 7272): #ip y puerto del se
      rvidor
186                         print "Could not connect to server at localhost port 7272
      , exiting"
187                         Aria.exit(1);
188                 print "cliente: Se tardo %ld msec en connectarse\n" % (startTime.
      mSecSince())
189                 
190                 client.runAsync()
191                 client.addHandler("updateNumbers",self.valores)
192                 client.addHandler("getSensorList",self.lista_sonares)
193                 client.addHandler("pose",self.valores_sonares)
194                 if client.dataExists("ratioDrive"): #supuestamente devuelve la in
      fo del robot con odometria
195                         print "ratioDrive si existe"
196                 else:
197                         Aria.exit(1);
198                 #client=envio_ratioDrive(client,TransRatio,RotRatio,LatRatio) #fi
      jar los valores para mover
199                 #client=uC_comandos_movi(client,comando,parametro) #Lo hace de un
      a manera directa anulando las demas operaciones
200                 #client.requestOnce("updateNumbers")
201                 #client.requestOnce("stop") #parada de emergencia
                return client
\end{DoxyCode}


\hypertarget{classcliente__lib_1_1cliente__lib_a3ca67c0c9d7f0a622abd740c780f64d1}{
\index{cliente\_\-lib::cliente\_\-lib@{cliente\_\-lib::cliente\_\-lib}!devuelve\_\-valorf@{devuelve\_\-valorf}}
\index{devuelve\_\-valorf@{devuelve\_\-valorf}!cliente_lib::cliente_lib@{cliente\_\-lib::cliente\_\-lib}}
\subsubsection[{devuelve\_\-valorf}]{\setlength{\rightskip}{0pt plus 5cm}def cliente\_\-lib.cliente\_\-lib.devuelve\_\-valorf (
\begin{DoxyParamCaption}
\item[{}]{ self}
\end{DoxyParamCaption}
)}}
\label{classcliente__lib_1_1cliente__lib_a3ca67c0c9d7f0a622abd740c780f64d1}


Devuelve el variable valor\_\-fisico, con usa espera de 100ms. 


\begin{DoxyParams}{Par�metros}
\item[{\em self}]este parametro no es necesario escribir \end{DoxyParams}


Definici�n en la l�nea 218 del archivo cliente\_\-lib.py.




\begin{DoxyCode}
219                                  :
220                 ArUtil.sleep(100)
                return self.valor_fisico
\end{DoxyCode}


\hypertarget{classcliente__lib_1_1cliente__lib_a127d026872fbd11f4f5bbe4a73424b77}{
\index{cliente\_\-lib::cliente\_\-lib@{cliente\_\-lib::cliente\_\-lib}!devuelve\_\-valors@{devuelve\_\-valors}}
\index{devuelve\_\-valors@{devuelve\_\-valors}!cliente_lib::cliente_lib@{cliente\_\-lib::cliente\_\-lib}}
\subsubsection[{devuelve\_\-valors}]{\setlength{\rightskip}{0pt plus 5cm}def cliente\_\-lib.cliente\_\-lib.devuelve\_\-valors (
\begin{DoxyParamCaption}
\item[{}]{ self}
\end{DoxyParamCaption}
)}}
\label{classcliente__lib_1_1cliente__lib_a127d026872fbd11f4f5bbe4a73424b77}


Devuelve el variable valor\_\-sonares, con usa espera de 100ms. 


\begin{DoxyParams}{Par�metros}
\item[{\em self}]este parametro no es necesario escribir \end{DoxyParams}


Definici�n en la l�nea 226 del archivo cliente\_\-lib.py.




\begin{DoxyCode}
227                                  :
228                 ArUtil.sleep(100)
229                 return self.valor_sonares

\end{DoxyCode}


\hypertarget{classcliente__lib_1_1cliente__lib_ac3e89d3066207b05b217a50f549c239a}{
\index{cliente\_\-lib::cliente\_\-lib@{cliente\_\-lib::cliente\_\-lib}!envio\_\-consulta\_\-fisica@{envio\_\-consulta\_\-fisica}}
\index{envio\_\-consulta\_\-fisica@{envio\_\-consulta\_\-fisica}!cliente_lib::cliente_lib@{cliente\_\-lib::cliente\_\-lib}}
\subsubsection[{envio\_\-consulta\_\-fisica}]{\setlength{\rightskip}{0pt plus 5cm}def cliente\_\-lib.cliente\_\-lib.envio\_\-consulta\_\-fisica (
\begin{DoxyParamCaption}
\item[{}]{ self, }
\item[{}]{ client, }
\item[{}]{ mensaje}
\end{DoxyParamCaption}
)}}
\label{classcliente__lib_1_1cliente__lib_ac3e89d3066207b05b217a50f549c239a}


Sirve para mandar ordenes al servidor utilizando paquetes ArNetPacket con comandos {\bfseries pose} y {\bfseries updateNumbers} 


\begin{DoxyParams}{Par�metros}
\item[{\em self}]este parametro no es necesario escribir \item[{\em client}]Se debe trar el objeto cliente a la definicion para poder utilizar el enlace del cliente para enviar el paquete al servidor \item[{\em mensaje}]puede ser cualquier comando del servidor que no devuelva informacion a exepcion de pose y updatenumbers \end{DoxyParams}
\begin{DoxyReturn}{Devuelve}
client 
\end{DoxyReturn}


Definici�n en la l�nea 167 del archivo cliente\_\-lib.py.




\begin{DoxyCode}
168                                                       :
169                 ## se puede usar pose y updateNumbers
170                 client.requestOnce(mensaje)
171                 return client
                
\end{DoxyCode}


\hypertarget{classcliente__lib_1_1cliente__lib_acfc22af72a1668db28d18ab4ff40909e}{
\index{cliente\_\-lib::cliente\_\-lib@{cliente\_\-lib::cliente\_\-lib}!envio\_\-ratioDrive@{envio\_\-ratioDrive}}
\index{envio\_\-ratioDrive@{envio\_\-ratioDrive}!cliente_lib::cliente_lib@{cliente\_\-lib::cliente\_\-lib}}
\subsubsection[{envio\_\-ratioDrive}]{\setlength{\rightskip}{0pt plus 5cm}def cliente\_\-lib.cliente\_\-lib.envio\_\-ratioDrive (
\begin{DoxyParamCaption}
\item[{}]{ self, }
\item[{}]{ client, }
\item[{}]{ TransRatio, }
\item[{}]{ RotRatio, }
\item[{}]{ LatRatio}
\end{DoxyParamCaption}
)}}
\label{classcliente__lib_1_1cliente__lib_acfc22af72a1668db28d18ab4ff40909e}


Sirve para realizar la teleoperacion, mandando los parametros. 


\begin{DoxyParams}{Par�metros}
\item[{\em self}]este parametro no es necesario escribir \item[{\em client}]Se debe trar el objeto cliente a la definicion para poder utilizar el enlace del cliente para enviar el paquete al servidor \item[{\em TransRatio}]Velocidad de traslacion \item[{\em RotRatio}]Velocidad de rotacion \item[{\em LatRatio}]velocidad lateral para el modelo Pioneer P3-\/DX no se necesario puede ser 0 \end{DoxyParams}
\begin{DoxyReturn}{Devuelve}
client 
\end{DoxyReturn}


Definici�n en la l�nea 130 del archivo cliente\_\-lib.py.




\begin{DoxyCode}
131                                                                       :
132                 myTransRatio=TransRatio
133                 myRotRatio=RotRatio
134                 myLatRatio=LatRatio
135                 packet=ArNetPacket()
136                 packet.doubleToBuf(myTransRatio)
137                 packet.doubleToBuf(myRotRatio)
138                 packet.doubleToBuf(50) # use half of the robot's maximum.
139                 packet.doubleToBuf(myLatRatio)
140                 client.requestOnce("ratioDrive", packet)
141                 return client
          
\end{DoxyCode}


\hypertarget{classcliente__lib_1_1cliente__lib_ac0a4410b48b4c759028bec6ae1c641e8}{
\index{cliente\_\-lib::cliente\_\-lib@{cliente\_\-lib::cliente\_\-lib}!lista\_\-sonares@{lista\_\-sonares}}
\index{lista\_\-sonares@{lista\_\-sonares}!cliente_lib::cliente_lib@{cliente\_\-lib::cliente\_\-lib}}
\subsubsection[{lista\_\-sonares}]{\setlength{\rightskip}{0pt plus 5cm}def cliente\_\-lib.cliente\_\-lib.lista\_\-sonares (
\begin{DoxyParamCaption}
\item[{}]{ self, }
\item[{}]{ packet}
\end{DoxyParamCaption}
)}}
\label{classcliente__lib_1_1cliente__lib_ac0a4410b48b4c759028bec6ae1c641e8}


Sirve para leer el paquete arNetPacket con la lista del sonar. 

este comando no es necesario utilizarlo es usado solo por el cliente para procesar el paquete 
\begin{DoxyParams}{Par�metros}
\item[{\em self}]este parametro no es necesario escribir \item[{\em packet}]este parametro no es necesario escribir \end{DoxyParams}
\begin{DoxyReturn}{Devuelve}
nada 
\end{DoxyReturn}


Definici�n en la l�nea 94 del archivo cliente\_\-lib.py.




\begin{DoxyCode}
95                                       :
96                 c="                                   "
97                 numSensor=packet.bufToByte2()
98                 numSensor2=packet.bufToStr(c,15)
99                 print str(numSensor)+" "+str(c.strip())

\end{DoxyCode}


\hypertarget{classcliente__lib_1_1cliente__lib_ade1f44e9270c8835c284832a72b96b6c}{
\index{cliente\_\-lib::cliente\_\-lib@{cliente\_\-lib::cliente\_\-lib}!uC\_\-comandos\_\-movi@{uC\_\-comandos\_\-movi}}
\index{uC\_\-comandos\_\-movi@{uC\_\-comandos\_\-movi}!cliente_lib::cliente_lib@{cliente\_\-lib::cliente\_\-lib}}
\subsubsection[{uC\_\-comandos\_\-movi}]{\setlength{\rightskip}{0pt plus 5cm}def cliente\_\-lib.cliente\_\-lib.uC\_\-comandos\_\-movi (
\begin{DoxyParamCaption}
\item[{}]{ self, }
\item[{}]{ client, }
\item[{}]{ comando, }
\item[{}]{ parametro}
\end{DoxyParamCaption}
)}}
\label{classcliente__lib_1_1cliente__lib_ade1f44e9270c8835c284832a72b96b6c}


Sirve para mandar ordenes de movimiento directamente al controlador de la plataforma movil. 


\begin{DoxyParams}{Par�metros}
\item[{\em self}]este parametro no es necesario escribir \item[{\em client}]Se debe trar el objeto cliente a la definicion para poder utilizar el enlace del cliente para enviar el paquete al servidor \item[{\em comando}]es un numero de 1-\/255 que representa una funcion esta informacion se puede encontrar en el API de ARIA \item[{\em parametro}]el parametro de la funcion en caso de no tener se deja el valor en blanco \end{DoxyParams}
\begin{DoxyReturn}{Devuelve}
client 
\end{DoxyReturn}


Definici�n en la l�nea 151 del archivo cliente\_\-lib.py.




\begin{DoxyCode}
152                                                            :
153                 mi_comando=comando #comando 8 es MOVE parametro un valor de 5000 
      a -4999 es en mm, 11 LEV y su parametro es velocidad +o- mm/s
154                 mi_parametro=parametro #parametro
155                 packet=ArNetPacket()
156                 packet.strToBuf(mi_comando+" "+mi_parametro)
157                 client.requestOnce("MicroControllerMotionCommand", packet) #Micro
      ControllerMotionCommand
158                 return client

\end{DoxyCode}


\hypertarget{classcliente__lib_1_1cliente__lib_af7b751bcf94c96150b23bacb5e477956}{
\index{cliente\_\-lib::cliente\_\-lib@{cliente\_\-lib::cliente\_\-lib}!valores@{valores}}
\index{valores@{valores}!cliente_lib::cliente_lib@{cliente\_\-lib::cliente\_\-lib}}
\subsubsection[{valores}]{\setlength{\rightskip}{0pt plus 5cm}def cliente\_\-lib.cliente\_\-lib.valores (
\begin{DoxyParamCaption}
\item[{}]{ self, }
\item[{}]{ packet}
\end{DoxyParamCaption}
)}}
\label{classcliente__lib_1_1cliente__lib_af7b751bcf94c96150b23bacb5e477956}


Sirve cuando se manda el comando \char`\"{}updateNumbers\char`\"{} en el paquete. 

este comando no es necesario utilizarlo es usado solo por el cliente para procesar el paquete 
\begin{DoxyParams}{Par�metros}
\item[{\em self}]este parametro no es necesario escribir \item[{\em packet}]el paquete que recibe el cliente del servidor, no es necesario escribir \end{DoxyParams}
\begin{DoxyReturn}{Devuelve}
Nada, pero guarda en self.valores\_\-fisico \mbox{[}voltaje\_\-bateria,myX,myY,myTh,myVel,myRotVel,myLatVel,myTemperature\mbox{]} 
\end{DoxyReturn}


Definici�n en la l�nea 71 del archivo cliente\_\-lib.py.




\begin{DoxyCode}
72                                 :
73                 #devuelve los valores voltaje_bateria,myX,myY,myTh,myVel,myRotVel
      ,myLatVel,myTemperature
74                 voltaje_bateria=packet.bufToByte2()/10
75                 myX = packet.bufToByte4()#
76                 myY = packet.bufToByte4()
77                 myTh = packet.bufToByte2()
78                 myVel = packet.bufToByte2()
79                 myRotVel = packet.bufToByte2()
80                 myLatVel = packet.bufToByte2()
81                 myTemperature = packet.bufToByte()
82                 #print "X= "+str(myX)+" y="+str(myY)+" th="+str(myTh)
83                 self.valor_fisico=(voltaje_bateria,myX,myY,myTh,myVel,myRotVel,my
      LatVel,myTemperature)
84                 self.valor_fisico=(voltaje_bateria,myX,myY,myTh,myVel,myRotVel,my
      LatVel,0)
85                 #print valor

\end{DoxyCode}


\hypertarget{classcliente__lib_1_1cliente__lib_abcf28c2207cb5519090654484137db23}{
\index{cliente\_\-lib::cliente\_\-lib@{cliente\_\-lib::cliente\_\-lib}!valores\_\-sonares@{valores\_\-sonares}}
\index{valores\_\-sonares@{valores\_\-sonares}!cliente_lib::cliente_lib@{cliente\_\-lib::cliente\_\-lib}}
\subsubsection[{valores\_\-sonares}]{\setlength{\rightskip}{0pt plus 5cm}def cliente\_\-lib.cliente\_\-lib.valores\_\-sonares (
\begin{DoxyParamCaption}
\item[{}]{ self, }
\item[{}]{ packet}
\end{DoxyParamCaption}
)}}
\label{classcliente__lib_1_1cliente__lib_abcf28c2207cb5519090654484137db23}


Sirve para leer el paquete arNetPacket con los valores del sonar. 

este comando no es necesario utilizarlo es usado solo por el cliente para procesar el paquete 
\begin{DoxyParams}{Par�metros}
\item[{\em self}]este parametro no es necesario escribir \item[{\em packet}]este parametro no es necesario escribir \end{DoxyParams}
\begin{DoxyReturn}{Devuelve}
nada 
\end{DoxyReturn}


Definici�n en la l�nea 108 del archivo cliente\_\-lib.py.




\begin{DoxyCode}
109                                         :
110                 c="                                                              
                                                          "
111                 cantidad=packet.bufToDouble()
112                 dato=range(int(cantidad))
113                 for j in range(int(cantidad)):
114                         dato[j]=c #ya que pasa el buffer a la variable esta debe 
      tener la longitud necesaria y deben ser guardada en cada lectura
115                 for i in range(int(cantidad)):
116                         packet.bufToStr(dato[i],50)
117                         dato[i]=dato[i].replace("\x00"," ") #Porque al final anad
      e un NONE del string
118                         dato[i]=dato[i].strip() 
119                 self.valor_sonares=[cantidad,dato]

\end{DoxyCode}




\subsection{Documentaci�n de los datos miembro}
\hypertarget{classcliente__lib_1_1cliente__lib_a675dd8430aa2eeb33240b8b07ed61543}{
\index{cliente\_\-lib::cliente\_\-lib@{cliente\_\-lib::cliente\_\-lib}!ip@{ip}}
\index{ip@{ip}!cliente_lib::cliente_lib@{cliente\_\-lib::cliente\_\-lib}}
\subsubsection[{ip}]{\setlength{\rightskip}{0pt plus 5cm}{\bf cliente\_\-lib.cliente\_\-lib.ip}}}
\label{classcliente__lib_1_1cliente__lib_a675dd8430aa2eeb33240b8b07ed61543}


Definici�n en la l�nea 61 del archivo cliente\_\-lib.py.

\hypertarget{classcliente__lib_1_1cliente__lib_abba3409f89ee8dcec8b180c90aa5d77c}{
\index{cliente\_\-lib::cliente\_\-lib@{cliente\_\-lib::cliente\_\-lib}!valor\_\-fisico@{valor\_\-fisico}}
\index{valor\_\-fisico@{valor\_\-fisico}!cliente_lib::cliente_lib@{cliente\_\-lib::cliente\_\-lib}}
\subsubsection[{valor\_\-fisico}]{\setlength{\rightskip}{0pt plus 5cm}{\bf cliente\_\-lib.cliente\_\-lib.valor\_\-fisico}}}
\label{classcliente__lib_1_1cliente__lib_abba3409f89ee8dcec8b180c90aa5d77c}


Definici�n en la l�nea 59 del archivo cliente\_\-lib.py.

\hypertarget{classcliente__lib_1_1cliente__lib_aadea6e24bd3a01b0500fcc67543a97e9}{
\index{cliente\_\-lib::cliente\_\-lib@{cliente\_\-lib::cliente\_\-lib}!valor\_\-sonares@{valor\_\-sonares}}
\index{valor\_\-sonares@{valor\_\-sonares}!cliente_lib::cliente_lib@{cliente\_\-lib::cliente\_\-lib}}
\subsubsection[{valor\_\-sonares}]{\setlength{\rightskip}{0pt plus 5cm}{\bf cliente\_\-lib.cliente\_\-lib.valor\_\-sonares}}}
\label{classcliente__lib_1_1cliente__lib_aadea6e24bd3a01b0500fcc67543a97e9}


Definici�n en la l�nea 60 del archivo cliente\_\-lib.py.



La documentaci�n para esta clase fue generada a partir del siguiente fichero:\begin{DoxyCompactItemize}
\item 
\hyperlink{cliente__lib_8py}{cliente\_\-lib.py}\end{DoxyCompactItemize}

\hypertarget{classgamepad_1_1gamepad}{
\section{Referencia de la Clase gamepad.gamepad}
\label{classgamepad_1_1gamepad}\index{gamepad::gamepad@{gamepad::gamepad}}
}
\subsection*{Métodos públicos}
\begin{DoxyCompactItemize}
\item 
def \hyperlink{classgamepad_1_1gamepad_af68047f7eb80547eed320d45f34c2a05}{\_\-\_\-init\_\-\_\-}
\item 
def \hyperlink{classgamepad_1_1gamepad_a8e9b61aced2fb5f8fd25127ae19aa3d5}{gamepad\_\-init}
\item 
def \hyperlink{classgamepad_1_1gamepad_ab0e62e91d6e1e08101897640b4dc4a27}{gamepad\_\-lectura}
\end{DoxyCompactItemize}
\subsection*{Atributos públicos}
\begin{DoxyCompactItemize}
\item 
\hyperlink{classgamepad_1_1gamepad_a06303222a83dcb4e07be733da7ab35cd}{a}
\end{DoxyCompactItemize}


\subsection{Descripción detallada}


Definición en la línea 29 del archivo gamepad.py.



\subsection{Documentación del constructor y destructor}
\hypertarget{classgamepad_1_1gamepad_af68047f7eb80547eed320d45f34c2a05}{
\index{gamepad::gamepad@{gamepad::gamepad}!\_\-\_\-init\_\-\_\-@{\_\-\_\-init\_\-\_\-}}
\index{\_\-\_\-init\_\-\_\-@{\_\-\_\-init\_\-\_\-}!gamepad::gamepad@{gamepad::gamepad}}
\subsubsection[{\_\-\_\-init\_\-\_\-}]{\setlength{\rightskip}{0pt plus 5cm}def gamepad.gamepad.\_\-\_\-init\_\-\_\- (
\begin{DoxyParamCaption}
\item[{}]{self}
\end{DoxyParamCaption}
)}}
\label{classgamepad_1_1gamepad_af68047f7eb80547eed320d45f34c2a05}


Definición en la línea 30 del archivo gamepad.py.


\begin{DoxyCode}
31                           :
32                 self.a=0

\end{DoxyCode}


\subsection{Documentación de las funciones miembro}
\hypertarget{classgamepad_1_1gamepad_a8e9b61aced2fb5f8fd25127ae19aa3d5}{
\index{gamepad::gamepad@{gamepad::gamepad}!gamepad\_\-init@{gamepad\_\-init}}
\index{gamepad\_\-init@{gamepad\_\-init}!gamepad::gamepad@{gamepad::gamepad}}
\subsubsection[{gamepad\_\-init}]{\setlength{\rightskip}{0pt plus 5cm}def gamepad.gamepad.gamepad\_\-init (
\begin{DoxyParamCaption}
\item[{}]{self}
\end{DoxyParamCaption}
)}}
\label{classgamepad_1_1gamepad_a8e9b61aced2fb5f8fd25127ae19aa3d5}


Definición en la línea 33 del archivo gamepad.py.


\begin{DoxyCode}
34                               :
35                 #Abre el dispositivo js0 como si fuera un archivo de lectura
36                 pipe=open('/dev/input/js0','r')
37                 msg=[]
                return pipe,msg
\end{DoxyCode}
\hypertarget{classgamepad_1_1gamepad_ab0e62e91d6e1e08101897640b4dc4a27}{
\index{gamepad::gamepad@{gamepad::gamepad}!gamepad\_\-lectura@{gamepad\_\-lectura}}
\index{gamepad\_\-lectura@{gamepad\_\-lectura}!gamepad::gamepad@{gamepad::gamepad}}
\subsubsection[{gamepad\_\-lectura}]{\setlength{\rightskip}{0pt plus 5cm}def gamepad.gamepad.gamepad\_\-lectura (
\begin{DoxyParamCaption}
\item[{}]{self, }
\item[{}]{pipe, }
\item[{}]{msg}
\end{DoxyParamCaption}
)}}
\label{classgamepad_1_1gamepad_ab0e62e91d6e1e08101897640b4dc4a27}


Definición en la línea 38 del archivo gamepad.py.


\begin{DoxyCode}
39                                           :
40                 #time.sleep(5)
41                 a=0
42                 #Para cada caracter leidos desde el /dev/input/js0 pipe..
43                 for char in pipe.read(1):
44                         #Agrega la representacion entera de la lectura de caracte
      res Unicode en la lista msg.
45                         msg+=[ord(char)]
46                         #print len(msg)
47                         #Si el tamanio de la lista msg es 8
48                         if len(msg) == 8:
49                                 #Boton evento si el byte 6 es 1
50                                 if msg[6] == 1: #1
51                                         if msg[4] == 0:
52                                                 if msg[7] == 0:
53                                                         #print "Boton 1"
54                                                         a='b1'
55                                                 elif msg[7] == 1:
56                                                         #print "Boton 2"
57                                                         a='b2'
58                                                 elif msg[7] == 2:
59                                                         #print "Boton 3"
60                                                         a='b3'
61                                                 elif msg[7] == 3:
62                                                         #print "Boton 4"
63                                                         a='b4'
64                                                 elif msg[7] == 4:
65                                                         #print "Boton L1"
66                                                         a='l1'
67                                                 elif msg[7] == 5:
68                                                         #print "Boton R1"
69                                                         a='r1'
70                                                 elif msg[7] == 6:
71                                                         #print "Boton L2"
72                                                         a='l2'
73                                                 elif msg[7] == 7:
74                                                         #print "Boton R2"
75                                                         a='r2'
76                                 #ejes evento if el byte 6 es 2
77                                 elif msg[6] == 2:
78                                         if msg[5] == 128:
79                                                 if msg[7] == 1:
80                                                         #print "Boton Arriba"
81                                                         a='ar'
82                                                 elif msg[7] == 0:
83                                                         #print "Boton Izquierda"
84                                                         a='iz'
85                                         elif msg[5] == 127:
86                                                 if msg[7] == 0:
87                                                         #print "Boton Derecha"
88                                                         a='de'
89                                                 elif msg[7] == 1:
90                                                         #print "Boton Abajo"
91                                                         a='ab'
92                                 #resetear msg como una lista vacia
93                                 msg[0:] = []
94                 self.a=a
                return a
\end{DoxyCode}


\subsection{Documentación de los datos miembro}
\hypertarget{classgamepad_1_1gamepad_a06303222a83dcb4e07be733da7ab35cd}{
\index{gamepad::gamepad@{gamepad::gamepad}!a@{a}}
\index{a@{a}!gamepad::gamepad@{gamepad::gamepad}}
\subsubsection[{a}]{\setlength{\rightskip}{0pt plus 5cm}{\bf gamepad.gamepad.a}}}
\label{classgamepad_1_1gamepad_a06303222a83dcb4e07be733da7ab35cd}


Definición en la línea 30 del archivo gamepad.py.



La documentación para esta clase fue generada a partir del siguiente fichero:\begin{DoxyCompactItemize}
\item 
\hyperlink{gamepad_8py}{gamepad.py}\end{DoxyCompactItemize}

\hypertarget{classinicio_1_1prueba__adqui}{
\section{Referencia de la Clase inicio.prueba\_\-adqui}
\label{classinicio_1_1prueba__adqui}\index{inicio::prueba\_\-adqui@{inicio::prueba\_\-adqui}}
}


es la clase encargada del entorno grafico y enlace con \hyperlink{namespacecliente__lib}{cliente\_\-lib}  


\subsection*{Métodos públicos}
\begin{DoxyCompactItemize}
\item 
def \hyperlink{classinicio_1_1prueba__adqui_adab3c8bf1d5be6b8523065ccd474765a}{\_\-\_\-init\_\-\_\-}
\begin{DoxyCompactList}\small\item\em para cargar el XML de gtk+ y sus senales \end{DoxyCompactList}\item 
def \hyperlink{classinicio_1_1prueba__adqui_af626267088da2ac8ad96d85b91850c05}{izq\_\-clicked}
\item 
def \hyperlink{classinicio_1_1prueba__adqui_ad5348d604e42fe8466b4c1f7338068bf}{der\_\-clicked}
\item 
def \hyperlink{classinicio_1_1prueba__adqui_add1484bca105dc4a14533c487e6f7a78}{arriba\_\-clicked}
\item 
def \hyperlink{classinicio_1_1prueba__adqui_abac92d1dcf8c253686ddbf0ef741f692}{abajo\_\-clicked}
\item 
def \hyperlink{classinicio_1_1prueba__adqui_a133dddf01f8261f6cca38064b8ae006f}{parar\_\-clicked}
\item 
def \hyperlink{classinicio_1_1prueba__adqui_a903f8caf83ea99a6a35a31089cee5944}{button1\_\-clicked}
\item 
def \hyperlink{classinicio_1_1prueba__adqui_a24ae8befd66a50575811e300464e7f1a}{generador\_\-mapa}
\item 
def \hyperlink{classinicio_1_1prueba__adqui_a448fd223767febb13d3f929fcddc85cf}{on\_\-maps\_\-destroy}
\end{DoxyCompactItemize}
\subsection*{Atributos públicos}
\begin{DoxyCompactItemize}
\item 
\hyperlink{classinicio_1_1prueba__adqui_a7c099b095d3893076bcd6dcc22aa5ce4}{aument}
\item 
\hyperlink{classinicio_1_1prueba__adqui_a32103b146bd571e752923638e680db8b}{teleoper}
\item 
\hyperlink{classinicio_1_1prueba__adqui_abc25d678f19639848ca3b7509a842566}{boton}
\item 
\hyperlink{classinicio_1_1prueba__adqui_a1f0a7213982dc7228773c19258e425f4}{area}
\item 
\hyperlink{classinicio_1_1prueba__adqui_a1f0c7dfba4bed8128426c5058c255dbe}{imagen}
\item 
\hyperlink{classinicio_1_1prueba__adqui_a073ec339511e7a2660c9ae92c613f293}{nombre\_\-archivo}
\item 
\hyperlink{classinicio_1_1prueba__adqui_a5a6710e93f733c84b360e42513fdd4a9}{a}
\item 
\hyperlink{classinicio_1_1prueba__adqui_a3519a8ae5deb71530289fc1274500b7e}{cliente}
\item 
\hyperlink{classinicio_1_1prueba__adqui_aafd8544e61c02137d45d1202e4330da5}{valores\_\-f}
\end{DoxyCompactItemize}


\subsection{Descripción detallada}
es la clase encargada del entorno grafico y enlace con \hyperlink{namespacecliente__lib}{cliente\_\-lib} 

Definición en la línea 55 del archivo inicio.py.



\subsection{Documentación del constructor y destructor}
\hypertarget{classinicio_1_1prueba__adqui_adab3c8bf1d5be6b8523065ccd474765a}{
\index{inicio::prueba\_\-adqui@{inicio::prueba\_\-adqui}!\_\-\_\-init\_\-\_\-@{\_\-\_\-init\_\-\_\-}}
\index{\_\-\_\-init\_\-\_\-@{\_\-\_\-init\_\-\_\-}!inicio::prueba_adqui@{inicio::prueba\_\-adqui}}
\subsubsection[{\_\-\_\-init\_\-\_\-}]{\setlength{\rightskip}{0pt plus 5cm}def inicio.prueba\_\-adqui.\_\-\_\-init\_\-\_\- (
\begin{DoxyParamCaption}
\item[{}]{self, }
\item[{}]{dimensiones, }
\item[{}]{nombre\_\-archivo, }
\item[{}]{dimensiones\_\-robot}
\end{DoxyParamCaption}
)}}
\label{classinicio_1_1prueba__adqui_adab3c8bf1d5be6b8523065ccd474765a}


para cargar el XML de gtk+ y sus senales 


\begin{DoxyParams}{Parámetros}
{\em self} & no se necesita incluirlo al utilizar la funcion ya que se lo pone solo por ser la definicion de una clase \\
\hline
\end{DoxyParams}
\begin{DoxyReturn}{Devuelve}
self 
\end{DoxyReturn}


Definición en la línea 62 del archivo inicio.py.


\begin{DoxyCode}
63                                                                        :
64                 self.aument=0
65                 builder = gtk.Builder() #El archivo de glade debe estar en gtkbui
      lder
66                 builder.add_from_file("glade/GUI.glade") #Carga el archivo glade
67                 builder.connect_signals(self) #Toma todas las senales de glade
68                 self.teleoper = builder.get_object("map") #Ventana principal
69                 self.boton = builder.get_object("button1")
70                 self.area=builder.get_object("mapa")
71                 self.teleoper.show()
72                 #Inicializa cliente d-bus
73                 #self.client_dbus = DBusClient()
74                 #
75                 #Inicializar visor
76                 #self.client_dbus.visor()
77                 #
78                 #imnicio opencv
79                 self.imagen=mapa(dimensiones,nombre_archivo,dimensiones_robot)
80                 self.nombre_archivo=nombre_archivo
81                 self.imagen.cargar()
82                 #
83                 self.a=cliente_lib();
84                 self.a.ip="192.168.1.102"
85                 self.cliente=self.a.cliente_inicio()
86                 #self.a.cliente_apaga(self.cliente)
87                 try:
88                         self.cliente=self.a.envio_consulta_fisica(self.cliente,"p
      ose")
89                         #valor_x,valor_y,valor_t=self.a.devuelve_valors()
90                         valor_x=self.a.x
91                         valor_y=self.a.y
92                         ArUtil.sleep(100) #cambiar estos comandos por temporisado
      res de python
93                         #sleep(0.1)
94                         self.cliente=self.a.envio_consulta_fisica(self.cliente,"u
      pdateNumbers")
95                         self.valores_f=self.a.devuelve_valorf()
96                         mi_x=self.valores_f[1]
97                         mi_y=self.valores_f[2]
98                         mi_th=self.valores_f[3]
99                         #print valor_x
100                         #print valor_y
101                         print len(valor_x)
102                         self.generador_mapa(valor_x,valor_y,mi_x,mi_y,mi_th)
103                 except:
104                         print "Fallo inicio"
                        pass
\end{DoxyCode}


\subsection{Documentación de las funciones miembro}
\hypertarget{classinicio_1_1prueba__adqui_abac92d1dcf8c253686ddbf0ef741f692}{
\index{inicio::prueba\_\-adqui@{inicio::prueba\_\-adqui}!abajo\_\-clicked@{abajo\_\-clicked}}
\index{abajo\_\-clicked@{abajo\_\-clicked}!inicio::prueba_adqui@{inicio::prueba\_\-adqui}}
\subsubsection[{abajo\_\-clicked}]{\setlength{\rightskip}{0pt plus 5cm}def inicio.prueba\_\-adqui.abajo\_\-clicked (
\begin{DoxyParamCaption}
\item[{}]{self, }
\item[{}]{widget, }
\item[{}]{data = {\ttfamily None}}
\end{DoxyParamCaption}
)}}
\label{classinicio_1_1prueba__adqui_abac92d1dcf8c253686ddbf0ef741f692}


Definición en la línea 177 del archivo inicio.py.


\begin{DoxyCode}
178                                                   :
179                 try:
180                         print "Presiono abajo"
181                         TransRatio,RotRatio,LatRatio = [-50,0,0]
182                         self.cliente=self.a.envio_ratioDrive(self.cliente,TransRa
      tio,RotRatio,LatRatio) #fijar los valores para mover
183                         ArUtil.sleep(100)
184                         #sleep(0.1)
185                         self.cliente=self.a.envio_consulta_fisica(self.cliente,"p
      ose")
186                         #valor_x,valor_y,valor_t=self.a.devuelve_valors()
187                         valor_x=self.a.x
188                         valor_y=self.a.y
189                         self.cliente=self.a.envio_consulta_fisica(self.cliente,"u
      pdateNumbers")
190                         self.valores_f=self.a.devuelve_valorf()
191                         mi_x=self.valores_f[1]
192                         mi_y=self.valores_f[2]
193                         mi_th=self.valores_f[3]
194                         #print self.valores_f
195                         #print valor_x
196                         #print valor_y
197                         print len(valor_x)
198                         self.generador_mapa(valor_x,valor_y,mi_x,mi_y,mi_th)
199                 except:
200                         print "Fallo"
                        pass
\end{DoxyCode}
\hypertarget{classinicio_1_1prueba__adqui_add1484bca105dc4a14533c487e6f7a78}{
\index{inicio::prueba\_\-adqui@{inicio::prueba\_\-adqui}!arriba\_\-clicked@{arriba\_\-clicked}}
\index{arriba\_\-clicked@{arriba\_\-clicked}!inicio::prueba_adqui@{inicio::prueba\_\-adqui}}
\subsubsection[{arriba\_\-clicked}]{\setlength{\rightskip}{0pt plus 5cm}def inicio.prueba\_\-adqui.arriba\_\-clicked (
\begin{DoxyParamCaption}
\item[{}]{self, }
\item[{}]{widget, }
\item[{}]{data = {\ttfamily None}}
\end{DoxyParamCaption}
)}}
\label{classinicio_1_1prueba__adqui_add1484bca105dc4a14533c487e6f7a78}


Definición en la línea 153 del archivo inicio.py.


\begin{DoxyCode}
154                                                    :
155                 try:
156                         print "Presiono arriba"
157                         TransRatio,RotRatio,LatRatio = [50,0,0]
158                         self.cliente=self.a.envio_ratioDrive(self.cliente,TransRa
      tio,RotRatio,LatRatio) #fijar los valores para mover
159                         ArUtil.sleep(100)
160                         #sleep(0.1)
161                         self.cliente=self.a.envio_consulta_fisica(self.cliente,"p
      ose")
162                         #valor_x,valor_y,valor_t=self.a.devuelve_valors()
163                         valor_x=self.a.x
164                         valor_y=self.a.y
165                         self.cliente=self.a.envio_consulta_fisica(self.cliente,"u
      pdateNumbers")
166                         self.valores_f=self.a.devuelve_valorf()
167                         mi_x=self.valores_f[1]
168                         mi_y=self.valores_f[2]
169                         mi_th=self.valores_f[3]
170                         #print self.valores_f
171                         #print valor_x
172                         #print valor_y
173                         print len(valor_x)
174                         self.generador_mapa(valor_x,valor_y,mi_x,mi_y,mi_th)
175                 except:
176                         print "Fallo"
                        pass
\end{DoxyCode}
\hypertarget{classinicio_1_1prueba__adqui_a903f8caf83ea99a6a35a31089cee5944}{
\index{inicio::prueba\_\-adqui@{inicio::prueba\_\-adqui}!button1\_\-clicked@{button1\_\-clicked}}
\index{button1\_\-clicked@{button1\_\-clicked}!inicio::prueba_adqui@{inicio::prueba\_\-adqui}}
\subsubsection[{button1\_\-clicked}]{\setlength{\rightskip}{0pt plus 5cm}def inicio.prueba\_\-adqui.button1\_\-clicked (
\begin{DoxyParamCaption}
\item[{}]{self, }
\item[{}]{widget, }
\item[{}]{data = {\ttfamily None}}
\end{DoxyParamCaption}
)}}
\label{classinicio_1_1prueba__adqui_a903f8caf83ea99a6a35a31089cee5944}


Definición en la línea 224 del archivo inicio.py.


\begin{DoxyCode}
225                                                     :
226                 try:
227                         self.cliente=self.a.envio_consulta_fisica(self.cliente,"p
      ose")
228                         ArUtil.sleep(100)
229                         #sleep(0.1)
230                         #valor_x,valor_y,valor_t=self.a.devuelve_valors()
231                         valor_x=self.a.x
232                         valor_y=self.a.y
233                         self.cliente=self.a.envio_consulta_fisica(self.cliente,"u
      pdateNumbers")
234                         self.valores_f=self.a.devuelve_valorf()
235                         mi_x=self.valores_f[1]
236                         mi_y=self.valores_f[2]
237                         mi_th=self.valores_f[3]
238                         #print self.valores_f
239                         #print valor_x
240                         #print valor_y
241                         print len(valor_x)
242                         self.generador_mapa(valor_x,valor_y,mi_x,mi_y,mi_th)
243                 except:
244                         print "Fallo paro"
245                         pass

\end{DoxyCode}
\hypertarget{classinicio_1_1prueba__adqui_ad5348d604e42fe8466b4c1f7338068bf}{
\index{inicio::prueba\_\-adqui@{inicio::prueba\_\-adqui}!der\_\-clicked@{der\_\-clicked}}
\index{der\_\-clicked@{der\_\-clicked}!inicio::prueba_adqui@{inicio::prueba\_\-adqui}}
\subsubsection[{der\_\-clicked}]{\setlength{\rightskip}{0pt plus 5cm}def inicio.prueba\_\-adqui.der\_\-clicked (
\begin{DoxyParamCaption}
\item[{}]{self, }
\item[{}]{widget, }
\item[{}]{data = {\ttfamily None}}
\end{DoxyParamCaption}
)}}
\label{classinicio_1_1prueba__adqui_ad5348d604e42fe8466b4c1f7338068bf}


Definición en la línea 129 del archivo inicio.py.


\begin{DoxyCode}
130                                                 :
131                 try:
132                         print "presiono derecha"
133                         TransRatio,RotRatio,LatRatio = [0,-90,-90]
134                         self.cliente=self.a.envio_ratioDrive(self.cliente,TransRa
      tio,RotRatio,LatRatio) #fijar los valores para mover
135                         ArUtil.sleep(100)
136                         #sleep(0.1)
137                         self.cliente=self.a.envio_consulta_fisica(self.cliente,"p
      ose")
138                         #valor_x,valor_y,valor_t=self.a.devuelve_valors()
139                         valor_x=self.a.x
140                         valor_y=self.a.y
141                         self.cliente=self.a.envio_consulta_fisica(self.cliente,"u
      pdateNumbers")
142                         self.valores_f=self.a.devuelve_valorf()
143                         mi_x=self.valores_f[1]
144                         mi_y=self.valores_f[2]
145                         mi_th=self.valores_f[3]
146                         #print self.valores_f
147                         #print valor_x
148                         #print valor_y
149                         print len(valor_x)
150                         self.generador_mapa(valor_x,valor_y,mi_x,mi_y,mi_th)
151                 except:
152                         print "Fallo"
                        pass
\end{DoxyCode}
\hypertarget{classinicio_1_1prueba__adqui_a24ae8befd66a50575811e300464e7f1a}{
\index{inicio::prueba\_\-adqui@{inicio::prueba\_\-adqui}!generador\_\-mapa@{generador\_\-mapa}}
\index{generador\_\-mapa@{generador\_\-mapa}!inicio::prueba_adqui@{inicio::prueba\_\-adqui}}
\subsubsection[{generador\_\-mapa}]{\setlength{\rightskip}{0pt plus 5cm}def inicio.prueba\_\-adqui.generador\_\-mapa (
\begin{DoxyParamCaption}
\item[{}]{self, }
\item[{}]{valor\_\-x, }
\item[{}]{valor\_\-y, }
\item[{}]{mi\_\-x, }
\item[{}]{mi\_\-y, }
\item[{}]{mi\_\-th}
\end{DoxyParamCaption}
)}}
\label{classinicio_1_1prueba__adqui_a24ae8befd66a50575811e300464e7f1a}


Definición en la línea 246 del archivo inicio.py.


\begin{DoxyCode}
247                                                                 :
248                 try:
249                         print valor_x
250                         if (len(valor_x)>0):
251                                 for a in range(len(valor_x)): #SIP en c. 
252                                         self.imagen.anadir_punto((valor_x['x%d' %
       a]/100+300,valor_y['y%d' % a]/100+200),radio=2)
253                         self.imagen.crear_imagen()
254                         self.imagen.rotacion_y_posicion_robot(mi_x/100+300,mi_y/1
      00+200,mi_th)
255                         self.area.set_from_file(self.nombre_archivo)
256                 except:
257                         print "Fallo mapa"
258                         pass
        
\end{DoxyCode}
\hypertarget{classinicio_1_1prueba__adqui_af626267088da2ac8ad96d85b91850c05}{
\index{inicio::prueba\_\-adqui@{inicio::prueba\_\-adqui}!izq\_\-clicked@{izq\_\-clicked}}
\index{izq\_\-clicked@{izq\_\-clicked}!inicio::prueba_adqui@{inicio::prueba\_\-adqui}}
\subsubsection[{izq\_\-clicked}]{\setlength{\rightskip}{0pt plus 5cm}def inicio.prueba\_\-adqui.izq\_\-clicked (
\begin{DoxyParamCaption}
\item[{}]{self, }
\item[{}]{widget, }
\item[{}]{data = {\ttfamily None}}
\end{DoxyParamCaption}
)}}
\label{classinicio_1_1prueba__adqui_af626267088da2ac8ad96d85b91850c05}


Definición en la línea 105 del archivo inicio.py.


\begin{DoxyCode}
106                                                 :
107                 try:
108                         print "presiono izquierda"
109                         TransRatio,RotRatio,LatRatio = [0,90,90]
110                         self.cliente=self.a.envio_ratioDrive(self.cliente,TransRa
      tio,RotRatio,LatRatio) #fijar los valores para mover
111                         ArUtil.sleep(100)
112                         #sleep(0.1)
113                         self.cliente=self.a.envio_consulta_fisica(self.cliente,"p
      ose")
114                         #valor_x,valor_y,valor_t=self.a.devuelve_valors()
115                         valor_x=self.a.x
116                         valor_y=self.a.y
117                         self.cliente=self.a.envio_consulta_fisica(self.cliente,"u
      pdateNumbers")
118                         self.valores_f=self.a.devuelve_valorf()
119                         mi_x=self.valores_f[1]
120                         mi_y=self.valores_f[2]
121                         mi_th=self.valores_f[3]
122                         #print self.valores_f
123                         #print valor_x
124                         #print valor_y
125                         print len(valor_x)
126                         self.generador_mapa(valor_x,valor_y,mi_x,mi_y,mi_th)
127                 except:
128                         print "Fallo init"
                        pass
\end{DoxyCode}
\hypertarget{classinicio_1_1prueba__adqui_a448fd223767febb13d3f929fcddc85cf}{
\index{inicio::prueba\_\-adqui@{inicio::prueba\_\-adqui}!on\_\-maps\_\-destroy@{on\_\-maps\_\-destroy}}
\index{on\_\-maps\_\-destroy@{on\_\-maps\_\-destroy}!inicio::prueba_adqui@{inicio::prueba\_\-adqui}}
\subsubsection[{on\_\-maps\_\-destroy}]{\setlength{\rightskip}{0pt plus 5cm}def inicio.prueba\_\-adqui.on\_\-maps\_\-destroy (
\begin{DoxyParamCaption}
\item[{}]{self, }
\item[{}]{widget, }
\item[{}]{data = {\ttfamily None}}
\end{DoxyParamCaption}
)}}
\label{classinicio_1_1prueba__adqui_a448fd223767febb13d3f929fcddc85cf}


Definición en la línea 259 del archivo inicio.py.


\begin{DoxyCode}
260                                                     :
261                 print self.a.cliente_apaga(self.cliente)
                gtk.main_quit()
\end{DoxyCode}
\hypertarget{classinicio_1_1prueba__adqui_a133dddf01f8261f6cca38064b8ae006f}{
\index{inicio::prueba\_\-adqui@{inicio::prueba\_\-adqui}!parar\_\-clicked@{parar\_\-clicked}}
\index{parar\_\-clicked@{parar\_\-clicked}!inicio::prueba_adqui@{inicio::prueba\_\-adqui}}
\subsubsection[{parar\_\-clicked}]{\setlength{\rightskip}{0pt plus 5cm}def inicio.prueba\_\-adqui.parar\_\-clicked (
\begin{DoxyParamCaption}
\item[{}]{self, }
\item[{}]{widget, }
\item[{}]{data = {\ttfamily None}}
\end{DoxyParamCaption}
)}}
\label{classinicio_1_1prueba__adqui_a133dddf01f8261f6cca38064b8ae006f}


Definición en la línea 201 del archivo inicio.py.


\begin{DoxyCode}
202                                                   :
203                 try:
204                         print "Presiono alto"
205                         self.cliente.requestOnce("stop") #parada de emergencia
206                         ArUtil.sleep(100)
207                         #sleep(0.1)
208                         self.cliente=self.a.envio_consulta_fisica(self.cliente,"p
      ose")
209                         #valor_x,valor_y,valor_t=self.a.devuelve_valors()
210                         valor_x=self.a.x
211                         valor_y=self.a.y
212                         self.cliente=self.a.envio_consulta_fisica(self.cliente,"u
      pdateNumbers")
213                         self.valores_f=self.a.devuelve_valorf()
214                         mi_x=self.valores_f[1]
215                         mi_y=self.valores_f[2]
216                         mi_th=self.valores_f[3]
217                         #print self.valores_f
218                         #print valor_x
219                         #print valor_y
220                         print len(valor_x)
221                         self.generador_mapa(valor_x,valor_y,mi_x,mi_y,mi_th)
222                 except:
223                         print "Fallo"
                        pass
\end{DoxyCode}


\subsection{Documentación de los datos miembro}
\hypertarget{classinicio_1_1prueba__adqui_a5a6710e93f733c84b360e42513fdd4a9}{
\index{inicio::prueba\_\-adqui@{inicio::prueba\_\-adqui}!a@{a}}
\index{a@{a}!inicio::prueba_adqui@{inicio::prueba\_\-adqui}}
\subsubsection[{a}]{\setlength{\rightskip}{0pt plus 5cm}{\bf inicio.prueba\_\-adqui.a}}}
\label{classinicio_1_1prueba__adqui_a5a6710e93f733c84b360e42513fdd4a9}


Definición en la línea 62 del archivo inicio.py.

\hypertarget{classinicio_1_1prueba__adqui_a1f0a7213982dc7228773c19258e425f4}{
\index{inicio::prueba\_\-adqui@{inicio::prueba\_\-adqui}!area@{area}}
\index{area@{area}!inicio::prueba_adqui@{inicio::prueba\_\-adqui}}
\subsubsection[{area}]{\setlength{\rightskip}{0pt plus 5cm}{\bf inicio.prueba\_\-adqui.area}}}
\label{classinicio_1_1prueba__adqui_a1f0a7213982dc7228773c19258e425f4}


Definición en la línea 62 del archivo inicio.py.

\hypertarget{classinicio_1_1prueba__adqui_a7c099b095d3893076bcd6dcc22aa5ce4}{
\index{inicio::prueba\_\-adqui@{inicio::prueba\_\-adqui}!aument@{aument}}
\index{aument@{aument}!inicio::prueba_adqui@{inicio::prueba\_\-adqui}}
\subsubsection[{aument}]{\setlength{\rightskip}{0pt plus 5cm}{\bf inicio.prueba\_\-adqui.aument}}}
\label{classinicio_1_1prueba__adqui_a7c099b095d3893076bcd6dcc22aa5ce4}


Definición en la línea 62 del archivo inicio.py.

\hypertarget{classinicio_1_1prueba__adqui_abc25d678f19639848ca3b7509a842566}{
\index{inicio::prueba\_\-adqui@{inicio::prueba\_\-adqui}!boton@{boton}}
\index{boton@{boton}!inicio::prueba_adqui@{inicio::prueba\_\-adqui}}
\subsubsection[{boton}]{\setlength{\rightskip}{0pt plus 5cm}{\bf inicio.prueba\_\-adqui.boton}}}
\label{classinicio_1_1prueba__adqui_abc25d678f19639848ca3b7509a842566}


Definición en la línea 62 del archivo inicio.py.

\hypertarget{classinicio_1_1prueba__adqui_a3519a8ae5deb71530289fc1274500b7e}{
\index{inicio::prueba\_\-adqui@{inicio::prueba\_\-adqui}!cliente@{cliente}}
\index{cliente@{cliente}!inicio::prueba_adqui@{inicio::prueba\_\-adqui}}
\subsubsection[{cliente}]{\setlength{\rightskip}{0pt plus 5cm}{\bf inicio.prueba\_\-adqui.cliente}}}
\label{classinicio_1_1prueba__adqui_a3519a8ae5deb71530289fc1274500b7e}


Definición en la línea 62 del archivo inicio.py.

\hypertarget{classinicio_1_1prueba__adqui_a1f0c7dfba4bed8128426c5058c255dbe}{
\index{inicio::prueba\_\-adqui@{inicio::prueba\_\-adqui}!imagen@{imagen}}
\index{imagen@{imagen}!inicio::prueba_adqui@{inicio::prueba\_\-adqui}}
\subsubsection[{imagen}]{\setlength{\rightskip}{0pt plus 5cm}{\bf inicio.prueba\_\-adqui.imagen}}}
\label{classinicio_1_1prueba__adqui_a1f0c7dfba4bed8128426c5058c255dbe}


Definición en la línea 62 del archivo inicio.py.

\hypertarget{classinicio_1_1prueba__adqui_a073ec339511e7a2660c9ae92c613f293}{
\index{inicio::prueba\_\-adqui@{inicio::prueba\_\-adqui}!nombre\_\-archivo@{nombre\_\-archivo}}
\index{nombre\_\-archivo@{nombre\_\-archivo}!inicio::prueba_adqui@{inicio::prueba\_\-adqui}}
\subsubsection[{nombre\_\-archivo}]{\setlength{\rightskip}{0pt plus 5cm}{\bf inicio.prueba\_\-adqui.nombre\_\-archivo}}}
\label{classinicio_1_1prueba__adqui_a073ec339511e7a2660c9ae92c613f293}


Definición en la línea 62 del archivo inicio.py.

\hypertarget{classinicio_1_1prueba__adqui_a32103b146bd571e752923638e680db8b}{
\index{inicio::prueba\_\-adqui@{inicio::prueba\_\-adqui}!teleoper@{teleoper}}
\index{teleoper@{teleoper}!inicio::prueba_adqui@{inicio::prueba\_\-adqui}}
\subsubsection[{teleoper}]{\setlength{\rightskip}{0pt plus 5cm}{\bf inicio.prueba\_\-adqui.teleoper}}}
\label{classinicio_1_1prueba__adqui_a32103b146bd571e752923638e680db8b}


Definición en la línea 62 del archivo inicio.py.

\hypertarget{classinicio_1_1prueba__adqui_aafd8544e61c02137d45d1202e4330da5}{
\index{inicio::prueba\_\-adqui@{inicio::prueba\_\-adqui}!valores\_\-f@{valores\_\-f}}
\index{valores\_\-f@{valores\_\-f}!inicio::prueba_adqui@{inicio::prueba\_\-adqui}}
\subsubsection[{valores\_\-f}]{\setlength{\rightskip}{0pt plus 5cm}{\bf inicio.prueba\_\-adqui.valores\_\-f}}}
\label{classinicio_1_1prueba__adqui_aafd8544e61c02137d45d1202e4330da5}


Definición en la línea 62 del archivo inicio.py.



La documentación para esta clase fue generada a partir del siguiente fichero:\begin{DoxyCompactItemize}
\item 
\hyperlink{inicio_8py}{inicio.py}\end{DoxyCompactItemize}

\hypertarget{classrenderizado_1_1renderizado}{
\section{Referencia de la Clase renderizado.renderizado}
\label{classrenderizado_1_1renderizado}\index{renderizado::renderizado@{renderizado::renderizado}}
}


es la clase encargada de la generacion del mapa atravez de comandos.  


\subsection*{Métodos públicos}
\begin{DoxyCompactItemize}
\item 
def \hyperlink{classrenderizado_1_1renderizado_a4c17dcbfcfe999944d4bb4829b833b66}{\_\-\_\-init\_\-\_\-}
\begin{DoxyCompactList}\small\item\em Inicio de variable de uso de la libreria. \end{DoxyCompactList}\item 
def \hyperlink{classrenderizado_1_1renderizado_a64752e268753f383d7b9d252070c65f4}{cargar}
\begin{DoxyCompactList}\small\item\em genera el los lienzos del mapa y del robot \end{DoxyCompactList}\item 
def \hyperlink{classrenderizado_1_1renderizado_a52891772883b0606be993163a87b70ae}{anadir\_\-punto}
\begin{DoxyCompactList}\small\item\em crear un punto en el lienzo del mapa \end{DoxyCompactList}\item 
def \hyperlink{classrenderizado_1_1renderizado_a43054b88bb47c71371321e31c406003e}{graficar}
\begin{DoxyCompactList}\small\item\em Comando solo utilizado para la prueba del paquete para visualizar los lienzos. \end{DoxyCompactList}\item 
def \hyperlink{classrenderizado_1_1renderizado_a890510a320079ad0fee8e2e87adda69e}{crear\_\-imagen}
\begin{DoxyCompactList}\small\item\em Crea la imagen para cargarla en el programa con el lienzo ya finalizado. \end{DoxyCompactList}\item 
def \hyperlink{classrenderizado_1_1renderizado_a1b2c9f08cee1a5ee13e0e130f598efc0}{rotacion\_\-y\_\-posicion\_\-robot}
\begin{DoxyCompactList}\small\item\em graficar el robot en el lienzo de mapa \end{DoxyCompactList}\end{DoxyCompactItemize}
\subsection*{Atributos públicos}
\begin{DoxyCompactItemize}
\item 
\hyperlink{classrenderizado_1_1renderizado_af7128f70b3c3a2131bde3e83dbb28da0}{nombre\_\-archivo}
\item 
\hyperlink{classrenderizado_1_1renderizado_a99de1ed3c1a24b5d9a3da2b2905bd571}{nombre\_\-archivo1}
\item 
\hyperlink{classrenderizado_1_1renderizado_a70a61493b255b1624edf30547356c83e}{dimeniones}
\item 
\hyperlink{classrenderizado_1_1renderizado_ab835e8ffdd4128f40fd35258f3bdcfce}{dimensiones\_\-robot}
\item 
\hyperlink{classrenderizado_1_1renderizado_a54c4ebbaf251f0aa7e540f58f5b0f5eb}{image}
\item 
\hyperlink{classrenderizado_1_1renderizado_af50ef1daafbc8bf96205e5487dc54503}{robot}
\end{DoxyCompactItemize}


\subsection{Descripción detallada}
es la clase encargada de la generacion del mapa atravez de comandos. 

Definición en la línea 51 del archivo renderizado.py.



\subsection{Documentación del constructor y destructor}
\hypertarget{classrenderizado_1_1renderizado_a4c17dcbfcfe999944d4bb4829b833b66}{
\index{renderizado::renderizado@{renderizado::renderizado}!\_\-\_\-init\_\-\_\-@{\_\-\_\-init\_\-\_\-}}
\index{\_\-\_\-init\_\-\_\-@{\_\-\_\-init\_\-\_\-}!renderizado::renderizado@{renderizado::renderizado}}
\subsubsection[{\_\-\_\-init\_\-\_\-}]{\setlength{\rightskip}{0pt plus 5cm}def renderizado.renderizado.\_\-\_\-init\_\-\_\- (
\begin{DoxyParamCaption}
\item[{}]{self, }
\item[{}]{dimensiones, }
\item[{}]{nombre\_\-archivo, }
\item[{}]{dimensiones\_\-robot}
\end{DoxyParamCaption}
)}}
\label{classrenderizado_1_1renderizado_a4c17dcbfcfe999944d4bb4829b833b66}


Inicio de variable de uso de la libreria. 


\begin{DoxyParams}{Parámetros}
{\em self} & no se necesita incluirlo al utilizar la funcion ya que se lo pone solo por ser la definicion de una clase \\
\hline
{\em dimensiones} & Es una tupla con las dimensiones de la imagen a generar \\
\hline
{\em nombre\_\-archivo} & Es un string con el nombre de la imagen a crear como por ejemplo imagen.jpg \\
\hline
{\em dimensiones\_\-robot} & Las dimensiones en pixeles de la imagen robot \\
\hline
\end{DoxyParams}
\begin{DoxyReturn}{Devuelve}
Nada 
\end{DoxyReturn}


Definición en la línea 61 del archivo renderizado.py.


\begin{DoxyCode}
62                                                                        :
63                 self.nombre_archivo=nombre_archivo
64                 self.nombre_archivo1=nombre_archivo.replace(".jpg",".acdc")+".jpg
      "
65                 self.dimeniones=dimensiones
66                 self.dimensiones_robot=dimensiones_robot
67                 self.image=cv.CreateImage(dimensiones,8,3)
68                 cv.Not(self.image,self.image)
                cv.SaveImage(self.nombre_archivo1,self.image)
\end{DoxyCode}


\subsection{Documentación de las funciones miembro}
\hypertarget{classrenderizado_1_1renderizado_a52891772883b0606be993163a87b70ae}{
\index{renderizado::renderizado@{renderizado::renderizado}!anadir\_\-punto@{anadir\_\-punto}}
\index{anadir\_\-punto@{anadir\_\-punto}!renderizado::renderizado@{renderizado::renderizado}}
\subsubsection[{anadir\_\-punto}]{\setlength{\rightskip}{0pt plus 5cm}def renderizado.renderizado.anadir\_\-punto (
\begin{DoxyParamCaption}
\item[{}]{self, }
\item[{}]{punto, }
\item[{}]{color = {\ttfamily 0}, }
\item[{}]{radio = {\ttfamily 1}}
\end{DoxyParamCaption}
)}}
\label{classrenderizado_1_1renderizado_a52891772883b0606be993163a87b70ae}


crear un punto en el lienzo del mapa 


\begin{DoxyParams}{Parámetros}
{\em self} & no se necesita incluirlo al utilizar la funcion ya que se lo pone solo por ser la definicion de una clase \\
\hline
{\em punto} & la coordenada donde se va a graficar el circulo \\
\hline
{\em color} & para cambiar el color del circulo, si no se incluye se grafica en negro \\
\hline
{\em radio} & para definir el radio del circulo, si no se incluye es tomado como 1 \\
\hline
\end{DoxyParams}
\begin{DoxyReturn}{Devuelve}
Nada 
\end{DoxyReturn}


Definición en la línea 87 del archivo renderizado.py.


\begin{DoxyCode}
88                                                     :
89                 #font = cv.InitFont(cv.CV_FONT_HERSHEY_SIMPLEX, 1, 1, 0, 3, 8) #C
      reates a font
90                 #x = 30
91                 #y = 40
92                 #cv.PutText(self.image,"Hello World!!!", (x,y),font, 255) #Draw t
      he text
93                 #cv.Line(image,(a*50,60),9(90,90),cv.RGB(17*a, 110-a, 255))
94                 cv.Circle(self.image,punto,radio,cv.RGB(255*color, 255*color, 255
      *color),-1) # ultimo parametro para grosor de la linea negativo lleno
                #self.image1=self.image
\end{DoxyCode}
\hypertarget{classrenderizado_1_1renderizado_a64752e268753f383d7b9d252070c65f4}{
\index{renderizado::renderizado@{renderizado::renderizado}!cargar@{cargar}}
\index{cargar@{cargar}!renderizado::renderizado@{renderizado::renderizado}}
\subsubsection[{cargar}]{\setlength{\rightskip}{0pt plus 5cm}def renderizado.renderizado.cargar (
\begin{DoxyParamCaption}
\item[{}]{self}
\end{DoxyParamCaption}
)}}
\label{classrenderizado_1_1renderizado_a64752e268753f383d7b9d252070c65f4}


genera el los lienzos del mapa y del robot 


\begin{DoxyParams}{Parámetros}
{\em self} & no se necesita incluirlo al utilizar la funcion ya que se lo pone solo por ser la definicion de una clase \\
\hline
\end{DoxyParams}
\begin{DoxyReturn}{Devuelve}
Nada 
\end{DoxyReturn}


Definición en la línea 75 del archivo renderizado.py.


\begin{DoxyCode}
76                         :
77                 self.image=cv.LoadImage(self.nombre_archivo1, cv.CV_LOAD_IMAGE_CO
      LOR) #Load the image
                self.robot=cv.LoadImage("robot.jpg", cv.CV_LOAD_IMAGE_COLOR) #Loa
      d the image
\end{DoxyCode}
\hypertarget{classrenderizado_1_1renderizado_a890510a320079ad0fee8e2e87adda69e}{
\index{renderizado::renderizado@{renderizado::renderizado}!crear\_\-imagen@{crear\_\-imagen}}
\index{crear\_\-imagen@{crear\_\-imagen}!renderizado::renderizado@{renderizado::renderizado}}
\subsubsection[{crear\_\-imagen}]{\setlength{\rightskip}{0pt plus 5cm}def renderizado.renderizado.crear\_\-imagen (
\begin{DoxyParamCaption}
\item[{}]{self}
\end{DoxyParamCaption}
)}}
\label{classrenderizado_1_1renderizado_a890510a320079ad0fee8e2e87adda69e}


Crea la imagen para cargarla en el programa con el lienzo ya finalizado. 


\begin{DoxyParams}{Parámetros}
{\em self} & no se necesita incluirlo al utilizar la funcion ya que se lo pone solo por ser la definicion de una clase \\
\hline
\end{DoxyParams}
\begin{DoxyReturn}{Devuelve}
Nada 
\end{DoxyReturn}


Definición en la línea 116 del archivo renderizado.py.


\begin{DoxyCode}
117                               :
118                 cv.SaveImage(self.nombre_archivo1, self.image) #Saves the image#
                
\end{DoxyCode}
\hypertarget{classrenderizado_1_1renderizado_a43054b88bb47c71371321e31c406003e}{
\index{renderizado::renderizado@{renderizado::renderizado}!graficar@{graficar}}
\index{graficar@{graficar}!renderizado::renderizado@{renderizado::renderizado}}
\subsubsection[{graficar}]{\setlength{\rightskip}{0pt plus 5cm}def renderizado.renderizado.graficar (
\begin{DoxyParamCaption}
\item[{}]{self, }
\item[{}]{tiempo\_\-ms, }
\item[{}]{imagen = {\ttfamily 0}}
\end{DoxyParamCaption}
)}}
\label{classrenderizado_1_1renderizado_a43054b88bb47c71371321e31c406003e}


Comando solo utilizado para la prueba del paquete para visualizar los lienzos. 


\begin{DoxyParams}{Parámetros}
{\em self} & no se necesita incluirlo al utilizar la funcion ya que se lo pone solo por ser la definicion de una clase \\
\hline
{\em tiempo\_\-ms} & tiempo en milisegundos que se va a mostrar la imagen \\
\hline
{\em imagen} & es el lienzo que va a ser graficado si no se lo incluye es tomado como cero. \\
\hline
\end{DoxyParams}
\begin{DoxyReturn}{Devuelve}
Nada 
\end{DoxyReturn}


Definición en la línea 103 del archivo renderizado.py.


\begin{DoxyCode}
104                                              :
105                 if imagen==0:
106                         cv.ShowImage('Mapa Robot',self.image) #Show the image
107                         cv.WaitKey(tiempo_ms)
108                 else:
109                         cv.ShowImage('Mapa Robot',imagen) #Show the image
                        cv.WaitKey(tiempo_ms)
\end{DoxyCode}
\hypertarget{classrenderizado_1_1renderizado_a1b2c9f08cee1a5ee13e0e130f598efc0}{
\index{renderizado::renderizado@{renderizado::renderizado}!rotacion\_\-y\_\-posicion\_\-robot@{rotacion\_\-y\_\-posicion\_\-robot}}
\index{rotacion\_\-y\_\-posicion\_\-robot@{rotacion\_\-y\_\-posicion\_\-robot}!renderizado::renderizado@{renderizado::renderizado}}
\subsubsection[{rotacion\_\-y\_\-posicion\_\-robot}]{\setlength{\rightskip}{0pt plus 5cm}def renderizado.renderizado.rotacion\_\-y\_\-posicion\_\-robot (
\begin{DoxyParamCaption}
\item[{}]{self, }
\item[{}]{robo\_\-x = {\ttfamily 200}, }
\item[{}]{robo\_\-y = {\ttfamily 100}, }
\item[{}]{robo\_\-th = {\ttfamily 80}}
\end{DoxyParamCaption}
)}}
\label{classrenderizado_1_1renderizado_a1b2c9f08cee1a5ee13e0e130f598efc0}


graficar el robot en el lienzo de mapa 


\begin{DoxyParams}{Parámetros}
{\em self} & no se necesita incluirlo al utilizar la funcion ya que se lo pone solo por ser la definicion de una clase \\
\hline
{\em robo\_\-x} & coordenada x de la odometria del robot \\
\hline
{\em robo\_\-y} & coordenada y de la odometria del robot \\
\hline
{\em robo\_\-th} & Valor Th del robot \\
\hline
\end{DoxyParams}
\begin{DoxyReturn}{Devuelve}
Nada 
\end{DoxyReturn}


Definición en la línea 128 del archivo renderizado.py.


\begin{DoxyCode}
129                                                                             :
130                 image_mapa=cv.LoadImage(self.nombre_archivo1, cv.CV_LOAD_IMAGE_CO
      LOR)
131                 dimensiones_robot=self.dimensiones_robot
132                 image1=cv.CreateImage(dimensiones_robot,8,3)
133                 image_mascara=cv.CreateImage(dimensiones_robot,8,1)
134                 
135                 ##rotacion
136                 #Rotar el robot
137                 src_center=dimensiones_robot[0]/2,dimensiones_robot[1]/2
138                 rot_mat=cv.CreateMat( 2, 3, cv.CV_32FC1 )
139                 cv.GetRotationMatrix2D(src_center, robo_th, 1.0,rot_mat);
140                 cv.WarpAffine(self.robot,image1,rot_mat)
141                 #crear filtro para negro
142                 cv.InRangeS(image1,cv.RGB(0,0,0),cv.RGB(14,14,14),image_mascara)
143                 cv.Not(image_mascara,image_mascara)
144                 #cv.ReleaseImage(image1)
145                 
146                 #reducir y posicion
147                 cv.SetImageROI(image_mapa,(robo_x,robo_y, dimensiones_robot[0], d
      imensiones_robot[1]));
148                 cv.Copy(image1,image_mapa,mask=image_mascara)
149                 cv.ResetImageROI(image_mapa);
150                 cv.SaveImage(self.nombre_archivo, image_mapa) #Saves the image#
151                 #self.graficar(2000,image_mapa)
152                 

\end{DoxyCode}


\subsection{Documentación de los datos miembro}
\hypertarget{classrenderizado_1_1renderizado_a70a61493b255b1624edf30547356c83e}{
\index{renderizado::renderizado@{renderizado::renderizado}!dimeniones@{dimeniones}}
\index{dimeniones@{dimeniones}!renderizado::renderizado@{renderizado::renderizado}}
\subsubsection[{dimeniones}]{\setlength{\rightskip}{0pt plus 5cm}{\bf renderizado.renderizado.dimeniones}}}
\label{classrenderizado_1_1renderizado_a70a61493b255b1624edf30547356c83e}


Definición en la línea 61 del archivo renderizado.py.

\hypertarget{classrenderizado_1_1renderizado_ab835e8ffdd4128f40fd35258f3bdcfce}{
\index{renderizado::renderizado@{renderizado::renderizado}!dimensiones\_\-robot@{dimensiones\_\-robot}}
\index{dimensiones\_\-robot@{dimensiones\_\-robot}!renderizado::renderizado@{renderizado::renderizado}}
\subsubsection[{dimensiones\_\-robot}]{\setlength{\rightskip}{0pt plus 5cm}{\bf renderizado.renderizado.dimensiones\_\-robot}}}
\label{classrenderizado_1_1renderizado_ab835e8ffdd4128f40fd35258f3bdcfce}


Definición en la línea 61 del archivo renderizado.py.

\hypertarget{classrenderizado_1_1renderizado_a54c4ebbaf251f0aa7e540f58f5b0f5eb}{
\index{renderizado::renderizado@{renderizado::renderizado}!image@{image}}
\index{image@{image}!renderizado::renderizado@{renderizado::renderizado}}
\subsubsection[{image}]{\setlength{\rightskip}{0pt plus 5cm}{\bf renderizado.renderizado.image}}}
\label{classrenderizado_1_1renderizado_a54c4ebbaf251f0aa7e540f58f5b0f5eb}


Definición en la línea 61 del archivo renderizado.py.

\hypertarget{classrenderizado_1_1renderizado_af7128f70b3c3a2131bde3e83dbb28da0}{
\index{renderizado::renderizado@{renderizado::renderizado}!nombre\_\-archivo@{nombre\_\-archivo}}
\index{nombre\_\-archivo@{nombre\_\-archivo}!renderizado::renderizado@{renderizado::renderizado}}
\subsubsection[{nombre\_\-archivo}]{\setlength{\rightskip}{0pt plus 5cm}{\bf renderizado.renderizado.nombre\_\-archivo}}}
\label{classrenderizado_1_1renderizado_af7128f70b3c3a2131bde3e83dbb28da0}


Definición en la línea 61 del archivo renderizado.py.

\hypertarget{classrenderizado_1_1renderizado_a99de1ed3c1a24b5d9a3da2b2905bd571}{
\index{renderizado::renderizado@{renderizado::renderizado}!nombre\_\-archivo1@{nombre\_\-archivo1}}
\index{nombre\_\-archivo1@{nombre\_\-archivo1}!renderizado::renderizado@{renderizado::renderizado}}
\subsubsection[{nombre\_\-archivo1}]{\setlength{\rightskip}{0pt plus 5cm}{\bf renderizado.renderizado.nombre\_\-archivo1}}}
\label{classrenderizado_1_1renderizado_a99de1ed3c1a24b5d9a3da2b2905bd571}


Definición en la línea 61 del archivo renderizado.py.

\hypertarget{classrenderizado_1_1renderizado_af50ef1daafbc8bf96205e5487dc54503}{
\index{renderizado::renderizado@{renderizado::renderizado}!robot@{robot}}
\index{robot@{robot}!renderizado::renderizado@{renderizado::renderizado}}
\subsubsection[{robot}]{\setlength{\rightskip}{0pt plus 5cm}{\bf renderizado.renderizado.robot}}}
\label{classrenderizado_1_1renderizado_af50ef1daafbc8bf96205e5487dc54503}


Definición en la línea 75 del archivo renderizado.py.



La documentación para esta clase fue generada a partir del siguiente fichero:\begin{DoxyCompactItemize}
\item 
\hyperlink{renderizado_8py}{renderizado.py}\end{DoxyCompactItemize}

\chapter{Documentación de archivos}
\hypertarget{cliente__lib_8py}{
\section{Referencia del Archivo cliente\_\-lib.py}
\label{cliente__lib_8py}\index{cliente\_\-lib.py@{cliente\_\-lib.py}}
}
\subsection*{Clases}
\begin{DoxyCompactItemize}
\item 
class \hyperlink{classcliente__lib_1_1cliente__lib}{cliente\_\-lib.cliente\_\-lib}
\begin{DoxyCompactList}\small\item\em es la clase encargada del cliente \end{DoxyCompactList}\end{DoxyCompactItemize}
\subsection*{Paquetes}
\begin{DoxyCompactItemize}
\item 
package \hyperlink{namespacecliente__lib}{cliente\_\-lib}


\begin{DoxyCompactList}\small\item\em libreria para realizar el cliente \end{DoxyCompactList}

\end{DoxyCompactItemize}
\subsection*{Funciones}
\begin{DoxyCompactItemize}
\item 
def \hyperlink{namespacecliente__lib_afb746084e43cb9c21db470d7b4990cae}{cliente\_\-lib.main}
\begin{DoxyCompactList}\small\item\em Sirve para realizar pruebas de conexion. \end{DoxyCompactList}\end{DoxyCompactItemize}

\hypertarget{gamepad_8py}{
\section{Referencia del Archivo gamepad.py}
\label{gamepad_8py}\index{gamepad.py@{gamepad.py}}
}
\subsection*{Clases}
\begin{DoxyCompactItemize}
\item 
class \hyperlink{classgamepad_1_1gamepad}{gamepad.gamepad}
\end{DoxyCompactItemize}
\subsection*{Paquetes}
\begin{DoxyCompactItemize}
\item 
package \hyperlink{namespacegamepad}{gamepad}
\end{DoxyCompactItemize}
\subsection*{Variables}
\begin{DoxyCompactItemize}
\item 
tuple \hyperlink{namespacegamepad_ac2d3a197e612c4d688cc70b717d040f4}{gamepad.c} = gamepad()
\item 
tuple \hyperlink{namespacegamepad_a71ea7805d13c1a442407730008cf5247}{gamepad.a} = c.gamepad\_\-lectura(pipe,msg)
\end{DoxyCompactItemize}

\hypertarget{inicio_8py}{
\section{Referencia del Archivo inicio.py}
\label{inicio_8py}\index{inicio.py@{inicio.py}}
}
\subsection*{Clases}
\begin{DoxyCompactItemize}
\item 
class \hyperlink{classinicio_1_1prueba__teleoperacion}{inicio.prueba\_\-teleoperacion}
\begin{DoxyCompactList}\small\item\em Se lo crea como objeto para poder trabajar con las senales de la interfaz grafica. \end{DoxyCompactList}\end{DoxyCompactItemize}
\subsection*{Paquetes}
\begin{DoxyCompactItemize}
\item 
package \hyperlink{namespaceinicio}{inicio}
\end{DoxyCompactItemize}
\subsection*{Variables}
\begin{DoxyCompactItemize}
\item 
tuple \hyperlink{namespaceinicio_a3d259595825f914437642eb35265f3ad}{inicio.app} = prueba\_\-teleoperacion()
\end{DoxyCompactItemize}

\hypertarget{mainpage_8dox}{
\section{Referencia del Archivo mainpage.dox}
\label{mainpage_8dox}\index{mainpage.dox@{mainpage.dox}}
}

\hypertarget{renderizado_8py}{
\section{Referencia del Archivo renderizado.py}
\label{renderizado_8py}\index{renderizado.py@{renderizado.py}}
}
\subsection*{Clases}
\begin{DoxyCompactItemize}
\item 
class \hyperlink{classrenderizado_1_1renderizado}{renderizado.renderizado}
\begin{DoxyCompactList}\small\item\em es la clase encargada de la generacion del mapa atravez de comandos. \end{DoxyCompactList}\end{DoxyCompactItemize}
\subsection*{Paquetes}
\begin{DoxyCompactItemize}
\item 
package \hyperlink{namespacerenderizado}{renderizado}


\begin{DoxyCompactList}\small\item\em programa que utiliza la libreria openCV para la generacion del lienzo \end{DoxyCompactList}

\end{DoxyCompactItemize}
\subsection*{Variables}
\begin{DoxyCompactItemize}
\item 
tuple \hyperlink{namespacerenderizado_a67d5237e7cc5eff92a6e3b9d4782723e}{renderizado.a} = renderizado((600,400),\char`\"{}imagen.jpg\char`\"{},(13,13))
\end{DoxyCompactItemize}

\hypertarget{servidor__novo_8py}{
\section{Referencia del Archivo servidor\_\-novo.py}
\label{servidor__novo_8py}\index{servidor\_\-novo.py@{servidor\_\-novo.py}}
}
\subsection*{Paquetes}
\begin{DoxyCompactItemize}
\item 
package \hyperlink{namespaceservidor__novo}{servidor\_\-novo}


\begin{DoxyCompactList}\small\item\em Servidor para el Pioneer P3-\/DX. \end{DoxyCompactList}

\end{DoxyCompactItemize}
\subsection*{Funciones}
\begin{DoxyCompactItemize}
\item 
def \hyperlink{namespaceservidor__novo_a994b4e4dc22448a609ecd86baa9ff6da}{servidor\_\-novo.requestCallback}
\begin{DoxyCompactList}\small\item\em Sirve cuando se manda el comando \char`\"{}test\char`\"{} en el paquete. \end{DoxyCompactList}\item 
def \hyperlink{namespaceservidor__novo_ae02aab4783a9544975ff25e7d2fde80a}{servidor\_\-novo.movimiento}
\begin{DoxyCompactList}\small\item\em Sirve cuando se manda el comando \char`\"{}mover\char`\"{} en el paquete. \end{DoxyCompactList}\item 
def \hyperlink{namespaceservidor__novo_a8af363dfe669e9277832c4491d253db3}{servidor\_\-novo.rotar}
\begin{DoxyCompactList}\small\item\em Sirve cuando se manda el comando \char`\"{}rotar\char`\"{} en el paquete. \end{DoxyCompactList}\item 
def \hyperlink{namespaceservidor__novo_a6bb7f16cf80ec7b45c715da84a62257d}{servidor\_\-novo.posicion}
\begin{DoxyCompactList}\small\item\em Sirve cuando se manda el comando \char`\"{}pose\char`\"{} en el paquete. \end{DoxyCompactList}\end{DoxyCompactItemize}
\subsection*{Variables}
\begin{DoxyCompactItemize}
\item 
tuple \hyperlink{namespaceservidor__novo_ae604d92f7f5c43d2a993ae9402534424}{servidor\_\-novo.robot} = ArRobot()
\item 
tuple \hyperlink{namespaceservidor__novo_a2b0e3a19479a3a456184ce1f64a648a0}{servidor\_\-novo.gyro} = ArAnalogGyro(robot)
\item 
tuple \hyperlink{namespaceservidor__novo_a966ece9821b12a79fc5b83f628987f7d}{servidor\_\-novo.sonarDev} = ArSonarDevice(7)
\item 
tuple \hyperlink{namespaceservidor__novo_a538b68fbb948f9cf0573a119042c9e0d}{servidor\_\-novo.server} = ArServerBase()
\item 
tuple \hyperlink{namespaceservidor__novo_a4644b693a5931c2f1a9d68500534d955}{servidor\_\-novo.packet} = ArNetPacket()
\item 
tuple \hyperlink{namespaceservidor__novo_a201a5b59bc2f5a75d2a4c82d81f17180}{servidor\_\-novo.con} = ArSimpleConnector(sys.argv)
\item 
tuple \hyperlink{namespaceservidor__novo_aae2aad1feed83599e4783f4d39086ee6}{servidor\_\-novo.serverInfoRobot} = ArServerInfoRobot(server, robot)
\item 
tuple \hyperlink{namespaceservidor__novo_a31795a5817e65297017799e1fdbc6fdd}{servidor\_\-novo.serverInfoSensor} = ArServerInfoSensor(server, robot)
\item 
tuple \hyperlink{namespaceservidor__novo_aaddb28ac7dd14daa90eaadd8fde37bbd}{servidor\_\-novo.drawings} = ArServerInfoDrawings(server)
\item 
tuple \hyperlink{namespaceservidor__novo_a63f8dd02ccddbc7be4de022c23246578}{servidor\_\-novo.modeStop} = ArServerModeStop(server, robot)
\item 
tuple \hyperlink{namespaceservidor__novo_a17ac782014a1ecc8ae25509094a6812b}{servidor\_\-novo.modeRatioDrive} = ArServerModeRatioDrive(server, robot)
\item 
tuple \hyperlink{namespaceservidor__novo_a1f0f40c9f2b17d9579e189b4ec8f4ffe}{servidor\_\-novo.modeWander} = ArServerModeWander(server, robot)
\item 
tuple \hyperlink{namespaceservidor__novo_a1cd792305e65be0c6e06c0cd885b1593}{servidor\_\-novo.commands} = ArServerHandlerCommands(server)
\item 
tuple \hyperlink{namespaceservidor__novo_a9cc0dcde8396c1ff04833e8811a1df74}{servidor\_\-novo.uCCommands} = ArServerSimpleComUC(commands, robot)
\item 
tuple \hyperlink{namespaceservidor__novo_a0407c97de55cf4fa7aac903f2813aa50}{servidor\_\-novo.loggingCommands} = ArServerSimpleComMovementLogging(commands, robot)
\item 
tuple \hyperlink{namespaceservidor__novo_a20fff7251fe63ba8161a62b254023d04}{servidor\_\-novo.gyroCommands} = ArServerSimpleComGyro(commands, robot, gyro)
\item 
tuple \hyperlink{namespaceservidor__novo_a0b77e4ad4ff622636ae781f9e76c0bba}{servidor\_\-novo.configCommands} = ArServerSimpleComLogRobotConfig(commands, robot)
\end{DoxyCompactItemize}

\printindex
\end{document}
